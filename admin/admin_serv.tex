
\documentclass[ucs]{beamer}

\usetheme{GSyC}
%\usebackgroundtemplate{\includegraphics[width=\paperwidth]{gsyc-bg.png}}


\usepackage[greek,spanish]{babel}   %greek permite usar euro 
\usepackage[utf8x]{inputenc}
\usepackage{graphicx}
\usepackage{amssymb} % Simbolos matematicos


% Metadatos del PDF, por defecto en blanco, pdftitle no parece funcionar
   \hypersetup{%
     pdftitle={Administración de Servicios y Aplicaciones},%
     %pdfsubject={Diseño y Administración de Sistemas y Redes},%
     pdfauthor={GSyC},%
     pdfkeywords={},%
   }
%


% Para colocar un logo en la esquina inferior de todas las transpas
%   \pgfdeclareimage[height=0.5cm]{gsyc-logo}{gsyc}
%   \logo{\pgfuseimage{gsyc-logo}}


% Para colocar antes de cada sección una página de recuerdo de índice
%\AtBeginSection[]{
%  \begin{frame}<beamer>{Contenidos}
%    \tableofcontents[currentframetitle]
%  \end{frame}
%}



\begin{document}

% Entre corchetes como argumento opcional un título o autor abreviado
% para los pies de transpa
\title[Administración de Servicios y Aplicaciones]{Administración de Servicios y Aplicaciones}
%\subtitle{Diseño y Administración de Sistemas y Redes}
\author[GSyC]{Escuela Tec. Sup. Ingeniería de Telecomunicación}
\institute{gsyc-profes (arroba) gsyc.es}
\date[2014]{Diciembre de 2014}


%% TÍTULO
\begin{frame}
  \titlepage
  % Oportunidad para poner otro logo si se usó la opción nologo
  % \includegraphics[width=2cm]{logoesp}  
\end{frame}



%% LICENCIA DE REDISTRIBUCIÓN DE LAS TRANSPAS
%% Nota: la opción b al frame le dice que justifique el texto
%% abajo (por defecto c: centrado)
\begin{frame}[b]
\begin{flushright}
{\tiny
\copyright \insertshortdate~\insertshortauthor \\
  Algunos derechos reservados. \\
  Este trabajo se distribuye bajo la licencia \\
  Creative Commons Attribution Share-Alike 4.0\\
}
\end{flushright}  
\end{frame}



%% ÍNDICE
%\begin{frame}
%  \frametitle{Contenidos}
%  \tableofcontents
%\end{frame}





%%---------------------------------------------------
\section{Empaquetado de ficheros}
%%---------------------------------------------------

\begin{frame}[fragile]
\frametitle{Empaquetado de ficheros}

Almacenar varios ficheros en uno solo, no necesariamente
con compresión

Utilidad:
\begin{itemize}
\item
Más cómodo de manejar (copiar, enviar por correo, etc)
\item
Conservar metainformación (permisos) o incluso mayúsculas/minúsculas, tildes, etc
si los ficheros van a pasar por un sistema de ficheros diferente
\begin{itemize}
\item
ISO9660 (cdrom)
\item
vfat (Windows, discos externos,   pendrives)
\item
ntfs (Windows)
\end{itemize}

\end{itemize}



\end{frame}
%%----------------------------------------------
\begin{frame}[fragile]

\frametitle{gzip}


Comprime o descomprime 1 fichero

Extensión: \verb|fichero.z    fichero.gz|
\begin{itemize}	
\item
Comprimir y descomprimir (borrando el original):

\verb|gzip fichero |

\verb|gunzip fichero.gz|

\end{itemize}
\end{frame}
%%---------------------------------------------------------------
\begin{frame}[fragile]
\begin{itemize}
\item
Comprimir y descomprimir (manteniendo el original):

\verb|gzip -c fichero > fichero.gz   | 

\verb|zcat fichero.gz > fichero| 

\verb@zcat fichero.gz  | less @ 

% zcat = gunzip -c


\end{itemize}

\end{frame}
%%---------------------------------------------------------------
\begin{frame}[fragile]

\frametitle{tar + gzip}
Comprime o descomprime varios ficheros, directorios\\
Extensión: \verb|fichero.tar.gz    fichero.tgz|
\begin{itemize}	
\item
Comprimir:\\ \verb|tar -cvzf fichero.tgz fichero1 fichero2| 
\item
Descomprimir:\\ \verb|tar -xvzf fichero.tgz |
\item
Mostrar contenido:\\ \verb|tar -tzf fichero.tgz| 
\end{itemize}


\end{frame}
%%---------------------------------------------------------------
\begin{frame}[fragile]


\frametitle{WinZip}
\begin{itemize}	
\item 
Por motivos de licencias, originalmente no había compresores para
Linux.
(Pero las aplicaciones Windows saben descomprimir descomprimir \verb|.tgz|)
\item 
Descomprimir: 
\verb|   unzip fichero.zip |
\end{itemize}

\end{frame}


%%---------------------------------------------------------------
\begin{frame}[fragile]
\frametitle{bz2}
Formato que ofrece compresión más alta que .gz, (empleando 
más CPU y memoria)
% En windows el WinImp lo abre, el winzip se lía
\begin{itemize}	
\item 
Comprimir y descomprimir 1 fichero, borrando el original

\verb|bzip2 fichero|

\verb|bunzip2 fichero.bz2 |
\item 
Comprimir y descomprimir 1 fichero, manteniendo el original

\verb|bzip2 -c fichero > fichero.bz2|

\verb|bunzip2 -c fichero.bz2 >  fichero |
\item 
Comprimir y descomprimir varios ficheros, manteniendo el original

\verb@tar -c fichero1 fichero2  |  bzip2 > fichero.bz2  @


\verb@tar -xjf fichero.bz2@
\end{itemize}
\end{frame}




%%---------------------------------------------------------------
\begin{frame}[fragile]
\frametitle{Disquetes }

\begin{itemize}
\item
Podemos montar la disquetera como un sistema de ficheros Unix

\item
Pero suele ser más práctico copiar un \verb|.tgz| en modo \verb|msdos|

\verb|   mdir a:  |

\verb|   mcopy fichero a:  |

\verb|   mcopy a:fichero   |

\end{itemize}

\end{frame}

%-------------------------------------------

\begin{frame}[fragile]

\frametitle{Fragmentación de ficheros }
Si necesitas trocear una imagen de gran tamaño en ficheros que quepan en un 
\emph{pendrive} o cdrom

\begin{itemize}
\item
Empaquetar y comprimir un directorio:\\ \verb|tar -cvzf mi_imagen.tgz mi_directorio|
\item
Mostrar contenido:\\ \verb|tar -tzf mi_imagen.tgz|

\item
Trocear:\\
\verb|#     tamaño    fichero        prefijo|
\verb|split -b 500MB  mi_imagen.tgz  mi_imagen.tgz.|
\begin{footnotesize}
(Observa que el segundo parámetro es igual al primero, pero
añadiendo un punto)
\end{footnotesize}


\item
Habremos generado
  \begin{footnotesize}
  \begin{verbatim}
mi_imagen.tgz.aa  mi_imagen.tgz.ab  mi_imagen.tgz.ac
  \end{verbatim}
  \end{footnotesize}
\end{itemize}
\end{frame}



%%----------------------------------------------
\begin{frame}[fragile]
En la máquina destino (no importa si en el \emph{host} el S.O. es distinto)
\begin{itemize}
\item
Unir los fragmentos
\verb|cat mi_imagen.tgz.* > mi_imagen.tgz|

(En MS Windows para este paso podemos emplear HjSplit, Free File Splitter o cualquier otro programa similar)
\item
Descomprimir y desempaquetar:\\ \verb|tar -xvzf mi_imagen.tgz |

(En MS Windows podemos usar 7-Zip  o similares)
\end{itemize}

\end{frame}


%%---------------------------------------------------------------
\section{Instalación de paquetes}
%%---------------------------------------------------------------
\begin{frame}[fragile]

\frametitle{Instalación de paquetes}
\begin{itemize}	
\item
Método clásico para instalar programas:

Formato .tgz

Descomprimir y seguir las instrucciones del fichero README

Suele ser del estilo de
\begin{verbatim}
./configure
make compile
make install
\end{verbatim}
\item 
Sistema de gestión de paquetes

Colección de herramientas que automatizan
la instalación, actualización y eliminación de programas.
\end{itemize}


\end{frame}

%%---------------------------------------------------------------
\begin{frame}[fragile]

\begin{itemize}	

\item 
Gestión de paquetes, Debian y derivados 

Paquetes en formato .\verb|deb|

Se pueden manejar directamente con \verb|dpkg|, o con \verb|apt-get, aptitude,|
\verb|dselect|, o \verb|synaptic| 

\item
Gestión de paquetes, RedHat y derivados

Paquetes en formato .\verb|rpm|

Se pueden manejar directamente con \verb|rpm|, o con \verb|up2date| o \verb|yum| 
%up2date son los repositorios de redhat y de los primeros fedores
%yum es para fedora


\end{itemize}


\end{frame}
%---------------------------------------------
\subsection{El sistema de paquetes de Debian}
%---------------------------------------------

\begin{frame}[fragile]
\frametitle{El sistema de paquetes de Debian}

%\begin{itemize}
%\item Dos tipos de paquetes:
%  \begin{itemize}
%  \item Binarios (.deb)
%  \item Fuentes (.dsc, .orig.tar.gz, .diff.gz)
%  \end{itemize}
%\item 
Los paquetes mantienen \emph{dependencias} entre sí, de forma que
  la instalación de un paquete puede:
  \begin{itemize}
  \item \emph{depender} de que se instale también otro
  \item \emph{recomendar} que se instale también otro
%se instala por omisión. El paquete podría funcionar sin ello,
%pero para hacer algo util normalmente lo querrán isntalar
  \item \emph{sugerir} que se instale también otro
  \item \emph{entrar en conflicto} con otro actualmente instalado
  \end{itemize}
%\item Los nombres de los paquetes Debian siguen el siguiente convenio:\\
%\verb|<nombre>_<NúmeroVersión>-<NúmeroRevisiónDebian>.deb|
%\end{itemize}


\end{frame}



%-----------------------------------------------------------------
\subsection{dpkg}
%-----------------------------------------------------------------

\begin{frame}[fragile]
\frametitle{dpkg}

\begin{itemize}
\item Es la herramienta básica de gestión de paquetes, que es usada
  por las otras (dselect, apt-get, aptitude, synaptic).
\item Usos principales:
  \begin{itemize}
  \item 
\verb|dpkg -i paquete_VVV-RRR.deb|

Instala un paquete
  \item 
\verb|dpkg -r paquete|

Desinstala  (\emph{remove}) un paquete, elimina todo excepto los ficheros de configuración
  \item 
\verb|dpkg -P paquete| 

Purga un paquete, eliminando incluso los ficheros de configuración
  \end{itemize}
\item Tiene muchas opciones. Puede esquivarse el esquema de
  dependencias (peligroso) con las opciones que empiezan por
  \verb|--force-...|
\end{itemize}

\end{frame}


%%---------------------------------------------------------------
\begin{frame}[fragile]

Versiones de Ubuntu:


\begin{verbatim}	
nombre año.mes 

Warty Warthog 4.10     Hoary Hedgehog 5.04
Breezy Badger 5.10     Dapper Drake 6.04 
Edgy Eft 6.10          Feisty Fawn 7.04
Gutsy Gibbon 7.10      Hardy Heron  8.04  LTS
Intrepid Ibex 8.10     Jaunty Jackalope 9.04
Karmic Koala 9.10      Lucid Lynx 10.04 LTS
Maverick Meerkat 10.10 Natty Narwhal 11.04
Oneiric Ocelot 11.10   Precise Pangolin 12.04 LTS
Quantal Quetzal 12.10  Raring Ringtail 13.04
Saucy Salamander 13.10 Trusty Tahr 14.04 LTS
Utopic Unicorn 14.10   Vivid Vervet 15.04
\end{verbatim}

Versión estándar: Desde 13.04, soportada durante 9 meses
(18 meses en las versiones anteriores)

LTS: Long Term Support: 3 años en escritorio y 5 en servidor

\end{frame}
%%----------------------------------------------
\begin{frame}[fragile]
Ubuntu Desktop / Ubuntu Server Edition / Ubuntu Server Edition JeOS

Variantes de Ubuntu: Kubuntu, Xubuntu, Gobuntu, Ubuntu Studio 



\end{frame}


%--------------------------------------------------------------
\subsection{apt}
%--------------------------------------------------------------

\begin{frame}[fragile]
\frametitle{apt}

\begin{itemize}
\item La herramienta más sencilla de usar y más potente.
\item Usa \emph{repositorios}: sitios centralizados
donde se almacenan paquetes
\item Las direcciones de los repositorios se indican en el       
  fichero 

\verb|/etc/apt/sources.list|
\item
Los repositorios de ubuntu se dividen en 4 componentes
\begin{enumerate}
\item
\emph{Main}. Soportado oficialmente por ubuntu. Libre
\item
\emph{Restricted}. Soportado oficialmente. No libre
\item
\emph{Universe}. No soportado oficialmente. Libre
\item
\emph{Multiverse}. No soportado oficialmente. No libre
\end{enumerate}

Además, se pueden añadir componentes de terceros 

\end{itemize}

\end{frame}

%%%%%%%%%%%%%%%%%%%%%%%%%%%%%%%%%%%%%%%%%%%%%%%%%%%%%%%%%%%%%%%%%%%

\begin{frame}[fragile]

\begin{scriptsize}
\begin{verbatim}
# deb cdrom:[Ubuntu 6.06 _Dapper Drake_ - Release i386 (20060531)]/ dapper main restricted
deb http://archive.ubuntu.com/ubuntu edgy main restricted
deb http://security.ubuntu.com/ubuntu edgy-security main restricted
deb http://archive.ubuntu.com/ubuntu edgy-updates main restricted

## All community supported packages, including security- and other updates
deb http://archive.ubuntu.com/ubuntu edgy universe multiverse
deb http://security.ubuntu.com/ubuntu edgy-security universe multiverse
deb http://archive.ubuntu.com/ubuntu edgy-updates universe multiverse

# Google Picasa for Linux repository
deb http://dl.google.com/linux/deb/ stable non-free
\end{verbatim}
\end{scriptsize}

\end{frame}

%%%%%%%%%%%%%%%%%%%%%%%%%%%%%%%%%%%%%%%%%%%%%%%%%%%%%%%%%%%%%%%%%%%

\begin{frame}[fragile]
\frametitle{Uso básico de apt}

El primer \emph{front-end} fue \verb|dselect|, muy potente y con un pésimo
interfaz de usuario

Desde línea de mandatos se puede usar \verb|apt-get|

A partir de 2005, debian recomienda usar \verb|aptitude|, que tiene la
misma sintaxis que \verb|apt-get|

\begin{itemize}
\item 
\verb|aptitude update|  $\equiv$  \verb|apt-get update|

Actualizar lista de paquetes: 
\item 
\verb|aptitude safe-upgrade|

Actualizar todos los paquetes instalados a la última versión disponible
(sin cambiar de distribución)
\item 
\verb|aptitude install paquete|

Instalar un paquete (resolviendo conflictos)  

\end{itemize}
\end{frame}


%%---------------------------------------------------------------
\begin{frame}[fragile]

Aunque indiquemos a nuestro sistema de paquetería que instale
la última versión de un paquete, tal vez no sea posible. Se dice
que el paquete está \emph{retenido} (\emph{hold})

\begin{itemize}	
\item
El paquete depende de otro no incluido en la distribución actual
\item 
El administrador lo ha retenido \emph{a mano} (no le gusta,
da problemas...)

Un paquete retenido para \verb|apt-get| puede no estar retenido
para \verb|aptitude|. Y viceversa. 
\begin{itemize}	
\item
\begin{verbatim}
aptitude:
sudo aptitude hold nombre_del paquete
sudo aptitude unhold nombre_del paquete
\end{verbatim}


\item
\begin{verbatim}
apt-get:
sudo install feta
sudo feta hold nombre_del paquete
sudo feta unhold nombre_del paquete
\end{verbatim}


\end{itemize}

\end{itemize}


\end{frame}


%%---------------------------------------------------------------
\begin{frame}[fragile]

\begin{itemize}	
\item 
\verb|aptitude remove paquete|

Desinstalar un paquete (resolviendo conflictos)  

\item 
\verb|aptitude --purge remove paquete|

Purgar un paquete (resolviendo conflictos)  
\item 
\verb|aptitude  dist-upgrade|

Actualiza \emph{agresivamente} todos los paquetes instalados, lo que
puede incluir el paso a la versión más reciente de la distribución

\item 
\verb|aptitude clean|

Borrar las copias descargadas de los .deb
\end{itemize}

\end{frame}

%%%%%%%%%%%%%%%%%%%%%%%%%%%%%%%%%%%%%%%%%%%%%%%%%%%%%%%%%%%%%%%%%%%

\begin{frame}[fragile]
\frametitle{Otros mandatos interesantes}
En los repositorios hay muchos paquetes
¿Cómo saber cuál necesito?

\begin{itemize}
\item 
\verb|aptitude search cadena| 

Buscar una cadena en el nombre o descripción de un paquete. Indica
el estado del paquete (instalado, no instalado, borrado...)
%(usa regexps)

%ejemplo
%  apt-cache search ssh --names-only|grep server


%\item 
%\verb|apt-cache search cadena.* --names-only| 
%
%Buscar una cadena en el nombre de paquete
\item 
\verb|aptitude show paquete|

Muestra descripción del paquete

\item 
\verb|dpkg-reconfigure paquete|

Reconfigurar un paquete

%\item 
%Buscar si existe un cierto nombre de paquete.  (usa comodines)
%
%\verb|dpkg -l paquete*|\
\end{itemize}

\end{frame}

%%%%%%%%%%%%%%%%%%%%%%%%%%%%%%%%%%%%%%%%%%%%%%%%%%%%%%%%%%%%%%%%%%%

%\begin{frame}[fragile]

%\begin{itemize}
%\item 
%\verb|dpkg -L paquete|
%
%Listar los contenidos de un paquete instalado

%\item 
%\verb|dpkg -S fich|
%
%Buscar a qué paquete pertenece un fichero
%\end{itemize}

%\end{frame}






%%---------------------------------------------------------------


%\begin{frame}[fragile]
%%\section{Instalación de paquetes rpm}
%Paquetes RPM
%\begin{itemize}
%\item
%Instalar paquete\\
%\verb|rpm -Uvh  nombre.rpm|
%\item
%Desinstalar\\
%\verb|rpm -e nombre.rpm|
%\item
%Preguntar si está instalado\\
%\verb|rpm -q paquete.rpm  | 
%\item
%Mostrar todos los paquetes instalados\\
%\verb|rpm -qa | 
%\end{itemize}

%\end{frame}


%%---------------------------------------------------------------
\begin{frame}[fragile]
\subsection{Sistemas de paquetes en OS X}
\frametitle{Sistemas de paquetes en OS X}

Apple no tiene previsto el uso de estos sistemas para
usuarios \emph{normales}. Pero sí son muy útiles para usuarios
con perfil de desarrollador o administrador. Se usan prácticamente
igual que apt-get o aptitude

Actualmente podemos 
optar por tres sistemas
\begin{enumerate}
\item
fink 

El más antiguo. Basado en apt-get. Poco usado hoy
\item
macports

Basado en los ports de FreeBSD. Muy completo. Muy independiente de Apple

\item
homebrew

El más moderno y mejor integrado con Apple. Tal vez el más popular
hoy
\end{enumerate}

\end{frame}


%%---------------------------------------------------------------
\section{Localizar ficheros}
%%---------------------------------------------------------------
\begin{frame}[fragile]
\frametitle{Localizar ficheros}
\begin{itemize}
\item 
\verb|find                   | Busca un fichero  \\
\verb!find . | grep fichero  ! Filtra la búsqueda
\item 
\verb|locate     | Busca un fichero (en una base de datos)
\item 
\verb|updatedb   | Actualiza la base de datos

\end{itemize}
\end{frame}


%%---------------------------------------------------------------
\section{Hora. Parada del sistema}
%%---------------------------------------------------------------

\begin{frame}[fragile]

\frametitle{Hora. Parada del sistema}
  \begin{itemize}
        \item \texttt{\textbf{shutdown -P now}} $\equiv$
          \texttt{\textbf{poweroff}}

Apaga el sistema
        \item \texttt{\textbf{shutdown -r now}} $\equiv$
          \texttt{\textbf{reboot}} 

Reinicia el sistema
\item 
\verb|sleep n|

Duerme la shell segundos  
\item 
\verb!sleep 28800 ; halt ! \\Detiene la máquina al cabo de 8 horas
  \end{itemize}
\end{frame}


%%---------------------------------------------------------------

\begin{frame}[fragile]
\begin{itemize}
  
\item 
Poner fecha y hora:
\begin{itemize}
\item
Automáticamente: Demonio \emph{ntpd}, cliente de \emph{Network Time Protocol}

\item
Manualmente

\verb|date -s AAAA-MM-DD| \\
\verb|date -s HH:MM| 
\end{itemize}

\end{itemize}
\end{frame}



%%---------------------------------------------------------------
\begin{frame}[fragile]
Casi siempre hay varias soluciones para una tarea. Generales o particulares
\begin{itemize}	
\item 
\verb@find . |grep cadena@

\verb@find . -name cadena\*@
\item 
\verb@sleep 60 | shutdown -h now@

\verb@shutdown -h 1@

\item 
etc, etc
\end{itemize}
Todas sirven. ¿No? ¿Cuál es \emph{mejor}?
\begin{itemize}	
\item 
Cuando somos novatos en un sistema, con una solución general
sabremos resolver ese problema y otros parecidos
\item 
Cuando conocemos mejor un sistema y dominamos las soluciones
generales, las soluciones particulares
suelen ser más eficientes
\end{itemize}

\end{frame}

%%---------------------------------------------------------------
%\begin{frame}[fragile]
%\section{rsync}
%sincroniza un directorio respecto a otro

%\verb|rsync [opciones] [origen] [destino]|

%\begin{itemize}
%\item 
%pueden estar en diferentes máquinas
%\item 
%útil para \emph{backups} y \emph{mirrors}
%\end{itemize}

%\begin{scriptsize}
%\begin{verbatim}
%#!/bin/bash
%rsync -e ssh --verbose --archive --delete --exclude *.mp3 --exclude *.MP3  \
%mortuno@pantuflo:/tmp/aa /tmp/bb
%
%# a:archive. preserva permisos, subdirectorios...
%#  --delete: borra en el destino lo que no esté en el origen
%\end{verbatim}
%\end{scriptsize}
%\end{frame}





%%---------------------------------------------------------------
\section{Copias de seguridad}
%%---------------------------------------------------------------
\begin{frame}[fragile]
\frametitle{Copias de seguridad}

\begin{itemize}	
\item
\verb|tar| o similares

Problema: Siempre se duplican los directorios enteros
\item 
\verb|rsync|

\emph{Mirror} unidireccional. Permite mantener una réplica 
de un directorio. Solo se actualizan las novedades.
No permite modificar la réplica
\item 
\verb|FreeFileSync, Synkron|

Herramientas libres para sincronización bidireccional (Windows, Linux, OS X). 

Sincronizan dos (o más) directorios: Cualquiera de los dos directorios puede modificarse

\end{itemize}
\end{frame}

%-------------------------
\begin{frame}[fragile]
\begin{itemize}


\item 
Sistemas de almacenamiento permanente

\verb|Time Machine| (OS X)

\verb|dumpfs| (bsd) 

\verb|pdumpfs| (Linux, Windows)

\verb|TimeVault, FlyBack| (Linux)

%timevault es para gnome ubuntu
%flyback es para linux en general
\verb|venti| (Plan 9)

\begin{itemize}
\item 
Se registran los cambios en los ficheros, sin borrar nunca nada
\item 
Mantienen una 
\emph{foto} del estado diario del sistema de ficheros, en
un directorio con formato yyyy/mm/dd 
\item 
Parece mucho, pero hoy el almacenamiento es muy barato. P.e. si 
generamos 10 Mb diarios, necesitamos unos 4Gb
anuales
\end{itemize}
\end{itemize}


\end{frame}




%%---------------------------------------------------------------
\section{Administración de los demonios}
%%---------------------------------------------------------------
\begin{frame}[fragile]
\frametitle{Administración de los demonios}

Los demonios son programas relativamente \emph{normales}, con algunas particularidades
\begin{itemize}	
\item
Ofrecen servicios (impresión, red, ejecución periódica de tareas, logs, etc)
\item 
Suelen estar creados por el proceso de arranque \emph{init}  (ppid=1)
\item 
Sus nombres suelen acaban en \emph{d}
\item 
Se ejecutan en \emph{background}
\item 
No están asociados a un usuario en una terminal
\item
El grueso de su configuración suele hacerse desde un 
único fichero 

En el caso de debian,
\verb|/etc/midemonio.conf|
\item 
Se inician y se detienen de manera uniforme
%Se inicia y se detiene desde un script en \verb|/etc/init.d|
\end{itemize}

Pueden organizarse de dos maneras: 

\begin{itemize}
\item
System V
\item
Upstart
\end{itemize}

\end{frame}


%%---------------------------------------------------------------
\begin{frame}[fragile]
\frametitle{Unix System V}
Versión de Unix comercializada en 1983 por AT\&T 


\begin{itemize}
\item
La mayoría de los Unix, incluyendo Linux, son derivados de System V
\item
Otros Unix son derivados del Unix BSD de aquella época: esto incluye los BSD actuales
y OS X (Apple) 
\end{itemize}


System V introduce
una forma de organizar los demonios basada en 
\begin{itemize}
\item
\emph{Niveles de ejecución}
\item
Scripts en \verb|/etc/init.d|
\item
Ordenación lineal de sucesos: 

Las tareas se ordenan secuencialmente en orden preestablecido, solo cuando una está completamente
acabada empieza la siguiente
\end{itemize}

\end{frame}

%%---------------------------------------------------------------
\begin{frame}[fragile]
\frametitle{Upstart}

%http://www.netsplit.com/blog/articles/2006/08/26/upstart-in-universe
\subsection{upstart}

\begin{itemize}
\item
El sistema de arranque tradicional de Linux (System V)
no es adecuado para las máquinas actuales
\begin{itemize}
\item
Son externos: aparecen y desaparecen
\item
Están en red
\item
Ahorran energía
\item \ldots
\end{itemize}

% Storage buses allow more than a fixed number of drives, so they must be scanned for; this operation frequently does not block.
% Firmware may need to be loaded after the device has been detected, but before it is usable by the system.
% Mounting a partition in /etc/fstab may require tools in /usr which is a network filesystem that cannot be mounted until after networking has been brought up.

\item
\emph{Upstart} es un sistema de arranque basado en eventos
(pueden suceder en cualquier orden, puede haber tareas
en paralelo)
\item
Mantiene una capa adicional de sofware para que las órdenes
al estilo System V sigan funcionando



\end{itemize}
\end{frame}
%%----------------------------------------------
\begin{frame}[fragile]

\begin{itemize}

\item
Desarrollado por Ubuntu, con el propósito de extenderlo
a todos los Linux

Aparece en Ubuntu 6.10 \emph{edgy} (Octubre de 2006)
\item
También lo usan los S.O. de google (chrome, chromium) y algunas versiones
de Red Hat

La mayoría de los Linux siguen empleando System V
\item
En Fedora y Debian es opcional y poco frecuente. Upstart es
una de las mayores diferencias entre Debian y Ubuntu

\item
Alternativas: \emph{launchd} (OS X), \emph{initng},  SMF, systemd 
%smf: solaris

%\item
%Está previsto que reemplace a \emph{cron} y tal vez a  \emph{inetd}, manteniendo siempre la compatibilidad
\end{itemize}
En 2014, Debian anuncia que pasará a usar systemd, y Ubuntu anuncia que en el futuro abandonará upstart y también usará systemd


\end{frame}

%%---------------------------------------------------------------
\begin{frame}[fragile]
\frametitle{Ficheros de configuración}
Los ficheros donde se configura un demonio son:
\begin{itemize}
\item
Fichero principal de configuración 

\verb|/etc/midemonio.conf|

\item
Configuración de puesta en marcha y parada 


\begin{itemize}
\item
System V

\verb|/etc/init.d/midemonio|
\item
Upstart

\verb|/etc/init/midemonio.conf|

\end{itemize}


\item
Configuración del administrador local

\verb|/etc/default/midemonio|

Solo existe en Debian y
derivados (también en Ubuntu con Upstart). No lo usan todos los
paquetes

\end{itemize}
\end{frame}


%%---------------------------------------------------------------
\begin{frame}[fragile]
\frametitle{Directorio /etc/default}
\begin{itemize}
\item
La inmensa mayoría de los parámetros de \verb|/etc/midemonio.conf|
y de 
\verb|/etc/init.d/midemonio|
o de
\verb|/etc/init/midemonio.conf|
los ha escrito el desarrollador del demonio o el empaquetador de
la distribución, es normal que el administrador local
de cada máquina concreta solo modifique unos pocos
\item
En algún momento habrá que actualizar el demonio a una versión nueva,
que frecuentemente incluirá cambios en sus ficheros de configuración,
escritos por el desarrollador o el empaquetador


\begin{itemize}
\item
¿Instalamos los ficheros nuevos y  \emph{machacamos} los viejos?

Problema: se pierde la configuración que ha personalizado el
administrador local
\item
¿Mantenemos los viejos y descartamos los nuevos?

Problema: se pierden los cambios de la versión actual, el fichero
de configuración (antiguo) podría incluso se incompatible con el demonio
(actual)
\end{itemize}
\end{itemize}

\end{frame}



%%---------------------------------------------------------------
\begin{frame}[fragile]
\frametitle{}
Solución: fichero \verb|/etc/default/midemonio|
\begin{itemize}
\item
Es un fichero muy corto, con muy pocos parámetros, muy importantes,
que se sabe que serán modificados por el administrador local
\item
Cuando se instalan versiones nuevas del demonio, este fichero se mantiene
\item
Los cambios introducidos por las nuevas versiones de los demonios estarán
en \verb|/etc/midemonio.conf| o en 
\verb|/etc/init.d/midemonio|
o 
\verb|/etc/init/midemonio.conf|


\end{itemize}

\end{frame}



%%---------------------------------------------------------------
\begin{frame}[fragile]
\frametitle{Administración estilo System V}
El código de un demonio puede estar en cualquier lugar
del sistema de ficheros.
Pero siempre se coloca en \verb|/etc/init.d/midemonio| un script
para manejarlo

\begin{itemize}	
\item
\verb|/etc/init.d/midemonio start|

Inicia el servicio
\item
\verb|/etc/init.d/midemonio stop|

Detiene el servicio
\item 
\verb|/etc/init.d/midemonio restart|

Detiene e inicia el servicio. Suele ser 
%\textcolor{red}{necesario para releer los ficheros de configuración}
\textbf{\textcolor{red}{necesario para releer los ficheros de configuración}}
si se han modificado (\verb|/etc/midemonio.conf|)


\end{itemize}

Con frecuencia también está disponible
\begin{itemize}	
\item
\verb|/etc/init.d/midemonio reload|

Lee el fichero de configuración sin detener el servicio
\end{itemize}
\end{frame}

%%---------------------------------------------------------------
\subsection{Niveles de ejecución}
%%---------------------------------------------------------------

\begin{frame}[fragile]
\frametitle{Niveles de ejecución}
%Esto es el System V Init
%Otras distribuciones, como Gentoo o Slackware, tienen el método de inicio bastante distinto (Gentoo es SystemV modificado, y Slackware usa el BSD Init),

%http://en.wikipedia.org/wiki/Runlevel
%http://tldp.org/HOWTO/From-PowerUp-To-Bash-Prompt-HOWTO-6.html#ss6.1

% http://www.debian.org/doc/debian-policy/ch-opersys.html
¿Qué demonios se ponen en marcha cuando se inicia el sistema?

%La mayoría de los Linux usan el sistema de arranque de System V

%BSD tiene su propio sistema (que también lo emplea Slackware)

Un Nivel de ejecución (\emph{runlevel}) es una configuración de arranque.
Para cada nivel, se define un conjunto de demonios que deben ejecutarse


\end{frame}


%%---------------------------------------------------------------
\begin{frame}[fragile]
Supongamos una fábrica. 
Diferentes niveles (estados), no secuenciales.
Al entrar en un nivel se apagan ciertos sistemas y se encienden otros
\begin{scriptsize}
\begin{verbatim}
    Nivel 1 - Noche
        Al entrar en este nivel apagar
            01 motores
            02 luces principales
        Al entrar en este nivel encender
            01 alarma
            02 luces_auxiliares
    Nivel 2 - Producción normal
        Al entrar en este nivel apagar
            01 alarma
            02 luces auxiliares
        Al entrar en este nivel encender
            ....
    Nivel 3-  Mantenimiento     
        Al entrar en este nivel apagar
            01 motores
            ....
\end{verbatim}
\end{scriptsize}
    %Nivel 4 - Vacaciones, solo mantenimiento
\end{frame}


%%---------------------------------------------------------------
\begin{frame}[fragile]
El responsable de conectar y desconectarlo todo será
el vigilante de seguridad, así que hay que dejarle
unas instrucciones muy claras:

\vspace{0.6cm}

\begin{scriptsize}

\verb%ordinal            [encender|apagar]           nombre_del_sistema%
\vspace{0.2cm}

\begin{tabular}[h]{|l|l|l|}\hline
\bf{Código} & \bf{Significado} & \bf{Mandato } \\
\hline \hline
S10motor-ppal &1º encender motor principal & \verb|/etc/init.d/motor-ppal start|\\ \hline
S20motor-aux & 2º encender motor auxiliar & \verb|/etc/init.d/motor-aux start|\\ \hline
\hline
K10alarma    & 1º apagar alarma& \verb|/etc/init.d/alarma stop|\\ \hline
  \end{tabular}

\end{scriptsize}
\vspace{0.6cm}

Dentro de cada nivel, las tareas se ordenan desde 00 hasta 99 (con un cero a la izquierda para
los valores del 0 al 9)

\end{frame}

%-------------------------
\begin{frame}[fragile]

El \emph{vigilante de seguridad} es el proceso \verb|init|.  

Hay un directorio por nivel:

Debian y derivados
\begin{scriptsize}
    \begin{verbatim}
    /etc/rc0.d
    /etc/rc1.d
    ...
\end{verbatim}
\end{scriptsize}

Red Hat y derivados
\begin{scriptsize}
\begin{verbatim}
    /etc/rc.d/rc0.d
    /etc/rc.d/rc1.d
    ...
\end{verbatim}
\end{scriptsize}


Hay otro directorio cuyos servicios se activan siempre, en cualquier nivel

\begin{scriptsize}
    \begin{verbatim}
     /etc/rcS.d
\end{verbatim}
\end{scriptsize}

\end{frame}

%-------------------------
\begin{frame}[fragile]
%\item \texttt{/etc/rcS.d}, \texttt{/etc/rc0.d}, \texttt{/etc/rc1.d}, \texttt{/etc/rc.d/rc2.d} \dots

Dentro de los directorios hay enlaces simbólicos
\begin{itemize}
\item
Apuntan al script en \verb|/etc/init.d| que controla el demonio
\item
Cada nombre del enlace indica conexión/desconexión, ordinal y script a manejar
\end{itemize}

Cuando entra en el nivel N, el proceso init se encarga de

\begin{itemize}

\item Ejecutar por orden
todos los scripts en \verb|/etc/rcN.d| que empiezen por K (de Kill).
Les pasa el parámetro \emph{stop}

\item A continuación, ejecuta por orden
todos los scripts en \verb|/etc/rcN.d| que empiezen por S (de Start). 
Les pasa
el parámetro \emph{start}




\end{itemize}
\begin{verbatim}
who -r
\end{verbatim}
Indica el nivel de ejecución actual
\end{frame}

%-------------------------
\begin{frame}[fragile]
Niveles comunes:
\begin{scriptsize}
\begin{verbatim}
      0   Halt (Parada del sistema)
      1   Modo monousuario, usuario root, sin red
      2-5 Diversos modos multiusuario
      6   Reboot (Reiniciar el sistema) 
\end{verbatim}
\end{scriptsize}

Red Hat y derivados:
\begin{scriptsize}
\begin{verbatim}
      2   No se utiliza (definible por el administrador)
      3   Modo multiusuario completo, solo consola de texto
      4   No se utiliza (definible por el administrador)
      5   Modo multiusuario completo, con X Window
\end{verbatim}
\end{scriptsize}

Debian y derivados:

\begin{scriptsize}
\begin{verbatim}
      No hay diferencia del 2 al 5. Se usa el 2 por omisión,
      el resto queda libre para el administrador
\end{verbatim}
\end{scriptsize}

\end{frame}


%%---------------------------------------------------------------
\begin{frame}[fragile]

Ejemplo del contenido de \verb|/etc/rc2.d/|

\begin{scriptsize}
\begin{verbatim}
S10acpid            S18hplip         S20postfix        S89atd              
S10powernowd.early  S19cupsys        S20powernowd      S89cron
S10sysklogd         S20apmd          S20rsync          S90binfmt-support
S10wacom-tools      S20festival      S20ssh            S98usplash
\end{verbatim}
\end{scriptsize}

Ejemplo del contenido de \verb|/etc/rc6.d/|
\begin{scriptsize}
\begin{verbatim}
K19cupsys       K25mdadm            S15wpa-ifupdown     S50lvm
K19setserial    K25nfs-user-server  S20sendsigs         S50mdadm-raid
K20dbus         K30etc-setserial    S30urandom          S60umountroot
K20laptop-mode  K50alsa-utils       S31umountnfs.sh     S90halt
\end{verbatim}
\end{scriptsize}


\end{frame}



%%% Local Variables: 
%%% mode: latex
%%% TeX-master: "admin-main"
%%% End: 


%%---------------------------------------------------------------
  \section{Tareas periódicas}
%%---------------------------------------------------------------

\begin{frame}[fragile]
  \frametitle{Tareas periódicas}

  \begin{itemize}
  \item Automatizan la gestión del sistema
  \item Fiabilidad. Protegen frente a olvidos 
  \item Se ejecutan en el momento preciso (día y hora)
  \item Ayudan o detectan situaciones de error
  \item Facilitan el control del sistema
  \item Programas: 
\begin{itemize}
\item
\texttt{\textbf{cron}} 
\item
\texttt{\textbf{anacron}}.
Permite ejecutar algo programado para un momento
en que el sistema estaba apagado
% solo puede usarlo el root. Se ejecuta en algún momento
% del día
\item
\texttt{\textbf{at}}.
Ejecuta una tarea a la hora indicada por \emph{stdin}
\end{itemize}
  \end{itemize}
\end{frame}

%%---------------------------------------------------------------

\begin{frame}[fragile]
Usos de las tareas periódicas
%%  \subsection{Usos de las tareas periódicas}

  \begin{itemize}
  \item Generación de informes periódicos (fin de mes, etc.)
  \item Estado de las comunicaciones
  \item Borrado de ficheros temporales (\texttt{\textbf{/tmp}},
    \texttt{\textbf{/var/tmp}})
  \item Tareas de respaldo de información
  \item Control de los procesos presentes en el sistema
  \item Parada del sistema según horarios de trabajo
  \item Recordatorios
  \item Descarga de \emph{software} en horarios de poco tráfico
  \end{itemize}
\end{frame}


%%---------------------------------------------------------------
\begin{frame}[fragile]
  \subsection{cron}
% man 5 cron

  \begin{itemize}
  \item Es uno de los demonios esenciales de un sistema, siempre está
    arrancado (\texttt{\textbf{/usr/sbin/cron}})
  \item Se encarga de ejecutar tareas programadas para un determinado
    momento, bajo la identidad del usuario que lo programó y con
    precisión de 1 minuto
  \item Se controla a través del uso de determinados ficheros de
    configuración (solo para el superusuario) y mediante el uso de la
    orden ``\texttt{\textbf{crontab}}'' (para todos los usuarios).
  \end{itemize}
\end{frame}
%%---------------------------------------------------------------
\begin{frame}[fragile]
\subsection{Tabla de cron}


\begin{scriptsize}
\begin{verbatim}
SHELL=/bin/bash
MAILTO=koji
PATH=/usr/local/bin:/usr/bin:/bin
#   m    h   dayofmonth  month  dow   command
   16    *       *        *     *    ping 193.147.71.119  -c 1
   0     9       4        8     *    echo "regar plantas"
   0   15,18     *        *    1-5   echo "hora de salir" | wall
\end{verbatim}
\end{scriptsize}

m: Minuto. De 0 a 59\\
h: Hora. De 0 a 23\\
dayofmonth: de 0 a 31\\
month: de 1 a 12\\
dayofweek: de 0 a 7. 0=7=domingo, 1=lunes, 2=martes\ldots

Cada línea es una tarea
\end{frame}
%%---------------------------------------------------------------
\begin{frame}[fragile]

\begin{itemize}
\item
Se pueden poner comentarios con
\verb|#| pero no en cualquier posición, solo siguiendo
el patrón \emph{principio de línea, 0 o más espacios, almohadilla}
\item
En las asignaciones \verb|variable=valor|, el valor no se expande.
Por tanto, no pueden hacerse cosas como p.e. \verb|PATH = $HOME/bin:$PATH|
\item
Es necesario dejar una linea en blanco al final de la tabla
%Un rango como 23-7 no es legal, porque $23>7$
\end{itemize}
\end{frame}
%%---------------------------------------------------------------
%\begin{frame}[fragile]
%  \subsection{Lista de tareas para \texttt{\textbf{cron}}}
%
%  \begin{itemize}
%  \item Cada tarea se especifica en una línea diferente, en la que se
%    indica el momento en el que debe realizarse, y qué se debe
%    ejecutar.
%  \item Cada línea contiene 6 campos separados por espacios:
%    \begin{itemize}
%    \item minuto (0--59)
%    \item hora (0--23)
%    \item día del mes (1--31)
%    \item mes (1--12)
%    \item día de la semana (0--7). 0=7=domingo, 1=lunes, 2=martes\ldots
%    \item orden a ejecutar
%    \end{itemize}
%  \end{itemize}
%\end{frame}
%%---------------------------------------------------------------

\begin{frame}[fragile]
\begin{small}
\begin{verbatim}
*         -> todos            
1-4       -> 1,2,3 y 4    
1,4       -> 1 y 4 
*/3       -> cada 3 
1-15/3    -> los primeros 15, cada 3 
\end{verbatim}

Ejemplos y contraejemplos:

\begin{scriptsize}
\begin{verbatim}
#   m    h   dayofmonth  month  dow   command
   *   14-15     *        *     *    echo "OJO: de 14 a 15:59"
   *   23-7      *        *     *    echo "RANGO ILEGAL, 23>7"
\end{verbatim}
\end{scriptsize}


\end{small}
\end{frame}


%--------------------------
\begin{frame}[fragile]


\begin{itemize}
\item 
\verb|crontab -e|   \\Edita la tabla de cron del usuario. Usa el editor
por omisión (normalmente vi). Podemos usar otro cambiando la variable 
de entorno EDITOR
\item 
\verb|crontab -l|   \\Muestra tabla de cron
\item 
\verb|crontab mi_tabla|   \\El fichero \emph{mi\_tabla} pasa a ser nueva tabla de cron
\end{itemize}

\end{frame}

%--------------------------------------
%
%\begin{frame}[fragile]
%  \subsection{\texttt{\textbf{crontab}}}
%
%\begin{verbatim}
%       crontab [ -u usuario ] fichero
%       crontab [ -u usuario ] { -l | -r | -e }
%\end{verbatim}
%
%  \begin{itemize}
%  \item ``\texttt{\textbf{-u usuario}}'': disponible solo para
%    \texttt{\textbf{root}}, permite ver o modificar las tareas
%    programadas para otro usuario.
%  \item ``\texttt{\textbf{fichero}}'': Reemplaza la lista de tareas
%    programadas por las que aparezcan en el fichero indicado.
%  \item ``\texttt{\textbf{-l}}'': Muestra la lista de todas las tareas
%    programadas.
%  \item ``\texttt{\textbf{-r}}'': Elimina todas las tareas
%    programadas.
%  \item ``\texttt{\textbf{-e}}'': Edita la lista de tareas
%    programadas.
%  \end{itemize}
%\end{frame}




%%---------------------------------------------------------------
%\begin{frame}[fragile]
%  \subsection{Formato de cada línea en \texttt{\textbf{cron}}}
%
%  Cada campo numérico (del primero al quinto de cada línea), puede tener:
%    \begin{itemize}
%    \item Un ``\texttt{\textbf{*}}'', que indica ``todos''.
%    \item Un número, que indica un momento exacto (``\texttt{\textbf{4}}'').
%    \item Lista de números separadas por comas (``\texttt{\textbf{3,4,6}}'').
%    \item Rango de números separados por guiones (``\texttt{\textbf{3-7}}'').
%    \end{itemize}
%
%    Ejemplo: \\
%    \texttt{\textbf{0 9,18 * * 1-5 echo "hora de comer" | wall}} \\
%    (ejecuta la orden de lunes a viernes a las 9:00 y a las 18:00).
%\end{frame}

%%---------------------------------------------------------------
\begin{frame}[fragile]

  \frametitle{Ambigüedades en la especificación del momento de ejecución}

  \begin{itemize}
  \item El día en el que se ejecuta cada orden se puede indicar de 2
    maneras:
    \begin{itemize}
    \item día del mes ($3^{er}$ campo)
    \item día de la semana ($5^o$ campo)
    \end{itemize}

    En caso de aparecer los dos campos (esto es, que ninguno es
    ``\texttt{\textbf{*}}''), la interpretación que hace
    \texttt{\textbf{cron}} es que la orden debe ejecutarse cuando se
    cumpla \emph{cualquiera} de ellos

    Ejemplo: \\
    \texttt{\textbf{0,30 * 13 * 5 echo 'Viernes 13!' | wall}} \\
    %(ejecuta la orden cada media hora, \textbf{todos los viernes} y
    %además \textbf{todos los días 13 de cada mes}).
    (ejecuta la orden cada media hora, todos los viernes y
    además todos los días 13 de cada mes)
  \end{itemize}
\end{frame}

%%---------------------------------------------------------------

\begin{frame}[fragile]

  \frametitle{Momentos ``especiales'' (solo Linux)}

  En lugar de especificar los 5 primeros campos, se puede usar
  una cadena de las siguientes:
  \begin{itemize}
  \item \texttt{\textbf{@reboot}}: Se ejecuta al iniciarse la máquina.
  \item \texttt{\textbf{@yearly}}: Se ejecuta una vez al año.
  \item \texttt{\textbf{@monthly}}: Se ejecuta una vez al mes.
  \item \texttt{\textbf{@weekly}}: Se ejecuta una vez por semana.
  \item \texttt{\textbf{@daily}}: Se ejecuta una vez al día.
  \item \texttt{\textbf{@hourly}}: Se ejecuta una vez por hora.
  \end{itemize}
\end{frame}

%%---------------------------------------------------------------

\begin{frame}[fragile]
  \frametitle{Entorno de ejecución de las tareas}
  \begin{itemize}
  \item Cada tarea de cron se ejecuta por una \emph{shell}
 \texttt{\textbf{/bin/sh}}. (a menos que definamos otra cosa en SHELL)
  \item Causa de
\textbf{\textcolor{red}{errores frecuentes:}}
 El PATH con el que cron busca
 el mandato no es el del usuario, sino \verb|/usr/bin:/bin|. Soluciones:

\begin{itemize}
\item Indicar PATH en la tabla
\item Especificar el path abosoluto del mandato (p.e. \verb|/usr/local/bin/mimandato|)
\end{itemize}

  \item
Quien ejecuta las tareas no es el dueño de la tabla, sino cron.
Aunque emplea algunas variables de entorno del dueño
de la tabla, como
    \texttt{\textbf{LOGNAME}} y
    \texttt{\textbf{HOME}}.
  
  \item La entrada estándar de cada tarea se redirige de
    \texttt{\textbf{/dev/null}}, la salida estándar y la de error se
    envían por correo electrónico al propietario de la tarea (si 
hay servidor de correo)
  \end{itemize}
\end{frame}

%%---------------------------------------------------------------

%\begin{frame}[fragile]
%  \subsection{\texttt{\textbf{cron}} para el administrador}
%
%  \begin{itemize}
%  \item La orden \texttt{\textbf{crontab}} almacena la lista de tareas
%    para cada usuario en el fichero
%    ``\texttt{\textbf{/var/spool/cron/crontabs/\emph{usuario}}}''.
%  \item \texttt{\textbf{cron}} examina periódicamente estos ficheros,
%    y además el contenido del fichero \texttt{\textbf{/etc/crontab}},
%    que tiene una lista de tareas periódicas de \emph{administración}.
%  \item Estas tareas periódicas de administración tienen el mismo
%    formato que la lista de tareas de usuarios, pero con un campo más:
%    el nombre del usuario que ha de llevar a cabo la tarea.
%    Ejemplo: \\
%    \texttt{\textbf{0 8 * * * root /etc/init.d/gdm restart}} \\
%    (reinicia el servicio ``\texttt{\textbf{gdm}}'', como usuario
%    \texttt{\textbf{root}}, todos los días a las 8:00).
%  \end{itemize}
%\end{frame}


%%---------------------------------------------------------------
%\begin{frame}[fragile]
%%  \subsection{\texttt{\textbf{cron}} para el administrador (2)}
%
%  \begin{itemize}
%  \item Además, lee todos los ficheros del directorio
%    \texttt{\textbf{/etc/cron.d}} como extensiones del
%    \texttt{\textbf{/etc/crontab}} (esto es, contienen también el
%    campo ``usuario'').
%  \end{itemize}
%
%  Normalmente, el contenido del fichero \texttt{\textbf{/etc/crontab}}
%  es siempre el mismo, y similar a este:
%  {\flushleft\fontsize{8pt}{8pt}\selectfont
%\begin{verbatim}
%17 *    * * *   root    run-parts --report /etc/cron.hourly
%25 6    * * *   root    run-parts --report /etc/cron.daily
%47 6    * * 7   root    run-parts --report /etc/cron.weekly
%52 6    1 * *   root    run-parts --report /etc/cron.monthly
%\end{verbatim}
%  }

%  Esto es, ejecuta con la periodicidad indicada todos los
%  \emph{scripts} que se encuentren en esos directorios.
%
%  Si algún paquete del sistema necesita realizar algún tipo de tarea
%  periódica, añade \emph{scripts} al directorio
%  \texttt{\textbf{/etc/cron.*}} adecuado, o, si sus necesidades no se
%  ajustan a ninguno de ellos, añaden un fichero con formato de
%  \texttt{\textbf{cron}} al directorio \texttt{\textbf{/etc/cron.d}}.
%\end{frame}

%%---------------------------------------------------------------
%
%\begin{frame}[fragile]
%  \subsection{\texttt{\textbf{anacron}}}
%
%  Las tareas periódicas indicadas en los directorios
%  \texttt{\textbf{/etc/cron.daily}},
%  \texttt{\textbf{/etc/cron.weekly}}\ldots, suelen ser tareas
%  rutinarias del sistema que conviene realizar de vez en cuando, como
%  limpiar los temporales, gestionar colas de correo, realizar
%  \emph{back-up}s, etc.
%
%  Estas tareas se realizan en horas de poco uso de la máquina (de
%  madrugada).
%
%  Pero es habitual que haya máquinas que no estén encendidas 24 horas
%  al día: estaciones de trabajo que se encienden y apagan para cada
%  uso, y no suelen estar encendidas en el momento planificado para
%  llevar a cabo estas tareas periódicas.
%
%  \texttt{\textbf{anacron}}: Servicio que se inicia al encender la
%  máquina, se da cuenta de qué tareas periódicas deberían haberse
%  realizado desde la última vez que se apagó esta, y las realiza.
%\end{frame}

%%---------------------------------------------------------------
%\begin{frame}[fragile]
%  \subsection{Ficheros de configuración de \texttt{\textbf{crontab}}}
%
%  Normalmente, la orden \texttt{\textbf{crontab}} puede ejecutarla
%  cualquier usuario, permitiendo que incluya tareas periódicas en el
%  sistema, pero se puede limitar su uso con
%  \texttt{\textbf{/etc/cron.allow}} y
%  \texttt{\textbf{/etc/cron.deny}}.
%  \begin{itemize}
%  \item Si no exiten estos ficheros, todos pueden usar cron
%  \item Si existe el fichero
%``\texttt{\textbf{/etc/cron.allow}}'', solo los usuarios aquí indicados
%pueden ejecutar cron (un usuario por línea)
%
%  \item ``\texttt{\textbf{/etc/cron.deny}}''. Si no existe
%    \texttt{\textbf{/etc/cron.allow}}, y existe este fichero, los
%    usuarios incluidos en él no tienen permiso para utilizar
%    \texttt{\textbf{crontab}}.
%  \end{itemize}
%\end{frame}

%%---------------------------------------------------------------


%\begin{frame}[fragile]
%  \subsection{\texttt{\textbf{at}}}
%
%  \begin{itemize}
%  \item El demonio \texttt{\textbf{cron}} se utiliza para realizar
%    tareas periódicas, que se realizan en un momento determinado y se
%    vuelven a realizar pasado un período de tiempo.
%  \item \texttt{\textbf{at}} y \texttt{\textbf{atd}} se utilizan para
%    ejecutar una tarea concreta \emph{en un instante determinado}, y
%    no volver a ejecutarla más.
%  \item ``\texttt{\textbf{atd}}'' es un demonio, que siempre está
%    arrancado, y se encarga de ejecutar las tareas programadas.
%  \item ``\texttt{\textbf{at}}'' se ejecuta para añadir tareas nuevas a la
%    lista de tareas por realizar.
%  \item ``\texttt{\textbf{atq}}'' lista las tareas pendientes para el
%    usuario que las ejecuta.
%  \item ``\texttt{\textbf{atrm}}'' elimina una tarea de la lista de
%    tareas pendientes.
%  \end{itemize}
%\end{frame}


%%---------------------------------------------------------------
%\begin{frame}[fragile]
%  \subsection{Ficheros de configuración de \texttt{\textbf{at}}}
%
%  Son similares a los de \texttt{\textbf{crontab}}:
%  \begin{itemize}
%  \item ``\texttt{\textbf{/etc/at.allow}}''. Si existe, indica que
%    solamente la lista de usuarios incluidos en ese fichero (un
%    usuario por línea) tiene permiso para utilizar
%    \texttt{\textbf{at}}, \texttt{\textbf{atq}} y
%    \texttt{\textbf{atrm}}.
%  \item ``\texttt{\textbf{/etc/at.deny}}''. Si no existe
%    \texttt{\textbf{/etc/at.allow}}, y existe este fichero, los
%    usuarios incluidos en él no tienen permiso para utilizar
%    \texttt{\textbf{at}}, \texttt{\textbf{atq}} y
%    \texttt{\textbf{atrm}}.
%  \item Si ninguno de los dos ficheros existe, solo el superusuario
%    puede utilizar estas órdenes.
%  \end{itemize}
%
%  Normalmente, ``\texttt{\textbf{at.allow}}'' no existe, y
%  ``\texttt{\textbf{at.deny}}'' es un fichero vacío, lo cual permite
%  el uso de \texttt{\textbf{at}} para todos los usuarios del sistema.
%\end{frame}
%
%%%---------------------------------------------------------------
%\begin{frame}[fragile]
%  \subsection{Uso de \texttt{\textbf{at}}}
%
%  \begin{itemize}
%  \item \texttt{\textbf{at \emph{fecha/hora}}}. Lee órdenes de la
%    entrada estándar, para que se realicen en la \emph{fecha/hora}
%    especificadas.
%
%    \texttt{\textbf{\emph{fecha/hora}}} puede ser:
%    \begin{itemize}
%    \item \texttt{\textbf{HH[:]MM[am|pm]}} [\texttt{\textbf{Mes día}}]
%    \item ``\texttt{\textbf{now}}'', ``\texttt{\textbf{midnight}}'',
%      ``\texttt{\textbf{noon}}'', ``\texttt{\textbf{teatime}}'',
%      ``\texttt{\textbf{today}}'' o ``\texttt{\textbf{tomorrow}}''.
%    \item \texttt{\textbf{\emph{fecha/hora} + \emph{número}
%          (minutes,hours,days,weeks)}}
%    \end{itemize}

%%    Ejemplos:
%    \begin{itemize}
%    \item \texttt{\textbf{1550 Feb 14 + 3 days}}
%    \item \texttt{\textbf{10am Jul 31}}
%    \item \texttt{\textbf{1am tomorrow}}
%    \end{itemize}
%  \end{itemize}
%\end{frame}


\end{document}



\documentclass[ucs]{beamer}

\usetheme{GSyC}
%\usebackgroundtemplate{\includegraphics[width=\paperwidth]{gsyc-bg.png}}


\usepackage[greek,spanish]{babel}   %greek permite usar euro 
\usepackage[utf8x]{inputenc}
\usepackage{graphicx}
\usepackage{amssymb} % Simbolos matematicos
%\usepackage {hthtml}



% Metadatos del PDF, por defecto en blanco, pdftitle no parece funcionar
   \hypersetup{%
     pdftitle={Administración de Usuarios},%
     %pdfsubject={Diseño y Administración de Sistemas y Redes},%
     pdfauthor={GSyC},%
     pdfkeywords={},%
   }
%


% Para colocar un logo en la esquina inferior de todas las transpas
%   \pgfdeclareimage[height=0.5cm]{gsyc-logo}{gsyc}
%   \logo{\pgfuseimage{gsyc-logo}}


% Para colocar antes de cada sección una página de recuerdo de índice
%\AtBeginSection[]{
%  \begin{frame}<beamer>{Contenidos}
%    \tableofcontents[currentframetitle]
%  \end{frame}
%}


\definecolor{darkred}{rgb}  {1.0, 0.0, 0.0}
\definecolor{darkgreen}{rgb}{0.0, 0.4, 0.0}
\definecolor{darkblue}{rgb} {0.0, 0.0, 0.8}

% for resalted text
\newcommand{\res}[1]{\textcolor{darkred}{#1}}
% for different text
\newcommand{\dif}{\textsl}
% for reserved words
\newcommand{\rw}[1]{\textrm{\textbf{#1}}}
% for commands
\newcommand{\com}[1]{\textrm{\textbf{#1}}}



\begin{document}

% Entre corchetes como argumento opcional un título o autor abreviado
% para los pies de transpa
\title[Administración de Usuarios]{Administración de Usuarios }
%\subtitle{Diseño y Administración de Sistemas y Redes}
\author[GSyC]{Escuela Técnica Superior de Ingeniería de Telecomunicación\\
Universidad Rey Juan Carlos}
\institute{gsyc-profes (arroba) gsyc.urjc.es}
\date[2017]{Octubre de 2017}




%% TÍTULO
\begin{frame}
  \titlepage
  % Oportunidad para poner otro logo si se usó la opción nologo
  % \includegraphics[width=2cm]{logoesp}  
\end{frame}



%% LICENCIA DE REDISTRIBUCIÓN DE LAS TRANSPAS
%% Nota: la opción b al frame le dice que justifique el texto
%% abajo (por defecto c: centrado)
\begin{frame}[b]
\begin{flushright}
{\tiny
\copyright \insertshortdate~\insertshortauthor \\
  Algunos derechos reservados. \\
  Este trabajo se distribuye bajo la licencia \\
  Creative Commons Attribution Share-Alike 3.0\\
}
\end{flushright}  
\end{frame}



%% ÍNDICE
%\begin{frame}
%  \frametitle{Contenidos}
%  \tableofcontents
%\end{frame}




%%---------------------------------------------------------------
  \section{Usuarios y grupos}
%%---------------------------------------------------------------

\begin{frame}[fragile]
  \frametitle{Usuarios y grupos}
%http://www.unix.org.ua/orelly/networking/puis/ch04_01.htm
  \begin{itemize}
    \item Cada usuario tiene un identificador (\emph{UID}), un grupo
      principal (o primario) al que pertenece (\emph{GID}), una serie de grupos
      adicionales, un nombre de usuario (\emph{login}), un directorio
      de trabajo  (\emph{home})
%(\texttt{$HOME})
    \item La orden \texttt{\textbf{id}} nos da el UID, el GID y los
      grupos adicionales de un usuario

    \item Cada usuario puede tener dos tipos de recursos en un sistema
      UNIX: Procesos y Ficheros
%  \begin{enumerate}
%    \item Procesos. Cada proceso tiene asociado un UID, un GID y 
%      grupos adicionales
%    \item Ficheros. Cada fichero tiene asociado un UID y un GID
%  \end{enumerate}
  \end{itemize}
\end{frame}

%%---------------------------------------------------------------
\begin{frame}[fragile]
  \begin{itemize}
    \item Cada UID y cada GID puede tener asociado un nombre,
      especificado en los ficheros \texttt{/etc/passwd} y \texttt{/etc/group},
      respectivamente
    \item La información de \verb+/etc/passwd+ y \verb+/etc/group+ la utilizan
      diversas órdenes de administración.
      Ambos ficheros deben existir y ser coherentes para que el sistema funcione correctamente
 \item 
No es recomendable editar estos ficheros directamente, sino mediante mandatos
como \verb|usermod|
\item
Si se edita \verb|/etc/passwd| y \verb|/etc/group/| directamente, debe usarse
\verb|vipw| y \verb|vigr|
  \end{itemize}
\end{frame}


%%---------------------------------------------------------------
\begin{frame}[fragile]
  \frametitle{Elección de la palabra clave}
  \begin{itemize}
    \item El campo \emph{passwd} en el \texttt{\textbf{/etc/passwd}} y el
      \texttt{\textbf{/etc/shadow}} se encuentra cifrado con una función \emph{hash} para evitar que los usuarios (y administradores) puedan
      conocer las contraseñas de otros usuarios.
    \item Se usa un cifrado de un solo sentido: no existe algoritmo para averiguar la contraseña a partir de estos ficheros.
    \item Pero se pueden probar varias contraseñas, hasta millones por segundo (John the Ripper).
    \item Es imprescindible elegir palabras clave seguras, que no aparezcan en diccionarios, evitando nombres
      o fechas significativas, combinando símbolos, y de la mayor longitud posible.
  \end{itemize}
\end{frame}



%%----------------------------------------------
\begin{frame}[fragile]


\begin{itemize}

\item
Ejemplos de malas contraseñas:
  \begin{footnotesize}
  \begin{verbatim}
123456
4312
toby
r2d2
tornillo
fromage
mostoles
  \end{verbatim}
  \end{footnotesize}
\item
Contraseñas que parecen buenas, pero son malas:
  \begin{footnotesize}
  \begin{verbatim}
XCV330
NCC-1701-A 
ARP2600V

  \end{verbatim}
  \end{footnotesize}

\end{itemize}
\end{frame}
%%----------------------------------------------
\begin{frame}[fragile]
\begin{itemize}

\item

Buenas contraseñas

Para usos ordinarios, una contraseña razonablemente buena será parecida (¡pero no igual!)
a una de estas


  \begin{footnotesize}
  \begin{verbatim}
Contraseña  Nemotécnico
----------------------------------------------------
00QuReMa:   Queridos Reyes Magos:
3x4igDoze   3x4=doce
1pt,yTp1    uno para todos,todos para uno
19Dy500n    19 dias y 500 noches
R,cmqht?0   Rascayú, cuando mueras que harás tú?
wali1YS!    we all live in a Yellow Submarine
Lh10knpr.   le hare una oferta que no podrá rechazar
  \end{verbatim}
  \end{footnotesize}

Para aplicaciones especialmente sensibles, es necesario ampliar
el \emph{keyspace} a 14 caracteres o más


\end{itemize}

\end{frame}


%%---------------------------------------------------------------
\begin{frame}[fragile]
\begin{itemize}
\item
Es conveniente que busquemos e inutilicemos las contraseñas débiles de nuestros usuarios, ya que suponen un primer punto de entrada en nuestro sistema
\item
En Unix, el root no puede leer las contraseñas. Pero en otros entornos sí. Y muchos usuarios emplean siempre la misma contraseña

Ejemplo:
\begin{enumerate}
\item Juan Torpe usa como contraseña  dgj441iU en juan.torpe@hotmail.com
\item Juan Torpe se apunta en \verb|www.politonosdebisbalgratis.com|, con su cuenta de correo y su contraseña de siempre
\item El administrador malicioso de este web ya conoce el nombre de Juan, su cuenta de correo y su contraseña.
\begin{itemize}
\item Puede usar la función \emph{¿contraseña olvidada?}  y colarse en cualquier otra cuentas de Juan
\end{itemize}
\end{enumerate}

Debemos instruir a nuestros usuarios sobre esto
\end{itemize}

\end{frame}



%%---------------------------------------------------------------
\begin{frame}[fragile]
\frametitle{}
\begin{itemize}
\item
Los usuarios sin duda olvidarán en ocasiones su contraseña y tendremos que generles una nueva, de forma segura
\item
Pero es muy poco profesional que nosotros como administradores olvidemos una contraseña. Debemos usar varias y guardarlas de
forma medianamente segura (gpg, lastpass, etc)
\end{itemize}

\end{frame}

%%---------------------------------------------------------------
\begin{frame}[fragile]
  \frametitle{\texttt{/etc/passwd}}
  \begin{itemize}
    \item Contiene la información de todos los usuarios del sistema.
    \item Contenido: líneas con campos separados por dos puntos:\\
      \texttt{\emph{login}:\emph{passwd}:\emph{UID}:\emph{GID}:\emph{info}:\emph{home-dir}:\emph{shell}}

    \item El campo ``\emph{login}'' puede tener hasta 32 caracteres en Linux, pero se recomienda
limitarlo a 8, como en los UNIX clásicos

    \item El campo ``\emph{passwd}'' contiene la contraseña cifrada (con DES o con MD5) y puede estar en
      otro fichero, en el \verb+/etc/shadow+.
    \item El campo ``\emph{info}'' contiene el nombre real del usuario e información
 adicional como el teléfono, etc.
Por (desafortunados) motivos históricos, también se le denomina GECOS

%Tenían una impresora en un ordenador usando gecos, para poder usar hacía
%falta cierto código dependiente de cada usuario, y no se les ocurrió mejor
% cosa que ponerlo aquí
%This field is optional and only used for informational purposes. Usually,
%t contains the full user name. GECOS means General Electric Comprehensive
%perating System, which has been renamed to GCOS when GE's large
%systems division was sold to Honeywell. Dennis Ritchie has reported:
%"Sometimes we sent printer output or batch jobs to the GCOS machine. The
%gcos field in the password file was a place to stash the information
%for the $IDENTcard. Not elegant."
%
    \item En algunos sistemas, puede haber información externa (NIS, LDAP\ldots)
    \item Programas que lo utilizan directamente: \texttt{\textbf{login}}, \texttt{\textbf{su}}, \texttt{\textbf{passwd}}.
  \end{itemize}
\end{frame}



%%---------------------------------------------------------------
\begin{frame}[fragile]
  \frametitle{\texttt{/etc/group}}
  \begin{itemize}
    \item Nombres de grupos del sistema, y miembros de cada grupo.
    \item Contenido: líneas con campos separados por dos puntos:\\
      \texttt{\emph{nombre}:\emph{passwd}:\emph{GID}:\emph{lista-logins}}
    \item ``\emph{lista-logins}'' son usuarios separador por comas que
      pertenecen a ese grupo.
    \item El campo ``\emph{passwd}'' no se suele utilizar. Permite ingresar en un grupo
en el que no se es miembro. 
% hay editar estos ficheros a mano, no hay mandato, una prueba más de lo poco que
% se usa
%, casi siempre está bloqueado (``\texttt{\textbf{*}}'').
%      Lo utilizan únicamente las órdenes \texttt{\textbf{sg}} y \texttt{\textbf{newgrp}}.
%      Si se usa, puede estar en otro fichero, en el \verb+/etc/gshadow+.
    \item En algunos sistemas, puede haber información externa (NIS, LDAP\ldots)
  \end{itemize}
\end{frame}

%%---------------------------------------------------------------
\begin{frame}[fragile]
  \frametitle{\texttt{/etc/shadow}}
  \begin{itemize}
    \item Si existe, contiene las contraseñas cifradas de los usuarios del sistema.
    \item Contenido: líneas con campos separados por dos puntos:\\
      \texttt{\emph{login}:\emph{passwd}:\emph{a}:\emph{b}:\emph{c}:\emph{d}:\emph{e}:\emph{f}:\emph{g}}
      \begin{itemize}
        \item[\texttt{\textbf{\emph{a}}}:] momento en que la \emph{passwd} fue cambiada por última vez.
        \item[\texttt{\textbf{\emph{b}}}:] días que deben pasar antes de que pueda cambiarse.
        \item[\texttt{\textbf{\emph{c}}}:] días después de los cuales la \emph{passwd} debe cambiarse.
        \item[\texttt{\textbf{\emph{d}}}:] días antes de la expiración para avisar al usuario.
        \item[\texttt{\textbf{\emph{e}}}:] días después de la expiración para desactivar la cuenta.
        \item[\texttt{\textbf{\emph{f}}}:] momento en que la cuenta se ha desactivado.
        \item[\texttt{\textbf{\emph{g}}}:] campo reservado.
      \end{itemize}
  \end{itemize}
\end{frame}


%%---------------------------------------------------------------
\begin{frame}[fragile]

  \frametitle{Para mejorar la seguridad se añade un "salt"}
\emph{salt} es un tipo de \emph{nounce}: Number used once. 2 bytes aleatorios
que se añaden a la contraseña

\res{Sin salt}
  \begin{footnotesize}
  \begin{verbatim}
password  -->  hash (password)
"sesamo" --> zv/coRb$PjGToGEqNZF434TmQ7bAH.rVi3i.o7IWQAI9qqzeGKe/JkJq
bDfQE2gBFYzBTDNCHyoxpZvSLhenkPT3L6aZN0
  \end{verbatim}
  \end{footnotesize}


El atacante puede usar una \emph{rainbow table}: El resultado de aplicar
hash a un diccionario completo. Si encuentra la hash en la tabla, conoce
la contraseña que fue usada

  \begin{footnotesize}
  \begin{verbatim}
rainbow table
hash(palabra1)
hash(palabra2)
hash(palabra3)
  \end{verbatim}
  \end{footnotesize}


\end{frame}
%%----------------------------------------------
\begin{frame}[fragile]

\res{Con salt}

  \begin{footnotesize}
  \begin{verbatim}
password+salt  -->  hash (password+salt)

rainbow table:
hash(palabra1+salt1)
hash(palabra1+salt2)
hash(palabra1+salt3)
  \end{verbatim}
  \end{footnotesize}


\begin{itemize}
\item
\emph{salt} se guarda en abierto: se añade al hash, son los primeros dos bytes
\item
Esto obliga a que la rainbow table sea mucho mayor, puede hacerla inviable
\end{itemize}

\end{frame}



%%---------------------------------------------------------------
%\begin{frame}[fragile]
%  \subsection{Añadir un usuario al sistema}
%  \begin{itemize}
%    \item Elegir un \emph{login}, un UID y un GID para el nuevo usuario.
%    \item Decidir a qué grupos adicionales debe pertenecer.
%    \item Modificar \texttt{\textbf{/etc/passwd}}, \texttt{\textbf{/etc/shadow}}, \texttt{\textbf{/etc/group}}.
%    \item Crear el directorio HOME (normalmente \texttt{\textbf{/home/\emph{login}}}), con el UID
%      y el GID adecuados.
%    \item Copiar los ficheros de inicio de la cuenta (desde \texttt{\textbf{/etc/skel}}).
%    \item Todas estas tareas están automatizadas y las lleva a cabo la orden \texttt{\textbf{adduser}}.
%  \end{itemize}
%\end{frame}

%%---------------------------------------------------------------
\begin{frame}[fragile]
  \frametitle{Desactivar un usuario del sistema}
  \begin{itemize}
    \item Bloquear su contraseña en el \texttt{\textbf{/etc/passwd}} o \texttt{\textbf{/etc/shadow}}
      (añadiendo un carácter ``\texttt{\textbf{-}}'' o ``\texttt{\textbf{*}}'', por ejemplo).
    \item Eliminar sus tareas periódicas (\texttt{\textbf{/var/spool/cron}}).
    \item Revisar \texttt{\textbf{/etc/aliases}} y \texttt{\textbf{.forward}} por si el usuario tuviera
      acciones a realizar con el correo recibido.
%    \item Revisar sus ficheros de personalización, haciendo hincapié en \texttt{\textbf{.forward .rhosts .shosts .ssh/authorized\_keys}}
  \end{itemize}
  
Eliminar un usuario           
  \begin{itemize}
    \item \texttt{\textbf{userdel}} 
    \item \texttt{\textbf{userdel -r}}   también borra su correo y su \emph{home}
  \end{itemize}
\end{frame}

%%---------------------------------------------------------------
%\begin{frame}[fragile]
%  \subsection{Eliminar un usuario del sistema}
%  \begin{itemize}
%    \item Hay que deshacer el proceso de creación de cuenta: modificación de \texttt{\textbf{/etc/passwd}},
%      \texttt{\textbf{/etc/shadow}}, \texttt{\textbf{/etc/group}}.
%    \item Eliminar \texttt{\textbf{/home/\emph{login}}} y cualquier otro fichero que pueda tener en el sistema
%      (en el \texttt{\textbf{/tmp}}, etc).
%    \item Eliminar sus tareas periódicas (\texttt{\textbf{/var/spool/cron}}).
%    \item Eliminar su correo (\texttt{\textbf{/var/mail/\emph{login}}}).
%    \item La orden \texttt{\textbf{deluser}} se encarga de hacer la mayor parte de estas tareas.
%  \end{itemize}
%\end{frame}



%%---------------------------------------------------------------
%\begin{frame}[fragile]
%  \frametitle{Ficheros de personalización}
%  \begin{itemize}
%    \item Residen en el directorio de inicio (\texttt{\textbf{\$HOME}}).
%    \item Cada intérprete, cada programa, cada orden puede usar ficheros distintos.
%    \item Su nombre suele comenzar por ``\texttt{\textbf{.}}''
%    \item \texttt{bash}: \texttt{\textbf{.profile .bash\_profile .bashrc .bash\_history}}
%    \item \texttt{ssh}: \texttt{\textbf{.rhosts .shosts .ssh/}}
%    \item \emph{e-mail}: \texttt{\textbf{.forward}}
%  \end{itemize}
%\end{frame}

%%---------------------------------------------------------------
\begin{frame}[fragile]
  \frametitle{Usuarios especiales}
  \begin{itemize}
    \item No todas las líneas del \texttt{\textbf{/etc/passwd}} corresponden con usuarios físicos.
    \item Super-usuario: uid=0 (su \emph{login} es normalemente \texttt{\textbf{root}}).
    \item Otros usuarios del sistema: se utilizan para:
      \begin{itemize}
        \item tareas específicas de administración
        \item propietarios de determinados ficheros del sistema
        \item ejecución de determinadas aplicaciones (bases de datos, servidores de web, ftp, e-mail, noticias, etc)
      \end{itemize}
    \item Normalmente, los usuarios normales tienen UIDs entre el 1000 y el 30000.
  \end{itemize}
\end{frame}

%%---------------------------------------------------------------
%\begin{frame}[fragile]
%  \subsection{Proceso de \emph{login}}
%  \begin{itemize}
%    \item Para que un usuario \emph{entre} en el sistema, algún servicio (\texttt{\textbf{getty}},
%      \texttt{\textbf{ssh}}, \texttt{\textbf{telnetd}}, etc) ejecuta un proceso de \texttt{\textbf{login}}:
%      \begin{itemize}
%        \item Se comprueba el nombre de usuario y la contraseña (en el \texttt{\textbf{/etc/passwd}}
%          y \texttt{\textbf{/etc/shadow}}).
%        \item Se comprueban posibles restricciones de acceso (contraseña caducada, etc) en el
%          \texttt{\textbf{/etc/shadow}}.
%        \item Registro de usuarios: \texttt{\textbf{/var/run/utmp}}, \texttt{\textbf{/var/log/wtmp}}.
%        \item \texttt{\textbf{setuid()}}, \texttt{\textbf{setgid()}}, \texttt{\textbf{setgroups()}}.
%        \item \texttt{\textbf{chdir(HOME)}}.
%        \item \texttt{\textbf{exec(SHELL)}}.
%      \end{itemize}
%  \end{itemize}
%\end{frame}

%%---------------------------------------------------------------
%\begin{frame}[fragile]
%  \subsection{Registro de usuarios}
%  \begin{itemize}
%    \item Cada vez que un usuario realiza un proceso de \emph{login} o de \emph{logout} en el sistema,
%      se modifican los ficheros \texttt{\textbf{/var/run/utmp}} y \texttt{\textbf{/var/log/wtmp}}.
%    \item Ambos son ficheros con contenido binario, no se pueden consultar ni editar directamente con un editor de texto.
%    \item \texttt{\textbf{/var/run/utmp}}: usuarios conectados.
%      \begin{itemize}
%        \item Se puede consultar con ``\texttt{\textbf{who}}'', ``\texttt{\textbf{w}}'' o ``\texttt{\textbf{finger}}''.
%      \end{itemize}
%    \item \texttt{\textbf{/var/log/wtmp}}: registro de usuarios pasados.
%      \begin{itemize}
%        \item Se puede consultar con ``\texttt{\textbf{last}}'' o ``\texttt{\textbf{who /var/log/wtmp}}''.
%      \end{itemize}
%  \end{itemize}
%\end{frame}

%%---------------------------------------------------------------
\begin{frame}[fragile]
  \frametitle{Cambio de contraseña}
    Para cambiar la contraseña y otros datos se utilizan las órdenes
      \texttt{\textbf{passwd}} (contraseña), \texttt{\textbf{chfn}} (info/gecos), \texttt{\textbf{chsh}} (\emph{shell}):
      \begin{itemize}
        \item Estas órdenes tienen \emph{set-uid} para que un usuario normal pueda modificar
          información privilegiada.
        \item Antes de nada, piden la \emph{passwd} del usuario para verificar que es quien dice ser.
        \item Bloquean cada fichero a modificar para asegurar exclusión de accesos.
        \item Realizan las modificaciones.
        \item Desbloquean ficheros.
      \end{itemize}

Para cambiar la contraseña de un usuario desde un script

\begin{itemize}
\item
  \begin{scriptsize}
  \begin{verbatim}
echo "jperez:sesamo" | chpasswd
  \end{verbatim}
  \end{scriptsize}
\end{itemize}



\end{frame}

%%---------------------------------------------------------------
\begin{frame}[fragile]
  \frametitle{Cambios de usuario y grupo}
  \begin{itemize}
    \item \texttt{\textbf{su}} ejecuta otra \emph{shell} bajo un usuario distinto.

  \begin{itemize}
        \item \texttt{\textbf{su jperez}} 

ejecuta otra shell, perteneciente al usuario \emph{jperez}
        \item \texttt{\textbf{su}} 

ejecuta shell con uid=0 (\texttt{\textbf{root}}).

        \item Pide la contraseña del usuario destino, excepto  salvo si el origen es \texttt{\textbf{root}}.

  \end{itemize}


    %\item \texttt{\textbf{sg}} y \texttt{\textbf{newgrp}} ejecutan una \emph{shell} con distinto GID.
    \item \texttt{\textbf{newgrp}} ejecuta una \emph{shell} con distinto GID.
      \begin{itemize}
        \item Tiene \emph{set-uid} 
%para poder realizar las llamadas al sistema
  %        \texttt{\textbf{setuid()}}, \texttt{\textbf{setgid()}} y \texttt{\textbf{setgroups()}}.
        \item  \texttt{\textbf{newgrp}} permite cambiar el GID a otro grupo al que
          pertenezcamos  (cambia el grupo primario)
      \end{itemize}
  \end{itemize}
\end{frame}




%%---------------------------------------------------------------
\begin{frame}[fragile]
Mandatos que sólo pueden ejecutarse como root
\begin{itemize}
\item \texttt{groupadd } \emph{grupo}\\  crea un grupo
\item \texttt{adduser} \emph{usuario}\footnote{En RedHat, useradd usuario y chfn usuario} \\añade un usuario. Copia en su home el directorio \verb|/etc/skel| \\
 \texttt{adduser} \emph{ usuario grupo } \\añade un usuario a un grupo
\item \texttt{usermod -g }  \emph{grupo\_primario  usuario}
\\ Cambia el grupo primario por omisión del usuario 
\item \texttt{passwd}  \emph{usuario}\\Cambia la contraseña de un usuario
%\item \texttt{su}  \emph{usuario}  \\Cambia el id. de usuario actual

\end{itemize}
\end{frame}

%%---------------------------------------------------------------
\begin{frame}[fragile]
\begin{itemize}

\item \texttt{chown} \emph{dueño fichero(s)}\\  cambia el dueño de un fichero
\item \texttt{chgrp} \emph{dueño fichero(s)}\\ cambia el grupo de un fichero
\end{itemize}

Mandatos para cualquier usuario
\begin{itemize}
\item \texttt{passwd} \\Cambia la contraseña del usuario
%\item \texttt{groups }  \emph{usuario}\\ muestra los grupos de un usuario
%\item \texttt{id }  \emph{usuario}\\ muestra id, gid, grupos
\item \texttt{newgrp}  \emph{grupo}\\Entre los grupos de un usuario, elige el grupo primario
\end{itemize}
\end{frame}





\end{document}

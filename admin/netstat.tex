
\documentclass[ucs]{beamer}

\usetheme{GSyC}
%\usebackgroundtemplate{\includegraphics[width=\paperwidth]{gsyc-bg.png}}


\usepackage[greek,spanish]{babel}   %greek permite usar euro 
\usepackage[utf8x]{inputenc}
\usepackage{graphicx}
\usepackage{amssymb} % Simbolos matematicos


% Metadatos del PDF, por defecto en blanco, pdftitle no parece funcionar
   \hypersetup{%
     pdftitle={Netstat},%
     %pdfsubject={Diseño y Administración de Sistemas y Redes},%
     pdfauthor={GSyC},%
     pdfkeywords={},%
   }
%


% Para colocar un logo en la esquina inferior de todas las transpas
%   \pgfdeclareimage[height=0.5cm]{gsyc-logo}{gsyc}
%   \logo{\pgfuseimage{gsyc-logo}}


%% Para colocar antes de cada sección una página de recuerdo de índice
%\AtBeginSection[]{
%  \begin{frame}<beamer>{Contenidos}
%    \tableofcontents[currentframetitle]
%  \end{frame}
%}


\definecolor{darkred}{rgb}  {1.0, 0.0, 0.0}
\definecolor{darkgreen}{rgb}{0.0, 0.4, 0.0}
\definecolor{darkblue}{rgb} {0.0, 0.0, 0.8}

% for resalted text
\newcommand{\res}[1]{\textcolor{darkred}{#1}}
% for different text
\newcommand{\dif}{\textsl}
% for reserved words
\newcommand{\rw}[1]{\textrm{\textbf{#1}}}
% for commands
\newcommand{\com}[1]{\textrm{\textbf{#1}}}



\begin{document}

% Entre corchetes como argumento opcional un título o autor abreviado
% para los pies de transpa
\title[netstat]{netstat }
%\subtitle{Diseño y Administración de Sistemas y Redes}
\author[GSyC]{Escuela Técnica Superior de Ingeniería de Telecomunicación\\
Universidad Rey Juan Carlos}
\institute{gsyc-profes (arroba) gsyc.urjc.es}
\date[2017]{Noviembre de 2017}






%% TÍTULO
\begin{frame}
  \titlepage
  % Oportunidad para poner otro logo si se usó la opción nologo
  % \includegraphics[width=2cm]{logoesp}  
\end{frame}



%% LICENCIA DE REDISTRIBUCIÓN DE LAS TRANSPAS
%% Nota: la opción b al frame le dice que justifique el texto
%% abajo (por defecto c: centrado)
\begin{frame}[b]
\begin{flushright}
{\tiny
\copyright \insertshortdate~\insertshortauthor \\
  Algunos derechos reservados. \\
  Este trabajo se distribuye bajo la licencia \\
  Creative Commons Attribution Share-Alike 4.0\\
}
\end{flushright}  
\end{frame}



%% ÍNDICE
%\begin{frame}
%  \frametitle{Contenidos}
%  \tableofcontents
%\end{frame}



%%---------------------------------------------------------------
\section{netstat}
%%---------------------------------------------------------------


\begin{frame}[fragile]
\frametitle{netstat}




\begin{itemize}
\item
\verb|netstat| es una herramienta básica en cualquier SSOO (Unix, Linux, MacOS, Windows...)
\item
Muestra 
información sobre 
conexiones de red, tablas de encaminamiento, estadísticas de los interfaces,
NAT y multicast

\end{itemize}

Nos permitirá comprobar qué demonios están funcionando en nuestra
máquina

\begin{itemize}
\item
Es una herramienta básica para depurar cualquier servicio

\item
Un principio básico de seguridad en cualquier sistema es tener
activos solo los servicios necesarios, cualquier nuevo
servicio siempre implica un cierto riesgo
\end{itemize}
\end{frame}


%%---------------------------------------------------------------
\subsection{Información sobre conexiones de red}
%%---------------------------------------------------------------
\begin{frame}[fragile]
\frametitle{Información sobre conexiones de red, Linux}
\begin{itemize}
\item
\verb|-tu|

Muestra información sobre TCP y UDP (y no sobre los
\emph{Unix domain sockets})
\item
\verb|-p|

Indica el programa a quien pertenece el socket

Si lo ejecuta un usuario ordinario, solo muestra algunos nombres

Si lo ejecuta root, muestra todos los nombres
\item
\verb|-a|

Muestra todos los sockets, no solamente las conexiones establecidas
sino también los sockets que están \emph{escuchando}

\item
\verb|-n|

Muestra direcciones IP y no nombres de máquina.

Muestra números de puerto, no nombre de servicio asociado (en los \emph{well know ports}).
No intenta resolver nombres de máquina. 

\end{itemize}

\end{frame}

%%---------------------------------------------------------------
\begin{frame}[fragile]
\frametitle{}

  \begin{tiny}
  \begin{verbatim}
koji@afrodita:~$ sudo netstat -tupan
Conexiones activas de Internet (servidores y establecidos)
Prot Recv-Q Send-Q Dirección_Local Dirección_Externa    Estado      PID/Program name
tcp    0     0 0.0.0.0:22          0.0.0.0:*            ESCUCHAR    27488/sshd      
tcp    0     0 127.0.0.1:631       0.0.0.0:*            ESCUCHAR    1320/cupsd      
tcp    0     0 193.147.71.120:22   193.147.71.62:56881  ESTABLECIDO 26653/sshd: koji 
tcp6   0     0 :::79               :::*                 ESCUCHAR    19514/xinetd    
tcp6   0     0 :::22               :::*                 ESCUCHAR    27488/sshd      
tcp6   0     0 ::1:631             :::*                 ESCUCHAR    1320/cupsd      
udp    0     0 0.0.0.0:49573       0.0.0.0:*                        32398/avahi-daemon:
udp    0     0 0.0.0.0:5353        0.0.0.0:*                        32398/avahi-daemon:

  \end{verbatim}
  \end{tiny}

\end{frame}

%    Recv-Q
%       The count of bytes not copied by the user program connected to this socket.
%
%   Send-Q
%       The count of bytes not acknowledged by the remote host.


%%---------------------------------------------------------------
\begin{frame}[fragile]
\frametitle{}


  \begin{tiny}
  \begin{verbatim}
koji@afrodita:~$ sudo netstat -tupa
Conexiones activas de Internet (servidores y establecidos)
Prot Recv-Q Send-Q Dirección_Local Dirección_Externa Estado      PID/Program name
tcp    0     0 *:ssh                   *:*           ESCUCHAR    27488/sshd      
tcp    0     0 localhost:ipp           *:*           ESCUCHAR    1320/cupsd      
tcp    0     0 afrodita:ssh            doublas:56881 ESTABLECIDO 26653/sshd: koji 
tcp6   0     0 [::]:finger             [::]:*        ESCUCHAR    19514/xinetd    
tcp6   0     0 [::]:ssh                [::]:*        ESCUCHAR    27488/sshd      
tcp6   0     0 ip6-localhost:ipp       [::]:*        ESCUCHAR    1320/cupsd      
udp    0     0 *:49573                 *:*                       32398/avahi-daemon:
udp    0     0 *:mdns                  *:*                       32398/avahi-daemon:

  \end{verbatim}
  \end{tiny}
Problema:

Netstat no comprueba que el demonio que está atado a un puerto sea
el demonio \emph{habitual} en ese puerto.

Ejemplo: si atamos un servicio cualquiera al puerto 80, netstat
lo tomará por un servidor web
\end{frame}

%%---------------------------------------------------------------
\begin{frame}[fragile]
\frametitle{}

\begin{itemize}
\item
Un demonio puede escuchar en un puerto (p.e. el 22) de cualquier
dirección de la máquina, por tanto, en cualquiera de los interfaces de la máquina
  \begin{tiny}
  \begin{verbatim}
tcp    0     0 0.0.0.0:22          0.0.0.0:*            ESCUCHAR    27488/sshd      
  \end{verbatim}
  \end{tiny}
\item
O bien puede escuchar en un puerto (p.e. el 631), pero no de todas 
las direcciones/todos los interfaces, sino de una en concreto
  \begin{tiny}
  \begin{verbatim}
tcp    0     0 127.0.0.1:631       0.0.0.0:*            ESCUCHAR    1320/cupsd      
  \end{verbatim}
  \end{tiny}
Este demonio está en la máquina afrodita, pero no atiende peticiones hechas a la
dirección pública de afrodita, solamente a \emph{localhost}/127.0.0.1
\item
Podemos usar \verb|grep| para filtrar la salida de netstat
  \begin{tiny}
  \begin{verbatim}
koji@afrodita:~$ netstat -tupan |grep 22
tcp        0      0 0.0.0.0:22              0.0.0.0:*               ESCUCHAR    -               
tcp        0      0 193.147.71.120:22       193.147.71.62:34285     ESTABLECIDO -               
tcp6       0      0 :::22                   :::*                    ESCUCHAR    -      
  \end{verbatim}
  \end{tiny}

\end{itemize}

\end{frame}



%%---------------------------------------------------------------
\begin{frame}[fragile]
\frametitle{Netsat en OS X}
Uso típico:

\verb|netstat -f inet -an|
\begin{itemize}
\item
Para ver solo las conexiones tcp y udp, la opción no es -tu sino
\verb|-f inet|

\item
\verb|-a| y \verb|-n| se comportan como en Linux

\item
Para saber qué proceso escucha en cada puerto, no hay un equivalente
a  \verb|-p|

Pero podemos usar

\verb|sudo lsof -iTCP:12345 -sTCP:LISTEN|

(Donde 12345 sería el puerto a consultar)

o bien

\verb@sudo lsof -i -n -P |grep UDP@

\end{itemize}

\end{frame}

\end{document}

%%---------------------------------------------------------------
\begin{frame}[fragile]
\frametitle{Referencias}
\begin{itemize}
\item
Tutorial de netstat

CIP Tudela 
\item
Netstat tutorial

Security Quick-Start HOWTO for Linux
\end{itemize}

\end{frame}


\end{document}



\documentclass[ucs]{beamer}

\usetheme{GSyC}
%\usebackgroundtemplate{\includegraphics[width=\paperwidth]{gsyc-bg.png}}


\usepackage[spanish]{babel}
\usepackage[utf8x]{inputenc}
\usepackage{graphicx}
\usepackage{amssymb} % Simbolos matematicos
\usepackage{enumerate}


% Metadatos del PDF, por defecto en blanco, pdftitle no parece funcionar
   \hypersetup{%
     pdftitle={DevOps},%
     %pdfsubject={Diseño y Administración de Sistemas y Redes},%
     pdfauthor={GSyC},%
     pdfkeywords={},%
   }
%


% Para colocar un logo en la esquina inferior de todas las transpas
%   \pgfdeclareimage[height=0.5cm]{gsyc-logo}{gsyc}
%   \logo{\pgfuseimage{gsyc-logo}}


% Para colocar antes de cada sección una página de recuerdo de índice
%\AtBeginSection[]{
%  \begin{frame}<beamer>{Contenidos}
%    \tableofcontents[currentframetitle]
%  \end{frame}
%}


%\newboolean{detallado}
%\setboolean{detallado}{false}
%\newcommand{\opcional}[1]{\ifthenelse{\boolean{detallado}}{#1}{}}


\definecolor{darkred}{rgb}  {1.0, 0.0, 0.0}
\definecolor{darkgreen}{rgb}{0.0, 0.4, 0.0}
\definecolor{darkblue}{rgb} {0.0, 0.0, 0.8}

% for resalted text
\newcommand{\res}[1]{\textcolor{darkred}{#1}}
% for different text
\newcommand{\dif}{\textsl}
% for reserved words
\newcommand{\rw}[1]{\textrm{\textbf{#1}}}
% for commands
\newcommand{\com}[1]{\textrm{\textbf{#1}}}






\begin{document}

% Entre corchetes como argumento opcional un título o autor abreviado
% para los pies de transpa
\title[Sistemas de Ficheros]{Sistemas de Ficheros}

%\subtitle{Diseño y Administración de Sistemas y Redes}
\author[GSyC]{Escuela Técnica Superior de Ingeniería de Telecomunicación\\
Universidad Rey Juan Carlos}
\institute{gsyc-profes (arroba) gsyc.urjc.es}
\date[2015]{Noviembre de 2015}



%% TÍTULO
\begin{frame}
  \titlepage
  % Oportunidad para poner otro logo si se usó la opción nologo
  % \includegraphics[width=2cm]{logoesp}  
\end{frame}



%% LICENCIA DE REDISTRIBUCIÓN DE LAS TRANSPAS
%% Nota: la opción b al frame le dice que justifique el texto
%% abajo (por defecto c: centrado)
\begin{frame}[b]
\begin{flushright}
{\tiny
\copyright \insertshortdate~\insertshortauthor \\
  Algunos derechos reservados. \\
  Este trabajo se distribuye bajo la licencia \\
  Creative Commons Attribution Share-Alike 4.0\\
}
\end{flushright}  
\end{frame}



%% ÍNDICE
\begin{frame}
  \frametitle{Contenidos}

%\begin{tiny}
  \tableofcontents
%\end{tiny}

\end{frame}


%%---------------------------------------------------
\section{Introducción}
%%---------------------------------------------------


\begin{frame}[fragile]
\frametitle{Introducción}
% Sistema:   2. m. Conjunto de cosas que relacionadas entre si ordenadamente
%   contribuyen a determinado objeto.
  
  \begin{itemize}
    \item Un sistema de ficheros es una forma de almacenar y organizar ficheros para
permitir su uso
 \item Pueden usar un dispositivo de almacenamiento (disco, cdrom), la red o ser sólo
un interfaz para acceder a datos
    \item Para poder empezar a almacenar información en un sistema de ficheros,
      éste tiene que ser \emph{inicializado}
    \item En Unix, para poder usarlo, hay que \emph{montarlo} en alguna parte de
      la jerarquía de directorios, un árbol cuya raíz es
      el directorio llamado \texttt{\textbf{/.}}
  \end{itemize}


\end{frame}

%--------------------------------------------------------------------

\begin{frame}[fragile]
\emph{On a UNIX system, everything is a file; if something is not a file, it is a process}
%http://www.faqs.org/docs/linux_intro/sect_03_01.html

Los ficheros pueden ser
 \begin{itemize}
\item
Ficheros normales
\item
Directorios
\item
Ficheros especiales (Entrada y salida. Están en \emph{/dev })
\item
Enlaces
\item
Fifos. (Pipes con nombre). Para comunicación entre procesos
\item
Sockets de dominio. Similares a los sockets TCP/IP
%(Domain) sockets: a special file type, similar to TCP/IP sockets, providing inter-process networking protected by the file system's access control.
  \end{itemize}

\end{frame}

%%---------------------------------------------------------------
\begin{frame}[fragile]


% http://tldp.org/LDP/intro-linux/html/sect_03_01.html
El primer caracter de \verb|ls -l| representa:
\begin{verbatim}
-       Regular file
d       Directory
c       Special file
l       Link
p       Named pipe
s       Socket
b       Block device
\end{verbatim}


\end{frame}



\begin{frame}[fragile]
\frametitle{Jerarquía del Sistema de Ficheros }
Para quien se acerca a Linux resulta confuso un \verb|ls -l / |
\begin{footnotesize}
\begin{verbatim}
drwxr-xr-x    2 root     root         4096 ene 30 20:34 bin
drwxr-xr-x    2 root     root         4096 mar 12 19:46 boot
drwxr-xr-x    5 root     root        24576 may 22 06:27 dev
drwxr-xr-x   66 root     root         4096 may 19 00:26 etc
drwxrwsr-x    7 root     staff        4096 abr 16 17:36 home
drwxr-xr-x    6 root     root         4096 feb  1 18:02 lib
drwxr-xr-x    2 root     root        16384 nov  7  2000 lost+found
dr-xr-xr-x    2 root     root         4096 nov 10  2000 mix
dr-xr-xr-x   67 root     root            0 may 19 02:25 proc
drwxr-xr-x   14 root     root         4096 feb 12 19:28 root
drwxr-xr-x    2 root     root         4096 ene 30 20:30 sbin
drwxrwxrwt    9 root     root         4096 may 22 10:19 tmp
drwxr-xr-x   15 root     root         4096 nov  8  2000 usr
drwxr-xr-x   16 root     root         4096 nov  9  2000 var
\end{verbatim}
\end{footnotesize}

\end{frame}


%%---------------------------------------------------------------

\begin{frame}[fragile]
\begin{itemize}

\item La estructura de todos los Unix se \emph{parece}
\item La estructura de todas las distribuciones Linux 
se \emph{parece mucho}
\end{itemize}
\end{frame}


%%---------------------------------------------------------------


\begin{frame}[fragile]

\frametitle{Jerarquía clásica}
La jerarquía actual puede resultar algo ilógica, 
pero hay motivos históricos

En los primeros Unix los discos eran más pequeños y más caros, 

en uno estaba lo \emph{imprescindible} para que el sistema funcionase:
\begin{verbatim}
/
/etc
/lib
/tmp
/bin
/root
\end{verbatim}
\end{frame}


%%---------------------------------------------------------------

\begin{frame}[fragile]

y en un segundo disco, se montaba \verb|/usr|
\begin{verbatim}
/usr/spool
/usr/bin
/usr/include
/usr/tmp
/usr/adrn
/usr/lib
\end{verbatim}
\end{frame}


%%---------------------------------------------------------------
\section{FHS Filesystem Hierarchy Standard}
%%---------------------------------------------------------------

\begin{frame}[fragile]
\frametitle{FHS Filesystem Hierarchy Standard}
Estándar propuesto para todos los Linux y para los UNIX que
quieran unirse. Año 1994.  Versión actual: 2.3 (enero 2004)

Dos criterios

¿Un fichero puede almacenarse en una máquina y usarse en otra?
\begin{itemize}	
\item
Sí: Compartibles. (\emph{shareable}) 
\item 
No: No compartibles. (\emph{unshareable})
% pj locks
\end{itemize}

¿Un fichero puede cambiar sin intervención del administrador?

\begin{itemize}	
\item 
Sí: Dinámicos. 
\item 
No: Estáticos. Pueden almacenarse el modo sólo-lectura. Copias
de seguridad menos frecuentes
\end{itemize}

\end{frame}


%%---------------------------------------------------------------


\begin{frame}[fragile]

\begin{enumerate}
\item Directorios de usuarios
\item Programas (incluyendo mandatos y librerías)
\item Configuración del sistema        
\item El Hardware
\item Documentación
\item Ficheros Temporales
\item Otros directorios relacionados con el S.O.
\item Puntos de montaje
\end{enumerate}

\end{frame}


%%---------------------------------------------------------------
\subsection{Directorios de usuarios}
%%---------------------------------------------------------------


\begin{frame}[fragile]
\frametitle{Directorios de usuarios}
\begin{itemize}

\item 

Directorio del administrador

\verb|/root|

\item 

Usuarios locales

\verb|/home/jperez|

o bien

\verb|/home/profesores|

\verb|/home/alumnos|
\item 
Usuarios NIS

\verb|/users/jperez|

\end{itemize}
\end{frame}


%\begin{frame}[fragile]

%No forma parte de la norma FHS pero es importante:

%\verb|~/.bash_profile|

%(No confundir con 
%\verb|/etc/profile|   )

%\end{frame}

%---------------------------------------------
\subsection{Programas y mandatos} 
%---------------------------------------------

\begin{frame}[fragile]

\frametitle{Programas y mandatos} 
\begin{itemize}

\item 
Mandatos útiles para todos los usuarios

/bin

/usr/bin

\item
Mandatos útiles para el root

/sbin

/usr/sbin

\end{itemize}
(Todo lo que haya bajo \verb|/usr| debería ser sólo lectura)
\end{frame}

%------------------------------------------

\begin{frame}[fragile]
\begin{itemize}

\item  Programas

        \begin{itemize}
\item

Software no incluido en la distribución Linux 

/usr/local



\item
Grandes aplicaciones en la distribución

/opt


        \end{itemize}
\end{itemize}
\end{frame}

%----------------------------------------------------

\begin{frame}[fragile]
\begin{itemize}


\item
Librerías estáticas y dinámicas

/lib

/usr/lib

/usr/local/lib

\item
Ficheros de cabecera (para compilar) 

/usr/include


\item
Ficheros independientes de la arquitectura

/usr/share



\end{itemize}
\end{frame}


%----------------------------------------------------
\subsection{Configuración del sistema}        
%----------------------------------------------------
\begin{frame}[fragile]
\frametitle{Configuración del sistema}        

Directorio \verb|/etc|
\begin{itemize}

\item  

Información sobre el sistema de ficheros (puntos de montaje, opciones)

/etc/fstab
\item  
cuentas de usuarios

/etc/passwd
\item  
Passwords de los usuarios

/etc/shadow
\item  
Scripts para arranque del sistema

/etc/init.d
\item  
...
\end{itemize}
\end{frame}


%----------------------------------------------------
\subsection{El Hardware}
%----------------------------------------------------

\begin{frame}[fragile]
\frametitle{El Hardware}

Los dispositivos del sistema
\verb|/dev|
\begin{verbatim}
/dev/hda     IDE primario master 
/dev/hdb     IDE primario slave
/dev/hdc     IDE secundario master 
/dev/hdd     IDE secundario slave 

/dev/hda1    Primera partición primaria del hda
/dev/hda2    ...

/dev/sda     Primer disco SCSI
/dev/sdb     Segundo disco SCSI
/dev/sda1    ...


\end{verbatim}
\end{frame}
%%----------------------------------------------
\begin{frame}[fragile]

\begin{verbatim}
/dev/cdrom
/dev/fd0     disquete 
/dev/audio   tarjeta sonido
/dev/modem
/dev/mouse
/dev/input/mouse0
/dev/ttyN    donde N es el nº de consola (no gráfica)
/dev/pts/N   Consola (X Window)

\end{verbatim}
El estándar no dice mucho sobre \verb|/dev|, es bastante variable

\end{frame}


%---------------------------------------------


\begin{frame}[fragile]

\begin{itemize}
  
\item 
Ficheros \emph{virtuales} que representan las estructuras del Kernel
en ejecución,  dan información sobre la cpu...


  \begin{footnotesize}
  \begin{verbatim}
/proc/cpuinfo     CPU
/proc/pci         Tarjetas PCI
/proc/ioports     Puertos I/O
/proc/meminfo     Información sobre la memoria
/proc/NN          Información sobre el proceso de pid NN
  \end{verbatim}
  \end{footnotesize}

\end{itemize}
  


Los directorios \texttt{\textbf{/proc}} y \texttt{\textbf{/sys}} no se corresponden
con discos físicos, sino que son un medio de enviar y recibir información directamente del \emph{kernel}.

Cuando se   lee o se escribe algún fichero del \texttt{\textbf{/proc}}, se está
pidiendo o recibiendo información del kernel
\end{frame}



%---------------------------------------------
\subsection{Documentación}
%---------------------------------------------

\begin{frame}[fragile]
\frametitle{Documentación}
\begin{itemize}

\item

/usr/share/doc

Documentación sobre el software del sistema
\item
/usr/man

Ficheros del mandato \emph{man}

\end{itemize}
\end{frame}

%----------------------------------------------------
\subsection{Ficheros Temporales}
%----------------------------------------------------

\begin{frame}[fragile]

\frametitle{Ficheros Temporales}
\begin{itemize}

\item 
Ficheros temporales 

(se borran cuando la máquina arranca)

\verb|/tmp|
\item
Fragmentos de ficheros recuperados

\verb|/lost+found|
\end{itemize}
\end{frame}



\begin{frame}[fragile]
\begin{itemize}

\item
Ficheros que cambian con frecuencia

\verb|/var|

  \begin{footnotesize}
  \begin{verbatim}
/var/log/syslog        bitácora principal del sistema  
/var/log/messages      otra bitácora con diversos mensajes
/var/log/dmesg         mensajes del sistema al arrancar
/var/spool/lpd/lp      spool de la impresora
/var/tmp               Ficheros temporales
/var/mail              Correo de los usuarios
/var/run               PID de programas en ejecución
  \end{verbatim}
  \end{footnotesize}

\end{itemize}
\end{frame}

%----------------------------------
\subsection{Otros directorios relacionados con el S.O.}
%----------------------------------

\begin{frame}[fragile]

\frametitle{Otros directorios relacionados con el S.O.}
\begin{itemize}

\item 
\verb|/boot|

Todo lo requerido para el arranque, antes de que el sistema ejecute
programas de usuario


\item
Código fuente
        \begin{itemize}
\item 
Código fuente del software de sistema

\verb|/usr/src|

\item 
Código fuente del kernel linux

\verb|/usr/src/linux|
        \end{itemize}

\end{itemize}

\end{frame}




%%---------------------------------------------------------------
\subsection{Puntos de Montaje}
%%---------------------------------------------------------------
\begin{frame}[fragile]
\frametitle{Puntos de Montaje}

Unidades extraibles: Disquetes, cdrom, \emph{pendrives}

Solían colocarse en el raiz
p.e. \verb|/cdrom|. Pero esto llena el raiz de directorios

En FHS 2.3 (año 2004) aparece 
\verb|/media|

\verb|/media/cdrom  /media/cdrecorder  /media/zip  /media/floppy|


\begin{itemize}
\item 
Si solo hay uno de un tipo:

\verb|/media/cdrom|

\item 
Si hay más de uno del mismo tipo

\verb|/media/cdrom0|

\verb|/media/cdrom1|

\verb|/media/cdrom -> /media/cdrom1|

\end{itemize}


\end{frame}



%%---------------------------------------------------------------
\begin{frame}[fragile]
\begin{verbatim}
/mnt
\end{verbatim}

Directorio vacío para que el administrador monte 
un sistema de ficheros que necesita temporalmente. Los programas no deberían usarlo

\begin{itemize}	
\item
\verb|/mnt/cdrom      |    ¡No es estándar!

Es una costumbre reciente, va contra el estándar. Dentro  de \verb|/mnt| 
debe estar directamente el sistema de ficheros temporal, sin subdirectorios
% el estándar es una costumbre mucho más antigua
\end{itemize}



\end{frame}

%---------------------------------------
  \section{Montaje de sistemas de ficheros }
%---------------------------------------
\begin{frame}[fragile]
  \frametitle{Montaje de sistemas de ficheros }
  \begin{itemize}
  \item Normalmente, no todos los ficheros del árbol de directorios se
    encuentran en el mismo disco.
  \item \emph{Punto de montaje}: directorio que pertenece a un disco
    (o \emph{partición}) distinto, junto con todo su contenido
    (excluyendo otros puntos de montaje).
%  \item El \emph{kernel} incluye información de todas las particiones
%    y puntos de montaje asociados en el fichero
%    \texttt{\textbf{/proc/mounts}} (el fichero
%    \texttt{\textbf{/etc/mtab}} contiene prácticamente la misma
%    información).
  \item Se pueden consultar los puntos de montaje junto con
    los discos o particiones que están \emph{montadas} en ellos con
    las órdenes \emph{mount} y \emph{df}
  \end{itemize}
\end{frame}

%---------------------------------------
  \subsection{mount, df}
%---------------------------------------
\begin{frame}[fragile]
  \frametitle{mount, df}
  \begin{itemize}
  \item \texttt{\textbf{mount}}: Muestra las particiones, puntos de
    montaje, tipo de partición y opciones de cada una de ellas:
    {\flushleft\fontsize{7pt}{7pt}\selectfont
\begin{verbatim}
/dev/hda2 on / type ext3 (rw,noatime)
proc on /proc type proc (rw)
sysfs on /sys type sysfs (rw)
devpts on /dev/pts type devpts (rw,gid=5,mode=620)
/dev/hda5 on /scratch type ext3 (ro,noatime)
tmpfs on /tmp type tmpfs (rw)
\end{verbatim}
    }
  \item \texttt{\textbf{df}}: Muestra cada una de las particiones
    \emph{con ficheros reales} montadas en el sistema, el punto en el que
    está montada, su capacidad y su uso:
    {\flushleft\fontsize{7pt}{7pt}\selectfont
\begin{verbatim}
Filesystem           1K-blocks      Used Available Use% Mounted on
/dev/hda2             28842780   6957692  20419960  26% /
/dev/hda5             38448276  32838556   3656620  90% /scratch
tmpfs                   517960      1196    516764   1% /tmp
\end{verbatim}
    }
  \end{itemize}
\end{frame}


\begin{frame}[fragile]
  Para montar un sistema de ficheros 
  \begin{itemize}
  \item Crear el directorio si no existe:

    \texttt{\textbf{mkdir /var}}
  \item Hacer visible el sistema de ficheros bajo ese directorio:

    \texttt{\textbf{mount -t ext2 -o rw /dev/hda3 /var}}

(es más habitual indicar las opciones en \verb|/etc/fstab|)

  \item Si queremos desmontar (o hacer invisible) un sistema de
    ficheros que esté montado en el directorio \texttt{\textbf{/var}}:

    \texttt{\textbf{umount /var}}
  \end{itemize}
\end{frame}

%---------------------------------------
\begin{frame}[fragile]
  %\subsection{\texttt{\textbf{/etc/fstab}}}
  \subsection{/etc/fstab}
%http://www.tuxfiles.org/linuxhelp/fstab.html

    {\flushleft\fontsize{7pt}{7pt}\selectfont
\begin{verbatim}
# <filesystem> <mount point>   <type>  <options>         <dump><pass>
proc           /proc           proc   defaults              0    0
/dev/hda2      /               ext3   noatime               0    1
/dev/hda5      /scratch        ext3   noatime,ro            0    1
/dev/hda6      none            swap   sw                    0    0
tmpfs          /tmp            tmpfs  defaults              0    0
/dev/sda1      /media/pendrive vfat   defaults,user,noauto  0    0

\end{verbatim}
    }

  \begin{itemize}
  \item \verb|mount -a| monta todo lo indicado en este fichero
  \item En el arranque se ejecuta \verb|mount -a|
  \item \verb|mount /media/pendrive|  

monta el pendrive con todas
  las opciones indicadas en fstab

  \end{itemize}


 \verb|<dump>|
¿Incluir en las copias de seguridad hechas con \emph{dump}? (Normalmente no)

 \verb|<pass>|
Orden para el \verb|fsck| del arranque (0: desactivado).
\end{frame}
%---------------------------------------
\begin{frame}[fragile]
%  \subsection{Opciones de los sistemas de ficheros}
\begin{verbatim}
<options>
\end{verbatim}
  \begin{itemize}
  \item \texttt{\textbf{rw}}: Permisos de lectura y escritura.
  \item \texttt{\textbf{ro}}: Sólo lectura.
  %\item \texttt{\textbf{loop}}: Permite montar un \emph{fichero} en lugar de un dispositivo de bloques.
  \item \texttt{\textbf{auto/noauto}}: ¿Montar automáticamente con \verb|mount -a|?
  \item \texttt{\textbf{user/nouser}}: ¿Los usuarios normales pueden montar y desmontar? 
(o hace falta ser \texttt{\textbf{root}}) 
  \item \texttt{\textbf{exec/noexec}}: ¿Se pueden ejecutar binarios?
  \item \texttt{\textbf{sync}}: Al modificar un fichero, se escribe físicamente de inmediato
  \item \texttt{\textbf{async}}: Se usan buffers 
  \item \texttt{\textbf{defaults}}: \verb|rw, suid, dev, exec, auto, nouser, async|
  \item \ldots 
  \end{itemize}
\end{frame}

% dev /nodev : ¿se puede hacer mknod? Esto es ¿puede haber dispositivos en ese
% sistema de ficheros

%---------------------------------------
  \subsection{Tipos de sistemas de ficheros}
%---------------------------------------
\begin{frame}[fragile]
  \frametitle{Tipos de sistemas de ficheros}
  \begin{itemize}
%  \item Los que reconoce el \emph{kernel} están en \texttt{\textbf{/proc/filesystems}}.
  \item Tradicionales
    \begin{itemize}
    \item \texttt{\textbf{msdos}}: El usado por MS-DOS y Windows
      pre-95, sin permisos ni dueños, nombres de fichero de 8
      caracteres con extensiones de 3 caracteres
    \item \texttt{\textbf{vfat}}: Usado a partir de Windows-95,
      compatible con MS-DOS pero con posibilidad de nombres de fichero
      largos
    \item \texttt{\textbf{ntfs}}: Desde Windows NT hasta Windows XP. Añade
      características de seguridad (permisos, dueños, etc). Los primeros 
     \emph{drivers} para Linux tenían limitaciones, en la actualidad se puede
      leer y escribir con normalidad 
% en marzo 2007 el driver de serie permite leer y modificar ficheros. No crear
% nuevos ni borrarlos. Hay otros drivers beta que sí pueden
% redhat no incluye este driver por dudar legales
% MS anunció en 2003 aprox el sistema de ficheros winfs, un sistema de ficheros
% como el de las BD relaciones. En 2006 lo abandonó
% vista lleva un nuevo sistema de ficheros, llamado txfs, que es un ntfs con
% journaling
    \item \texttt{\textbf{iso9660}}: Sistema de fichero utilizado en
      los CDs de datos
    \item \texttt{\textbf{minix}}: usado por MINIX y por los
      primeros Linux
    \item \texttt{\textbf{ext2}}: Sistema de ficheros tradicional    
      en Linux
    \end{itemize}
  \end{itemize}
\end{frame}

%---------------------------------------
\begin{frame}[fragile]
  \begin{itemize}
  \item Con \emph{journal}
%http://en.wikipedia.org/wiki/Journaling_file_system
    \begin{itemize}
    \item \texttt{\textbf{ext3}}: Siguiente versión del
      \texttt{\textbf{ext2}}, idéntico pero con adición de
      \emph{journal}. El más utilizado actualmente
    \item \texttt{\textbf{reiserfs}}, \texttt{\textbf{jfs}}, \texttt{\textbf{xfs}}

    Mejores prestaciones, pero incompatibles con   
    \texttt{\textbf{ext2}}
    \end{itemize}
  \item Con características especiales :

 \verb|romfs, cramfs, autofs, umsdos|
  \item No asociados a dispositivo

 \verb|proc, sysfs, devfs, devpts, tmpfs, ramfs, usbfs|
  \end{itemize}
\end{frame}
%---------------------------------------
%\begin{frame}[fragile]
%  \subsection{Tipos de sistemas de ficheros (3)}
%  \begin{itemize}
%  \item Con características especiales:
%    \begin{itemize}
%    \item \texttt{\textbf{romfs}}: ideado para discos pequeños y de
%      solo lectura, típicamente discos grabados en ROM o en EPROM.
%    \item \texttt{\textbf{cramfs}}: similar al
%      \texttt{\textbf{romfs}}, pero incluyendo compresión de los
%      datos.
%    \item \texttt{\textbf{autofs}}: pseudo-sistema de ficheros para
%      permitir el auto-montaje de unidades extraíbles.
%    \item \texttt{\textbf{umsdos}}: Compatible con discos MS-DOS,
%      añade características esenciales para sistemas UNIX (dueños y
%      grupos, permisos, nombres largos, enlaces, etc).
%    \end{itemize}
%  \end{itemize}
%\end{frame}

%---------------------------------------
%\begin{frame}[fragile]
%  \subsection{Tipos de sistemas de ficheros (4)}
%  \begin{itemize}
%  \item No asociados a dispositivo:
%    \begin{itemize}
%    \item \texttt{\textbf{proc}}, \texttt{\textbf{sysfs}}: Información
%      proporcionada por el \emph{kernel} a los procesos de usuario.
%    \item \texttt{\textbf{devfs}}, \texttt{\textbf{devpts}}:
%      Sistema de ficheros en el que se generan automáticamente
%      los dispositivos reconocidos por el \emph{kernel} en cada
%      momento.
%    \item \texttt{\textbf{tmpfs}}, \texttt{\textbf{ramfs}}: Sistema de
%      ficheros de lectura/escritura, con contenido en memoria, con
%      carácter temporal y para permitir un acceso más rápido.
%    \item \texttt{\textbf{usbfs}}: lista de dispositivos conectados al
%      \emph{bus} USB y características de cada uno de ellos.
%    \end{itemize}
%  \end{itemize}
%\end{frame}

%---------------------------------------
\begin{frame}[fragile]
  \begin{itemize}
  \item Remotos:
    \begin{itemize}
    \item \texttt{\textbf{nfs}}: \emph{Network File System},
      desarrollado por SUN, el más usado entre los sistemas ficheros
      remotos en UNIX
%.  proporcionada por el \emph{kernel} a los procesos de usuario.
    \item \texttt{\textbf{smb}}/\texttt{\textbf{cifs}}: Sistema de
      ficheros remotos usado por Microsoft
%cifs, 1996 es una mejoría para smb
    \item \texttt{\textbf{ncp}}: \emph{Netwate Core Protocol},
      protocolo sobre IPX para montar sistemas de ficheros de Novell
      Netware
%    \item \texttt{\textbf{coda}}: Sistema de ficheros remoto avanzado,
%      con muchas ventajas sobre NFS, pero muy complicado de poner en
%      funcionamiento.
%    \item \texttt{\textbf{afs}}: \emph{Andrew filesystem}.
    \item \texttt{\textbf{sshfs}}: \emph{Secure SHell FileSystem},
protocolo basado en ssh
    \end{itemize}

%SSHFS (Secure SHell FileSystem) is a file system for Linux (and other
%operating systems with a FUSE implementation, such as Mac OS X or FreeBSD)
%capable of operating on files on a remote computer using just a secure
%shell login on the remote computer. On the local computer where the
%SSHFS is mounted, the implementation makes use of the FUSE (Filesystem
%in Userspace) kernel module. The practical effect of this is that the
%end user can seamlessly interact with remote files being securely served
%over SSH just as if they were local files on his/her computer. On the
%remote computer the SFTP subsystem of SSH is used.



\item
Soporte de otras plataformas:

    \texttt{\textbf{hfs}} (Apple Macintosh),
 \texttt{\textbf{bfs}} (\emph{Boot File System}, SCO),
    \texttt{\textbf{efs}} (SGI, IRIX), \texttt{\textbf{jffs}}
    (\emph{Journaling Flash File System}), 
\texttt{\textbf{hpfs}} (OS/2),
    \texttt{\textbf{qnx4}}, \texttt{\textbf{sysv}} (System V),
    \texttt{\textbf{ufs}} (SunOS, FreeBSD, NetBSD, OpenBSD)\ldots

  \end{itemize}
\end{frame}


%%  en 2003 Ms presentó Winfs, previsto para vista. 
%% todo era una una base de datos, muchos metadatos
%% muere oficialmente en junio de 2006, donando organos a sqlserver



%%---------------------------------------------------------------
\subsection{Sistemas de Ficheros en Espacio de usuario}
%%---------------------------------------------------------------
\begin{frame}[fragile]
\frametitle{Sistemas de Ficheros en Espacio de usuario}
\begin{itemize}	
\item
Los sistemas de ficheros tradicionales están implementados 
en el núcleo. Añadir uno sistema de ficheros es complicado, 
y puede comprometer la integridad del sistema.
\item 
Los sistemas de ficheros en espacio de usuario son aplicaciones
\emph{normales}
\item 
Para Linux, FreeBSD, NetBSD, OpenSolaris y Mac OS X exite FUSE
\emph{Filesystem in Userspace}. Es un módulo del núcleo que
actúa de puente entre el núcleo y el código del sistema de
ficheros

\end{itemize}
\end{frame}



%%---------------------------------------------------------------
\begin{frame}[fragile]
Ejemplos de sistemas de ficheros FUSE
\begin{itemize}	
\item
sshfs
\item 
GmailFS. Almacena los datos sobre correos de gmail. No es fiable
porque no está aprobado por google. (Tampoco prohibido, al menos
explícitamente)
\item 
Acceso a ficheros empaquetados (tgz, zip, etc)
\item
Almacenamiento en Bases de Datos
\item
Encriptación
\item
Hardware poco común
\item
Sistemas de versiones de ficheros (CVS, SVN...)
\item
Monitorización de sistemas de ficheros
\end{itemize}

\end{frame}

%%---------------------------------------------------------------
\begin{frame}[fragile]
\subsection{sshfs}
Secure SHell FileSystem. Basado en FUSE. Sistema de ficheros de red
\begin{itemize}	
\item
Menos eficiente pero más seguro que NFS
\item
En el servidor basta disponer del demonio ssh convencional
\item 
En el cliente basta instalar el paquete \verb|sshfs|
\end{itemize}

Montar el \emph{home} remoto:

\verb|sshfs -C usuario@maquina: /punto/de/montaje|


Montar un directorio remoto


\verb|sshfs -C usuario@maquina:/un/directorio  /punto/de/montaje|


Desmontar:

\verb|fusermount -u /punto/de/montaje|


\end{frame}



%%---------------------------------------------------------------
\begin{frame}[fragile]

%http://www.netsplit.com/blog/articles/2006/08/26/upstart-in-universe
\subsection{upstart}

\begin{itemize}
\item
El sistema de arranque tradicional de Linux (System V)
no es adecuado para las máquinas actuales
\begin{itemize}
\item
Son externos: aparecen y desaparecen
\item
Están en red
\item
Ahorran energía
\item \ldots
\end{itemize}

% Storage buses allow more than a fixed number of drives, so they must be scanned for; this operation frequently does not block.
% Firmware may need to be loaded after the device has been detected, but before it is usable by the system.
% Mounting a partition in /etc/fstab may require tools in /usr which is a network filesystem that cannot be mounted until after networking has been brought up.

\item
\emph{Upstart} es un sistema de arranque basado en eventos,
desarrollado por Ubuntu, con el propósito de extenderlo
a todos los Linux

Aparece en Ubuntu 6.10 \emph{edgy} (Octubre de 2006)

\item
Alternativas: \emph{launchd} (MacOS X), \emph{initng},  SMF 
%smf: solaris

\item
Está previsto que reemplace a \emph{cron} y tal vez a  \emph{inetd}, manteniendo siempre la compatibilidad
\end{itemize}


\end{frame}


%%---------------------------------------------------------------
\begin{frame}[fragile]

En \emph{upstart} se modifica la columna \verb|<filesystem>| de \verb|/etc/fstab|,
%(a menos que esté el \emph{sticky bit} activado 
%\verb|   chmod [+-]t dir|)

incorporando un \emph{Universally Unique Identifier }
\begin{tiny}
\begin{verbatim}
# <file system> <mount point>   <type>                      <options><dump><pass>
proc            /proc           proc                               defaults 0 0
UUID=e8a76033-f833-490d-8a55-ceca132c2ba7 / ext3 defaults,errors=remount-ro 0 1
UUID=e38c8abf-1af7-49be-bba5-bcf45dab8dc2 /home               ext3 defaults 0 2
UUID=967cf88c-7b0b-42a9-bf93-deb7b710aad2 /media/sda6         ext3 defaults 0 2
UUID=f5c3bc51-7795-4bc9-b18e-4a16b7496e93 none                      swap sw 0 0
/dev/hda        /media/cdrom0                       udf,iso9660 user,noauto 0 0
\end{verbatim}

\end{tiny}



\end{frame}




%%---------------------------------------------------------------
\section{Codificación de caracteres}
%%---------------------------------------------------------------
%http://en.wikipedia.org/wiki/Code_pages
\begin{frame}[fragile]
\frametitle{Codificación de caracteres}
Correspondencia entre un carácter de lenguaje natural
y un símbolo en otro sistema de representación. En informática,
uno o más octetos

A veces se llama \emph{code pages} (IBM, Microsoft)

\subsection{Codificaciones clásicas}

\begin{itemize}	
\item 
EBCDIC: Extended Binary Coded Decimal Interchange Code. IBM, año 1963. 8 bits. 
Se usa en algunos equipos IBM.
Diferentes versiones incompatibles entre sí
\item 
ASCII: American Standard Code for Information Interchange. 
ANSI, American National Standards Institute, año 1963). 7 bits. Solo inglés
\end{itemize}
\end{frame}


%%---------------------------------------------------------------
\subsection{ASCII extendido}
%%---------------------------------------------------------------
\begin{frame}[fragile]
\frametitle{ASCII extendido}
8 bits.  Cada conjunto
de idiomas necesita su propia variante. Compatible con ASCII
\begin{itemize}	
\item 
Code Pages 437. Inglés. Primeros IBM PC, MS-DOS

Code Pages 850. Europa occidental. Primeros IBM PC, MS-DOS
\item 
ISO-8859 (Organización Internacional para la Estandarización), año 1992. 
Habitual en linux hasta mediados de los años \emph{cerenta}

ISO-8859-1, informalmente conocido como Latin-1

ISO-8859-2 europa central, ISO-8859-5 cirílico , ISO-8859-6 árabe, ...

ISO-8859-15 o Latin-9. Año 1998. Muy parecido a Latin-1, incluye
%\textgreek{\euro} 
el símbolo del euro
\item
windows-1252. Parecido a ISO-8859-1. Se confunden con frecuencia. 
Se empleaba en los primeros Windows
\end{itemize}
\end{frame}
%%---------------------------------------------------------------

%las codificaciones anteriores permiten trabajar en tu lengua local
%y en ingles. Pero no entre varias lenguas distintas al tiempo.
% iso 8859, sin información extra no se puede saber en qué idioma está.



% unicode incluye élfico, klingon, geroflíficos egipcios y el rongorongo:
% unos geroglíficos usados en la isla de pascua en la ¿edad media?
% que ni siquiera se han podido descifrar


\begin{frame}[fragile]
\subsection{Unicode}
Estándar industrial. \emph{Unicode Consortium}, año 1991. 
Compatible con ISO 10646.

Asocia un número a cada carácter empleado por algún lenguaje escrito del mundo.
Más de 100.000 caracteres

Se puede codificar de diferentes maneras
\begin{itemize}	
\item 
UTF-8 es la forma en Unix de codificar unicode. 

Compatible
con ASCII. Cada carácter ocupa entre 1 y 4 octetos
% lo hizo ken thomoson, para plan 9. tiene la gracia de que se autosincroniza:
% si lees un chorro de bytes puedes entenderlo si saber dónde ha empezado
% (en otros codificaciones no sabrías donde está el límite entre caracteres)
\item 
UTF-16. Cada carácter ocupa entre 2 y 4 octetos.

Nativo en Windows desde Windows 2000, aunque
se seguía usando windows-1252.
\item 
Punycode. RFC 3492. Empleado en la Internacionalización 
de Nombres de Dominio en Aplicaciones (IDNA). Años 2003-2005. Permite nombres de dominio en unicode.

\verb|españa.es   ->   xn--espaa-rta.es|

\verb|ortuño.es   ->   xn--ortuo-rta.es|

\item 
UCS-2, UCS-4, SCSU, ...
\end{itemize}

\end{frame}
%%----------------------------------------------
\subsection{recode}
%%----------------------------------------------
\begin{frame}[fragile]

\frametitle{recode}
%El mandato \emph{recode} 
Orden que convierte ficheros entre diferentes codificaciones
\begin{itemize}	
\item
\verb|recode utf-8|

Lee \emph{stdin}, convierte desde utf-8 hasta las locales actuales
y escribe en \emph{stdout}
\item
\verb|recode latin-1..utf-8|

Lee \emph{stdin}, convierte desde latin-1 hasta utf-8 
y escribe en \emph{stdout}
\item
\verb|recode  utf-8..windows-1252 fichero|

Modifica el fichero, convirtiendo desde utf-8 hasta 
windows-1252

\end{itemize}
\end{frame}







%---------------------------------------
%\begin{frame}[fragile]
%  \subsection{Tipos de sistemas de ficheros (6)}
%  \begin{itemize}
%  \item Además de los mencionados, en Linux hay soporte para muchos
%    otros sistemas de ficheros, la mayor parte de ellos utilizados por
%    otras plataformas para guardar su información.
%  \item Entre ellos están: \texttt{\textbf{udf}},
%    \texttt{\textbf{adfs}} (\emph{Acorn Disc Filing System}),
%    \texttt{\textbf{affs}} (\emph{Amiga Fast File System}),
%    \texttt{\textbf{asfs}} (\emph{Amiga Smart File System}),
%    \texttt{\textbf{hfs}} (Apple Macintosh), \texttt{\textbf{befs}}
%    (BeOS), \texttt{\textbf{bfs}} (\emph{Boot File System}, SCO),
%    \texttt{\textbf{efs}} (SGI, IRIX), \texttt{\textbf{jffs}}
%    (\emph{Journaling Flash File System}), \texttt{\textbf{vxfs}}
%    (VERITAS VxFS), \texttt{\textbf{hpfs}} (OS/2),
%    \texttt{\textbf{qnx4}}, \texttt{\textbf{sysv}} (System V),
%    \texttt{\textbf{ufs}} (SunOS, FreeBSD, NetBSD, OpenBSD)\ldots
%  \end{itemize}
%\end{frame}


%---------------------------------------
%\begin{frame}[fragile]
%  \subsection{Cómo añadir un disco al sistema}

%  Para añadir un disco nuevo al sistema, es necesiario:
%  \begin{enumerate}
%  \item Preparar las particiones del disco
%  \item Crear el sistema de ficheros
%  \item Elegir (y crear) el punto de montaje
%  \item Montar el nuevo disco,
%  \item Añadir contenido al disco.
%  \item Modificar \texttt{\textbf{/etc/fstab}} para que se monte
%    automáticamente en el arranque.
%  \end{enumerate}
%\end{frame}
%
%---------------------------------------
%\begin{frame}[fragile]
%  \subsection{1. Preparar las particiones del disco}
%  \begin{itemize}
%  \item No siempre es necesario: en disquetes no se utilizan nunca, y
%    en llaveros USB, a veces tampoco se hacen particiones y se utiliza
%    todo el disco (\texttt{\textbf{/dev/sda}} en lugar de
%    \texttt{\textbf{/dev/sda1}}).
%  \item Uso de las órdenes ``\texttt{\textbf{fdisk}}'',
%    ``\texttt{\textbf{cfdisk}}'' o ``\texttt{\textbf{sfdisk}}'' para
%    modificar la tabla de particiones.
%    \begin{itemize}
%    \item \texttt{\textbf{fdisk}}: Herramienta original, es necesario
%      hacer todo ``a mano'', algo complicada.
%    \item \texttt{\textbf{cfdisk}}: Con menús a toda pantalla, bonito,
%      fácil de usar.
%    \item \texttt{\textbf{sfdisk}}: Orientado para ser usado desde
%      \emph{scripts}.
%    \end{itemize}
%  \end{itemize}
%\end{frame}
%
%%---------------------------------------
%\begin{frame}[fragile]
%  \subsection{2. Crear el sistema de ficheros}
%  \begin{itemize}
%  \item Hay que elegir el tipo de sistema de ficheros más adecuado.
%  \item Normalmente, a menos que haya necesidades especiales, se
%    utilizará el mismo sistema para todas las particiones. Se suele
%    elegir ``\texttt{\textbf{ext3}}'', o bien algún otro sistema con
%    \emph{journaling}.
%  \item Creación del sistema de ficheros con la orden
%    ``\texttt{\textbf{mkfs}}'':
%
%    \texttt{\textbf{mkfs -t ext3 /dev/hda3}}
%  \end{itemize}
%\end{frame}
%
%%---------------------------------------
%\begin{frame}[fragile]
%  \subsection{3. Elegir (y crear) el punto de montaje}
%  \begin{itemize}
%  \item Los puntos de montaje más habituales, entre los directorios
%    ``estándar'', son: \texttt{\textbf{/usr}}, \texttt{\textbf{/var}},
%    \texttt{\textbf{/home}}, \texttt{\textbf{/tmp}},
%    \texttt{\textbf{/var/tmp}}.
%  \item Pero se puede crear cualquier otro, dependiendo de nuestras
%    necesidades, por ejemplo \texttt{\textbf{/musica}}.
%  \item Antes de poder \emph{montar} un nuevo sistema de ficheros en
%    un directorio, es necesario que este directorio exista: hay que
%    crearlo con \texttt{\textbf{mkdir}}.
%  \item Los permisos que tenga este directorio no son muy relevantes,
%    ya que desaparecerán una vez montado el nuevo sistema de ficheros.
%  \end{itemize}
%\end{frame}
%
%%---------------------------------------
%\begin{frame}[fragile]
%  \subsection{4. Montar el nuevo disco}
%  \begin{itemize}
%  \item Si queremos montar un disco de manera temporal, para
%    examinarlo o para comprobar que ha sido creado correctamente, se
%    suele hacer en un directorio espeacial para montajes temporales,
%    el \texttt{\textbf{/mnt}}:
%
%    \texttt{\textbf{mount -t ext3 /dev/hda3 /mnt}}
%  \end{itemize}
%\end{frame}
%
%%---------------------------------------
%\begin{frame}[fragile]
%  \subsection{5. Añadir contenido al disco}
%  \begin{itemize}
%  \item El contenido original del punto de montaje, al igual que los
%    permisos que tenga, desaparecerán una vez montado el nuevo sistema
%    de ficheros.
%  \item Si queremos que el nuevo disco tenga la misma información que
%    tenía ese directorio, hay que hacer una copia (con
%    \texttt{\textbf{cp}}, \texttt{\textbf{rsync}}\ldots):
%
%    \texttt{\textbf{rsync -avHWSx /var /mnt}}
%  \end{itemize}
%\end{frame}

%%---------------------------------------
%\begin{frame}[fragile]
%  \subsection{6. Modificar \texttt{\textbf{/etc/fstab}}}
%  \begin{itemize}
%  \item Por cada punto de montaje que tengamos en el sistema, es
%    necesaria una línea nueva en el fichero
%    \texttt{\textbf{/etc/fstab}}.
%  \item Normalmente, las opciones no se modifican (``\texttt{\textbf{defaults}}'').
%
%  \item El contenido original del punto de montaje, al igual que los
%    permisos que tenga, desaparecerán una vez montado el nuevo sistema
%    de ficheros.
%  \item Si queremos que el nuevo disco tenga la misma información que
%    tenía ese directorio, hay que hacer una copia (con
%    \texttt{\textbf{cp}}, \texttt{\textbf{rsync}}\ldots)
%  \end{itemize}
%\end{frame}

%%---------------------------------------
%\begin{frame}[fragile]
%  \subsection{Ejemplo de montaje de un nuevo disco}
%
%  Estas acciones deberán realizarse preferiblemente en modo
%  monousuario, para evitar que haya procesos que interfieran con
%  nuestras acciones.
%
%  Si queremos crear una nueva partición, \texttt{\textbf{/dev/hda3}},
%  para montarla en el directorio \texttt{\textbf{/var}}:
%
%  {\flushleft\fontsize{8pt}{8pt}\selectfont
%\begin{verbatim}
%# cfdisk /dev/hda
%# mkfs -t ext3 /dev/hda3
%# mount -t ext3 /dev/hda3 /mnt
%# rsync -avHWSx /var /mnt
%# rm -rf /var
%# mkdir /var
%# umount /mnt
%# echo "/dev/hda3 /mnt ext3 defaults 0 2" >> /etc/fstab
%# mount /var
%\end{verbatim}
%  }
%\end{frame}
%
%---------------------------------------
%\begin{frame}[fragile]
%  \subsection{Integridad del sistema de ficheros}
%  \begin{itemize}
%  \item En el arranque, se hace una comprobación rutinaria de los
%    sistemas de ficheros, buscando fallos que puedan ser corregidos.
%  \item Esta comprobación se realiza con la orden
%    ``\texttt{\textbf{fsck}}'', y puede realizarse a mano (siempre con
%    el disco sin montar):
%
%    \texttt{\textbf{fsck -t ext3 /dev/hda3}}
%  \end{itemize}
%\end{frame}



\end{document}

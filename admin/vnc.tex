
\documentclass[ucs]{beamer}

\usetheme{GSyC}
%\usebackgroundtemplate{\includegraphics[width=\paperwidth]{gsyc-bg.png}}


\usepackage[spanish]{babel}   
\usepackage[utf8x]{inputenc}
\usepackage{graphicx}
\usepackage{amssymb} % Simbolos matematicos


% Metadatos del PDF, por defecto en blanco, pdftitle no parece funcionar
   \hypersetup{%
     pdftitle={VNC},%
     %pdfsubject={Diseño y Administración de Sistemas y Redes},%
     pdfauthor={GSyC},%
     pdfkeywords={},%
   }
%


% Para colocar un logo en la esquina inferior de todas las transpas
%   \pgfdeclareimage[height=0.5cm]{gsyc-logo}{gsyc}
%   \logo{\pgfuseimage{gsyc-logo}}


%% Para colocar antes de cada sección una página de recuerdo de índice
%\AtBeginSection[]{
%  \begin{frame}<beamer>{Contenidos}
%    \tableofcontents[currentframetitle]
%  \end{frame}
%}


\definecolor{darkred}{rgb}  {1.0, 0.0, 0.0}
\definecolor{darkgreen}{rgb}{0.0, 0.4, 0.0}
\definecolor{darkblue}{rgb} {0.0, 0.0, 0.8}

% for resalted text
\newcommand{\res}[1]{\textcolor{darkred}{#1}}
% for different text
\newcommand{\dif}{\textsl}
% for reserved words
\newcommand{\rw}[1]{\textrm{\textbf{#1}}}
% for commands
\newcommand{\com}[1]{\textrm{\textbf{#1}}}



\begin{document}

% Entre corchetes como argumento opcional un título o autor abreviado
% para los pies de transpa
\title[Sesiones gráficas remotas]{Sesiones gráficas remotas }
%\subtitle{Diseño y Administración de Sistemas y Redes}
\author[GSyC]{Escuela Técnica Superior de Ingeniería de Telecomunicación\\
Universidad Rey Juan Carlos}
\institute{gsyc-profes (arroba) gsyc.urjc.es}
\date[2017]{Noviembre de 2017}




%% TÍTULO
\begin{frame}
  \titlepage
  % Oportunidad para poner otro logo si se usó la opción nologo
  % \includegraphics[width=2cm]{logoesp}  
\end{frame}



%% LICENCIA DE REDISTRIBUCIÓN DE LAS TRANSPAS
%% Nota: la opción b al frame le dice que justifique el texto
%% abajo (por defecto c: centrado)
\begin{frame}[b]
\begin{flushright}
{\tiny
\copyright \insertshortdate~\insertshortauthor \\
  Algunos derechos reservados. \\
  Este trabajo se distribuye bajo la licencia \\
  Creative Commons Attribution Share-Alike 4.0\\
}
\end{flushright}  
\end{frame}



%% ÍNDICE
%\begin{frame}
%  \frametitle{Contenidos}
%  \tableofcontents
%\end{frame}





%%---------------------------------------------------------------
\section{Sesiones gráficas remotas}
%%---------------------------------------------------------------

\begin{frame}[fragile]
\frametitle{Sesiones gráficas remotas}

Definiciones 

\begin{itemize}
\item
Máquina local

Equipo en el que trabaja el usuario, donde tiene su pantalla, teclado, y posiblemente, ratón

\item
Máquina remota

Máquina donde se ejecuta la aplicación a usar. El usuario no tiene acceso a su
teclado, pantalla ni ratón
\end{itemize}


Sesiones de texto / Sesiones remotas

\begin{itemize}
\item
El protocolo ssh nos permite trabajar cómodamente en máquinas
Unix remotas, con sesiones de texto 


\item
También se puede ssh usar en Microsoft Windows, aunque con muchas limitaciones, resulta
poco natural

\item
Pero habrá ocasiones en que será conveniente o imprescindible usar
sesiones gráficas en máquinas remotas

\end{itemize}
\end{frame}



%%----------------------------------------------
\begin{frame}[fragile]
\frametitle{ Protocolos para sesiones gráficas remotas (1)}
\begin{itemize}
\item
Soluciones propietarias como LogMeIn o TeamViewer

\item
RDP. \emph{Remote Desktop Services}, también conocido como  \emph{Terminal Services}.  

Nativo en Microsoft Windows, usable desde otras plataformas, p.e. \verb|gnome-rdp|, \verb|vinagre| (clientes) 
o \verb|xrdp| (servidor) 

Puerto por omisión: 3389 TCP
\end{itemize}

\end{frame}


%%---------------------------------------------------------------
\begin{frame}[fragile]
\frametitle{ Protocolos para sesiones gráficas remotas (2)}
\begin{itemize}
\item
X11 forwarding. Forma parte de X Window.

Tradicional y nativo en Linux/UNIX, usable en Microsoft Windows. 

Normalmente
no trabaja sobre un escritorio
completo sino con ventanas individuales.

Anticuado auque sigue disponible. Intrínsecamente poco seguro.
Puerto por omisión: 6000 TCP

\item
VNC

Protocolo abierto, multiplataforma. Nativo en muchos Linux.  Disponible en Microsoft Windows, Unix, 
*BSD, MacOS, Android, iOS, ...

\end{itemize}
\end{frame}




%%---------------------------------------------------------------
\begin{frame}[fragile]
\frametitle{X11 Forwarding}
\begin{itemize}
\item
X Window (año 1984) es el protocolo tradicional en Unix para mostrar gráficos. En local
y también en remoto

Nada que ver con Microsoft Windows

\item
En 1987 aparece la versión 11 de X Window, sigue siendo la versión actual. De ahí el
nombre X11


\item
X Window es un protocolo cliente-servidor. Aunque la terminología puede ser
anti-intuitiva:


\begin{itemize}
\item
La máquina local es el servidor. En ella están los gráficos.
Normalmente lo llamaríamos \emph{cliente},
pero es el \emph{servidor X11} (ofrece el servicio de representación gráfica)

\item
En la máquina remota está
el \emph{cliente X Window}. Aunque en esta máquina 
están los procesos que usan los gráficos, esto es,
los servicios. (Los servicios son los clientes de los
gráficos) 

\end{itemize}


\end{itemize}
\end{frame}
%%----------------------------------------------
\begin{frame}[fragile]
\begin{itemize}


\item
En la máquina local puede ser necesario configurar los permisos del servidor


\verb|xhost +| 

permite que cualquier cliente X Window
desde cualquier máquina lance ventanas en nuestro servidor
X Window local. 


\begin{itemize}
\item
También se pueden dar permisos más específicos
\end{itemize}

\item
Entre dos máquinas Ubuntu con el mismo usuario en ambas
máquinas, mediante ssh, no es necesario dar permisos adicionales

\end{itemize}


Para lanzar una aplicación gráfica en una máquina remota:

\begin{itemize}
\item
Desde la máquina local hacemos ssh a la máquina remota con la opción
\verb|-X| (mayúscula)


\verb|jperez@gamma12:~$ ssh -X alpha| 


\item
Lanzamos la aplicación en segundo plano

p.e

\verb|jperez@alpha:~$ xeyes&|

\end{itemize}
\end{frame}




%%---------------------------------------------------------------
\begin{frame}[fragile]
\frametitle{VNC}
VNC, \emph{Virtual Network Compute}
es un protocolo para abrir sesiones gráficas en máquinas remotas


\begin{itemize}
\item
Arquitectura cliente-servidor, desarrollado por The Olivetti \& Oracle Research Lab
\item
La terminología es la habitual, no la de X Windows. El cliente está en la máquina local, el servidor,
en la máquina remota
\item
Implementación liberada como software libre en 2002
\item
Muy popular. Muchas implementaciones para cualquier plataforma
(Microsoft Windows, Linux, Unix, MacOS, Android, iOS, Raspberry Pi ...)

Cualquier servidor de cualquier plataforma puede trabajar con cualquier
cliente en cualquier otra

\end{itemize}

\end{frame}
%%----------------------------------------------
\begin{frame}[fragile]


En Ubuntu


\begin{itemize}
\item
Tenemos un servidor integrado por omisión, llamado \verb|vino|
\item
Hay un cliente llamado \verb|vinagre|

\item
También podemos usar 
\emph{TightVNC}, entre otras implementaciones

\end{itemize}


En Windows

\begin{itemize}
\item
Podemos usar 
\emph{TightVNC}, entre otras implementaciones
\end{itemize}


En MacOS

\begin{itemize}
\item
Como cliente podemos usar el navegador nativo de MacOS, Safari. Basta escribir en
la barra de direcciones la dirección del servidor:

\verb|vnc://maquina:puerto|

\item
Otra opción (con cliente y servidor) es RealVNC. Tiene una versión comercial
y otra libre (RealVNC Open Edition)

\end{itemize}


\end{frame}

%%---------------------------------------------------------------
\begin{frame}[fragile]
\frametitle{vinagre}
El cliente de VNC oficial de Ubuntu es
\verb|vinagre| 
\begin{itemize}
\item
Está desarrollado conjuntamente con
\verb|vino|, algunas opciones avanzadas pueden funcionar mejor con este servidor

\item
También tiene soporte para RDP 
\end{itemize}

Uso:

\verb|vinagre <MAQUINA>:<PUERTO>| 
\end{frame}





%%---------------------------------------------------------------
\begin{frame}[fragile]
\frametitle{vino}
\begin{itemize}
\item
En Ubuntu, 
el servidor
\verb|vino| 
está instalado por omisión en las máquinas con escritorio 
gráfico
\item
Por omisión
trabaja en el puerto 5900 TCP

\item

Para cambiar el puerto del servidor:

    \begin{enumerate}
    \item
Instalamos dconf-editor

\verb|sudo apt install dconf-editor|
    \item
Lanzamos dconf-editor

    \item

En 


\verb% org | gnome | desktop | remote-access%

Cambiamos \verb|alternative-port|

Activamos \verb|use-alternative-port|
    \end{enumerate}
\end{itemize}

\end{frame}






%%---------------------------------------------------------------
\begin{frame}[fragile]
\frametitle{}
Normalmente usaremos vino cuando ya tenemos una sesión gráfica 
abierta en un Ubuntu con escritorio tradicional


Pero no es adecuado para abrir:
\begin{itemize}
\item
Una segunda sesión gráfica en la misma máquina

\item
Una sesión gráfica en una máquina remota a la que no
tenemos acceso físico o en la que no queremos abrir una sesión gráfica
tradicional

\item
Una sesión en una máquina donde no queramos un escritorio tan pesado
como Unity o Gnome

\end{itemize}


En cualquiera de estos casos

\begin{itemize}
\item
En vez de usar 
\emph{vino},
 podremos usar 
\emph{TightVNC}

\end{itemize}

\end{frame}



%%---------------------------------------------------------------
\begin{frame}[fragile]
\frametitle{Servidor de TightVNC en Ubuntu}

El servidor de TightVNC es 
\verb|vncserver|.
Para usarlo necesitaremos, además del propio 
\verb|vncserver|, 
un escritorio. Podríamos emplear Gnome o Unity, pero son muy pesados. Generalmente será
más adecuado

\begin{itemize}
\item
O bien un escritorio ligero como Xfce4
\item
O bien un gestor de ventanas como Openbox

(podemos considerar a Openbox como un escritorio ultra-ligero)
\end{itemize}


Los pasos son:

    \begin{enumerate}
    \item
Instalación de vncserver

    \item
Instalación del escritorio

    \item
Preparación del escritorio

    \item
Lanzamiento del servidor

    \item
Lanzamiento del cliente

    \end{enumerate}



\end{frame}
%%----------------------------------------------
\begin{frame}[fragile]

\frametitle{1 Instalación de vncserver}


En caso de que TightVNC no esté instalado en nuestro sistema


    \begin{enumerate}
    \item
Como en cualquier instalación de un paquete nuevo, suele ser recomendable
actualizar el sistema

\verb|sudo apt update; sudo apt upgrade|

    \item
Instalamos el paquete

\verb|sudo apt install tightvncserver|
    \end{enumerate}

\end{frame}

%%---------------------------------------------------------------
\begin{frame}[fragile]
\frametitle{2 Instalación del escritorio}
Si vamos a usar Openbox

\begin{itemize}
\item
Para saber si está instalado, intentamos ejecutar

\verb|openbox-session|

\item
Para instalarlo

\verb|sudo apt install openbox|
\end{itemize}


Si vamos a usar Xfce4

\begin{itemize}
\item
Para saber si está instalado, intentamos ejecutar

\verb|xfce4-session|

\item
Para instalarlo

\verb|sudo apt install xfce4|
\end{itemize}

\end{frame}



%%---------------------------------------------------------------
\begin{frame}[fragile]
\frametitle{3 Preparación del escritorio}

    \begin{enumerate}
    \item
En el servidor escribiremos un fichero
\verb|~/.vnc/xstartup|

con el siguiente contenido
\begin{itemize}
\item
Si vamos a usar Xfce4

  \begin{scriptsize}
  \begin{verbatim}
#!/bin/bash
xrdb ~/.Xresources
/usr/bin/xfce4-session
  \end{verbatim}
  \end{scriptsize}

\item
Si vamos a usar Openbox


  \begin{scriptsize}
  \begin{verbatim}
#!/bin/bash
xrdb ~/.Xresources
/usr/bin/openbox-session
  \end{verbatim}
  \end{scriptsize}

\end{itemize}

    \item
Como a cualquier script, le damos permiso de ejecución, p.e.

\verb|chmod 755 ~/.vnc/xstartup|
    \end{enumerate}
\end{frame}





%%---------------------------------------------------------------
\begin{frame}[fragile]
\frametitle{4 Lanzamiento del servidor}

Para lanzar el servidor ejecutamos la orden \verb|vncserver|, indicando
el tamaño de la pantalla y la profundidad del color, especificada en número
de bits

Ejemplo:

\verb|vncserver  -geometry 1024x768 -depth 16|

Si vamos a usarlo con frecuencia, puede ser conveniente poner esta
orden en un script de shell, p.e.

\verb|~/bin/mi_vncserver|
\end{frame}


%%---------------------------------------------------------------
\begin{frame}[fragile]
\frametitle{}
\begin{itemize}
\item
La primera vez que lancemos 
\verb|vncserver|, nos preguntará
la contraseña de la sesión.

Quedará almacenada en el fichero

\verb|~/.vnc/passwd|
\item
Si necesitamos cambiar la contraseña, ejecutamos

\verb|vncpasswd|

\item
También podemos cambiar la contraseña desde un script.

P.e. la contraseña \emph{sesamo} sería:

\verb@echo "sesamo\nsesamo\n\n" | vncpasswd@

\end{itemize}

\end{frame}



%%---------------------------------------------------------------
\begin{frame}[fragile]
\frametitle{}


Como cualquier demonio,
\verb|vncserver|
deberá atarse a un puerto para aceptar peticiones, en su caso a un puerto TCP 

\begin{itemize}
\item
El puerto por omisión es el 5900 TCP

\item
Atención, el servidor de VNC no emplea el concepto \emph{puerto TCP}, 
sino 
\emph{display port}, donde

\emph{puerto TCP} = 5900 + \emph{display port} 

\item
Sin embargo, en el cliente normalmente sí indicaremos el puerto TCP, no el
\emph{display  port}

\end{itemize}


\end{frame}
%%----------------------------------------------
\begin{frame}[fragile]


Si no indicamos otra cosa, el demonio
\verb|vncserver|
 intentará usar el
\emph{display port} 0, puerto TCP  5900.
\begin{itemize}
\item

Si está libre, mostrará el mensaje
\begin{scriptsize}
\begin{verbatim}
New 'X' desktop is gamma:0
\end{verbatim}
\end{scriptsize}


\item
Si está ocupado (tal vez por vino), lo intentará en el
\emph{display port} 1,  puerto TCP
 5901. Si 
tiene éxito el mensaje será

\begin{scriptsize}
\begin{verbatim}
New 'X' desktop is gamma:1
\end{verbatim}
\end{scriptsize}

y así sucesivamente con los
\emph{display port} 2, 3, etc
(puertos TCP 5902, TCP 5903, etc)

\end{itemize}

\end{frame}


%%---------------------------------------------------------------
\begin{frame}[fragile]
\frametitle{}

\begin{itemize}
\item
También podemos indicar 
explícitamente el  \emph{display port}
que usará el servidor:


Ejemplo: 
\emph{display port} 10 (puerto TCP 5910)

\verb|vncserver :10 -geometry 1024x768 -depth 16|

\item
La sesión permanecerá abierta hasta que la matemos explícitamente
con


\verb|vncserver -kill :10|


\begin{itemize}
\item
Esta orden mata el proceso
\verb|vncserver|, la sesión X11 en 
\verb|/tmp/.X11-unix|
y los logs en 
\verb|~/.vnc/|

\end{itemize}
\end{itemize}





\end{frame}

%%---------------------------------------------------------------
\begin{frame}[fragile]
\frametitle{5 Lanzamiento del cliente}
En la máquina local ejecutamos

\verb|vinagre <SERVIDOR>:<PUERTO|

p.e.

\verb|vinagre gamma:5901|

Observa que en este caso indicamos el puerto TCP, no el \emph{display port}



\begin{itemize}
\item
Recuerda que para cerrar la sesión hay que matar el servidor, no basta
con cerrar el cliente
\end{itemize}



\end{frame}



%%---------------------------------------------------------------
\begin{frame}[fragile]
\frametitle{Uso de VNC en Docker}
VNC es una forma conveniente de abrir sesiones gráficas dentro de un contenedor.

Configuración de ejemplo para el 
\emph{display port} 0, contraseña 
\emph{sesamo}

\verb|entrypoint.sh:|

  \begin{scriptsize}
  \begin{verbatim}
#!/bin/bash
vncserver :0 -geometry 1024x768 -depth 16
/usr/bin/xterm&
/bin/bash
  \end{verbatim}
  \end{scriptsize}


\end{frame}
%%----------------------------------------------
\begin{frame}[fragile]

\verb|Dockerfile:|

  \begin{scriptsize}
  \begin{verbatim}
FROM ubuntu:16.04
ENV USER root

RUN apt-get update && DEBIAN_FRONTEND=noninteractive &&  \
    apt-get install -y \
    tightvncserver openbox \
    xterm

# Expose VNC port
EXPOSE 5900

#set password for vnc
RUN echo "sesamo\nsesamo\n\n" | vncpasswd

COPY entrypoint.sh /
ENTRYPOINT ["/entrypoint.sh"]
  \end{verbatim}
  \end{scriptsize}

\begin{tiny}
\begin{flushright}
\url{http://ortuno.es/Dockerfile_vnc.txt}
\end{flushright}
\end{tiny}
\end{frame}


%%---------------------------------------------------------------
\begin{frame}[fragile]
\frametitle{}
Lanzamiento del contendor:

  \begin{scriptsize}
  \begin{verbatim}
#!/bin/bash
DISPLAY_NUMBER=0
PORT=$((DISPLAY_NUMBER+5900))
IMAGEN=vnc
NOMBRE=jper${IMAGEN}01
USUARIO=jperez

docker run -it --rm -h ${NOMBRE}  --name ${NOMBRE} -p ${PORT}:${PORT} \
    -e DISPLAY=:${DISPLAY_NUMBER} ${USUARIO}/${IMAGEN}
  \end{verbatim}
  \end{scriptsize}

\begin{tiny}
\begin{flushright}
\url{http://ortuno.es/lanza_vnc.txt}
\end{flushright}
\end{tiny}
\end{frame}


%%---------------------------------------------------------------
\end{document}

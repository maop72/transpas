\documentclass[ucs]{beamer}
\usetheme{GSyC}
\usepackage[spanish]{babel}   
\usepackage[utf8x]{inputenc}
\usepackage{graphicx}
\usepackage{amssymb} % Simbolos matematicos
\usepackage{lmodern,textcomp}  % para usar el carácter € tal cual
\usepackage{enumerate}


% Metadatos del PDF, por defecto en blanco, pdftitle no parece funcionar
   \hypersetup{%
     pdftitle={Ajax},
     %pdfsubject={Diseño y Administración de Sistemas y Redes},%
     pdfauthor={GSyC},%
     pdfkeywords={},%
   }
%


% Para colocar un logo en la esquina inferior de todas las transpas
%   \pgfdeclareimage[height=0.5cm]{gsyc-logo}{gsyc}
%   \logo{\pgfuseimage{gsyc-logo}}


% Para colocar antes de cada sección una página de recuerdo de índice
%\AtBeginSection[]{
%  \begin{frame}<beamer>{Contenidos}
%    \tableofcontents[currentframetitle]
%  \end{frame}
%}



\begin{document}

% Entre corchetes como argumento opcional un título o autor abreviado
% para los pies de transpa
\title[Ajax]{Ajax}
%\subtitle{Diseño y Administración de Sistemas y Redes}
\author[GSyC]{Escuela Técnica Superior de Ingeniería de Telecomunicación\\
Universidad Rey Juan Carlos}
\institute{gsyc-profes (arroba) gsyc.urjc.es}
\date[2017]{Diciembre de 2017}


%% TÍTULO
\begin{frame}
  \titlepage
  % Oportunidad para poner otro logo si se usó la opción nologo
  % \includegraphics[width=2cm]{logoesp}  
\end{frame}




%% LICENCIA DE REDISTRIBUCIÓN DE LAS TRANSPAS
%% Nota: la opción b al frame le dice que justifique el texto
%% abajo (por defecto c: centrado)
\begin{frame}[b]
\begin{flushright}
{\tiny
\copyright \insertshortdate~\insertshortauthor \\
  Algunos derechos reservados. \\
  Este trabajo se distribuye bajo la licencia \\
  Creative Commons Attribution Share-Alike 4.0
}
\end{flushright}
\end{frame}


%% ÍNDICE
%\begin{frame}
%  \frametitle{Contenidos}
%  \tableofcontents
%\end{frame}

% $Id$
%



\section{Ajax}
%%---------------------------------------------------------------
\begin{frame}[fragile]
\frametitle{Ajax}

Es un conjunto de técnicas para enviar información asíncrona entre
servidor web y cliente

\begin{itemize}
\item
Aunque hay antecedentes desde 1996, el Ajax actual lo crea google en 2004, para gmail
y google maps
\item
Evita que el usuario necesite pulsar \emph{enviar}. La aplicación
web recibe información continuamente, resultando una experiencia de usuario similar
a la de una aplicación de escritorio
\item

Originalmente significaba Asynchronous JavaScript and XML, pero cuando
empieza a usarse sin JavaScript y sin XML, el acrónimo AJAX se abandona
para usar la palabra Ajax

\end{itemize}

\end{frame}


%%---------------------------------------------------------------
\begin{frame}[fragile]
\frametitle{Funcionamiento de Ajax}
\begin{enumerate}
%   richardson pg 316
\item
El usuario solicita una URL desde su navegador

\item
El servidor web devuelve la página, que contiene un script

\item
El navegador presenta la página y ejecuta el script

\item
El script hace peticiones HTTP asíncronas a una URI del servidor, sin que
el usuario intervenga. El servidor suele ser RESTful

\item
El script interpreta las respuestas HTTP y modifica el HTML, sin que
el usuario intervenga
\end{enumerate}
\end{frame}


%%---------------------------------------------------------------
\begin{frame}[fragile]
\frametitle{Ajax con jQuery}
jQuery facilita mucho el uso de Ajax. El método principal es \verb|ajax()|

Recibe:

\begin{itemize}
\item
Un \emph{plain object} que puede tener varios atributos. El principal es 
\verb|url|, la dirección del servicio
\end{itemize}

Devuelve:

    \begin{itemize}
\item
Un objeto de tipo \verb|jqXHR|, que tiene varios atributos. Los principales
son:


    \begin{itemize}
    \item
Un método
llamado \verb|done| al que le 
pasamos la función que se invocará si la petición tiene éxito. Tendrá un
parámetro, que en nuestro caso será un objeto JSON

    \item
Un método
llamado \verb|fail| al que le 
pasamos la función a invocar en caso de error
    \end{itemize}
    \end{itemize}

%El método \verb|load()| recibe como argumento una URI, la invoca,
%y reemplaza el contenido indicado por el selector por la respuesta
%\end{itemize}



\end{frame}

%%----------------------------------------------
\begin{frame}[fragile]
El siguiente ejemplo llama a 
\url{http://fixer.io}, un servicio que ofrece tipos de cambio
de divisas

    \begin{itemize}
    \item
Este sitio ofrece una respuesta en JSONP, lo contrario incumpliría la 
\emph{same-origin policy}

    \item
jQuery se ocupa del manejo de JSONP, resulta transparente para el programador

    \item
Para los servicios sin soporte de JSONP, necesitaríamos un proxy en el
mismo sitio web que sirve el HTML

    \end{itemize}
\end{frame}


%%---------------------------------------------------------------
\begin{frame}[fragile]
\frametitle{}

  \begin{scriptsize}
  \begin{verbatim}
    $(document).ready(function() {
      let urlServicio = 'https://api.fixer.io/latest';

      peticion = $.ajax({
        url: urlServicio
      })

      peticion.done(manejaRespuesta);
      peticion.fail(manejaError);

      function manejaRespuesta(json) {
        $("#div01").text(JSON.stringify(json));
      };

      function manejaError(jqXHR) {
        $("#div01").text("Error: " + jqXHR.status);
      };
    });
  \end{verbatim}
  \end{scriptsize}
\begin{tiny}
\begin{flushright}
\url{http://ortuno.es/ajax.html}
\end{flushright}
\end{tiny}
\end{frame}


%%---------------------------------------------------------------
\begin{frame}[fragile]
\frametitle{Cambio de divisas}
\begin{itemize}
\item
Cada divisa tiene un código ISO 4217, que es un identificador de 3 letras mayúsculas.
Por ejemplo USD (United States Dollar) para el dólar, EUR para el euro, GBP para la libra esterlina,
etc

\item
El precio de una divisa frente a otra se indica mediante la concatenación de sus dos códigos ISO 4217.
La primera se denomina \emph{divisa base}
y la segunda,
\emph{divisa cotizada}

\item
Ejemplo EURUSD = 1.13

Aquí la divisa base es el euro, y la cotizada, el dólar. 

Este valor (1.13) indica cuántas
unidades de la divisa cotizada hacen falta para
comprar una unidad de la divisa base 

\item
Dicho de otro modo, la primera moneda es la que queremos
y la segunda, la que tenemos, la que usamos para pagar.

\end{itemize}
\end{frame}




%%---------------------------------------------------------------
\begin{frame}[fragile]
\frametitle{setInterval}

Para que la consulta Ajax se repita periódicamente, podemos
usar la función 
\verb|setInterval()|

Recibe
\begin{itemize}
\item
Como primer argumento, una función

\item
Como segundo argumento, un intervalo de tiempo en milisegundos

\end{itemize}

Como resultado, evalúa la función periódicamente, con el intervalo
especificado.
El siguente ejemplo se ejecutaría cada 60 segundos

  \begin{scriptsize}
  \begin{verbatim}
setInterval(function() { 
               miTexto=actualizaTexto();
               $("#p01").text(miTexto);
            }, 
            60000
);
  \end{verbatim}
  \end{scriptsize}

\end{frame}

\end{document}

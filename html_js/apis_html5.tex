\documentclass[ucs]{beamer}
\usetheme{GSyC}
\usepackage[spanish]{babel}   
\usepackage[utf8x]{inputenc}
\usepackage{graphicx}
\usepackage{amssymb} % Simbolos matematicos
\usepackage{lmodern,textcomp}  % para usar el carácter € tal cual
\usepackage{enumerate}


% Metadatos del PDF, por defecto en blanco, pdftitle no parece funcionar
   \hypersetup{%
     pdftitle={APIs de HTML 5},
     %pdfsubject={Diseño y Administración de Sistemas y Redes},%
     pdfauthor={GSyC},%
     pdfkeywords={},%
   }
%


% Para colocar un logo en la esquina inferior de todas las transpas
%   \pgfdeclareimage[height=0.5cm]{gsyc-logo}{gsyc}
%   \logo{\pgfuseimage{gsyc-logo}}


% Para colocar antes de cada sección una página de recuerdo de índice
%\AtBeginSection[]{
%  \begin{frame}<beamer>{Contenidos}
%    \tableofcontents[currentframetitle]
%  \end{frame}
%}



\begin{document}

% Entre corchetes como argumento opcional un título o autor abreviado
% para los pies de transpa
\title[APIs de HTML5]{APIs de HTML5}
%\subtitle{Diseño y Administración de Sistemas y Redes}
\author[GSyC]{Escuela Técnica Superior de Ingeniería de Telecomunicación\\
Universidad Rey Juan Carlos}
\institute{gsyc-profes (arroba) gsyc.urjc.es}
\date[2017]{Noviembre de 2017}


%% TÍTULO
\begin{frame}
  \titlepage
  % Oportunidad para poner otro logo si se usó la opción nologo
  % \includegraphics[width=2cm]{logoesp}  
\end{frame}




%% LICENCIA DE REDISTRIBUCIÓN DE LAS TRANSPAS
%% Nota: la opción b al frame le dice que justifique el texto
%% abajo (por defecto c: centrado)
\begin{frame}[b]
\begin{flushright}
{\tiny
\copyright \insertshortdate~\insertshortauthor \\
  Algunos derechos reservados. \\
  Este trabajo se distribuye bajo la licencia \\
  Creative Commons Attribution Share-Alike 4.0
}
\end{flushright}
\end{frame}


%% ÍNDICE
%\begin{frame}
%  \frametitle{Contenidos}
%  \tableofcontents
%\end{frame}

% $Id$
%

\section{APIs de HTML5}

%%---------------------------------------------------------------
\begin{frame}[fragile]
\frametitle{}
HTML 5 define algunas nuevas APIS. Podemos englobar en este conjunto
otras tecnologías que aunque en rigor no son parte de HTML5,
están muy relacionadas y definidas por el W3C 
\begin{itemize}
\item
Web Storage
\item
Web Workers
\item
Geolocation
\item
Canvas

API para dibujar imágenes vectoriales 2D y 
\emph{bitmap}. 

Hay librerías como jCanvas que facilitan su uso

\item
Web Messaging 

Para comunicación entre documentos de distintos orígenes, de manera
razonablemente segura y sin las limitaciones de la
\emph{same-origin policy}

\item
y algunas otras

\url{https://en.wikipedia.org/wiki/HTML5#New_APIs}
\end{itemize}

\end{frame}




\section{Web Storage}
%%---------------------------------------------------------------
\begin{frame}[fragile]
\frametitle{Web Storage}


\emph{Web Storage}
 es una API del W3C que permite almacenar información en el navegador 
en una forma que 
generalmente resulta más conveniente que las cookies


\begin{itemize}
\item
Espacio reservado para este almacenamiento depende del navegador, normalmente
entre 2.5Mb y 5 Mb por origen 

    \begin{itemize}
    \item
Mucho más que las cookies, limitadas a 4 K
    \end{itemize}

\item

La información de 
\emph{Web Storage} 
la escribe el navegador y permanece en el navegador, solo se envía al servidor
si se programa explícitamente

    \begin{itemize}
    \item
Las cookies las escribe el servidor y luego viajan en cada petición,
consumen ancho de banda. 
    \end{itemize}
\end{itemize}
\end{frame}

%%----------------------------------------------
\begin{frame}[fragile]

Por todo ello

    \begin{itemize}
    \item
Si la información la necesita el cliente, 
\emph{Web Storage} 
es mucho más conveniente

    \item
Si la información la necesita el servidor, 
\emph{Web Storage} 
no es tan útil, porque
el servidor no tiene acceso directamente a ella
%, hay que enviarla
%(p.e. en un campo oculto de un formulario o con Ajax)
    \end{itemize}


\begin{tiny}
\begin{flushright}
\url{https://stackoverflow.com/questions/3220660/local-storage-vs-cookies}
\end{flushright}
\end{tiny}
\end{frame}

%%---------------------------------------------------------------
\begin{frame}[fragile]
\frametitle{}

\emph{Web Storage} 
es una estructura de diccionario: clave-valor


Incluye dos mecanismos:
\begin{itemize}
\item

\emph{sessionStorage}

Un diccionario que dura tanto como la sesión. Se borra al cerrar el navegador


\item
\emph{localStorage} 

Un diccionario persistente, el diccionario se mantiene aunque se cierre el navegador. Solo
se borra si el usuario o la página lo borran explícitamente

\end{itemize}

Cada uno de estos diccionarios es distinto para cada origen (protocolo, host, puerto)
\end{frame}




%%---------------------------------------------------------------
\begin{frame}[fragile]
\frametitle{Almacenamiento de un valor}


  \begin{verbatim}
localStorage.setItem(clave) = valor;
sesionStorage.setItem(clave) = valor;
  \end{verbatim}


\begin{itemize}
\item
El valor y la clave han de ser cadenas 

\item
Si se intenta usar otro tipo de datos para el valor,
muchos navegadores lo convierten a cadena 

\item
Es preferible convertir el dato en JSON explícitamente
\end{itemize}
\end{frame}


%%---------------------------------------------------------------
\begin{frame}[fragile]
\frametitle{Recuperación de un valor}


  \begin{verbatim}
localStorage.getItem(clave);
sesionStorage.getItem(clave);
  \end{verbatim}

\begin{itemize}
\item
Estos métodos devuelven el valor asociado a la clave

\item
O \emph{undefined} si la clave no existe
\end{itemize}
\end{frame}


%%---------------------------------------------------------------
\begin{frame}[fragile]
\frametitle{Borrado de valores}

\verb|localStorage.removeItem(clave)|

\verb|sesionStorage.removeItem(clave)|

    \begin{itemize}
    \item
Borran la clave
    \item
Devuelven 
\emph{undefined}, tanto si la clave existía como si no
    \end{itemize}


\verb|localStorage.clear()|

\verb|sesionStorage.clear()|

\begin{itemize}
\item
Borran el diccionario completo del origen actual (protocolo, host, puerto)
\end{itemize}
\end{frame}



%%---------------------------------------------------------------
\begin{frame}[fragile]
\frametitle{}
Este programa pregunta al usuario su nombre solamente
la primera vez que se ejecuta

  \begin{scriptsize}
  \begin{verbatim}
  <script>
    'use strict'
    let nombreUsuario = localStorage.getItem('nombreUsuario');
    if (!nombreUsuario) {
      let input= prompt("¿Cómo te llamas?");
      localStorage.setItem('nombreUsuario',input);
    }
    for (let clave in localStorage) {
      let valor = localStorage[clave];
      console.log(clave+": "+valor)
    }
  </script>

  \end{verbatim}
  \end{scriptsize}

\begin{flushright}
\url{http://ortuno.es/localStorage.html}
\end{flushright}
\end{frame}


%%---------------------------------------------------------------
\begin{frame}[fragile]
\frametitle{}
Este programa borra el nombre de usuario, si estaba definido

  \begin{scriptsize}
  \begin{verbatim}
    let clave = 'nombreUsuario';
    let valor=localStorage.getItem(clave);
    if (valor){
        localStorage.removeItem(clave);
        alert(clave+ ' valía '+valor+ '. Ahora lo he borrado');
    }else{
        alert(clave+" no definido");
    };
  \end{verbatim}
  \end{scriptsize}
\begin{flushright}
\url{http://ortuno.es/localStorage2.html}
\end{flushright}

\end{frame}


%%---------------------------------------------------------------
\begin{frame}[fragile]
\frametitle{Referencias}
\begin{itemize}
\item
\url{https://en.wikipedia.org/wiki/Web_storage}


\item
\begin{tiny}
\url{https://developer.mozilla.org/en-US/docs/Web/API/Window/localStorage}
\end{tiny}

\item
\begin{tiny}
\url{https://developer.mozilla.org/en-US/docs/Web/API/Web_Storage_API/Using_the_Web_Storage_API}
\end{tiny}
\end{itemize}

\end{frame}



\section{Web Workers}
%%---------------------------------------------------------------
\begin{frame}[fragile]
\frametitle{Web Workers}
Un
\emph{web worker} es un programa en JavaScript que se ejecuta en el navegador web en segundo plano, independiente
de la página principal


    \begin{itemize}
    \item
Es útil para realizar tareas intensivas en CPU

    \item
Es útil para aprovechar los ordenadores multi núcleo (prácticamente todos los actuales)


    \item
Se crea como instancia del objeto/la función \verb|Worker()|, que recibe
como argumento el nombre del script

    \item
El web worker se comunica con el documento mediante el paso de mensajes:
basta asignar el manejador a la propiedad \emph{onmessage} del worker

    \item
Se puede finalizar desde la página principal invocando al método \verb|.terminate()|
del worker

    \item
Por motivos de seguridad, algunos navegadores como Chrome no permiten que se ejecuten
en páginas cargadas localmente, solo en aquellas servidas desde un web
    \end{itemize}

\end{frame}



%%---------------------------------------------------------------
\begin{frame}[fragile]
\frametitle{}
El siguiente ejemplo calcula dos números aleatorios, y cuando 
coinciden, envía un mensaje


  \begin{scriptsize}
  \begin{verbatim}
'use strict'
function random(x){
    return Math.floor((Math.random() * x) + 1);
}

let tamanio=1000000;
let x,y;
let c=100;
while (c>0){
    x=random(tamanio);
    y=random(tamanio);
    if (x===y) {
          postMessage(x);
          c-=1;
    }
}
  \end{verbatim}
  \end{scriptsize}
\begin{tiny}
\begin{flushright}
\url{http://ortuno.es/worker.js.txt}
\end{flushright}
\end{tiny}

\end{frame}


%%---------------------------------------------------------------
\begin{frame}[fragile]
\frametitle{}

  \begin{scriptsize}
  \begin{verbatim}
<body>
  <span id=resultado>--</span>
  <script>
  'user strict'
    $(document).ready(function() {
          let worker = new Worker('js/worker.js');
          worker.onmessage = function(event) {
            $("#resultado").text(event.data);
          };
    });
  </script>
</body>
  \end{verbatim}
  \end{scriptsize}

\begin{tiny}
\begin{flushright}
\url{http://ortuno.es/worker.html}
\end{flushright}
\end{tiny}
\end{frame}




\section{Geolocation}
%%---------------------------------------------------------------
\begin{frame}[fragile]
\frametitle{Geolocation}
Los navegadores modernos pueden conocer, si el usuario lo permite,
su ubicación geográfica

Posiblemente el más preciso es Google Chrome
\begin{itemize}
\item
En dispositivos con GPS, la información proviende del GPS

\item
En conexiones cableadas, obtiene información a partir de la dirección IP

\item
En conexiones WiFi

    \begin{itemize}
    \item
Obtiene las MAC de los Access Point WiFi colindantes
    \item
La compara con la base de datos de Google de la posición de los Access Point
    \end{itemize}

\item
En conexiones de telefonía móvil
    \begin{itemize}
    \item
Compara el identificador de la celda con su base de datos de posiciones
de celdas de telefonía
    \end{itemize}

\end{itemize}

\end{frame}




%%---------------------------------------------------------------
\begin{frame}[fragile]
\frametitle{}
¿Cómo obtiene Google la base de datos de Access Point y celdas de
telefonía móvil?
\begin{itemize}
\item
Con los coches que capturan datos para Google Maps. (Aunque esto le causó
problemas legales)

\item
Con la información de los millones de dispositivos Android con GPS
\end{itemize}

\end{frame}



%%---------------------------------------------------------------
\begin{frame}[fragile]
\frametitle{}
Para aceder a las coordenadas basta usar el método
\verb|navigator.geolocation.getCurrentPosition()|
pasándole como parámetro

\begin{itemize}
\item
La función que procesará las coordenadas

\item
La función que se invocará en caso de error

\item
Un objeto con opciones
\end{itemize}

\end{frame}

%%---------------------------------------------------------------
\begin{frame}[fragile]
\frametitle{}

  \begin{scriptsize}
  \begin{verbatim}
      let options = {
        enableHighAccuracy: true,
        timeout: 5000,
        maximumAge: 0
      };
      function success(pos) {
        let x = pos.coords;
        let mensaje = 'Posición actual\n';
        mensaje += 'Latitud :' + x.latitude;
        mensaje += '\nLongitud :' + x.longitude;
        mensaje += '\nPrecisión :' + x.accuracy + " metros";
        alert(mensaje);
      }
      function error(err) {
        console.warn(`ERROR(${err.code}): ${err.message}`);
      };
      navigator.geolocation.getCurrentPosition(success, error, options);
  \end{verbatim}
  \end{scriptsize}
\begin{tiny}
\begin{flushright}
\url{http://ortuno.es/geoloc.html}
\end{flushright}
\end{tiny}
\end{frame}

\end{document}


\documentclass[ucs]{beamer}

\usetheme{GSyC}
%\usebackgroundtemplate{\includegraphics[width=\paperwidth]{gsyc-bg.png}}


\usepackage[spanish]{babel}   
\usepackage[utf8x]{inputenc}
\usepackage{graphicx}
\usepackage{amssymb} % Simbolos matematicos
\usepackage{lmodern,textcomp}  % para usar el carácter € tal cual



% Metadatos del PDF, por defecto en blanco, pdftitle no parece funcionar
   \hypersetup{%
     pdftitle={HTML},%
     %pdfsubject={Diseño y Administración de Sistemas y Redes},%
     pdfauthor={GSyC},%
     pdfkeywords={},%
   }
%


% Para colocar un logo en la esquina inferior de todas las transpas
%   \pgfdeclareimage[height=0.5cm]{gsyc-logo}{gsyc}
%   \logo{\pgfuseimage{gsyc-logo}}


% Para colocar antes de cada sección una página de recuerdo de índice
%\AtBeginSection[]{
%  \begin{frame}<beamer>{Contenidos}
%    \tableofcontents[currentframetitle]
%  \end{frame}
%}



\begin{document}

% Entre corchetes como argumento opcional un título o autor abreviado
% para los pies de transpa
\title[HTML]{HTML }
%\subtitle{Diseño y Administración de Sistemas y Redes}
\author[GSyC]{Escuela Técnica Superior de Ingeniería de Telecomunicación\\
Universidad Rey Juan Carlos}
\institute{gsyc-profes (arroba) gsyc.urjc.es}
\date[2017]{Septiembre de 2017}


%% TÍTULO
\begin{frame}
  \titlepage
  % Oportunidad para poner otro logo si se usó la opción nologo
  % \includegraphics[width=2cm]{logoesp}  
\end{frame}



%% LICENCIA DE REDISTRIBUCIÓN DE LAS TRANSPAS
%% Nota: la opción b al frame le dice que justifique el texto
%% abajo (por defecto c: centrado)
\begin{frame}[b]



\vspace{1cm}
\begin{footnotesize}

\begin{flushright}
Las transparencias del tema ``CSS - Hojas de estilo'' \\
están basadas en el libro de librosweb.es \\
disponible en http://www.librosweb.es/css/ \\
Se ha pedido explícitamente autorización al autor original \\
para realizar esta obra derivada con fines educativos.

\copyright Javier Eguiluz - Librosweb.es \\
\vspace{1cm}
\end{flushright}



\begin{flushright}
{
\copyright 2002-2017 Jesús M. González Barahona, Gregorio Robles. \\

Este documento (o uno muy similar) está disponible en \\
\url{http://cursosweb.github.io}
  Algunos derechos reservados. \\
  Este trabajo se distribuye bajo la licencia \\
  Creative Commons Attribution Share-Alike 4.0\\
}
\end{flushright}  


\end{footnotesize}
\end{frame}



%% ÍNDICE
%\begin{frame}
%  \frametitle{Contenidos}
%  \tableofcontents
%\end{frame}




%%---------------------------------------------------------------



\subsection*{CSS: Consideraciones adicionales}

\begin{frame}[fragile]
\frametitle{CSS: Consideraciones adicionales}

\begin{center}
{\Huge CSS \\ Consideraciones Adicionales}

{\footnotesize (transparencias de referencia)}

\end{center}


\end{frame}


%%---------------------------------------------------------------


\begin{frame}
\frametitle{Orden de visualización}

\begin{itemize}
  \item El relleno y el margen son transparentes, por lo que en el espacio ocupado por el relleno se muestra el color o imagen de fondo (si están definidos)
  \item En el espacio ocupado por el margen se muestra el color o imagen de fondo de su elemento padre (si están definidos)
  \item Si ningún elemento padre tiene definido un color o imagen de fondo, se muestra el color o imagen de fondo de la propia página (si están definidos)
  \item Si una caja define tanto un color como una imagen de fondo, la imagen tiene más prioridad y es la que se visualiza
  \item Si la imagen de fondo no cubre totalmente la caja del elemento o si la imagen tiene zonas transparentes, también se visualiza el color de fondo. 
\end{itemize}

\end{frame}



%%---------------------------------------------------------------

\begin{frame}
\frametitle{Atributo width}

\begin{center}
  \begin{table}
   \begin{tabular}{p{1.8cm}p{7.8cm}}
Atributo &\bf{width} \\ \hline
Valores & $<medida>$ | $<porcentaje>$ | auto | inherit \\ \hline
Se aplica a & Todos los elementos, salvo los elementos en línea que no sean imágenes, las filas de tabla y los grupos de filas de tabla \\ \hline
Valor inicial & auto \\ \hline
Descripción & Establece la anchura de un elemento \\ \hline
  \end{tabular}
   \caption{Definición del atributo \emph{width} de CSS}
 \end{table}
\end{center}

\end{frame}

%%---------------------------------------------------------------

\begin{frame}
\frametitle{Atributo height}

\begin{center}
  \begin{table}
   \begin{tabular}{p{1.8cm}p{7.8cm}}
Atributo &\bf{height} \\ \hline
Valores & $<medida>$ | $<porcentaje>$ | auto | inherit \\ \hline
Se aplica a & Todos los elementos, salvo los elementos en línea que no sean imágenes, las columnas de tabla y los grupos de columnas de tabla \\ \hline
Valor inicial & auto \\ \hline
Descripción & Establece la altura de un elemento \\ \hline
  \end{tabular}
   \caption{Definición del atributo \emph{height} de CSS}
 \end{table}
\end{center}

\end{frame}

%%---------------------------------------------------------------


\begin{frame}
\frametitle{Márgenes}


\begin{center}
\begin{figure}[p]
\includegraphics[width=0.45\textwidth]{figs/f0428.png}
\end{figure}
\end{center}

\begin{itemize}
  \item En vez de utilizar la etiqueta $<blockquote>$ de HTML, debería utilizarse el atributo margin-left de CS
  \item Los márgenes verticales (margin-top y margin-bottom) sólo se pueden aplicar a los elementos de bloque y las imágenes, mientras que los márgenes laterales (margin-left y margin-right) se pueden aplicar a cualquier elemento
\end{itemize}

\end{frame}


%%---------------------------------------------------------------

\begin{frame}
\frametitle{El margen vertical}

Es algo peculiar:

\begin{itemize}
  \item Cuando se juntan dos o más márgenes verticales, se fusionan de forma automática y la altura del nuevo margen será igual a la altura del margen más alto de los que se han fusionado.
  \item Si un elemento está contenido dentro de otro elemento, sus márgenes verticales se fusionan y resultan en un nuevo margen de la misma altura que el mayor margen de los que se han fusionado
  \item Si no se diera este comportamiento y se estableciera un determinado margen a todos los párrafos, el primer párrafo no mostraría un aspecto homogéneo respecto de los demás.
\end{itemize}

\end{frame}

%%---------------------------------------------------------------

\begin{frame}
\frametitle{Relleno}

\begin{center}
  \begin{table}
   \begin{tabular}{p{1.8cm}p{7.8cm}}
Atributos &\bf{padding-top}, \bf{padding-right}, \bf{padding-bottom}, \bf{padding-left} \\ \hline
Valores & $<medida>$ | $<porcentaje>$ | inherit \\ \hline
Se aplica a & Todos los elementos excepto algunos elementos de tablas como grupos de cabeceras y grupos de pies de tabla \\ \hline
Valor inicial & 0 \\ \hline
Descripción & Establece cada uno de los rellenos horizontales y verticales de un elemento \\ \hline
  \end{tabular}
   \caption{Definición del atributo padding-top, padding-right, padding-bottom, padding-left de CSS}
 \end{table}
\end{center}

\end{frame}

%%---------------------------------------------------------------

\begin{frame}
\frametitle{}

\begin{center}
  \begin{table}
   \begin{tabular}{p{1.8cm}p{7.8cm}}
Atributo &\bf{padding} \\ \hline
Valores & ( $<medida>$ | $<porcentaje>$ ) {1, 4} | inherit \\ \hline
Se aplica a & Todos los elementos excepto algunos elementos de tablas como grupos de cabeceras y grupos de pies de tabla \\ \hline
Valor inicial & - \\ \hline
Descripción & Establece de forma directa todos los rellenos de los elementos \\ \hline
  \end{tabular}
   \caption{Definición del atributo padding de CSS}
 \end{table}
\end{center}

\end{frame}

%%---------------------------------------------------------------

\begin{frame}
\frametitle{Anchura de los bordes}

\begin{center}
  \begin{table}
   \begin{tabular}{p{1.8cm}p{7.8cm}}
Atributos &\bf{border-top-width}, \bf{border-right-width}, \bf{border-bottom-width}, \bf{border-left-width} \\ \hline
Valores & ( $<medida>$ | thin | medium | thick ) | inherit \\ \hline
Se aplica a & Todos los elementos \\ \hline
Valor inicial & Medium \\ \hline
Descripción & Establece la anchura de cada uno de los cuatro bordes de los elementos \\ \hline
  \end{tabular}
   \caption{Definición del atributo border-top-width, border-right-width, border-bottom-width, border-left-width de CSS}
 \end{table}
\end{center}

\end{frame}

%%---------------------------------------------------------------

\begin{frame}
\frametitle{Anchura de los bordes (shorthand)}

\begin{center}
  \begin{table}
   \begin{tabular}{p{1.8cm}p{7.8cm}}
Atributo &\bf{border-width} \\ \hline
Valores & ( $<medida>$ | thin | medium | thick ) {1, 4} | inherit \\ \hline
Se aplica a & Todos los elementos \\ \hline
Valor inicial & Medium \\ \hline
Descripción & Establece la anchura de todos los bordes del elemento \\ \hline
  \end{tabular}
   \caption{Definición del atributo border-width de CSS}
 \end{table}
\end{center}

\end{frame}

%%---------------------------------------------------------------

\begin{frame}
\frametitle{Color de los bordes}

\begin{center}
  \begin{table}
   \begin{tabular}{p{1.8cm}p{7.8cm}}
Atributos & \bf{border-top-color}, \bf{border-right-color}, \bf{border-bottom-color}, \bf{border-left-color} \\ \hline
Valores & $<color>$ | transparent | inherit \\ \hline
Se aplica a & Todos los elementos \\ \hline
Valor inicial & - \\ \hline
Descripción & Establece el color de cada uno de los cuatro bordes de los elementos \\ \hline
  \end{tabular}
   \caption{Definición del atributo border-top-color, border-right-color, border-bottom-color, border-left-color de CSS}
 \end{table}
\end{center}

\end{frame}

%%---------------------------------------------------------------

\begin{frame}
\frametitle{Color de los bordes (shorthand)}

\begin{center}
  \begin{table}
   \begin{tabular}{p{1.8cm}p{7.8cm}}
Atributo &\bf{border-color} \\ \hline
Valores & ( $<color>$ | transparent ) {1, 4} | inherit \\ \hline
Se aplica a & Todos los elementos \\ \hline
Valor inicial & - \\ \hline
Descripción & Establece el color de todos los bordes del elemento \\ \hline
  \end{tabular}
   \caption{Definición del atributo border-color de CSS}
 \end{table}
\end{center}

\end{frame}

%%---------------------------------------------------------------

\begin{frame}
\frametitle{Estilo de los bordes}

\begin{center}
  \begin{table}
   \begin{tabular}{p{1.8cm}p{7.8cm}}
Atributos &\bf{border-top-style}, \bf{border-right-style}, \bf{border-bottom-style}, \bf{border-left-style} \\ \hline
Valores & none | hidden | dotted | dashed | solid | double | groove | ridge | inset | outset | inherit \\ \hline
Se aplica a & Todos los elementos \\ \hline
Valor inicial & none \\ \hline
Descripción & Establece el estilo de cada uno de los cuatro bordes de los elementos \\ \hline
  \end{tabular}
   \caption{Definición del atributo border-top-style, border-right-style, border-bottom-style, border-left-style de CSS}
 \end{table}
\end{center}

\end{frame}

%%---------------------------------------------------------------

\begin{frame}
\frametitle{Estilo de los bordes \emph{shorthand}}

\begin{center}
  \begin{table}
   \begin{tabular}{p{1.8cm}p{7.8cm}}
Atributo &\bf{border-style} \\ \hline
Valores & (none | hidden | dotted | dashed | solid | double | groove | ridge | inset | outset ) {1, 4} | inherit \\ \hline
Se aplica a & Todos los elementos \\ \hline
Valor inicial & - \\ \hline
Descripción & Establece el estilo de todos los bordes del elemento \\ \hline
  \end{tabular}
   \caption{Definición del atributo border-style de CSS}
 \end{table}
\end{center}

\end{frame}

%%---------------------------------------------------------------

\begin{frame}
\frametitle{Atributos \emph{shorthand} para bordes}

\begin{center}
  \begin{table}
   \begin{tabular}{p{1.8cm}p{7.8cm}}
Atributos &\bf{border-top}, \bf{border-right}, \bf{border-bottom}, \bf{border-left} \\ \hline
Valores & ( $<medida\_borde>$ || $<color\_borde>$ || $<estilo\_borde>$ ) | inherit \\ \hline
Se aplica a & Todos los elementos \\ \hline
Valor inicial & - \\ \hline
Descripción & Establece el estilo completo de cada uno de los cuatro bordes de los elementos \\ \hline
 \end{tabular}
   \caption{Definición del atributo border-top, border-right, border-bottom, border-left de CSS}
 \end{table}
\end{center}

\end{frame}

%%---------------------------------------------------------------

\begin{frame}
\frametitle{Atributo \emph{shorthand} para borde (global)}

\begin{center}
  \begin{table}
   \begin{tabular}{p{1.8cm}p{7.8cm}}
Atributo &\bf{border} \\ \hline
Valores & ( $<medida\_borde>$ || $<color\_borde>$ || $<estilo\_borde>$ ) | inherit \\ \hline
Se aplica a & Todos los elementos \\ \hline
Valor inicial & - \\ \hline
Descripción & Establece el estilo completo de todos los bordes de los elementos \\ \hline
 \end{tabular}
   \caption{Definición del atributo border de CSS}
 \end{table}
\end{center}

\end{frame}

%%---------------------------------------------------------------

\begin{frame}[fragile]
\frametitle{Más sobre bordes}

\begin{itemize}
  \item Como el valor por defecto del atributo border-style es none, si un atributo shorthand no establece explícitamente el estilo de un borde, el elemento no muestra ese borde
  \item Cuando los cuatro bordes no son idénticos pero sí muy parecidos, se puede utilizar el atributo border para establecer de forma directa los atributos comunes de todos los bordes y posteriormente especificar para cada uno de los cuatro bordes sus atributos particulares:
\begin{verbatim}
h1 {
  border: solid #000;
  border-top-width: 6px;
  border-left-width: 8px;
}
\end{verbatim}
\end{itemize}

\end{frame}

%%---------------------------------------------------------------

\begin{frame}
\frametitle{Fondos}

\begin{itemize}
  \item Puede ser un color simple o una imagen.
  \item Solamente se visualiza en el área ocupada por el contenido y su relleno, ya que el color de los bordes se controla directamente desde los bordes y las zonas de los márgenes siempre son transparentes
  \item Se puede establecer de forma simultánea un color y una imagen de fondo. En este caso, la imagen se muestra delante del color, por lo que solamente si la imagen contiene zonas transparentes es posible ver el color de fondo.
\end{itemize}

\end{frame}


%%---------------------------------------------------------------

\begin{frame}
\frametitle{Atributo background-color}

\begin{center}
  \begin{table}
   \begin{tabular}{p{1.8cm}p{7.8cm}}
Atributo & \bf{background-color} \\ \hline
Valores& $color>$ | transparent | inherit \\ \hline
Se aplica a& Todos los elementos \\ \hline
Valor inicial& transparent \\ \hline
Descripción& Establece un color de fondo para los elementos \\ \hline
  \end{tabular}
   \caption{Definición del atributo background-color de CSS}
 \end{table}
\end{center}


\end{frame}


%%---------------------------------------------------------------

\begin{frame}
\frametitle{Atributo background-image}

\begin{center}
  \begin{table}
   \begin{tabular}{p{1.8cm}p{7.8cm}}
Atributo & \bf{background-image} \\ \hline
Valores& $url>$ | none | inherit \\ \hline
Se aplica a& Todos los elementos \\ \hline
Valor inicial& none \\ \hline
Descripción& Establece una imagen como fondo para los elementos \\ \hline
  \end{tabular}
   \caption{Definición del atributo background-image de CSS}
 \end{table}
\end{center}


\end{frame}


%%---------------------------------------------------------------

\begin{frame}
\frametitle{Atributo background-repeat}

\begin{center}
  \begin{table}
   \begin{tabular}{p{1.8cm}p{7.8cm}}
Atributo & \bf{background-repeat} \\ \hline
Valores& repeat | repeat-x | repeat-y | no-repeat | inherit \\ \hline
Se aplica a& Todos los elementos \\ \hline
Valor inicial& repeat \\ \hline
Descripción& Controla la forma en la que se repiten las imágenes de fondo \\ \hline
  \end{tabular}
   \caption{Definición del atributo background-repeat de CSS}
 \end{table}
\end{center}


\end{frame}


%%---------------------------------------------------------------

\begin{frame}
\frametitle{Atributo background-position}

\begin{center}
  \begin{table}
   \begin{tabular}{p{1.8cm}p{7.8cm}}
Atributo & \bf{background-position} \\ \hline
Valores& ( ( $<porcentaje>$ | $<medida>$ | left | center | right ) ( $<porcentaje>$ | $<medida>$ | top | center | bottom )? ) | ( ( left | center | right ) || ( top | center | bottom ) ) | inherit \\ \hline
Se aplica a& Todos los elementos \\ \hline
Valor inicial& 0\% 0\% \\ \hline
Descripción& Controla la posición en la que se muestra la imagen en el fondo del elemento \\ \hline
  \end{tabular}
   \caption{Definición del atributo background-position de CSS}
 \end{table}
\end{center}


\end{frame}


%%---------------------------------------------------------------

\begin{frame}
\frametitle{Atributo background-attachment}

\begin{center}
  \begin{table}
   \begin{tabular}{p{1.8cm}p{7.8cm}}
Atributo & \bf{background-attachment} \\ \hline
Valores& scroll | fixed | inherit \\ \hline
Se aplica a& Todos los elementos \\ \hline
Valor inicial& scroll \\ \hline
Descripción& Controla la forma en la que se visualiza la imagen de fondo: permanece fija cuando se hace scroll en la ventana del navegador o se desplaza junto con la ventana \\ \hline
  \end{tabular}
   \caption{Definición del atributo background-attachment de CSS}
 \end{table}
\end{center}


\end{frame}


%%---------------------------------------------------------------

\begin{frame}
\frametitle{Atributo \emph{shorthand} background}

\begin{center}
  \begin{table}
   \begin{tabular}{p{1.8cm}p{7.8cm}}
Atributo & \bf{background} \\ \hline
Valores& ( $background-color$ || $background-image$ || $background-repeat$ || $background-attachment$ || $background-position$ ) | inherit \\ \hline
Se aplica a& Todos los elementos \\ \hline
Valor inicial& - \\ \hline
Descripción& Establece todas los atributos del fondo de un elemento \\ \hline
  \end{tabular}
   \caption{Definición del atributo background de CSS}
 \end{table}
\end{center}


\end{frame}



%%---------------------------------------------------------------
\subsubsection*{Posicionamiento y visualización}


\begin{frame}
\frametitle{Posicionamiento y visualización}

\begin{itemize}
  \item Los navegadores crean y posicionan de forma automática todas las cajas que forman cada página HTML
  \item El diseñador puede modificar la posición en la que se muestra cada caja.
  \item Existen {\bf cinco tipos de posicionamiento} definidos para las cajas
\end{itemize}

\end{frame}



%%---------------------------------------------------------------

\begin{frame}
\frametitle{Tipos de posicionamiento}

\begin{enumerate}
  \item {\bf Normal o estático}: posicionamientosi no se indica lo contrario.
  \item{\bf Relativo}: consiste en posicionar una caja según el posicionamiento normal y después desplazarla respecto de su posición original.
  \item {\bf Absoluto}: la posición de una caja se establece de forma absoluta respecto de su elemento contenedor y el resto de elementos de la página ignoran la nueva posición del elemento.
  \item {\bf Fijo}: variante del posicionamiento absoluto que convierte una caja en un elemento inamovible, de forma que su posición en la pantalla siempre es la misma independientemente del resto de elementos e independientemente de si el usuario sube o baja la página en la ventana del navegador.
  \item {\bf Flotante}: desplaza las cajas todo lo posible hacia la izquierda o hacia la derecha de la línea en la que se encuentran.
\end{enumerate}

\end{frame}

%%---------------------------------------------------------------

\begin{frame}
\frametitle{Atributo position}

\begin{center}
  \begin{table}
   \begin{tabular}{p{1.8cm}p{7.8cm}}
Atributo & \bf{position} \\ \hline
Valores& static | relative | absolute | fixed | inherit \\ \hline
Se aplica a& Todos los elementos \\ \hline
Valor inicial& static \\ \hline
Descripción& Selecciona el posicionamiento con el que se mostrará el elemento \\ \hline
  \end{tabular}
   \caption{Definición del atributo position de CSS}
 \end{table}
\end{center}


\end{frame}



%%---------------------------------------------------------------

\begin{frame}
\frametitle{Significados atributo position}

\begin{itemize}
  \item static: corresponde al posicionamiento normal o estático. Si se utiliza este valor, se ignoran los valores de los atributos top, right, bottom y left que se verán a continuación.
  \item relative: corresponde al posicionamiento relativo. El desplazamiento de la caja se controla con los atributos top, right, bottom y left.
  \item absolute: corresponde al posicionamiento absoluto. El desplazamiento de la caja también se controla con los atributos top, right, bottom y left, pero su interpretación es mucho más compleja, ya que el origen de coordenadas del desplazamiento depende del posicionamiento de su elemento contenedor.
  \item fixed: corresponde al posicionamiento fijo. El desplazamiento se establece de la misma forma que en el posicionamiento absoluto, pero en este caso el elemento permanece inamovible en la pantalla.
\end{itemize}

\end{frame}


%%---------------------------------------------------------------
\subsubsection*{Pseudo-clases}

\begin{frame}
\frametitle{Pseudo-clases}

Como con los atributos id o class no es posible aplicar diferentes estilos a un mismo elemento en función de su estado, CSS introduce un nuevo concepto llamado pseudo-clases. Por ejemplo, en enlaces:

\begin{itemize}
  \item {\bf :link}: enlaces que apuntan a páginas o recursos que aún no han sido visitados por el usuario.
  \item {\bf :visited}: enlaces que apuntan a recursos que han sido visitados anteriormente por el usuario. El historial de enlaces visitados se borra automáticamente cada cierto tiempo y el usuario también puede borrarlo manualmente.
  \item {\bf :hover}: enlace sobre el que el usuario ha posicionado el puntero del ratón.
  \item {\bf :active}: enlace que está pinchando el usuario. Los estilos sólo se aplican desde que el usuario pincha el botón del ratón hasta que lo suelta.
\end{itemize}

\end{frame}



%%---------------------------------------------------------------

\begin{frame}
\frametitle{Atributos top, right, bottom, left}

\begin{center}
  \begin{table}
   \begin{tabular}{p{1.8cm}p{7.8cm}}
Atributos& {\bf top}, {\bf right}, {\bf bottom}, {\bf left} \\ \hline
Valores& $<medida>$ | $<porcentaje>$ | auto | inherit \\ \hline
Se aplica a& Todos los elementos posicionados \\ \hline
Valor inicial& auto \\ \hline
Descripción& Indican el desplazamiento horizontal y vertical del elemento respecto de su posición original \\ \hline
  \end{tabular}
   \caption{Definición del atributo top, right, bottom, left de CSS}
 \end{table}
\end{center}


\end{frame}


%%---------------------------------------------------------------

\begin{frame}
\frametitle{Posicionamiento normal (o estático)}

\begin{itemize}
  \item Utilizado por defecto por los navegadores
  \item Sólo se tiene en cuenta si el elemento es de bloque o en línea, sus atributos width y height y su contenido.
  \item Las cajas se muestran una debajo de otra comenzando desde el principio del elemento contenedor. La distancia entre las cajas se controla mediante los márgenes verticales.
  \item Si un elemento se encuentra dentro de otro, el elemento padre se llama ``elemento contenedor'' y determina tanto la posición como el tamaño de todas sus cajas interiores.
\end{itemize}


\begin{center}
\begin{figure}[p]
\includegraphics[width=0.8\textwidth]{figs/f0502.png}
\end{figure}
\end{center}

\end{frame}


%%---------------------------------------------------------------

\begin{frame}
\frametitle{Posicionamiento normal (o estático) (y II)}

\begin{itemize}
  \item Los elementos en línea forman los ``contextos de formato en línea''. Las cajas se muestran una detrás de otra de forma horizontal comenzando desde la posición más a la izquierda de su elemento contenedor.
  \item Si las cajas en línea ocupan más espacio del disponible en su propia línea, el resto de cajas se muestran en las líneas inferiores. 
  \item Si las cajas en línea ocupan un espacio menor que su propia línea, se puede controlar la distribución de las cajas mediante el atributo text-align para centrarlas, alinearlas a la derecha o justificarlas.
\end{itemize}


\begin{center}
\begin{figure}[p]
\includegraphics[width=0.8\textwidth]{figs/f0503.png}
\end{figure}
\end{center}

\end{frame}


%%---------------------------------------------------------------

\begin{frame}
\frametitle{Posicionamiento relativo}

\begin{itemize}
  \item Desplaza una caja respecto de su posición original establecida mediante el posicionamiento normal. El desplazamiento de la caja se controla con los atributos top, right, bottom y left.
  \item el atributo top se emplea para mover las cajas de forma descendente, el atributo bottom mueve las cajas de forma ascendente, el atributo left se utiliza para desplazar las cajas hacia la derecha y el atributo right mueve las cajas hacia la izquierda. 
\end{itemize}

\begin{center}
\begin{figure}[p]
\includegraphics[width=0.8\textwidth]{figs/f0504.png}
\end{figure}
\end{center}

\end{frame}


%%---------------------------------------------------------------

\begin{frame}
\frametitle{Posicionamiento absoluto}

\begin{itemize}
  \item Se emplea para establecer de forma exacta la posición en la que se muestra la caja de un elemento. 
  \item Cuando una caja se posiciona de forma absoluta, el resto de elementos de la página se ven afectados y modifican su posición. 
\end{itemize}

\begin{center}
\begin{figure}[p]
\includegraphics[width=0.8\textwidth]{figs/f0516.png}
\end{figure}
\end{center}

\end{frame}


%%---------------------------------------------------------------

\begin{frame}
\frametitle{Posicionamiento absoluto}

\begin{itemize}
  \item El primer elemento contenedor que esté posicionado de cualquier forma diferente a position: static se convierte en la referencia que determina la posición de la caja posicionada de forma absoluta.
  \item Si ningún elemento contenedor está posicionado, la referencia es la ventana del navegador, que no debe confundirse con el elemento $<body>$ de la página.
  \item Una vez determinada la referencia del posicionamiento absoluto, la interpretación de los valores de los atributos top, right, bottom y left se realiza como sigue:
  \begin{itemize}
    \item Top: desplazamiento desde el borde superior del elemento contenedor que se utiliza como referencia.
    \item Right: ídem pero desde el borde derecho al borde derecho.
    \item Left:  ídem pero desde el borde izquierdo al borde izquierdo.
    \item Bottom: ídem pero desde el borde inferior al borde inferior.
  \end{itemize}
\end{itemize}

\end{frame}


%%---------------------------------------------------------------

\begin{frame}
\frametitle{Diferencias entre posicionamiento absoluto y relativo}

\begin{center}
\begin{figure}[p]
\includegraphics[width=0.6\textwidth]{figs/f0519.png}
\end{figure}
\end{center}

\begin{center}
\begin{figure}[p]
\includegraphics[width=0.6\textwidth]{figs/f0521.png}
\end{figure}
\end{center}

\end{frame}


%%---------------------------------------------------------------

\begin{frame}
\frametitle{Posicionamiento fijo}

\begin{itemize}
  \item Es un caso particular del posicionamiento absoluto, ya que sólo se diferencian en el comportamiento de las cajas posicionadas.
  \item La principal característica de una caja posicionada de forma fija es que su posición es inamovible dentro de la ventana del navegador.
  \item El posicionamiento fijo hace que las cajas no modifiquen su posición ni aunque el usuario suba o baje la página en la ventana de su navegador.
  \item Si la página se visualiza en un medio paginado (por ejemplo en una impresora) las cajas posicionadas de forma fija se repiten en todas las páginas.
\end{itemize}

\end{frame}


%%---------------------------------------------------------------

\begin{frame}
\frametitle{Posicionamiento flotante}

\begin{itemize}
  \item Cuando una caja se posiciona con el modelo de posicionamiento flotante, automáticamente se convierte en una caja flotante, lo que significa que se desplaza hasta la zona más a la izquierda o más a la derecha de la posición en la que originalmente se encontraba.
\end{itemize}


\begin{center}
\begin{figure}[p]
\includegraphics[width=0.8\textwidth]{figs/f0507.png}
\end{figure}
\end{center}

\end{frame}


%%---------------------------------------------------------------

\begin{frame}
\frametitle{Posicionamiento flotante (y II)}

\begin{itemize}
  \item Cuando se posiciona una caja de forma flotante:
  \begin{itemize}
    \item La caja deja de pertenecer al flujo normal de la página, lo que significa que el resto de cajas ocupan el lugar dejado por la caja flotante.
    \item La caja flotante se posiciona lo más a la izquierda o lo más a la derecha posible de la posición en la que se encontraba originalmente.
  \end{itemize}
\end{itemize}


\begin{center}
\begin{figure}[p]
\includegraphics[width=0.8\textwidth]{figs/f0508.png}
\end{figure}
\end{center}

\end{frame}


%%---------------------------------------------------------------

\begin{frame}
\frametitle{Posicionamiento flotante (y III)}

\begin{itemize}
  \item Si existen otras cajas flotantes, al posicionar de forma flotante otra caja, se tiene en cuenta el sitio disponible.
  \item Si no existiera sitio en la línea actual, la caja flotante baja a la línea inferior hasta que encuentra el sitio necesario para mostrarse lo más a la izquierda o lo más a la derecha posible en esa nueva línea
\end{itemize}


\begin{center}
\begin{figure}[p]
\includegraphics[width=0.8\textwidth]{figs/f0509.png}
\end{figure}
\end{center}

\end{frame}


%%---------------------------------------------------------------

\begin{frame}
\frametitle{Atributo float}

\begin{center}
  \begin{table}
   \begin{tabular}{p{1.8cm}p{7.8cm}}
Atributo & \bf{float} \\ \hline
Valores& left | right | none | inherit \\ \hline
Se aplica a& Todos los elementos \\ \hline
Valor inicial& none \\ \hline
Descripción& Establece el tipo de posicionamiento flotante del elemento \\ \hline
  \end{tabular}
   \caption{Definición del atributo float de CSS}
 \end{table}
\end{center}


\end{frame}


%%---------------------------------------------------------------

\begin{frame}
\frametitle{Posicionamiento flotante (y IV)}

\begin{itemize}
  \item Los elementos que se encuentran alrededor de una caja flotante adaptan sus contenidos para que fluyan alrededor del elemento posicionado
  \item Uno de los principales motivos para la creación del posicionamiento float fue precisamente la posibilidad de colocar imágenes alrededor de las cuales fluye el texto.
\end{itemize}


\begin{center}
\begin{figure}[p]
\includegraphics[width=0.8\textwidth]{figs/f0513.png}
\end{figure}
\end{center}

\end{frame}


%%---------------------------------------------------------------

\begin{frame}
\frametitle{Atributo clear}

\begin{itemize}
  \item El atributo clear indica el lado del elemento HTML que no debe ser adyacente a ninguna caja posicionada de forma flotante. Si se indica el valor left, el elemento se desplaza de forma descendente hasta que pueda colocarse en una línea en la que no haya ninguna caja flotante en el lado izquierdo.
  \item La especificación oficial de CSS explica este comportamiento como ``un desplazamiento descendente hasta que el borde superior del elemento esté por debajo del borde inferior de cualquier elemento flotante hacia la izquierda''.
\end{itemize}

\end{frame}


%%---------------------------------------------------------------

\begin{frame}
\frametitle{Atributo clear}

\begin{center}
  \begin{table}
   \begin{tabular}{p{1.8cm}p{7.8cm}}
Atributo & \bf{clear} \\ \hline
Valores& none | left | right | both | inherit \\ \hline
Se aplica a& Todos los elementos de bloque \\ \hline
Valor inicial& none \\ \hline
Descripción& Indica el lado del elemento que no debe ser adyacente a ninguna caja flotante \\ \hline
  \end{tabular}
   \caption{Definición del atributo clear de CSS}
 \end{table}
\end{center}


\end{frame}



%%---------------------------------------------------------------

\begin{frame}
\frametitle{Atributo display}

\begin{center}
  \begin{table}
   \begin{tabular}{p{1.8cm}p{7.8cm}}
Atributo & \bf{display} \\ \hline
Valores& inline | block | none | list-item | run-in | inline-block | table | inline-table | table-row-group | table-header-group | table-footer-group | table-row | table-column-group | table-column | table-cell | table-caption | inherit \\ \hline
Se aplica a& Todos los elementos \\ \hline
Valor inicial& inline \\ \hline
Descripción& Permite controlar la forma de visualizar un elemento e incluso ocultarlo \\ \hline
  \end{tabular}
   \caption{Definición del atributo display de CSS}
 \end{table}
\end{center}


\end{frame}




%%---------------------------------------------------------------

\begin{frame}
\frametitle{Atributo visibility}

\begin{center}
  \begin{table}
   \begin{tabular}{p{1.8cm}p{7.8cm}}
Atributo & \bf{visibility} \\ \hline
Valores& visible | hidden | collapse | inherit \\ \hline
Se aplica a& Todos los elementos \\ \hline
Valor inicial& visible \\ \hline
Descripción& Permite hacer visibles e invisibles a los elementos \\ \hline
  \end{tabular}
   \caption{Definición del atributo visibility de CSS}
 \end{table}
\end{center}


\end{frame}


%%---------------------------------------------------------------

\begin{frame}
\frametitle{Atributo overflow}

\begin{itemize}
  \item En algunas ocasiones el contenido de un elemento no cabe en el espacio reservado para ese elemento y se desborda.
  \item La situación más habitual en la que el contenido sobresale de su espacio reservado es cuando se establece la anchura y/o altura de un elemento mediante el atributo width y/o height. 
  \item Los valores del atributo overflow tienen el siguiente significado:
  \begin{itemize}
    \item visible: el contenido no se corta y se muestra sobresaliendo la zona reservada para visualizar el elemento. Este es el comportamiento por defecto.
    \item hidden: el contenido sobrante se oculta y sólo se visualiza la parte del contenido que cabe dentro de la zona reservada para el elemento.
    \item scroll: solamente se visualiza el contenido que cabe dentro de la zona reservada para el elemento, pero también se muestran barras de scroll que permiten visualizar el resto del contenido.
    \item auto: el comportamiento depende del navegador, aunque normalmente es el mismo que el atributo scroll.
  \end{itemize}
\end{itemize}

\end{frame}


%%---------------------------------------------------------------

\begin{frame}
\frametitle{Atributo overflow (y II)}

\begin{center}
\begin{figure}[p]
\includegraphics[width=0.8\textwidth]{figs/f0524.png}
\end{figure}
\end{center}

\end{frame}


%%---------------------------------------------------------------

\begin{frame}
\frametitle{Atributo overflow (y III)}

\begin{center}
  \begin{table}
   \begin{tabular}{p{1.8cm}p{7.8cm}}
Atributo & \bf{overflow} \\ \hline
Valores& visible | hidden | scroll | auto | inherit \\ \hline
Se aplica a& Elementos de bloque y celdas de tablas \\ \hline
Valor inicial& visible \\ \hline
Descripción& Permite controlar los contenidos sobrantes de un elemento \\ \hline
  \end{tabular}
   \caption{Definición del atributo overflow de CSS}
 \end{table}
\end{center}


\end{frame}



%%---------------------------------------------------------------

\begin{frame}
\frametitle{Atributo z-index}

\begin{itemize}
  \item CSS permite controlar la posición tridimensional de las cajas posicionadas
  \item Es posible indicar las cajas que se muestran delante o detrás de otras cajas cuando se producen solapamientos.
  \item Cuanto más alto sea el valor numérico, más cerca del usuario se muestra la caja.
\end{itemize}

\end{frame}


%%---------------------------------------------------------------

\begin{frame}
\frametitle{Atributo z-index (II)}


\begin{center}
\begin{figure}[p]
\includegraphics[width=0.66\textwidth]{figs/f0525.png}
\end{figure}
\end{center}

\end{frame}



%%---------------------------------------------------------------

\begin{frame}
\frametitle{Atributo z-index (y III)}

\begin{center}
  \begin{table}
   \begin{tabular}{p{1.8cm}p{7.8cm}}
Atributo & \bf{z-index} \\ \hline
Valores& auto | $<numero>$ | inherit \\ \hline
Se aplica a& Elementos que han sido posicionados explícitamente \\ \hline
Valor inicial& auto \\ \hline
Descripción& Establece el nivel tridimensional en el que se muestra el elemento \\ \hline
  \end{tabular}
   \caption{Definición del atributo z-index de CSS}
 \end{table}
\end{center}


\end{frame}

%%---------------------------------------------------------------

\begin{frame}[fragile]
\frametitle{Selectores avanzados}

\begin{enumerate}
  \item Selector de hijos
\begin{verbatim}
p > span { color: blue; }
\end{verbatim}
  \item Selector adyacente
\begin{verbatim}
p + p { text-indent: 1.5em; }
\end{verbatim}
  \item Selector de atributos
\begin{verbatim}
a[class="externo"] { color: blue; }
a[class~="externo"] { color: blue; }
*[lang=en] { ... }
*[lang|="es"] { color : red }
\end{verbatim}
\end{enumerate}

\end{frame}



%%---------------------------------------------------------------

\begin{frame}
\frametitle{Colisión de estilos}

El método seguido por CSS para resolver las colisiones de estilos se muestra a continuación:

\begin{enumerate}
  \item Determinar todas las declaraciones que se aplican al elemento para el medio CSS seleccionado.
  \item Ordenar las declaraciones según su origen (CSS de navegador, de usuario o de diseñador) y su prioridad (palabra clave !important).
  \item Ordenar las declaraciones según lo específico que sea el selector. Cuanto más genérico es un selector, menos importancia tienen sus declaraciones.
  \item Si después de aplicar las normas anteriores existen dos o más reglas con la misma prioridad, se aplica la que se indicó en último lugar.
\end{enumerate}

\end{frame}



%%---------------------------------------------------------------

\begin{frame}
\frametitle{Medios CSS}

\begin{itemize}
  \item Permiten definir diferentes estilos para diferentes medios o dispositivos: pantallas, impresoras, móviles, proyectores, etc.
  \item Define algunas atributos específicamente para determinados medios: la paginación y los saltos de página para los medios impresos o el volumen y tipo de voz para los medios de audio.
  \item Ejemplos:
  \begin{itemize}
    \item screen: Pantallas de ordenador
    \item print: Impresoras y navegadores en el modo ``Vista Previa para Imprimir''
    \item handheld:	Dispositivos de mano: móviles, PDA, etc.
  \end{itemize}
\end{itemize}

\end{frame}

%%---------------------------------------------------------------

\begin{frame}[fragile]
\frametitle{Formas de indicar el medio}

\begin{enumerate}
  \item Reglas de tipo @media
{\footnotesize
    \begin{verbatim}
@media print {
  body { font-size: 10pt }
}
@media screen {
  body { font-size: 13px }
}
    \end{verbatim}
}
  \item Reglas de tipo @import
{\footnotesize
    \begin{verbatim}
@import url("estilos_basicos.css") screen;
@import url("estilos_impresora.css") print;
    \end{verbatim}
}
  \item Medios definidos con la etiqueta
{\footnotesize
    \begin{verbatim}
<link rel="stylesheet" type="text/css" media="screen" href="basico.css" />
<link rel="stylesheet" type="text/css" media="print, handheld" href="especial.css" />
    \end{verbatim}
}
  \item Medios definidos mezclando varios métodos
{\footnotesize
    \begin{verbatim}
<link rel="stylesheet" type="text/css"  media="screen" href="basico.css" />
@import url("estilos_seccion.css") screen;
@media print {
  /* Estilos específicos para impresora */
}
    \end{verbatim}
}
\end{enumerate}

\end{frame}

%%---------------------------------------------------------------

\begin{frame}[fragile]
\frametitle{Comentarios}

\begin{itemize}
  \item El comienzo de un comentario se indica mediante los caracteres /* y el final del comentario se indica mediante */
    \begin{verbatim}
/* Este es un comentario en CSS */
    \end{verbatim}
  \item Pueden ocupar tantas líneas como sea necesario, pero no se puede incluir un comentario dentro de otro comentario
    \begin{verbatim}
/* Este es un
   comentario CSS de varias
   lineas */
    \end{verbatim}
\end{itemize}

\end{frame}


%%---------------------------------------------------------------
\subsubsection*{Texto}

\begin{frame}
\frametitle{Tipografía}

\begin{itemize}
  \item CSS define numerosos atributos para modificar la apariencia del texto
  \item color se utiliza para establecer el color de la letra
  \item Como el valor del atributo color se hereda, normalmente se establece el atributo color en el elemento body para establecer el color de letra de todos los elementos de la página
 \item font-family se utiliza para indicar el tipo de letra con el que se muestra el texto
  \item Suele definirse como una lista de tipos de letra alternativos separados por comas. El último valor de la lista es el nombre de la familia tipográfica genérica que más se parece al tipo de letra que se quiere utilizar.
\end{itemize}

\end{frame}


%%---------------------------------------------------------------

\begin{frame}
\frametitle{Atributo color}

\begin{center}
  \begin{table}
   \begin{tabular}{p{1.8cm}p{7.8cm}}
Atributo & \bf{color} \\ \hline
Valores& $<color>$ | inherit \\ \hline
Se aplica a& Todos los elementos \\ \hline
Valor inicial& Depende del navegador \\ \hline
Descripción& Establece el color de letra utilizado para el texto \\ \hline
  \end{tabular}
   \caption{Definición del atributo color de CSS}
 \end{table}
\end{center}


\end{frame}


%%---------------------------------------------------------------

\begin{frame}
\frametitle{Atributo font-family}

\begin{center}
  \begin{table}
   \begin{tabular}{p{1.8cm}p{7.8cm}}
Atributo & \bf{font-family} \\ \hline
Valores& (( $<nombre\_familia>$ | $<familia\_generica>$ ) (,$nombre\_familia>$ | $<familia\_generica$)* ) | inherit \\ \hline
Se aplica a& Todos los elementos \\ \hline
Valor inicial& Depende del navegador \\ \hline
Descripción& Establece el tipo de letra utilizado para el texto \\ \hline
  \end{tabular}
   \caption{Definición del atributo font-family de CSS}
 \end{table}
\end{center}


\end{frame}


%%---------------------------------------------------------------

\begin{frame}
\frametitle{Atributo font-size (I)}

\begin{itemize}
  \item Además de medida relativas, absolutas y de porcentajes, CSS permite utilizar una serie de palabras clave para indicar el tamaño de letra del texto:
  \begin{itemize}
    \item tamaño\_absoluto: indica el tamaño de letra de forma absoluta mediante alguna de las siguientes palabras clave: xx-small, x-small, small, medium, large, x-large, xx-large.
    \item tamaño\_relativo: indica de forma relativa el tamaño de letra del texto mediante dos palabras clave (larger, smaller) que toman como referencia el tamaño de letra del elemento padre.
  \end{itemize}
\end{itemize}


\begin{center}
\begin{figure}[p]
\includegraphics[width=0.6\textwidth]{figs/f0601.png}
\end{figure}
\end{center}

\end{frame}


%%---------------------------------------------------------------

\begin{frame}
\frametitle{Atributo font-size (y II)}

\begin{center}
  \begin{table}
   \begin{tabular}{p{1.8cm}p{7.8cm}}
Atributo & \bf{font-size} \\ \hline
Valores & $<tamano\_absoluto>$ | $<tamano\_relativo>$ | $<medida>$ | $<porcentaje>$ | inherit \\ \hline
Se aplica a & Todos los elementos \\ \hline
Valor inicial & medium \\ \hline
Descripción & Establece el tamaño de letra utilizado para el texto \\ \hline
  \end{tabular}
   \caption{Definición del atributo font-size de CSS}
 \end{table}
\end{center}

\end{frame}


%%---------------------------------------------------------------

\begin{frame}
\frametitle{Atributo font-weight}

\begin{center}
  \begin{table}
   \begin{tabular}{p{1.8cm}p{7.8cm}}
Atributo & \bf{font-weight} \\ \hline
Valores& normal | bold | bolder | lighter | 100 | 200 | 300 | 400 | 500 | 600 | 700 | 800 | 900 | inherit \\ \hline
Se aplica a& Todos los elementos \\ \hline
Valor inicial& normal \\ \hline
Descripción& Establece la anchura de la letra utilizada para el texto \\ \hline
  \end{tabular}
   \caption{Definición del atributo font-weight de CSS}
 \end{table}
\end{center}


\end{frame}


%%---------------------------------------------------------------

\begin{frame}
\frametitle{Atributo font-style}

\begin{center}
  \begin{table}
   \begin{tabular}{p{1.8cm}p{7.8cm}}
Atributo & \bf{font-style} \\ \hline
Valores& normal | italic | oblique | inherit \\ \hline
Se aplica a& Todos los elementos \\ \hline
Valor inicial& normal \\ \hline
Descripción& Establece el estilo de la letra utilizada para el texto \\ \hline
  \end{tabular}
   \caption{Definición del atributo font-style de CSS}
 \end{table}
\end{center}

\end{frame}


%%---------------------------------------------------------------

\begin{frame}
\frametitle{Atributo font-variant}

\begin{center}
  \begin{table}
   \begin{tabular}{p{1.8cm}p{7.8cm}}
Atributo & \bf{font-variant} \\ \hline
Valores& normal | small-caps | inherit \\ \hline
Se aplica a& Todos los elementos \\ \hline
Valor inicial& normal \\ \hline
Descripción& Establece el estilo alternativo de la letra utilizada para el texto \\ \hline
  \end{tabular}
   \caption{Definición del atributo font-variant de CSS}
 \end{table}
\end{center}


\end{frame}


%%---------------------------------------------------------------

\begin{frame}
\frametitle{Atributo \emph{short-hand} font}

\begin{center}
  \begin{table}
   \begin{tabular}{p{1.8cm}p{7.8cm}}
Atributo & \bf{font} \\ \hline
Valores& ( ( $<font-style>$ || $<font-variant>$ || $<font-weight>$ )? $<font-size>$ ( / $<line-height>$ )? $<font-family>$ ) | caption | icon | menu | message-box | small-caption | status-bar | inherit \\ \hline
Se aplica a& Todos los elementos \\ \hline
Valor inicial& - \\ \hline
Descripción& Permite indicar de forma directa todas los atributos de la tipografía de un texto \\ \hline
  \end{tabular}
   \caption{Definición del atributo font de CSS}
 \end{table}
\end{center}


\end{frame}


%%---------------------------------------------------------------

\begin{frame}[fragile]
\frametitle{Atributo \emph{short-hand} font (y II)}

\begin{itemize}
  \item El orden en el que se deben indicar los atributos del texto es el siguiente:
  \begin{itemize}
    \item En primer lugar y de forma opcional se indican el font-style, font-variant y font-weight en cualquier orden.
    \item A continuación, se indica obligatoriamente el valor de font-size seguido opcionalmente por el valor de line-height.
    \item Por último, se indica obligatoriamente el tipo de letra a utilizar.
  \end{itemize}
\end{itemize}

\begin{footnotesize}
\begin{verbatim}
font: bold 1em "Trebuchet MS",Arial,Sans-Serif;
font: normal 0.9em "Lucida Grande", Verdana, Arial, Helvetica, sans-serif;
font: normal 1.2em/1em helvetica, arial, sans-serif;
\end{verbatim}
\end{footnotesize}

\end{frame}


%%---------------------------------------------------------------

\begin{frame}
\frametitle{Texto}

\begin{itemize}
  \item Además de los atributos relativas a la tipografía del texto, CSS define numerosos atributos que determinan la apariencia del texto en su conjunto.
  \item Estos atributos adicionales permiten controlar
  \begin{itemize}
    \item la alineación del texto,
    \item el interlineado,
    \item la separación entre palabras,
    \item etc.
  \end{itemize}
\end{itemize}

\end{frame}


%%---------------------------------------------------------------

\begin{frame}
\frametitle{Atributo text-align}

\begin{center}
  \begin{table}
   \begin{tabular}{p{1.8cm}p{7.8cm}}
Atributo & \bf{text-align} \\ \hline
Valores& left | right | center | justify | inherit \\ \hline
Se aplica a& Elementos de bloque y celdas de tabla \\ \hline
Valor inicial& left \\ \hline
Descripción& Establece la alineación del contenido del elemento \\ \hline
  \end{tabular}
   \caption{Definición del atributo text-align de CSS}
 \end{table}
\end{center}


\end{frame}


%%---------------------------------------------------------------

\begin{frame}
\frametitle{Atributo line-height}

\begin{center}
  \begin{table}
   \begin{tabular}{p{1.8cm}p{7.8cm}}
Atributo & \bf{line-height} \\ \hline
Valores& normal | $<numero>$ | $<medida>$ | $<porcentaje>$ | inherit \\ \hline
Se aplica a& Todos los elementos \\ \hline
Valor inicial& normal \\ \hline
Descripción& Permite establecer la altura de línea de los elementos \\ \hline
  \end{tabular}
   \caption{Definición del atributo line-height de CSS}
 \end{table}
\end{center}


\end{frame}


%%---------------------------------------------------------------

\begin{frame}
\frametitle{Atributo text-decoration}

\begin{center}
  \begin{table}
   \begin{tabular}{p{1.8cm}p{7.8cm}}
Atributo & \bf{text-decoration} \\ \hline
Valores& none | ( underline || overline || line-through || blink ) | inherit \\ \hline
Se aplica a& Todos los elementos \\ \hline
Valor inicial& none \\ \hline
Descripción& Establece la decoración del texto (subrayado, tachado, parpadeante, etc.) \\ \hline
  \end{tabular}
   \caption{Definición del atributo text-decoration de CSS}
 \end{table}
\end{center}


\end{frame}


%%---------------------------------------------------------------

\begin{frame}
\frametitle{Atributo text-transform}

\begin{center}
  \begin{table}
   \begin{tabular}{p{1.8cm}p{7.8cm}}
Atributo & \bf{text-transform} \\ \hline
Valores& capitalize | uppercase | lowercase | none | inherit \\ \hline
Se aplica a& Todos los elementos \\ \hline
Valor inicial& none \\ \hline
Descripción& Transforma el texto original (lo transforma a mayúsculas, a minúsculas, etc.) \\ \hline
  \end{tabular}
   \caption{Definición del atributo text-transform de CSS}
 \end{table}
\end{center}


\end{frame}


%%---------------------------------------------------------------

\begin{frame}
\frametitle{Atributo vertical-align}

\begin{center}
  \begin{table}
   \begin{tabular}{p{1.8cm}p{7.8cm}}
Atributo & \bf{vertical-align} \\ \hline
Valores& baseline | sub | super | top | text-top | middle | bottom | text-bottom | $<porcentaje>$ | $<medida>$ | inherit \\ \hline
Se aplica a& Elementos en línea y celdas de tabla \\ \hline
Valor inicial& baseline \\ \hline
Descripción& Determina la alineación vertical de los contenidos de un elemento \\ \hline
  \end{tabular}
   \caption{Definición del atributo vertical-align de CSS}
 \end{table}
\end{center}


\end{frame}


%%---------------------------------------------------------------

\begin{frame}
\frametitle{Atributo text-indent}

\begin{center}
  \begin{table}
   \begin{tabular}{p{1.8cm}p{7.8cm}}
Atributo & \bf{text-indent} \\ \hline
Valores& $<medida>$ | $<porcentaje>$ | inherit \\ \hline
Se aplica a& Los elementos de bloque y las celdas de tabla \\ \hline
Valor inicial& 0 \\ \hline
Descripción& Tabula desde la izquierda la primera línea del texto original \\ \hline
  \end{tabular}
   \caption{Definición del atributo text-indent de CSS}
 \end{table}
\end{center}


\end{frame}


%%---------------------------------------------------------------

\begin{frame}
\frametitle{Atributo letter-spacing}

\begin{center}
  \begin{table}
   \begin{tabular}{p{1.8cm}p{7.8cm}}
Atributo & \bf{letter-spacing} \\ \hline
Valores& normal | $<medida>$ | inherit \\ \hline
Se aplica a& Todos los elementos \\ \hline
Valor inicial& normal \\ \hline
Descripción& Permite establecer el espacio entre las letras que forman las palabras del texto \\ \hline
  \end{tabular}
   \caption{Definición del atributo letter-spacing de CSS}
 \end{table}
\end{center}


\end{frame}


%%---------------------------------------------------------------

\begin{frame}
\frametitle{Atributo word-spacing}

\begin{center}
  \begin{table}
   \begin{tabular}{p{1.8cm}p{7.8cm}}
Atributo & \bf{word-spacing} \\ \hline
Valores& normal | $<medida>$ | inherit \\ \hline
Se aplica a& Todos los elementos \\ \hline
Valor inicial& normal \\ \hline
Descripción& Permite establecer el espacio entre las palabras que forman el texto \\ \hline
  \end{tabular}
   \caption{Definición del atributo word-spacing de CSS}
 \end{table}
\end{center}


\end{frame}


%%---------------------------------------------------------------

\begin{frame}
\frametitle{Atributo white-space}

\begin{center}
  \begin{table}
   \begin{tabular}{p{1.8cm}p{7.8cm}}
Atributo & \bf{white-space} \\ \hline
Valores& normal | pre | nowrap | pre-wrap | pre-line | inherit \\ \hline
Se aplica a& Todos los elementos \\ \hline
Valor inicial& normal \\ \hline
Descripción& Establece el tratamiento de los espacios en blanco del texto \\ \hline
  \end{tabular}
   \caption{Definición del atributo white-space de CSS}
 \end{table}
\end{center}


\end{frame}


%%---------------------------------------------------------------

\begin{frame}
\frametitle{Atributo white-space (y II)}

El significado de cada uno de los valores es el siguiente:
\begin{itemize}
  \item normal: comportamiento por defecto de HTML.
  \item pre: se respetan los espacios en blanco y las nuevas líneas (exactamente igual que la etiqueta $<pre>$). Si la línea es muy larga, se sale del espacio asignado para ese contenido.
  \item nowrap: elimina los espacios en blanco y las nuevas líneas. Si la línea es muy larga, se sale del espacio asignado para ese contenido.
  \item pre-wrap: se respetan los espacios en blanco y las nuevas líneas, pero ajustando cada línea al espacio asignado para ese contenido.
  \item pre-line: elimina los espacios en blanco y respeta las nuevas líneas, pero ajustando cada línea al espacio asignado para ese contenido.
\end{itemize}

\end{frame}



%%---------------------------------------------------------------

\subsubsection*{Listas}

\begin{frame}
\frametitle{Atributo list-style-type}

\begin{center}
  \begin{table}
   \begin{tabular}{p{1.8cm}p{7.8cm}}
Atributo & \bf{list-style-type} \\ \hline
Valores& disc | circle | square | decimal | decimal-leading-zero | lower-roman | upper-roman | lower-greek | lower-latin | upper-latin | armenian | georgian | lower-alpha | upper-alpha | none | inherit \\ \hline
Se aplica a& Elementos de una lista \\ \hline
Valor inicial& disc \\ \hline
Descripción& Permite establecer el tipo de viñeta mostrada para una lista \\ \hline
  \end{tabular}
   \caption{Definición del atributo list-style-type de CSS}
 \end{table}
\end{center}


\end{frame}


%%---------------------------------------------------------------

\begin{frame}
\frametitle{Atributo list-style-position}

\begin{center}
  \begin{table}
   \begin{tabular}{p{1.8cm}p{7.8cm}}
Atributo & \bf{list-style-position} \\ \hline
Valores& inside | outside | inherit \\ \hline
Se aplica a& Elementos de una lista \\ \hline
Valor inicial& outside \\ \hline
Descripción& Permite establecer la posición de la viñeta de cada elemento de una lista \\ \hline
  \end{tabular}
   \caption{Definición del atributo list-style-type de CSS}
 \end{table}
\end{center}


\end{frame}


%%---------------------------------------------------------------

\begin{frame}
\frametitle{Atributo list-style-image}

\begin{center}
  \begin{table}
   \begin{tabular}{p{1.8cm}p{7.8cm}}
Atributo & \bf{list-style-image} \\ \hline
Valores& $<url>$ | none | inherit \\ \hline
Se aplica a& Elementos de una lista \\ \hline
Valor inicial& none \\ \hline
Descripción& Permite reemplazar las viñetas automáticas por una imagen personalizada \\ \hline
  \end{tabular}
   \caption{Definición del atributo list-style-image de CSS}
 \end{table}
\end{center}


\end{frame}


%%---------------------------------------------------------------

\begin{frame}
\frametitle{Atributo \emph{shorthand} list-style}

\begin{center}
  \begin{table}
   \begin{tabular}{p{1.8cm}p{7.8cm}}
Atributo & \bf{list-style} \\ \hline
Valores& ( $<list-style-type>$ || $<list-style-position>$ || $<list-style-image>$ ) | inherit \\ \hline
Se aplica a& Elementos de una lista \\ \hline
Valor inicial& - \\ \hline
Descripción& Atributo que permite establecer de forma simultánea todas las opciones de una lista \\ \hline
  \end{tabular}
   \caption{Definición del atributo list-style de CSS}
 \end{table}
\end{center}


\end{frame}


%%---------------------------------------------------------------

\begin{frame}
\frametitle{Imágenes según el estilo del enlace}

\begin{itemize}
  \item En ocasiones, puede resultar útil incluir un pequeño icono al lado de un enlace para indicar el tipo de contenido que enlaza.
  \item Este tipo de imágenes son puramente decorativas en vez de imágenes de contenido, por lo que se deberían añadir con CSS y no con elementos de tipo $<img>$. Utilizando el atributo background (y background-image) se puede personalizar el aspecto de los enlaces para que incluyan un pequeño icono a su lado.
  \item La técnica consiste en mostrar una imagen de fondo sin repetición en el enlace y añadir el padding necesario al texto del enlace para que no se solape con la imagen de fondo.
\end{itemize}

\end{frame}


%%---------------------------------------------------------------

\begin{frame}[fragile]
\frametitle{Imágenes según el estilo del enlace (II)}

\begin{footnotesize}
\begin{verbatim}
a { margin: 1em 0; float: left; clear: left; }
 
.rss {
  color: #E37529;
  padding: 0 0 0 18px;
  background: #FFF url(imagenes/rss.gif) no-repeat left center;
}
 
.pdf {
  padding: 0 0 0 22px;
  background: #FFF url(imagenes/pdf.png) no-repeat left center;
}
 
<a href="#">Enlace con el estilo por defecto</a>
<a class="rss" href="#">Enlace a un archivo RSS</a>
<a class="pdf" href="#">Enlace a un documento PDF</a>
\end{verbatim}
\end{footnotesize}

\end{frame}

%%---------------------------------------------------------------

\begin{frame}
\frametitle{Imágenes según el estilo del enlace (y III)}


\begin{center}
\begin{figure}[p]
\includegraphics[width=0.8\textwidth]{figs/f0704.png}
\end{figure}
\end{center}

\end{frame}


%%---------------------------------------------------------------

\begin{frame}[fragile]
\frametitle{Mostrar los enlaces como si fueran botones}

\begin{footnotesize}
\begin{verbatim}
a { margin: 1em 0; float: left; clear: left; }
a.boton {
  text-decoration: none;
  background: #EEE;
  color: #222;
  border: 1px outset #CCC;
  padding: .1em .5em;
}
a.boton:hover {
  background: #CCB;
}
a.boton:active {
  border: 1px inset #000;
}
<a class="boton" href="#">Guardar</a>
<a class="boton" href="#">Enviar</a>
\end{verbatim}
\end{footnotesize}

\end{frame}



%%---------------------------------------------------------------

\begin{frame}
\frametitle{Crear un menú vertical}

\begin{enumerate}
  \item Definir anchura del menú
  \item Eliminar las viñetas automáticas y todos los márgenes y espaciados aplicados por defecto
  \item Añadir un borde al menú de navegación y establecer el color de fondo y los bordes de cada elemento del menú
  \item Aplicar estilos a los enlaces: mostrarlos como un elemento de bloque para que ocupen todo el espacio de cada $<$li> del menú, añadir un espacio de relleno y modificar los colores y la decoración por defecto
\end{enumerate}


\begin{center}
\begin{figure}[p]
\includegraphics[width=0.6\textwidth]{figs/f0908.png}
\end{figure}
\end{center}

\end{frame}


%%---------------------------------------------------------------

\begin{frame}
\frametitle{Menú horizontal básico}

\begin{enumerate}
  \item Aplicar los estilos CSS básicos para establecer el estilo del menú (similares a los del menú vertical)
  \item  Establecer la anchura de los elementos del menú. Si el menú es de anchura variable y contiene cinco elementos, se asigna una anchura del 20\% a cada elemento
  \item Establecer los bordes de los elementos que forman el menú
  \item Se elimina el borde derecho del último elemento de la lista, para evitar el borde duplicado
\end{enumerate}


\begin{center}
\begin{figure}[p]
\includegraphics[width=0.8\textwidth]{figs/f0909.png}
\end{figure}
\end{center}

\end{frame}




%%---------------------------------------------------------------

%\begin{frame}
%\frametitle{}

%\begin{itemize}
%  \item 
%\end{itemize}
%
%\end{frame}


\end{document}

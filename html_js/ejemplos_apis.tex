\documentclass[ucs]{beamer}
\usetheme{GSyC}
\usepackage[spanish]{babel}   
\usepackage[utf8x]{inputenc}
\usepackage{graphicx}
\usepackage{amssymb} % Simbolos matematicos
\usepackage{lmodern,textcomp}  % para usar el carácter € tal cual
\usepackage{enumerate}


% Metadatos del PDF, por defecto en blanco, pdftitle no parece funcionar
   \hypersetup{%
     pdftitle={Ejemplos de APIs},
     %pdfsubject={Diseño y Administración de Sistemas y Redes},%
     pdfauthor={GSyC},%
     pdfkeywords={},%
   }
%


% Para colocar un logo en la esquina inferior de todas las transpas
%   \pgfdeclareimage[height=0.5cm]{gsyc-logo}{gsyc}
%   \logo{\pgfuseimage{gsyc-logo}}


% Para colocar antes de cada sección una página de recuerdo de índice
%\AtBeginSection[]{
%  \begin{frame}<beamer>{Contenidos}
%    \tableofcontents[currentframetitle]
%  \end{frame}
%}



\begin{document}

% Entre corchetes como argumento opcional un título o autor abreviado
% para los pies de transpa
\title[Ejemplos de APIs]{Ejemplos de APIs en JavaScript}
%\subtitle{Diseño y Administración de Sistemas y Redes}
\author[GSyC]{Escuela Técnica Superior de Ingeniería de Telecomunicación\\
Universidad Rey Juan Carlos}
\institute{gsyc-profes (arroba) gsyc.urjc.es}
\date[2017]{Diciembre de 2017}


%% TÍTULO
\begin{frame}
  \titlepage
  % Oportunidad para poner otro logo si se usó la opción nologo
  % \includegraphics[width=2cm]{logoesp}  
\end{frame}




%% LICENCIA DE REDISTRIBUCIÓN DE LAS TRANSPAS
%% Nota: la opción b al frame le dice que justifique el texto
%% abajo (por defecto c: centrado)
\begin{frame}[b]
\begin{flushright}
{\tiny
\copyright \insertshortdate~\insertshortauthor \\
  Algunos derechos reservados. \\
  Este trabajo se distribuye bajo la licencia \\
  Creative Commons Attribution Share-Alike 4.0
}
\end{flushright}
\end{frame}


%% ÍNDICE
%\begin{frame}
%  \frametitle{Contenidos}
%  \tableofcontents
%\end{frame}

% $Id$
%

\section{YouTube}


%%---------------------------------------------------------------
\begin{frame}[fragile]
\frametitle{YouTube}
Insertar un reproductor de vídeo de YouTube en JavaScript es muy sencillo

\begin{itemize}
\item
En el cuerpo del HTML incluimos un div con un identificador, donde irá el reproductor

\item
Cargamos el script
\url{https://www.youtube.com/iframe_api}

\item
En la función
\verb|onYouTubeIframeAPIReady()|,
creamos una instancia de YT.Player, y en sus atributos
fijamos el tamaño del reproductor y el identificador del video

\item
En la función
\verb|onPlayerReady()|,
invocamos 

\verb|event.target.playVideo()|
\end{itemize}
\end{frame}


%%---------------------------------------------------------------
\begin{frame}[fragile]
\frametitle{}

  \begin{scriptsize}
  \begin{verbatim}
    // Carga asíncrona del API del IFrame Player de YouTube
    let tag = document.createElement('script');
    tag.src = "https://www.youtube.com/iframe_api";
    let firstScriptTag = document.getElementsByTagName('script')[0];
    firstScriptTag.parentNode.insertBefore(tag, firstScriptTag);

    // Creación de un iframe, con reproductor de YouTube
    let player;
    function onYouTubeIframeAPIReady() {
      player = new YT.Player('mi_reproductor', {
        height: '768',
        width: '1024',
        videoId: 'lKCCZTUx0sI',
        events: {
          'onReady': onPlayerReady,
        }
      });
    }

    // El API llama a esta función cuando el reproductor esté listo
    function onPlayerReady(event) {
      event.target.playVideo();
    }
  \end{verbatim}
  \end{scriptsize}

\begin{tiny}
\begin{flushright}
\url{http://ortuno.es/youtube.html}
\end{flushright}
\end{tiny}
\end{frame}



%%---------------------------------------------------------------
\begin{frame}[fragile]
\frametitle{OpenStreeMap}


OpenStreetMap (OSM) es un proyecto que tiene como objetivo crear
un mapa del mundo editable y libre

\begin{itemize}
\item
Creado por Steve Coast en el Reino Unido en el año 2004

\item
Muy similar a Wikipedia, por su filosofía y funcionamiento

\item
Muy similar a Google Maps en cuanto a funcionalidad y calidad


\item
Hay mucho software y muchas APIs relacionadas con este proyecto:
captura de datos, generación de mapas, de capas, etc


\item
El uso básico consiste en integrar un mapa y puntos de interés
en un página HTML.  Una de las APIs más populares para 
esto es leaflet (\url{http://leafletjs.com})
\end{itemize}

\end{frame}



%%---------------------------------------------------------------
\begin{frame}[fragile]
\frametitle{}
La representación de mapas se basa en \emph{tiles} (azulejos, baldosas)


\begin{itemize}
\item
Un 
\emph{tile} es un \emph{bitmap} / mapa de bits con un fragmento de
mapa. Su tamaño típico es 256x256 pixeles. Aunque en ocasiones se usa
64x64 (para dispositivos móvileS) o 512x512, para dispositivos de alta
resolución

\item
Los datos de OpenStreetMap son libres (y gratuitos). El servidor
de 
\emph{tiles}
no lo es

    \begin{itemize}
    \item
Podemos contratarlo o desplegar uno propio

    \item
Para usos experimentales, con poca carga de datos, podemos usar
alguno proporcionado por la propia organización, como 
\url{tile.osm.org}
    \end{itemize}

\end{itemize}
\end{frame}




%%---------------------------------------------------------------
\begin{frame}[fragile]
\frametitle{Representación de un mapa OSM con leaflet}
Para representar un mapa OpenStreetMap en una página HTML, usando
la librería leaflet
\begin{itemize}
\item
Incluimos la librería desde el CDN


  \begin{scriptsize}
  \begin{verbatim}
 <script src="https://unpkg.com/leaflet@1.2.0/dist/leaflet.js">
  \end{verbatim}
  \end{scriptsize}

Esto crea un objeto global llamado L

\item
En nuestro HTML definimos un DIV donde irá el mapa. 
Es necesario que tenga definido 
un atributo CSS \emph{height} con la altura


  \begin{scriptsize}
  \begin{verbatim}
  <style>
    #id_mapa {
      height: 400px;
    }
  </style

...

  <div id="id_mapa"></div
  \end{verbatim}
  \end{scriptsize}
\end{itemize}
\end{frame}



%%---------------------------------------------------------------
\begin{frame}[fragile]
\frametitle{}
\begin{itemize}
\item
Creamos un objeto de tipo 
\verb|L.map()|, pasando las coordenadas del centro
del mapa deseado y el zoom inicial 

(el zoom 0 es un mapa de toda la tierra, numeros mayores van
haciendo el mapa más detallado)

\item

Invocamos el método \verb|L.tileLayer()|, 
con la URL del mapa en el servidor de tiles
y el mensaje de atribución del mapa
\end{itemize}
\end{frame}



%%---------------------------------------------------------------
\begin{frame}[fragile]
\frametitle{}

  \begin{scriptsize}
  \begin{verbatim}
    $(document).ready(function() {
          let latitud=40.417; //coordenada y
          let longitud=-3.703; //coordenada x
          let zoom=16;
          let mi_mapa = L.map('id_mapa').setView([latitud, longitud], zoom);
          L.tileLayer('http://{s}.tile.osm.org/{z}/{x}/{y}.png', {
            attribution: '&copy; <a href="http://osm.org/copyright">OpenStreetMap</a>'
          }).addTo(mi_mapa);
        });
  \end{verbatim}
  \end{scriptsize}


\begin{tiny}
\begin{flushright}
\url{http://ortuno.es/openstreet.html}
\end{flushright}
\end{tiny}
\end{frame}




%%---------------------------------------------------------------
\begin{frame}[fragile]
\frametitle{Marcadores y mensajes popup}
\begin{itemize}
\item
Podemos añadir marcadores, invicando el método
\verb|L.marker(<COORDENADAS>).addTo(<OBJETO MAP)|

\item
Podemos añadir mensajes emergentes de tipo
\emph{popup}, con el método \verb|.bindPopup| de un
marcador
\end{itemize}

  \begin{scriptsize}
  \begin{verbatim}
      let mi_marcador = L.marker(coord_labIII).addTo(mi_mapa);
      mi_marcador.bindPopup("Laboratorios III").openPopup();
  \end{verbatim}
  \end{scriptsize}

\begin{tiny}
\begin{flushright}
\url{http://ortuno.es/openstreet2.html}
\end{flushright}
\end{tiny}
\end{frame}

\end{document}

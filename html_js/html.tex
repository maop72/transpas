\documentclass[ucs]{beamer}

\usetheme{GSyC}
%\usebackgroundtemplate{\includegraphics[width=\paperwidth]{gsyc-bg.png}}


\usepackage[spanish]{babel}   
\usepackage[utf8x]{inputenc}
\usepackage{graphicx}
\usepackage{amssymb} % Simbolos matematicos
\usepackage{lmodern,textcomp}  % para usar el carácter € tal cual



% Metadatos del PDF, por defecto en blanco, pdftitle no parece funcionar
   \hypersetup{%
     pdftitle={HTML},%
     %pdfsubject={Diseño y Administración de Sistemas y Redes},%
     pdfauthor={GSyC},%
     pdfkeywords={},%
   }
%


% Para colocar un logo en la esquina inferior de todas las transpas
%   \pgfdeclareimage[height=0.5cm]{gsyc-logo}{gsyc}
%   \logo{\pgfuseimage{gsyc-logo}}


% Para colocar antes de cada sección una página de recuerdo de índice
%\AtBeginSection[]{
%  \begin{frame}<beamer>{Contenidos}
%    \tableofcontents[currentframetitle]
%  \end{frame}
%}



\begin{document}

% Entre corchetes como argumento opcional un título o autor abreviado
% para los pies de transpa
\title[HTML]{HTML }
%\subtitle{Diseño y Administración de Sistemas y Redes}
\author[GSyC]{Escuela Técnica Superior de Ingeniería de Telecomunicación\\
Universidad Rey Juan Carlos}
\institute{gsyc-profes (arroba) gsyc.urjc.es}
\date[2017]{Diciembre de 2017}


%% TÍTULO
\begin{frame}
  \titlepage
  % Oportunidad para poner otro logo si se usó la opción nologo
  % \includegraphics[width=2cm]{logoesp}  
\end{frame}



%% LICENCIA DE REDISTRIBUCIÓN DE LAS TRANSPAS
%% Nota: la opción b al frame le dice que justifique el texto
%% abajo (por defecto c: centrado)
\begin{frame}[b]
\begin{flushright}
{\tiny
\copyright \insertshortdate~\insertshortauthor \\
  Algunos derechos reservados. \\
  Este trabajo se distribuye bajo la licencia \\
  Creative Commons Attribution Share-Alike 4.0\\
}
\end{flushright}  
\end{frame}



%% ÍNDICE
%\begin{frame}
%  \frametitle{Contenidos}
%  \tableofcontents
%\end{frame}



%the practice of system and network administration
%t limoncelli

%principles of network and system administration
%m burgess

\section{Introducción a HTML}
%%---------------------------------------------------------------
\begin{frame}[fragile]
\frametitle{Introducción a HTML (1)}
\begin{itemize}
\item
HTML \emph{Hypertext Markup Language} es un lenguaje de marcado, inicialmente se usa para que navegadores web
compongan páginas con diverso contenido: texto, enlaces, imágenes,
audio, vídeo, etc. Páginas \emph{estáticas}: se editaban generalmente a mano

\item
El éxito de internet y HTML como interfaz de usuario hace que 
a finales de los años 1990 aparezcan tecnologías que permiten 
páginas web dinámicas: CGI, PHP, ASP...

\item
En 1995, JavaScript permite programar scripts en el navegador web, típicamente
para componer código HTML

\item
Muy importante: no confundir HTML con HTTP


\end{itemize}
\end{frame}

%%----------------------------------------------
\begin{frame}[fragile]
\frametitle{Introducción a HTML (2)}
\begin{itemize}
\item
%año 2005
A mediados de los años 2000 se desarrolla una nueva generación de herramientas 
para facilitar la programación de aplicaciones web:
Ruby on Rails, Django... 

\item
En esta misma época, año 2005, la aparición de AJAX permite que las aplicaciones
web comiencen a semejarse a las aplicaciones de escritorio

\item
A principios de la década de 2010, la madurez de las tecnologías web hace
que empiecen a usarse también fuera del navegador. \emph{JavaScript everywhere}.
Nodejs.
HTML como plataforma para interfaz gráfico en el escritorio 


\end{itemize}

\end{frame}


\section{Lenguajes de marcado}
%%---------------------------------------------------------------
\begin{frame}[fragile]
\frametitle{Lenguajes de marcado}
HTML es un \emph{lenguaje de marcado}: un sistema que permite incluir metainformación en un documento, esto
es, información sobre la información

\begin{itemize}
\item
La metainformación tiene que
distinguirse sintácticamente del texto.


\item
Es la evolución del \emph{lapiz azul} con el que tradicionalmente se editaban
documentos cuando la tecnología era analógica
\begin{itemize}
\item
Ejemplo de lenguaje de marcado muy elemental: redacto un documento en un procesador de textos, 
lo imprimo y alguien lo revisa, incluyendo anotaciones a mano.
Las anotaciones (metainformación) se distingue fácilmente del texto original

Para hacer algo semejante de forma digital, es necesaria una sintaxis que separe el texto
de la metainformación
\end{itemize}


\item
Ejemplos de lenguajes de marcado: troff, LaTeX, JsonML, SGML, HTML, XML


\end{itemize}

\end{frame}



%%---------------------------------------------------------------
\begin{frame}[fragile]
\frametitle{XML}
\begin{itemize}
\item
XML \emph{Extensible Markup Language}
%Resumen de XML: http://www.diveintopython3.net/xml.html

Es una forma de describir datos jerárquicamente.
Estándar para transferir información entre distintos sistemas sin tener
que adaptarlos a cada plataforma concreta, y de forma que sea fácil
de leer por un humano y fácil de procesar por un ordenador

\item
Creado en 1996 por el W3C (\emph{World Wide Web Consortium})

Dos versiones: XML 1.0 y XML 1.1
%1.0 (Fifth Edition) (November 26, 2008; )
% 1.1 (Second Edition) (August 16, 2006; )
\item
Algunos autores lo consideran un lenguaje de marcado, otros, un metalenguaje
de marcado
\item
Proviene de
SGML, \emph{Standard Generalized Markup Language}, norma ISO 8879:1986


SGML es un metalenguaje, un lenguaje para definir lenguajes de marcado,
\end{itemize}
\end{frame}




%%---------------------------------------------------------------
\begin{frame}[fragile]
\frametitle{}
\begin{itemize}
\item
XML tiene una sintaxis muy similar a la de HTML porque ambos provienen de SGML
\item
XML y HTML no son lenguajes alternativos

\begin{itemize}
\item
XML está diseñado para describir y comunicar datos datos de máquina a máquina
\item
HTML está diseñado para presentar en pantalla datos con formato. De máquina a persona 
\end{itemize}

\end{itemize}

\end{frame}


%%---------------------------------------------------------------
\begin{frame}[fragile]
\frametitle{}
En HTML, como en cualquier lenguaje de marcado, es esencial separar 
los aspectos semánticos del texto
del formato de la representación gráfica


\begin{itemize}
\item
Con la aparición de las plataformas móviles (smartphones y tablets), esto se
vuelve aún más importante
\item
Las primeras versiones de HTML no eran muy rigurosas en esto,
pero se ha ido corrigiendo gradualmente en cada nueva especificación
\end{itemize}

\end{frame}



%%---------------------------------------------------------------
\section{Versiones de HTML}
\begin{frame}[fragile]
\frametitle{Internet moderno: HTML}
\begin{itemize}
\item
HTML es la base de la World Wide Web, y por tanto, de la internet moderna

\item
Para los usuarios, WWW e internet son sinónimos. Pero nosotros debemos distinguirlo

\end{itemize}

Internet antes del web:

    \begin{itemize}
    \item
Los predecesores de internet aparecen en los años 1960
    \item
TCP/IP se desarrolla durante la década de 1970
    \begin{itemize}
    \item
1 de enero de 1983:  \emph{flag day} en que la red ARPANET
migra desde NCP hasta TCP/IP
    \end{itemize}
    \item
En la internet primitiva se usaban servicios como el correo electrónico, ftp, telnet, usenet, gopher, irc y algunos otros

    \end{itemize}

\end{frame}


%%---------------------------------------------------------------
\begin{frame}[fragile]
\frametitle{Versiones de HTML (1)}
\begin{itemize}
\item
Años 1989-1991. Tim Berners-Lee, un físico del CERN (\emph{European Organization for Nuclear Research, 
Conseil Européen pour la Recherche Nucléaire}) crea un sistema
de hipertexto para internet al que llama WWW,
\emph{World Wide Web}.



    \begin{itemize}
    \item
Para ello desarrolla el lenguaje HTML
junto con el protocolo HTTP 
    \item
Lo que en origen era un servicio más para un ámbito muy específico, 
tiene un exito arrollador que cambia no solo internet y la informática,
sino las comunicaciones humanas
    \item
El resto de servicios de internet siguen diferenciándose del WWW, pero casi
todos ellos acaban teniendo un interfaz de usuario web, lo que hace que
el usuario lo perciba como la misma cosa
    \end{itemize}


\end{itemize}
\end{frame}

%%----------------------------------------------
\begin{frame}[fragile]
\frametitle{Versiones de HTML (2)}
\begin{itemize}

\item
Año 1995. HTML 2.0. Publicado por el IETF (Internet Engineering Task Force). Añade formularios, tablas,
y soporte para internacionalización, entre otros

\item
Año 1997. HTML 3.2. 
\emph{Guerra de los navegadores}. Microsoft Internet Explorer y Netscape Navigator
intentaban prevalecer en el mercado, añadiendo características propias.
HTML 3.2 busca que ambos navegadores vuelvan a ser compatibles.

Añade características  muy desacertadas, como la etiqueta 
\emph{font}
y el atributo
\emph{color} 


\end{itemize}

\end{frame}



%%---------------------------------------------------------------
\begin{frame}[fragile]
\frametitle{Versiones de HTML (3)}
\begin{itemize}
\item
Versión 4.0.
Año 1997 

Normaliza el uso de marcos \emph{frames}, disponible desde Netscape Navigator 2 (año 1995)

Los marcos eran documentos HTML dentro de documentos HTML. Concepto problemático, han ido
desapareciendo

Tres variantes de HTML 4.0
    \begin{itemize}
    \item
\emph{Strict}, donde se prohiben elementos obsoletos
    \item
\emph{Transitional}, que admite elementos obsoletos
    \item
\emph{Frameset}, solo marcos
    \end{itemize}

\item
Versión 4.1.
Año 2000 

La versión más usada, hasta la aparición de HTML 5

\end{itemize}
\end{frame}

%%----------------------------------------------
\begin{frame}[fragile]
\frametitle{Versiones de HTML (4)}
\begin{itemize}

\item
Año 2004. El W3C decide abandonar HTML y migrar a XHTML 

    \begin{itemize}
    \item
XHTML: Extensible Hypertext Markup Language. Lenguaje muy similar
a HTML, pero que sigue estrictamente la sintaxis de XML
    \end{itemize}


El desarrollo de HTML lo retoma el WHATWG
(Web Hypertext Application Technology Working Group: Google Apple, Mozilla, Opera)

\item
Año 2008. Primer borrador de HTML 5 publicado por WHATWG

\item
Año 2009. El W3C abandona XHTML y vuelve a HTML5

\end{itemize}

\end{frame}


%%---------------------------------------------------------------
\begin{frame}[fragile]
\frametitle{Versiones de HTML (5)}
\begin{itemize}
\item
Año 2014. HTML 5.0

    \begin{itemize}
    \item
Define de forma precisa qué hacer con páginas incorrectas
    \item
Muchas mejoras en interoperabilidad
    \item
Mucho mejor soporte para dispositivos móviles
    \item
Audio y video
    \item
Gráficos vectoriales
    \item
Muchas APIs nuevas, como la geolocalización

    \end{itemize}
\item
Año 2016. HTML 5.1
\end{itemize}

En la actualidad cualquier desarrollo debería centrarse en HTML 5

    \begin{itemize}
    \item
La compatibilidad con HTML 4 es bastante buena
    \item
Existen técnicas y herramientas que permiten que páginas HTML 5 se
muestren correctamente en navegadores antiguos
    \end{itemize}
Toda la información contenida en las transparencias de esta asignatura es válida en HTML 4 y HTML 5, salvo
indicación contraria

\end{frame}



%%---------------------------------------------------------------
\begin{frame}[fragile]
\frametitle{Adobe Flash}
Otra de las grandes ventajas de HTML 5 es que permite prescindir de Flash
\begin{itemize}
\item
Adobe Flash es una plataforma software desarrollada por Adobe Systems
para mostrar animaciones, gráficos vectoriales, vídeos, audio, contenido
interactivo...

\item
Muy popular entre los años 2000 y 2010

\item
Muy problemático. 
Ya en el año 2000 se publican artículos como
\emph{Flash: 99\% Bad}, (J.Nielsen)


No estándar. Dependencia del fabricante.
Anima a desarrollar contenido centrado en la apariencia gráfica
externa, no en la usabilidad y la semántica 

\item
No soportado por Apple
\end{itemize}
Anunciada su desaparición oficial para el año 2020
\end{frame}



\section{Sintaxis de HTML}
%%---------------------------------------------------------------
\begin{frame}[fragile]
\frametitle{Sintaxis de HTML}
Un documento HTML está formado por

    \begin{itemize}
    \item
Declaración de tipo
    \item
Un elemento \emph{html}, que contiene

    \begin{itemize}
    \item
Un elemento \emph{head}, que contiene

    \begin{itemize}
    \item
Título
    \item
Codificación de caracteres
    \end{itemize}
    \item
Un elemento \emph{body}
    \end{itemize}
    \end{itemize}




  \begin{footnotesize}
  \begin{verbatim}
<!DOCTYPE html>
<html>
  <head>
    <title>Hola mundo en HTML</title>
    <meta charset="utf-8">
  </head>
  <body>
    Hola, mundo.
  </body>
</html>
  \end{verbatim}
  \end{footnotesize}


\end{frame}




%%---------------------------------------------------------------
\begin{frame}[fragile]
\frametitle{}
\begin{itemize}
\item
Algunas etiquetas de estos elementos se pueden omitir en ciertas circunstancias y el 
documento sigue siendo válido, quedan
sobreentendidas (head, body, html). Pero siempre es preferible ponerlo todo

\item
Otros elementos como el título y la codificación de caracteres son obligatorios.
Si se omiten, herramientas como \verb|https://validator.w3.org| nos indicarán
que el documento es erróneo

    \begin{itemize}
    \item
A pesar de eso, los navegadores podrán mostrar el documento de forma satisfactoria,
con lo que es muy habitual que nadie se preocupe de corregir estos errores
    \item
El problema se agrava cuando distintos navegadores tratan los errores de forma distinta
    \end{itemize}


\end{itemize}

\end{frame}








%%---------------------------------------------------------------
\begin{frame}[fragile]
\frametitle{DOCTYPE}
La declaración 
\verb|<!DOCTYPE html>|
es obligatoria al comienzo de un documento HTML 5
% debe haber 1 y solo 1

\begin{itemize}
\item
En rigor no es parte de HTML (no es un elemento HTML) sino una instrucción
que le dice al navegador que lo que viene a continuación es un documento HTML, versión 5
\end{itemize}

En HTML 4.x y anteriores, esto era más complicado

Ejemplos:
    \begin{itemize}
    \item

  \begin{footnotesize}
  \begin{verbatim}
<!DOCTYPE HTML PUBLIC "-//W3C//DTD HTML 4.01 Transitional//EN" 
"http://www.w3.org/TR/html4/loose.dtd">
  \end{verbatim}
  \end{footnotesize}
    \item

  \begin{footnotesize}
  \begin{verbatim}
<!DOCTYPE html PUBLIC "-//W3C//DTD XHTML 1.0 Strict//EN" 
"http://www.w3.org/TR/xhtml1/DTD/xhtml1-strict.dtd">
  \end{verbatim}
  \end{footnotesize}
    \item

  \begin{footnotesize}
  \begin{verbatim}
<!DOCTYPE HTML PUBLIC "-//W3C//DTD HTML 4.01 Transitional//EN" 
"http://www.w3.org/TR/html4/loose.dtd">
  \end{verbatim}
  \end{footnotesize}
    \end{itemize}


\end{frame}



%%---------------------------------------------------------------
\begin{frame}[fragile]
\frametitle{Case insensitive}
HTML es insensible a mayúsculas (\emph{case insensitive}), aunque lo habitual es usar siempre las
minúsculas.

Hay una excepción:

  \begin{footnotesize}
  \begin{verbatim}
<!DOCTYPE html>
<!doctype html>
  \end{verbatim}
  \end{footnotesize}

Ambas formas son idénticas y correctas, pero lo habitual es usar la primera, posiblemente 
por influencia de XHTML (que es sensible a mayúsculas, y donde la única
forma válida es la primera)

\end{frame}





%%---------------------------------------------------------------
\begin{frame}[fragile]
\frametitle{Elementos HTML}
Un documento HTML está compuesto por elementos (\emph{elements})

Un elemento puede ir
    \begin{itemize}
\item
A continuación de otro elemento
\item
Dentro de otro elemento. Esto es muy habitual, los documentos típicos tienen muchos niveles anidados
    \end{itemize}

La mayoría de los elementos están formados por:
    \begin{itemize}
\item
 Etiqueta de apertura
(\emph{start tag})
\item
Contenido
\item
 Etiqueta de cierre 
(\emph{end tag})
    \end{itemize}

Ejemplo:

\verb|<h1>Este elemento es un título de nivel 1</h1>|

\end{frame}


%%---------------------------------------------------------------
\begin{frame}[fragile]
\frametitle{}

Una etiqueta de apertura sencilla está formada por:
\begin{itemize}
\item
Signo de menor
\item
Nombre de la etiqueta
\item
Signo de mayor
\end{itemize}


Ejemplo:

\verb|<h1>|

Una etiqueta de cierre está formada por:

\begin{itemize}
\item
Signo de menor
\item
Barra (\emph{slash})
\item
Nombre de la etiqueta
\item
Signo de mayor
\end{itemize}


Ejemplo:

\verb|</h1>|

\end{frame}




%%---------------------------------------------------------------
\begin{frame}[fragile]
\frametitle{}

Lo habitual y recomendable es no poner ningún espacio ni después del
\verb|<|
ni antes del
\verb|>|


    \begin{itemize}
    \item
Un espacio después del \verb|<| es un error

Ejemplo :

\verb|< h1>|
Esto es un ERROR
    \item
Un espacio antes del \verb|>| es legal, pero no es recomendable

Ejemplo :

\verb|<h1 >|

    \end{itemize}


\end{frame}

%%---------------------------------------------------------------
\begin{frame}[fragile]
\frametitle{}
\begin{itemize}
\item
En la mayoría de elementos la etiqueta de cierre es obligatoria 

Ejemplo:

  \begin{footnotesize}
  \begin{verbatim}
<pre>Texto preformateado, con fuente de ancho fijo. Se mantienen 
los saltos de línea y los espacios    consecutivos</pre>
  \end{verbatim}
  \end{footnotesize}
\item
En algunos elementos se puede omitir la etiqueta de cierre. Aunque no es recomendable. Ejemplo :


  \begin{footnotesize}
  \begin{verbatim}
<p>Esto es un párrafo correcto</p>
<p>Esto también es un párrafo correcto
  \end{verbatim}
  \end{footnotesize}


\item
En algunos elementos, los de tipo \emph{void}, no puede haber ni contenido ni etiqueta de cierre. Solo pueden tener, opcionalmente, atributos.
Ejemplo :


  \begin{footnotesize}
  \begin{verbatim}
<br></br>
  \end{verbatim}
  \end{footnotesize}
Esto es un ERROR

\end{itemize}

\end{frame}


%%---------------------------------------------------------------
\begin{frame}[fragile]
\frametitle{}



Elementos de tipo void muy habituales son: \emph{br, hr, meta, link, img, input}

\begin{itemize}
\item
En HTML 4.01 también son de tipo void:  \emph{area, base, col, param}
\item
HTML 5 añade: \emph{ keygen,  source }
\end{itemize}


% br  breaking line.  hr: horizontal rule
% link: no es un enlace de hipertesto (eso es a,anchor) sino un enlace en el head a una plantilla css
% area: mapas clicables. base: especifica la base para los enlaces relativos
% col: define las columnas de un grupo de columnas. input: para hacer formularios
% param: para pasar parámetros a objetos incustrador. P.e. el autoplay de un reproductor de audio
% keygen se usar para cifrar el contenido de los formularios. i
% source se usa para indicar dos orígenes
% del mismo documento multimedia, y que el navegador elija. p.e. ogg/mp3



% command estaba en html5 inicialmente, pero es obsoleto
%http://www.456bereastreet.com/archive/201005/void_empty_elements_and_self-closing_start_tags_in_html/

\end{frame}



%%---------------------------------------------------------------
\begin{frame}[fragile]
\frametitle{¿Etiqueta h1 o elemento h1?}

Como hemos visto, HTML está formado por elementos, que siempre tienen etiqueta
de apertura y que pueden tener contenido y etiqueta de cierre

Consideremos por ejemplo
\verb|<h1>Introducción</h1>|

\begin{itemize}
\item
En rigor deberíamos decir \emph{el elemento h1}, no \emph{la etiqueta h1}
\item
Pero en muchos contextos es habitual y por tanto aceptable hablar de \emph{la etiqueta h1},
se entiende que es una sinécdoque, nos estamos refiriendo a
\emph{el elemento que empieza por la etiqueta h1}
% sinécdoque, una metonimia (usar una cosa por otra) donde se toma la parte por el todo o el todo por la parte)
\end{itemize}

\end{frame}




%%---------------------------------------------------------------
\begin{frame}[fragile]
\frametitle{Etiquetas autocerradas}
En XML, cuando un elemento no tiene texto, hay dos alternativas posibles
\begin{itemize}
\item
Usar una etiqueta de cierre y otra de apertura

  \begin{footnotesize}
  \begin{verbatim}
<holamundo></holamundo>
  \end{verbatim}
  \end{footnotesize}
\item
Usar una etiqueta \emph{autocerrada}
  \begin{footnotesize}
  \begin{verbatim}
<holamundo/>
  \end{verbatim}
  \end{footnotesize}Signo de menor, nombre, barra, signo de mayor
\end{itemize}


Por influencia de XML, algunos desarrolladores o herramientas de HTML usan las etiquetas
autocerradas

    \begin{itemize}
    \item
En los elementos void son válidas. Pero no aportan nada, no tienen significado especial
    \item
En otros elementos son incorrectas. Aunque el navegador suele ignorarlos y mostrar la página igualmente
    \end{itemize}

Conclusión: no debemos usar etiquetas autocerradas
% https://stackoverflow.com/questions/3558119/are-non-void-self-closing-tags-valid-in-html5


\end{frame}

%%---------------------------------------------------------------
\begin{frame}[fragile,]
\frametitle{Comentarios}

Los comentarios son iguales que en XML. Se pueden poner en cualquier lugar del documento
  \begin{footnotesize}
  \begin{verbatim}
<!-- Esto es un comentario -->
  \end{verbatim}
  \end{footnotesize}

\end{frame}


%%---------------------------------------------------------------
\begin{frame}[fragile]
\frametitle{Atributos}
Dentro de la etiqueta de apertura puede haber uno o más
\emph{atributos},
que son modificadores del elemento

Ejemplo:

  \begin{footnotesize}
  \begin{verbatim}
<html lang="es-ES">
Esto es texto en español de España
</html>\end{verbatim}
  \end{footnotesize}


    \begin{itemize}
    \item
El atributo
\emph{lang} indica el idioma del texto, especificado en ISO 639-1
    \item
Su sintaxis es similar pero no idéntica a la variable LANG de Unix, donde se indicaría \verb|es_ES.UTF-8|

    \end{itemize}


\end{frame}

%%----------------------------------------------
\begin{frame}[fragile]

\begin{itemize}
\item
Un atributo es un par formado por un nombre y un valor. Su sintaxis es
    \begin{itemize}
    \item
Nombre del atributo
    \item
Signo igual
    \item
Valor del atributo 


    \begin{itemize}
    \item
Siempre es de tipo texto
    \item

Es recomendable meterlo siempre entre comilla dobles, aunque si el atributo no contiene espacios, se pueden omitir
    \end{itemize}
    \end{itemize}




\item
Si hay varios atributos, se separan por espacios

\verb|<img src="urjc.gif" alt="Logo de la URJC">|

\item
El nombre del atributo no se puede repetir dentro del mismo elemento. Sí puede aparece
el mismo nombre de atributo en un elemento distinto
\item
Los atributos no están ordenados, no hay garantía de que se mantenga
el orden
\end{itemize}




\end{frame}


\section{Cabecera del documento}
%%---------------------------------------------------------------
\begin{frame}[fragile]
\frametitle{Elemento head}

El elemento head es la cabecera del documento.

Es un contenedor para metadatos del documento HTML. Esta información nunca se muestra directamente.
Sus elementos son
  \begin{verbatim}
<title>, <base>, <style>, <link>, <meta>, <script>
  \end{verbatim}

%Metadatos principales: título, codificación de caracteres, hojas de estilo CSS y  código JavaScript


\begin{itemize}
\item
El elemento \verb|<title>| define el título del documento. Es de inclusión obligatoria
\item
El elemento \verb|<base>| define la base de las direcciones relativas
\item
Los elemento 
\verb|<style>| 
y
\verb|<link>| 
especifican las hojas de estilo CSS
\item
El elemento \verb|<meta>| se usa para añadir diversa metainformación
\item
El elemento \verb|<script>| contiene código JavaScript, o un enlace a una página con el código

\end{itemize}
\end{frame}

%%----------------------------------------------
\begin{frame}[fragile]
\frametitle{CSS}


CSS
(\emph{Cascading Style Sheets} 
es un lenguaje de diseño gráfico para crear hojas de estilo,
%Las hojas de estilo CSS (\emph{Cascading Style Sheets}
que son son una sucesión de reglas que especifican
el formato gráfico de un documento

Las hojas CSS pueden ubicarse

    \begin{itemize}
    \item
En el propio documento HTML, dentro del elemento 
\verb|<style>|
    \item
En un documento distinto, especificando con el elemento
\verb|<link>|

Ejemplo:
\verb|<link href="css/bootstrap.min.css" rel="stylesheet">|

    \begin{itemize}
\item
Es de tipo
\emph{void}

    \item
Puede aparecer varias veces, pero solo en la sección
\emph{head}
nunca en
\emph{body}


    \item
No confundir con los enlaces a otros documentos HTML dentro del cuerpo del documento, que se 
indican con \verb|<a>|
    \end{itemize}

    \end{itemize}


\end{frame}




%%---------------------------------------------------------------
\begin{frame}[fragile]
\frametitle{Elemento meta}

Contiene diversos atributos con metainformación

    \begin{itemize}
    \item
El único obligatorio es charset
    \end{itemize}

  \begin{footnotesize}
  \begin{verbatim}
<head>
  <meta charset="UTF-8">
  <meta name="description" content="Tutorial sobre tecnologías web">
  <meta name="keywords" content="HTML,CSS,Bootstrap, JavaScript">
  <meta name="author" content="Juan García">
</head>
  \end{verbatim}
  \end{footnotesize}



\end{frame}




%%---------------------------------------------------------------
\begin{frame}[fragile]
\frametitle{Codificación de caracteres}
En HTML antiguo lo habitual era emplear la codificación ISO-8859, también llamada latin1. O más bien
windows-1252, que es muy similar
\begin{itemize}
\item
En la actualidad la recomendación es usar UTF-8
    \item
UTF-8 es una forma de codificar unicode, la más habitual pero no la única
\end{itemize}



\end{frame}

%%----------------------------------------------
\begin{frame}[fragile]



La sintaxis de HTML 4 era muy farragosa

  \begin{scriptsize}
  \begin{verbatim}
<head>
<META http-equiv="Content-Type" content="text/html; charset=ISO-8859-5">
...
</head>

<head>
<meta http-equiv="Content-Type" content="text/html; charset=utf-8">
...
</head>
  \end{verbatim}
  \end{scriptsize}


En HTML 5 :

  \begin{footnotesize}
  \begin{verbatim}
<meta charset="UTF-8">
  \end{verbatim}
  \end{footnotesize}



\end{frame}

%%----------------------------------------------
\begin{frame}[fragile]

Problema:
Hay varios lugares donde indicar la codificación


\begin{itemize}
\item
Desde HTTP
(tal y como lo haya configurado el administrador del servidor web)

\item
Dentro del HTML
(tal y como lo configure el autor de la página)
\end{itemize}

Ambas informaciones pueden ser discrepantes.
El convenio es dar precedencia a HTTP.
Problema: un servidor que tenga páginas HTML 4 y HTML 5 mezcladas


    \begin{itemize}
    \item
Se puede configurar el servidor para usar una codificación distinta para
cada fichero o directorio, pero el autor y el administrador suelen
ser personas distintas, no siempre bien coordinadas
    \end{itemize}



\end{frame}



\section{Cuerpo del documento}
%%---------------------------------------------------------------


%%---------------------------------------------------------------
\begin{frame}[fragile]
\frametitle{Cuerpo del documento}
El elemento body contiene el cuerpo del documento, su contenido principal


\begin{itemize}
\item
El contenido del cuerpo incluye
parrafos de texto, hiperenlaces (también llamados hipervínculos y enlaces), imagénes, tablas, listas, etc

\item
Solo puede haber un elemento body

\item
En HTML 4 tenía atributos como 
\emph{background},
\emph{bgcolor},
\emph{link}
o
\emph{text} para modificar los colores, pero no se admiten en HTML 5
\end{itemize}

\end{frame}



\begin{frame}[fragile]
\frametitle{p}

Párrafo: secuencia de oraciones con unidad temática. Acaba en punto y aparte

El elemento
\verb|<p>| crea un párrafo



Entre un párrafo y otro hay

    \begin{itemize}
    \item
Un salto de línea
    \item
Una separación adicional
    \end{itemize}


La etiqueta de cierre \verb|</p>| es opcional, si no aparece, 
se considera implícita antes de la apertura
del siguiente elemento.
Pero es preferible cerrar explícitamente con la etiqueta de cierre


\end{frame}



%%---------------------------------------------------------------
\begin{frame}[fragile]
\frametitle{br}
La etiqueta \verb|<br>| define el elemento \emph{breacking line}

\begin{itemize}
\item
En un documento HTML, los saltos de línea se ignoran
\item
Si queremos forzar un salto de línea (sin definir un párrafo), usamos \verb|<br>|
\item
En cuanto al formato, es similar a un nuevo párrafo, pero solo con el salto de línea,  sin la separación adicional
\item
Es de tipo \emph{void} (no puede tener ni contenido ni etiqueta de cierre)

\end{itemize}

\end{frame}


%%---------------------------------------------------------------
\begin{frame}[fragile]
\frametitle{em}
La etiqueta \verb|<em>| define el elemento \emph{emphasized}

\begin{itemize}
\item
Típicamente el navegador mostrará el texto en cursiva
\end{itemize}


No confundir con la unidad
\emph{em},
que en español traduciríamos como
\emph{eme},
esto es, la letra eme

    \begin{itemize}
    \item
 Es una unidad clásica de tipografía, para referirse al ancho de la M mayúscula. \verb|1em, 1.2em|, etc
    \item
En la actualidad, las M mayúsculas suelen ser un poco más
estrechas que un em.  Así que em en la práctica significa \emph{letra de tamaño normal}
    \end{itemize}



\end{frame}

%<strong>	Defines important text
%code>	Defines a piece of computer code
%samp>	Defines sample output from a computer program
%kbd>	Defines keyboard input
%var>	Defines a variable





%%---------------------------------------------------------------
\begin{frame}[fragile]
\frametitle{pre}
La etiqueta \verb|<pre>| define el elemento \emph{texto preformateado}
\begin{itemize}
\item
El navegador muestra el texto respetando los espacios entre palabras y los saltos
de línea

\item
Normalmente se usa una fuente de ancho fijo, típicamente
courier

\end{itemize}

\end{frame}


%%---------------------------------------------------------------
\begin{frame}[fragile]
\frametitle{h1-h6}
Las etiquetas 
\verb|<h1>|, 
\verb|<h2>|, 
...
\verb|<h6>|, 
definen elementos 
\emph{heading}
(encabezado)
\begin{itemize}
\item
Están organizados jerárquicamente, siendo h1 el más importante y h6 el menos
\end{itemize}

\end{frame}



%%---------------------------------------------------------------
\begin{frame}[fragile]
\frametitle{a}
La etiqueta \verb|<a>| define el elemento \emph{anchor} (ancla), que sirve
para hacer anclas y también para hacer hiperenlaces

Su nombre es poco afortunado, no describe lo que hace

    \begin{itemize}
    \item
La invención del hipertexto se atribuye a 
D.Engelbart y 
T.Nelson, por separado, en 1962/1963

    \item
En el hipertexto original solo había \emph{anchors}: enlaces al documentos dentro
de la misma sede

    \item
Tim Berners-Lee inventa los enlaces a documentos en otros sitios (la WWW). Pero en el lenguaje
HTML mantiene el nombre \emph{anchor}, y por tanto, usa la etiqueta
\verb|<a>| 

    \item
En la actualidad, un \emph{anchor} es un enlace desde una parte de un documento a otra
parte del mismo documento. Pero el elemento
\verb|<a>| 
se usa para hiperenlaces
    \end{itemize}

\end{frame}



%%---------------------------------------------------------------
\begin{frame}[fragile]
\frametitle{}
\begin{itemize}
\item
Normalmente el elemento 
\verb|<a>| 
se usar para hacer hiperenlaces, con el atributo
\emph{href (hypertext reference)}


  \begin{footnotesize}
  \begin{verbatim}
<a href="https://www.urjc.es">Página de la URJC</a>
  \end{verbatim}
  \end{footnotesize}

\item

También se pueden enlazar un correo electrónico

  \begin{scriptsize}
  \begin{verbatim}
<a href="mailto:jperez@miempresa.com?Subject=Contacto%20web">
    Envíame un correo</a>
  \end{verbatim}
  \end{scriptsize}


(Aunque hoy esto no es recomendable porque la dirección de correo queda expuesta a los spammers)

\item
O enlazar un script

  \begin{scriptsize}
  \begin{verbatim}
<a href="javascript:alert('¡Hola, mundo!');">Holamundo en JavaScript</a>
  \end{verbatim}
  \end{scriptsize}

\end{itemize}
\end{frame}


%%---------------------------------------------------------------
\begin{frame}[fragile]
\frametitle{}
Los enlaces pueden ser 


\begin{itemize}
\item
Absolutos a una dirección
  \begin{footnotesize}
  \begin{verbatim}
<a href="http://linkedsite/url.html">Documento</a>
  \end{verbatim}
  \end{footnotesize}

\item
Absolutos dentro del mismo sitio 
  \begin{footnotesize}
  \begin{verbatim}
<a href="/url.html">Documento en mismo sitio web</a>
  \end{verbatim}
  \end{footnotesize}

\item
Relativos 
  \begin{footnotesize}
  \begin{verbatim}
<a href="url.html">Documento en mismo sitio web y "dir"</a>
  \end{verbatim}
  \end{footnotesize}
\end{itemize}

\end{frame}




%%---------------------------------------------------------------
\begin{frame}[fragile]
\frametitle{anchors}
Un anchor es una referencia a un punto concreto dentro de un documento
\begin{itemize}
\item
Un anchor tiene un nombre, con las mismas reglas de cualquier otro atributo

\item
Para usar un anchor, a su nombre se le antepone la almohadilla

\item
Ejemplo de dirección con anchor:
  \begin{scriptsize}
  \begin{verbatim}
https://gsyc.urjc.es/~mortuno/index_lagrs/index_lagrs.html#transpas
  \end{verbatim}
  \end{scriptsize}

\item
Ejemplo de hiperenlace con anchor absoluto
  \begin{scriptsize}
  \begin{verbatim}
<a href="https://www.urjc.es/universidad/organizacion#rector>rector</a>
  \end{verbatim}
  \end{scriptsize}


\item
Ejemplo de hiperenlace con anchor relativo
  \begin{scriptsize}
  \begin{verbatim}
<a href="#inicio>inicio</a>
  \end{verbatim}
  \end{scriptsize}


\end{itemize}

\end{frame}


%%---------------------------------------------------------------
\begin{frame}[fragile]
\frametitle{Anchor al estilo HTML 4}
En HTML 4, un anchor se creaba definiendo un elemento 
\verb|<a>|

\begin{itemize}
\item
Sin atributo href

\item
Con atributo name
\end{itemize}

Ejemplo

  \begin{footnotesize}
  \begin{verbatim}
   <a name="punto07">Esto es el punto 7</a>
 
   [.....]

   <a href="#punto07">Volver al punto 7</a>
  \end{verbatim}
  \end{footnotesize}

En HTML 5, esto sigue funcionando en los navegadores, pero
es obsoleto, no deberíamos usarlo
% en el w3c validator da un warning

\end{frame}

%%---------------------------------------------------------------
\begin{frame}[fragile]
\frametitle{Anchor al estilo HTML 5}
En HTML 5, un anchor se crea añadiendo el atributo
\emph{id}
a cualquier elemento contenedor
\emph{p, h1, div, pre...}

  \begin{footnotesize}
  \begin{verbatim}
    <h1 id="punto07">Punto 7</h1>

    [.....]

    <a href="#punto07">Volver al punto 7</a>
  \end{verbatim}
  \end{footnotesize}


    \begin{itemize}
    \item
Naturalmente, estos anchors pueden emplearse tanto de forma
relativa como de forma absoluta

    \item
Esta forma de crear anchors también solía funcionar en los navegadores HTML 4, 
aunque no era la más común 
    \end{itemize}


\end{frame}




%%---------------------------------------------------------------
\begin{frame}[fragile]
\frametitle{Atributo target}
Añadiendo a un enlance el atributo \verb|target="_blank"|, este
se abrirá en una nueva pestaña del navegador

  \begin{footnotesize}
  \begin{verbatim}
<a href="http://urjc.es" target="_blank">Abrir en nueva pestaña</a>
  \end{verbatim}
  \end{footnotesize}

\begin{itemize}
\item
En HTML 4 el atributo target tenía otros posibles valores relacionados
con el uso de marcos 
(
\emph{\_self, \_parent, \_top, framename}
),
pero en HTML 5 desaparecen los marcos
\end{itemize}

\end{frame}



%%---------------------------------------------------------------
\begin{frame}[fragile]
\frametitle{div}

La etiqueta \verb|<div>| define una división o sección dentro del documento
\begin{itemize}
\item
Su uso habitual es el de contenedor genérico: delimita un bloque de texto, al que luego
se le dará formato mediante reglas CSS
\end{itemize}

Ejemplo

  \begin{footnotesize}
  \begin{verbatim}
<div class="respuesta">Todas son falsas</div>
  \end{verbatim}
  \end{footnotesize}

En HTML 5, además de este elemento se definen otros con el mismo propósito, pero con una semántica
más específica

    \begin{itemize}
    \item
  \begin{footnotesize}
  \begin{verbatim}
<section>, <nav>, <article>, <aside>, <hgroup>, 
<header>, <footer>, <time>, <mark>
  \end{verbatim}
  \end{footnotesize}
    \end{itemize}

%http://diveintohtml5.info/semantics.html

\end{frame}


%%---------------------------------------------------------------
\begin{frame}[fragile]
\frametitle{span}
La etiqueta 
\verb|<span>|  (espacio, longitud, lapso)
define una división o sección dentro del documento
\begin{itemize}
\item
Muy similar a div, pero no crea un bloque nuevo y por tanto,
no crea una nueva línea

\item
Se usa típicamente para dar formato a un grupo de palabras
\end{itemize}

\end{frame}




%%---------------------------------------------------------------
\begin{frame}[fragile]
\frametitle{table, th, tr, td}
Las etiquetas 
\verb|<table>| (tabla),
\verb|<th>| (table header),
\verb|<tr>| (table row),
\verb|<td>| (table data)
permiten crear tablas

  \begin{footnotesize}
  \begin{verbatim}
<table>
  <tr>
    <th>Cabecera, primera columna</th>
    <th>Cabecera, segunda columna</th>
  </tr>
  <tr>
    <td>Primera fila, primera columna</td>
    <td>Primera fila, segunda columna</td>
  </tr>
  <tr>
    <td>Segunda fila, primera columna</td>
    <td>Segunda fila, segunda columna</td>
  </tr>
</table
  \end{verbatim}
  \end{footnotesize}


\end{frame}


%%---------------------------------------------------------------
\begin{frame}[fragile]
\frametitle{ol,li}
Las etiquetas 
\verb|<ol>| (ordered list),
\verb|<li>| (list item),
permiten crear listas numeradas 

  \begin{footnotesize}
  \begin{verbatim}
<ol>
  <li>Sota</li>
  <li>Caballo</li>
  <li>Rey</li>
</ol>
  \end{verbatim}
  \end{footnotesize}


    \begin{itemize}
    \item
Las listas numeradas usan, por omisión, números naturales para cada item

    \item

Añadiendo
el atributo \emph{type} al elemento \emph{ol} se puede cambiar el tipo de marcador

Los valores posibles son \verb|1, A, a, I, i| para emplear números, letras mayúsculas,
minúsculas, números romanos en mayúsculas y en minúsculas, respectivamente
    \end{itemize}

\end{frame}




%%---------------------------------------------------------------
\begin{frame}[fragile]
\frametitle{ul,li}
Las etiquetas 
\verb|<ul>| (unordered list),
\verb|<li>| (list item),
permiten crear listas sin numerar

  \begin{footnotesize}
  \begin{verbatim}
<ul>
  <li>Sota</li>
  <li>Caballo</li>
  <li>Rey</li>
</ul
  \end{verbatim}
  \end{footnotesize}

\end{frame}

%%---------------------------------------------------------------
\begin{frame}[fragile]
\frametitle{dl,dt,dd}
Las etiquetas 
\verb|<dl>| (description list),
\verb|<dt>| (description term),
\verb|<dd>| (description),
permiten crear listas de descripciones o definiciones de términos

  \begin{footnotesize}
  \begin{verbatim}
<dl>
   <dt>
      Nombre
   </dt>
   <dd>
      Juan García
   </dd>
   <dt>
      Centro de origen
   </dt>
   <dd>
      ESTIT-URJC
   </dd>
</dl>
  \end{verbatim}
  \end{footnotesize}

\end{frame}

%%---------------------------------------------------------------
\begin{frame}[fragile]
\frametitle{img}
La etiqueta 
\verb|<img>| 
permite insertar imágenes

Ejemplos
  \begin{footnotesize}
  \begin{verbatim}
<img src="urjc.png" alt="Logo de la URJC">
<img src="urjc.png" alt="Logo de la URJC" width="300" height="240">
  \end{verbatim}
  \end{footnotesize}

Este elemento tiene dos atributos obligatorios
\begin{itemize}
\item
\emph{src}, que
especifica el origen del fichero
\item
\emph{alt}, que
indica una descripción en texto del contenido de la imagen, para
los navegadores sin gráficos
\end{itemize}

\end{frame}



%%---------------------------------------------------------------
\begin{frame}[fragile]
\frametitle{}
Es muy habitual incluir una imagen dentro de un elemento 
\verb|<a>|, de esa forma la imagen se convierte en un hipervínculo

  \begin{footnotesize}
  \begin{verbatim}
<a href="https://www.urjc.es">
   <img src="images/urjc.png" width="120" alt="logo de la URJC">
</a>
  \end{verbatim}
  \end{footnotesize}


\end{frame}


\subsection{Formularios}
%%---------------------------------------------------------------
\begin{frame}[fragile]
\frametitle{form}

Un formulario HTML es un elemento que permite aceptar entrada de información por parte
del usuario.

  \begin{scriptsize}
  \begin{verbatim}
<form>
    (Elementos del formulario)
</form>
  \end{verbatim}
  \end{scriptsize}


Los elementos posibles son varios  
    \begin{itemize}
    \item
El principal es el elemento
\verb|<input>|
que puede de ser de diferentes tipos (
\verb|text, password, radio, checkbox|) entre otros

    \item
Otros elementos son

\verb|<fieldset>|,
\verb|<select>|,
\verb|<textarea>| y
\verb|<button>|

    \item
HTML5 añade los elementos
\verb|<datalist>| y
\verb|<output>|

    \end{itemize}

Ejemplos de formularios:

\url{http://ortuno.es/form}

\end{frame}



%%---------------------------------------------------------------
\begin{frame}[fragile]
\frametitle{input: text, password, submit}
\verb|input| es un elemento HTML de tipo void
\begin{itemize}
\item
El atributo name indica el nombre de campo

\item
Con el atributo value se pueden asignar valores por omisión
\end{itemize}

  \begin{scriptsize}
  \begin{verbatim}
<form action="/action_page.html">
  Nombre de usuario:<br>
  <input type="text" name="usuario" ><br>
  Contraseña:<br>
  <input type="password" name="contrasenya" ><br><br>
  País:<br>
  <input type="text" name="pais" value="España" ><br><br>

  <input type="submit">
</form>
  \end{verbatim}
  \end{scriptsize}
\end{frame}



%%---------------------------------------------------------------
\begin{frame}[fragile]
\frametitle{input: radio}

\verb|<input type="radio">|
define un \emph{radio button}, que permite
elegir una (y solo una) opción entre varias


  \begin{scriptsize}
  \begin{verbatim}
<form>
  <input type="radio" name="os" value="Linux" checked>Linux<br>
  <input type="radio" name="os" value="MacOS" >MacOS<br>
  <input type="radio" name="os" value="Windows">Windows<br>
  <input type="radio" name="os" value="other">Otro<br>
</form>
  \end{verbatim}
  \end{scriptsize}

\end{frame}




%%---------------------------------------------------------------
\begin{frame}[fragile]
\frametitle{input: checkbox}


Checkbox es un tipo de input que permite elegir 0 o más opciones de
una lista

  \begin{scriptsize}
  \begin{verbatim}
<form>
  <input type="checkbox" name="terminos" value="si">
        He leido los términos y condiciones<br>
  <input type="checkbox" name="publicidad" value="si">
        Deseo recibir comunicaciones comerciales<br>
</form>
  \end{verbatim}
  \end{scriptsize}

\end{frame}



%%---------------------------------------------------------------
\begin{frame}[fragile]
\frametitle{input: tipos de HTML5}
HTML 5 añade nuevos tipos de input
\begin{itemize}
\item
color (no soportado por todos los navegadores)
\item
date (no soportado por todos los navegadores)
\item
datetime-local (no soportado por todos los navegadores)
\item
email
\item
month
\item
number
\item
range
\item
search
\item
tel
\item
time
\item
url
\item
week
\end{itemize}

\end{frame}



%%---------------------------------------------------------------
\begin{frame}[fragile]
\frametitle{fieldset}
Un conjunto de entradas se pueden agrupar en un
\verb|<fieldset>|,
con un título indicado en un elemento
\verb|<legend>|

  \begin{scriptsize}
  \begin{verbatim}
<form>
  <fieldset>
    <legend>
      Datos personales
    </legend>

    Elija un color:
    <input type="color" name="favcolor">
    <br> Fecha de nacimiento:
    <input type="date" name="nacimiento">
    <br> Fecha y hora de nacimiento:
    <input type="datetime-local" name="nacimiento-hora">
    <br> E-mail:
    <input type="email" name="email">
    <br> Indica un número del 1 al 10:
    <input type="number" name="numero" min="1" max="10">
    <br>
    <input type="submit">
  </fieldset>
</form>

  \end{verbatim}
  \end{scriptsize}

\end{frame}



%%---------------------------------------------------------------
\begin{frame}[fragile]
\frametitle{select}
El elemento
\verb|<select>|
permite elegir una opción entre varias


  \begin{scriptsize}
  \begin{verbatim}
<form>
   Indique el departamento:
   <select name="departamento">
      <option value="Comercial">Comercial</option>
      <option value="Técnico">Técnico</option>
      <option value="Webmaster">Webmaster</option>
   </select>
   <input type="submit">
</form>
  \end{verbatim}
  \end{scriptsize}

\end{frame}



%%---------------------------------------------------------------
\begin{frame}[fragile]
\frametitle{textarea}

Con el elemento
\verb|<textarea>|
el usuario puede introducir varias líneas de texto


  \begin{scriptsize}
  \begin{verbatim}
<form>
   <textarea name="mensaje" rows="10" cols="30">
   Escriba aquí su mensaje.
   </textarea>
</form>
  \end{verbatim}
  \end{scriptsize}



\end{frame}



%%---------------------------------------------------------------
\begin{frame}[fragile]
\frametitle{label}

Para que el usuario sepa qué es cada
elemento de un formulario, se puede usar:
    \begin{itemize}
    \item
Texto HTML ordinario
    \item
Un elemento
\verb|<label>|

Ventajas:

    \begin{itemize}
    \item
Haciendo clic sobre esta etiqueta, se activa el elemento
    \item
Facilita su interpretación en entornos distintos a navegadores tradicionales, p.e. lectores de HTML
    \item
Facilita el estilo consistente
    \end{itemize}
    \end{itemize}



\end{frame}

%%----------------------------------------------
\begin{frame}[fragile]

Para usar
\verb|<label>|
hay que
    \begin{itemize}
    \item
Añadir un atributo
\verb|<id>|
al
elemento

    \item
Poner en el label un atributo
\verb|for|
cuyo valor sea el del id

\end{itemize}

  \begin{scriptsize}
  \begin{verbatim}
  <form>
    <label for="ciudad">Ciudad de procedencia:</label>
    <input type="text" name="ciudad" id="ciudad">
    <input type="submit">
  </form>
  \end{verbatim}
  \end{scriptsize}


    \begin{itemize}
    \item
\verb|name|
lo use el
servidor 
    \item
\verb|id|
lo usa el navegador
    \end{itemize}
Son independientes, pueden coincidir o no


\end{frame}

%%----------------------------------------------
\begin{frame}[fragile]

En el caso del 
\verb|<input type="radio">|
y el 
\verb|<input type="checkbox">|

    \begin{itemize}
    \item
Normalmente es innecesario usar 
\verb|<label>|,
el texto dentro del elemento suele ser
bastante descriptivo

    \item
Si queremos incluir un 
\verb|<label>|,
 se usa como en cualquie otro elemento,
un 
\verb|<label>|
por cada input

    \begin{itemize}
    \item
 Aunque es recomendable que en el HTML escribamos el 
\verb|<label>|
después del 
\verb|<input>|,
para que el navegador muestre la etiqueta a la derecha del 
\verb|<input>|,
no a la izquierda
    \end{itemize}
    \end{itemize}

\end{frame}




\subsection{Elementos obsoletos}
%%---------------------------------------------------------------
\begin{frame}[fragile]
\frametitle{Elementos y Atributos obsoletos en HTML5}
En HTML 4 había muchos elementos y atributos relacionados con el formato.
Algunos de los más habituales:
\begin{itemize}
\item
align (left, right, center)

\item
color

\item
font

\item
u  (underline)
\end{itemize}

Todos ellos han desaparecido en HTML 5, en su lugar debe usarse CSS

\end{frame}



\subsection{Entities}
%%---------------------------------------------------------------
\begin{frame}[fragile]
\frametitle{Entities}
%https://www.w3schools.com/html/html_entities.asp
Las 
\emph{entities} se usan para
\begin{itemize}
\item
Representar caracteres que coinciden con metacaracteres de HTML.
Es la única forma de indicar caracteres como
\verb|<|

\item
Representar caracteres que el desarrollador no tenga en su teclado.
Su uso es opcional, con UTF-8 siempre se puede escribir cualquier caracter
(mediante teclados virtuales o copiando y pegando desde otro sitio)
\end{itemize}

Cada entity tiene un nombre y un número, se puede usar cualquier de
las dos formas

    \begin{itemize}
    \item
\emph{Ampersand}, 
nombre, punto y coma

\verb|&lt;|
    \item
\emph{Ampersand}, 
almohadilla, número, punto y coma

\verb|&#60;|

    \end{itemize}

\end{frame}



%%---------------------------------------------------------------
\begin{frame}[fragile]
\frametitle{}
Algunas entities habituales
\begin{itemize}
\item
\verb|<|
  \begin{footnotesize}
  \begin{verbatim}
&lt;      &#60;
  \end{verbatim}
  \end{footnotesize}

\item
\verb|>|

  \begin{footnotesize}
  \begin{verbatim}
&gt;      &#62;
  \end{verbatim}
  \end{footnotesize}

\item
€

  \begin{footnotesize}
  \begin{verbatim}
&euro;    &#8364;
  \end{verbatim}
  \end{footnotesize}


\item
Ñ

  \begin{footnotesize}
  \begin{verbatim}
&Ntilde;  &#209;
  \end{verbatim}
  \end{footnotesize}


\item
ñ

  \begin{footnotesize}
  \begin{verbatim}
&ntilde;  &#241;
  \end{verbatim}
  \end{footnotesize}
\end{itemize}

\end{frame}




%%---------------------------------------------------------------
\begin{frame}[fragile]
\frametitle{}
Otra entity frecuente es
\verb|&nbsp;|
\emph{non-breaking space}

Es un espacio que:
\begin{itemize}
\item
Siempre se representa
\item
Nunca se usa para partir una linea.
Ejemplo: para escribir
2 € 
con garantías de que ambos símbolos estarán en la mísma línea:

\verb|2&nbsp;&euro;|
\end{itemize}

\end{frame}


%%---------------------------------------------------------------

\begin{frame}
\frametitle{Material complementario}

\begin{itemize}
\item HyperText Markup Language (Wikibook): \\
  \url{http://en.wikibooks.org/wiki/HTML_Programming}
\item HTML5: A tutorial for beginners: \\
  \url{http://www.html-5-tutorial.com/}
\item Dive into HTML5: \\
  \url{http://diveintohtml5.info}
\item HTML5 (Wikipedia): \\
  \url{http://en.wikipedia.org/wiki/HTML5}
\item Web Fundametals (Code Academy): \\
  \url{http://www.codecademy.com/tracks/web}
\end{itemize}


\end{frame}


%%---------------------------------------------------------------
\end{document}
%%---------------------------------------------------------------


\documentclass[ucs]{beamer}

\usetheme{GSyC}
%\usebackgroundtemplate{\includegraphics[width=\paperwidth]{gsyc-bg.png}}


\usepackage[spanish]{babel}   
\usepackage[utf8x]{inputenc}
\usepackage{graphicx}
\usepackage{amssymb} % Simbolos matematicos
\usepackage{lmodern,textcomp}  % para usar el carácter € tal cual
%\usepackage{hthtml}
%\usepackage{html}



% Metadatos del PDF, por defecto en blanco, pdftitle no parece funcionar
   \hypersetup{%
     pdftitle={JavaScript II},
     %pdfsubject={Diseño y Administración de Sistemas y Redes},%
     pdfauthor={GSyC},%
     pdfkeywords={},%
   }
%


% Para colocar un logo en la esquina inferior de todas las transpas
%   \pgfdeclareimage[height=0.5cm]{gsyc-logo}{gsyc}
%   \logo{\pgfuseimage{gsyc-logo}}


% Para colocar antes de cada sección una página de recuerdo de índice
%\AtBeginSection[]{
%  \begin{frame}<beamer>{Contenidos}
%    \tableofcontents[currentframetitle]
%  \end{frame}
%}



\begin{document}

% Entre corchetes como argumento opcional un título o autor abreviado
% para los pies de transpa
\title[JavaScript II]{JavaScript II}
%\subtitle{Diseño y Administración de Sistemas y Redes}
\author[GSyC]{Escuela Técnica Superior de Ingeniería de Telecomunicación\\
Universidad Rey Juan Carlos}
\institute{gsyc-profes (arroba) gsyc.urjc.es}
\date[2017]{Noviembre de 2017}


%% TÍTULO
\begin{frame}
  \titlepage
  % Oportunidad para poner otro logo si se usó la opción nologo
  % \includegraphics[width=2cm]{logoesp}  
\end{frame}




%% LICENCIA DE REDISTRIBUCIÓN DE LAS TRANSPAS
%% Nota: la opción b al frame le dice que justifique el texto
%% abajo (por defecto c: centrado)
\begin{frame}[b]
\begin{flushright}
{\tiny
\copyright \insertshortdate~\insertshortauthor \\
  Algunos derechos reservados. \\
  Este trabajo se distribuye bajo la licencia \\
  Creative Commons Attribution Share-Alike 4.0
}
\end{flushright}
\end{frame}



%% ÍNDICE
%\begin{frame}
%  \frametitle{Contenidos}
%  \tableofcontents
%\end{frame}

% $Id$
%

\section{Excepciones}
%%---------------------------------------------------------------
\begin{frame}[fragile]
\frametitle{Excepciones}
Las excepciones son similares a las de cualquier otro lenguaje
\begin{itemize}
\item
Con la particularidad de que el argumento de \verb|throw| (la excepción), 
puede ser una cadena, un número, un booleano o un objeto
\end{itemize}

  \begin{scriptsize}
  \begin{verbatim}
'use strict'

try {
  throw 'xxx27';
} catch (e) {
  console.log('capturada excepción ' + e);
     // capturada excepción xxx27
}
  \end{verbatim}
  \end{scriptsize}

\end{frame}

%%---------------------------------------------------------------
\begin{frame}[fragile]
\frametitle{}
Cuando el motor de JavaScript lanza una excepción, es un objeto,
con las propiedades 
\verb|name| y
\verb|message|

  \begin{scriptsize}
  \begin{verbatim}

'use strict'

try{
    console.log( no_definido);
} catch (e) {
    console.log("capturada excepción ",e.name,e.message);
}
capturada excepción  ReferenceError no_definido is not defined
  \end{verbatim}
  \end{scriptsize}

\end{frame}

%%---------------------------------------------------------------
\begin{frame}[fragile]
\frametitle{}

Captura de un tipo concreto de excepción. P.e. \verb|ReferenceError|
  \begin{scriptsize}
  \begin{verbatim}
'use strict'
try {
  console.log(no_definido);
} catch (e) {
  console.log("Capturada excepción");
  switch (e.name) {
    case ('ReferenceError'):
      console.log(e.name + " Objeto no definido");
      break;
    default:
      console.log(e.name + " Excepción inesperada");
      break;
  }
}
  \end{verbatim}
  \end{scriptsize}
\end{frame}

%%---------------------------------------------------------------
\begin{frame}[fragile]
\frametitle{}
Hay 7 nombres de error
    \begin{itemize}
    \item
Error

Unspecified Error
    \item
EvalError	

An error has occurred in the eval() function
    \item
RangeError

A number \emph{out of range} has occurred
    \item
ReferenceError

An illegal reference has occurred
    \item
SyntaxError	

A syntax error has occurred
    \item
TypeError	

A type error has occurred
    \item
URIError

An error in encodeURI() has occurred
    \end{itemize}

\end{frame}

%%---------------------------------------------------------------
\begin{frame}[fragile]
\frametitle{}
Aunque nuestros programas pueden lanzar simplemente cadenas, números o booleanos,
es preferible lanzar objetos 

Hay un constructor para cada nombre de error

  \begin{scriptsize}
  \begin{verbatim}
'use strict'

try{
    throw new RangeError('xxx28');
} catch (e) {
    console.log('capturada excepción ',e.name,e.message);
    // capturada excepción  RangeError xxx28
}
  \end{verbatim}
  \end{scriptsize}
\end{frame}

\section{Módulos}
%%---------------------------------------------------------------
\begin{frame}[fragile]
\frametitle{Módulos}
Un Módulo es

\begin{itemize}
\item
 Un fichero que contiene código JavaScript

\item
Que es invocado desde otro fichero distinto, que contiene el código principal

\end{itemize}
En otros lenguajes se les llama bibliotecas o librerías
\end{frame}

%%----------------------------------------------
\begin{frame}[fragile]

En ECMAScript 5 no había una forma nativa de usar módulos, pero se
desarrollaron diferentes herramientas, incompatibles entre sí, que permitían su uso

Las principales son:

    \begin{itemize}
    \item
CommonJS

Para Node.js

    \item
RequireJS

Para el navegador
    \end{itemize}

En ECMASCript 6 sí hay soporte nativo para módulos. Una mezcla de las dos sintaxis, 
ambas están soportadas


    \begin{itemize}
    \item
Pero en la actualidad, año 2017, tanto los navegadores como node.js tienen muy
mal soporte para los módulos de ECMAScript 6
    \end{itemize}

\end{frame}

%%---------------------------------------------------------------
\begin{frame}[fragile]
\frametitle{Librerias en node.js}

Fichero \verb|lib01.js|
  \begin{scriptsize}
  \begin{verbatim}
function f(){
    return "efe";
}
function g(){
    return "ge";
}
function h(){  // esta función no la exportamos
    return "hache";
}

module.exports={
    f,g,
};
  \end{verbatim}
  \end{scriptsize}

Fichero principal

  \begin{scriptsize}
  \begin{verbatim}
'use strict'

let lib01=require('./lib01.js');
console.log(lib01.f());
console.log(lib01.g());
  \end{verbatim}
  \end{scriptsize}

\end{frame}


\section{Fecha y Hora}
%%---------------------------------------------------------------
\begin{frame}[fragile]
\frametitle{Fecha y Hora }
La fecha (con su hora) se guarda en objetos de tipo 
\verb|Date|

\begin{itemize}
\item
Siempre es una hora UTC (Coordinated Universal Time), anteriormente
llamada hora de Greenwich 

\item
Se guarda como milisegundos transcurridos desde el 1 de enero de 1970.

    \begin{itemize}
    \item
Es recomendable almacenar, procesar y transmitir este formato en todo momento, y solo 
inmediatamente antes de presentarlo al usuario, realizar la conversión necesaria
    \end{itemize}

\item
Atención, no es el \emph{tiempo unix} (segundos transcurridos desde el 
instante \emph{epoch}), como en otros sistemas. Aquí son milisegundos

\end{itemize}
\end{frame}

%%---------------------------------------------------------------
\begin{frame}[fragile]
\frametitle{}

  \begin{scriptsize}
  \begin{verbatim}
// Objeto con la hora actual
d=new Date();
console.log(d); // (Fecha y hora UTC actual)

// Objeto con una hora concreta, expresada como hora local
// P.e. 1 de septiembre de 2017, a las 9 de España
// Cuidado: 0 es enero, 11 es diciembre
d=new Date(2017, 8, 1, 9, 0, 0);
console.log(d); // 2017-09-01T07:00:00.000Z

// Objeto con una hora concreta, expresada como hora UTC
d=Date.UTC(2017, 8, 1, 9, 0, 0);
console.log(d);

// Objeto con una hora concreta, expresada en tiempo Unix
d=new Date(0);
console.log(d); // 1970-01-01T00:00:00.000Z

let hora_unix=1504256400;
d=new Date(hora_unix*1000);  //JavaScript espera milisegundos
console.log(d); // 2017-09-01T09:00:00.000Z
  \end{verbatim}
  \end{scriptsize}

\end{frame}


%%---------------------------------------------------------------
\begin{frame}[fragile]
\frametitle{}
  \begin{scriptsize}
  \begin{verbatim}
d=new Date(2017, 8, 1, 9, 0, 0);  // 9:00 hora española
                                  // Mes 8: septiembre

// Acceso a cada unidad, expresada como hora local
console.log(d.getFullYear());   // 2017
console.log(d.getMonth());  // 8 (septiembre)
console.log(d.getDate());  // 1
console.log(d.getDay()); //5  (viernes)
console.log(d.getHours()); // 9  hora española 
console.log(d.getMinutes()); // 0
console.log(d.getSeconds()); // 0 
console.log(d.getMilliseconds()); // 

// Acceso a cada unidad, expresada como hora UTC
console.log(d.getUTCFullYear());   // 2017
console.log(d.getUTCMonth());  // 8 (septiembre)
console.log(d.getUTCDate());  // 1
console.log(d.getUTCDay()); //5  (viernes)
console.log(d.getUTCHours()); // 7  hora UTC
console.log(d.getUTCMinutes()); // 0
console.log(d.getUTCSeconds()); // 0 
console.log(d.getUTCMilliseconds()); // 0
  \end{verbatim}
  \end{scriptsize}

\end{frame}

%%---------------------------------------------------------------
\begin{frame}[fragile]
\frametitle{}
Cálculo del tiempo transcurrido entre dos fechas

  \begin{scriptsize}
  \begin{verbatim}
'use strict'
let d1,d2;

// 1 de septiembre de 2017, a las 9 de España
d1=new Date(2017, 8, 1, 9, 0, 0);

// las 9 y 10
d2=new Date(2017, 8, 1, 9, 10, 0);

console.log(d2-d1) // 600000 (600 segundos)
  \end{verbatim}
  \end{scriptsize}
\end{frame}

%%---------------------------------------------------------------
\begin{frame}[fragile]
\frametitle{}
  \begin{scriptsize}
  \begin{verbatim}
'use strict'
let d1,d2,s, ms_año, edad;

//  8 de julio de 1996, creación de la URJC
d1=new Date(1996, 6, 8, 0, 0, 0);
d2=new Date();
console.log(d1); //  1996-07-07T22:00:00.000Z
console.log(d2); // 2017-11-01T17:09:03.193Z

s= (d2-d1) ;  // milisegundos transcurridos

ms_año= 31536000000  // 1000*60*60*24*365
edad= (s/ms_año).toFixed(2);  // Redondeo a 2 decimales

console.log('Edad de la URJC:', edad);   // 21.33
  \end{verbatim}
  \end{scriptsize}
\end{frame}


\section{POO basada en prototipos}
%%---------------------------------------------------------------
\begin{frame}[fragile]
\frametitle{POO basada en herencia vs POO basada en prototipos}

La gran mayoría de lenguajes y herramientas diversas que emplean
POO (programación orientada a objetos) aplican POO basada en herencia

    \begin{itemize}
    \item
Es la forma tradicional.
De hecho, es frecuente considerar que
\emph{POO} necesariamente implica herencia
    \end{itemize}


En POO tradicional, hay un principio generalmente aceptado:
\emph{Favor object composition over class inheritance}
%E. Gamma et al.  \emph{Design Patterns}.Addison-Wesley, 1994

(E. Gamma et al.  \emph{Design Patterns}. 1994)

La POO basada en prototipos 
va un paso más allá

\begin{itemize}
\item
Para solucionar una serie de problemas bien conocidos en la POO tradicional. 
No \emph{prefiere}
la composición frente  a la herencia, sino que omite por completo
la herencia y se basa solo en composición.
\end{itemize}

\end{frame}

%%---------------------------------------------------------------
\begin{frame}[fragile]
\frametitle{Problemas de la herencia (1)}
\begin{itemize}
% programming javascript, pg 49

\item
Código yo-yo

En jerarquías de cierta longitud, el programador se ve
obligado a subir y bajar por las clases continuamente

\item
La herencia es una relación fuertemente acoplada. Los hijos
heredan todo sobre sus padres, necesitan conocer todo sobre
sus padres (y abuelos, bisabuelos...)

\item
Problema del gorila y el plátano

\emph{Yo solo quería un plátano, pero me dieron el plátano, el  gorila que sostenía
el plátano y la jungla entera}

\end{itemize}
\end{frame}

%%----------------------------------------------
\begin{frame}[fragile]
\frametitle{Problemas de la herencia (2)}

\begin{itemize}
\item
La herencia es inflexible 

Por muy bien que se diseñe una jerarquía
de clases, en dominios medianamente complejos acaban apareciendo casos no previstos
que  no encajan en la taxonomía inicial

\item
Herencia múltiple

Heredar de diferentes clases es deseable, y
teóricamente posible. Pero en la práctica
resulta
muy complicado

\item
Arquitectura frágil

Rediseñar una clase obliga a modificar todos sus descendientes
\end{itemize}
\end{frame}

%%---------------------------------------------------------------
\begin{frame}[fragile]
\frametitle{POO basada en prototipos}

\begin{itemize}
\item
Aparece en el lenguaje Self, a mediados de los años 1980

\item
No hay clases como instancias de objetos: solo hay objetos, que se producen
mediante funciones denominadas \emph{factorías de objetos}

\item
Un objeto es una unidad de código que soluciona un problema concreto. No es
necesario (o al menos no es crítico) pensar cómo usar variantes de ese objeto
en otros casos

\item
A partir de ese objeto, según se va necesitando, se crean nuevos objetos
añadiendo o eliminando las propiedades (datos y métodos) necesarias.

\item
Metáfora. 

Herencia: piezas de Ikea 

Prototipo: piezas de Lego
\end{itemize}
\end{frame}

%%---------------------------------------------------------------
\begin{frame}[fragile]
\frametitle{Problemas de la POO basada en prototipos (1)}
\begin{itemize}
\item
Poco conocida

Pocos programadores la usan bien, pocos libros
la explican bien

\item
No adecuada para principiantes 

Una vez que se conoce su uso resulta
sencillo, pero su curva de aprendizaje es pronunciada

Ejemplo: en JavaScript es necesario manejar al menos los siguiente conceptos:

    \begin{itemize}
    \item
Lambdas
    \item
Cierres
    \item
Funciones flecha
    \item
Prototipos
    \item
Factoria de objetos
    \item
Extensiones dinámicas de objetos: mezclas, composición, agregación
    \end{itemize}
\end{itemize}
\end{frame}

%%---------------------------------------------------------------
\begin{frame}[fragile]
\frametitle{Problemas de la POO basada en prototipos (2)}
\begin{itemize}
\item
La flexibilidad del paradigma añade complejidad al intérprete/compilador,
lo que afecta al rendimiento

    \begin{itemize}
    \item
Aunque los motores modernos son muy eficientes y este problema solo es relevante
en casos extremos
    \end{itemize}

\item
La flexibilidad del paradigma puede hacer más difícil el garantizar que los
resultados sean correctos

    \begin{itemize}
    \item
Crítica análoga a la que hacen los partidarios de lenguajes de tipado
estático frente a lenguajes de tipado dinámico

    \item
Una clase es un contrato estricto sobre lo que puede o no puede hacer un
objeto. Estas restricciones no existen con los prototipos

\emph{With great power comes great responsibility}

% aparece en la revolución francesa. Lo repiten Churchill y Roosevelt. Popularizado por spider-man
    \end{itemize}
\end{itemize}
\end{frame}

%%---------------------------------------------------------------
\begin{frame}[fragile]
\frametitle{Enlaces sobre POO basada en prototipos}
\begin{itemize}
\item
Composition over Inheritance

Video en YouTube del canal \emph{Fun Fun Funtions}

\begin{tiny}
\begin{flushright}
\url{https://youtu.be/wfMtDGfHWpA}
\end{flushright}
\end{tiny}

\item
Master the JavaScript Interview: What’s the Difference Between Class \& Prototypal Inheritance?

Eric Elliott
\begin{tiny}
\begin{flushright}
\url{http://tinyurl.com/zbtjruf}
\end{flushright}
\end{tiny}

\item
Programming JavaScript Applications

Eric Elliott. O'Reilly, 2015
\begin{tiny}
\begin{flushright}
\url{http://proquest.safaribooksonline.com/book/programming/javascript/9781491950289}
\end{flushright}
\end{tiny}

\item
The Principles of Object-Oriented JavaScript

Nicholas Zakas.
No Starch Press, 2014
\begin{tiny}
\begin{flushright}
\url{http://proquest.safaribooksonline.com/book/programming/javascript/9781457185304}
\end{flushright}
\end{tiny}
\end{itemize}
\end{frame}

\section{POO basada en herencia}
%%---------------------------------------------------------------
\begin{frame}[fragile]
\frametitle{POO basada en herencia en JavaScript}
El lenguaje JavaScript usa POO basada en prototipos.

Tras una cierta polémica,
ECMASCript 6 soporta POO basada en herencia

    \begin{itemize}
    \item
En realidad es \emph{azúcar sintáctico}. Internamente siguen siendo prototipos
    \end{itemize}

Los argumentos principales por los que se acepta este \emph{paso atrás}
son

    \begin{enumerate}
    \item
Muchos programadores insisten en usarla. Distintas librerías, distintas
soluciones ad-hoc. Es preferible una implementación oficial

    \item
Para principiantes con problemas sencillos, resulta más adecuado
    \end{enumerate}

Por este último motivo, en la presente asignatura veremos POO basada en herencia, no
POO basada en prototipos
\end{frame}

%%---------------------------------------------------------------
\begin{frame}[fragile]
\frametitle{Clases}
\begin{itemize}
\item
Definimos clases con la palabra reservada
\verb|class|, el nombre de la clase y entre llaves, sus propiedades

\item
Por convenio, los nombres de clase empiezan por letra mayúscula

\item
A diferencia de lo que ocurre en los objetos, las propiedades no van separadas
por comas

\item
Las propiedades de los objetos se crean en el método
\verb|constructor()|

\item
Se accede a las propiedes mediante la palabra reservada
\verb|this|, que representa al objeto

\item
Para crear una clase heredera de otra:
\verb| class Hija extends Madre{ ....} |

\item
Para llamar a un método de la clase padre, se usa la palabra reservada
\verb|super|
\end{itemize}
\end{frame}


%%---------------------------------------------------------------
\begin{frame}[fragile]
\frametitle{}
  \begin{scriptsize}
  \begin{verbatim}
'use strict'
class Circunferencia{
    constructor(x,y,r){
        this.x=x;
        this.y=x;
        this.r=r;
    }
    aCadena(){
        return '('+this.x+','+this.y+','+this.r+')';
    }
}
class Circulo extends Circunferencia{
    constructor(x,y,r,color){
        super(x,y,r);
        this.color=color;
    }
    aCadena(){
        return super.aCadena()+ " color:"+ this.color;
    }
}
let a=new Circunferencia(2,2,1);
console.log(a.aCadena());   // (2,2,1)
let b=new Circulo(2,2,1,"azul");
console.log(b.aCadena());  // (2,2,1) color: azul
  \end{verbatim}
  \end{scriptsize}

\end{frame}


%%---------------------------------------------------------------
\begin{frame}[fragile]
\frametitle{Métodos}
Para declarar los métodos de una clase no es necesario usar la palabra reservada \verb|function|

Las clases tienen tres tipos de métodos.
\begin{itemize}
\item
\verb|constructor()|

Es un método especial para crear inicializar los objetos de esta clase

\item
Métodos de prototipo (métodos \emph{normales})

\item
Métodos estático. (métodos \emph{de clase})

Se crean anteponiendo la palabra reservada
\verb|static|

Se invocan sin instanciar las clases en objetos, no pueden llamarse a través de una instancia
de clase (a través de un objeto), solo a través de la clase

¿En qué casos un método  debería ser estático?

    \begin{itemize}
    \item
Cuando tenga sentido sin que se haya declarado un objeto
    \item
Cuando procese 2 o más objetos, sin que uno tenga más relevancia que otro
    \end{itemize}
\end{itemize}
\end{frame}

%%---------------------------------------------------------------
\begin{frame}[fragile]
\frametitle{}
  \begin{scriptsize}
  \begin{verbatim}
'use strict'
class Circunferencia{
    constructor(x,y,r){
        this.x=x;
        this.y=x;
        this.r=r;
    }
    aCadena(){
        return '('+this.x+','+this.y+','+this.r+')';
    }
    static distanciaCentros( a, b){
        return Math.sqrt( Math.pow(a.x-b.x, 2) + Math.pow(a.y-b.y, 2));
    }
    longitud(){
        return 2*Math.PI*this.r;
    }
}

let p=new Circunferencia(0,0,1);
let q=new Circunferencia(1,1,1);
console.log(p.aCadena());  // (0,0,1)
console.log( Circunferencia.distanciaCentros(p,q));  // 1.4142135623730951
console.log(p.longitud()); // 6.283185307179586
  \end{verbatim}
  \end{scriptsize}
\end{frame}

%%---------------------------------------------------------------
\begin{frame}[fragile]
\frametitle{Enlaces sobre clases en JavaScript}
\begin{itemize}
\item
MDN Web Docs. Classes

\begin{tiny}
\begin{flushright}
\url{https://developer.mozilla.org/en-US/docs/Web/JavaScript/Reference/Classes}
\end{flushright}
\end{tiny}
\item
\emph{Exploring ES6. Upgrade to the next version of JavaScript}

Axel Rauschmayer

\begin{tiny}
\begin{flushright}
\url{http://exploringjs.com/es6/ch_classes.html#sec_overview-classes}
\end{flushright}
\end{tiny}
\end{itemize}
\end{frame}

\end{document}

\documentclass[ucs]{beamer}
\usetheme{GSyC}
\usepackage[spanish]{babel}   
\usepackage[utf8x]{inputenc}
\usepackage{graphicx}
\usepackage{amssymb} % Simbolos matematicos
\usepackage{lmodern,textcomp}  % para usar el carácter € tal cual
\usepackage{enumerate}


% Metadatos del PDF, por defecto en blanco, pdftitle no parece funcionar
   \hypersetup{%
     pdftitle={jQuery},
     %pdfsubject={Diseño y Administración de Sistemas y Redes},%
     pdfauthor={GSyC},%
     pdfkeywords={},%
   }
%


% Para colocar un logo en la esquina inferior de todas las transpas
%   \pgfdeclareimage[height=0.5cm]{gsyc-logo}{gsyc}
%   \logo{\pgfuseimage{gsyc-logo}}


% Para colocar antes de cada sección una página de recuerdo de índice
%\AtBeginSection[]{
%  \begin{frame}<beamer>{Contenidos}
%    \tableofcontents[currentframetitle]
%  \end{frame}
%}



\begin{document}

% Entre corchetes como argumento opcional un título o autor abreviado
% para los pies de transpa
\title[jQuery]{jQuery}
%\subtitle{Diseño y Administración de Sistemas y Redes}
\author[GSyC]{Escuela Técnica Superior de Ingeniería de Telecomunicación\\
Universidad Rey Juan Carlos}
\institute{gsyc-profes (arroba) gsyc.urjc.es}
\date[2017]{Noviembre de 2017}


%% TÍTULO
\begin{frame}
  \titlepage
  % Oportunidad para poner otro logo si se usó la opción nologo
  % \includegraphics[width=2cm]{logoesp}  
\end{frame}




%% LICENCIA DE REDISTRIBUCIÓN DE LAS TRANSPAS
%% Nota: la opción b al frame le dice que justifique el texto
%% abajo (por defecto c: centrado)
\begin{frame}[b]
\begin{flushright}
{\tiny
\copyright \insertshortdate~\insertshortauthor \\
  Algunos derechos reservados. \\
  Este trabajo se distribuye bajo la licencia \\
  Creative Commons Attribution Share-Alike 4.0
}
\end{flushright}
\end{frame}


%% ÍNDICE
%\begin{frame}
%  \frametitle{Contenidos}
%  \tableofcontents
%\end{frame}

% $Id$
%

\section{jQuery}
\subsection{Introducción}

%%---------------------------------------------------------------
\begin{frame}[fragile]
\frametitle{DOM: Document Object Model}


DOM, Document Object Model es un API que permite tratar una página web con estructura de árbol
\begin{itemize}
\item
Estándar de Internet, normalizado por el W3C

\item
Es el interfaz que emplean los navegadores web internamente

\item
Cuando un navegador carga una página HTML, la procesa para convertirla en la estructura del DOM. Desde ahí se representa en pantalla.

\item
Los cambios que pueda tener la página, p.e. desde JavaScript, se hacen directamente con el DOM, el HTML no se vuelve a utilizar
\end{itemize}

\end{frame}




%%---------------------------------------------------------------
\begin{frame}[fragile]
\frametitle{jQuery}


JQuery es una librería JavaScript que permite recorrer un documento, seleccionar objetos del DOM, hacer animaciones,
manejar enventos, usar Ajax y hacer plugins sobre JavaScript
    \begin{itemize}
    \item
Creada por John Resig
en 2006, es software libre

    \item
Aunque se puede procesar el DOM con JavaScript nativo, es poco frecuente. Es más conveniente y habitual
emplear jQuery

    \item
También cuenta con una librería similar a Bootstrap, jQuery-UI


    \begin{itemize}
    \item
En esta asignatura preferimos Bootstrap a jQuery-UI
    \item
Es frecuente preferir Boostrap. jQuery-UI no es tan popular como
Bootstrap 
    \end{itemize}
    \end{itemize}

\end{frame}



%%---------------------------------------------------------------
\begin{frame}[fragile]
\frametitle{Modificar el HTML ¿Para qué?}

Programando sobre el DOM se puede hacer prácticamente cualquier cosa con una
página

\begin{itemize}
\item
Normalmente lo que deberíamos buscar es
funcionalidad útil que mejore la experiencia de usuario

\item
Deberíamos evitar los
efectos
que llamen la atención gratuitamente, comportamiento no estándar o poco intuitivos,
adornos que acaban molestando, etc
\end{itemize}

\end{frame}



%%---------------------------------------------------------------
\begin{frame}[fragile]
\frametitle{}
Funcionalidad que realmente mejora la experiencia de usuario:
\begin{itemize}
\item
Es normal que una aplicación tenga muchos parámetros, difíciles
de asimilar para el usuario

Ocultar unos y mostrar otros facilita su trabajo


\begin{itemize}
    \item
Se puede ocultar y/o marcar como deshabilitado lo que en cierto momento no se puede hacer
    \item
Jerarquizar el interfaz. Por ejemplo modo básico, modo normal, modo experto
\end{itemize}

    \item
Ofrecer información y ayuda contextual

    \item
Presentar la información en distintos formatos o unidades

    \item
Validación de formularios 



\end{itemize}

\end{frame}

%%----------------------------------------------
\begin{frame}[fragile]

\begin{itemize}


    \item
Formularios mejorados

Ejemplos

    \begin{itemize}
    \item
Una entrada donde el usuario indica un porcentaje desplazando una barra, no introduciendo un número
    \item
Una entrada que inmediatamente actualiza otra información. \emph{Si gasta 20 entonces le quedan 80}
    \end{itemize}

    \item
Información sobre el progreso de lo que el usuario ha pedido. P.e \emph{progress bar}
en porcentaje, o en unidades de tiempo. O estimaciones del tiempo restante

    \item
Información en tiempo real sobre sucesos diversos
\item
Generación de gráficos 


HTML Canvas.
(No lo trataremos en esta asignatura)

\item

...
\end{itemize}

\end{frame}




%%---------------------------------------------------------------
\begin{frame}[fragile]
\frametitle{Uso de jQuery}

Es una librería compuesta por un único fichero

    \begin{itemize}
    \item
Se puede descargar en el sistema de ficheros local y cargarla con


  \begin{scriptsize}
  \begin{verbatim}
<script src="jquery.js"></script>
  \end{verbatim}
  \end{scriptsize}

    \item
Se puede usar un CDN. Por ejemplo el de gogle

  \begin{tiny}
  \begin{verbatim}
<script src="https://ajax.googleapis.com/ajax/libs/jquery/3.2.1/jquery.min.js"></script>
  \end{verbatim}
  \end{tiny}

\end{itemize}

JQuery es una única funcion, que a su vez incluye diversas funciones

    \begin{itemize}
    \item
Esto es posible porque JavaScript usa funciones de orden superior, \emph{higher-order functions}.
Nos puede ayudar considerarlo un objeto con métodos (en JavaScript las funciones son objetos)


    \item
Esta función normalmente recibe el nombre de \verb|$|

Típicamente recibe como primer argumento un selector CSS que indica sobre qué
debe actuar
\end{itemize}
\end{frame}



%%---------------------------------------------------------------
\begin{frame}[fragile]
\frametitle{Añadir y quitar clases}


Un patrón habitual en jQuery es
\begin{itemize}
\item
Definir una clase CSS con el aspecto deseado. Por ejemplo ocultar un elemento

\item
Invocar los métodos 
\verb|removeClass| y 
\verb|addClass| 
para añadir y quitar esa clase cuando se produzca cierto eventeo

\end{itemize}


  \begin{scriptsize}
  \begin{verbatim}
  <style>
    .oculto {
      display: none;
    }
  </style>

[...]

  <div id="marco_foto" class="oculto">
    <img src="images/foto.jpg" alt="Foto de prueba">
  </div>
  \end{verbatim}
  \end{scriptsize}


\end{frame}





%%---------------------------------------------------------------
\begin{frame}[fragile]
\frametitle{}

  \begin{scriptsize}
  \begin{verbatim}
  <script>
    function mostrar_marcofoto() {
      $("#marco_foto").removeClass("oculto")
    };

    function ocultar_marcofoto() {
      $("#marco_foto").addClass("oculto")
    };

    function main() {
      $("#boton01").click(mostrar_marcofoto);
      $("#boton02").click(ocultar_marcofoto);
    };

    $(document).ready(main());
  </script>
  \end{verbatim}
  \end{scriptsize}


La función
\verb|$(document).ready();|
acepta la función principal que se ejecutará cuando acabe de cargase el documento HTML

\begin{tiny}
\begin{flushright}
\url{http://ortuno.es/hola_jquery01.html}
\end{flushright}
\end{tiny}

\end{frame}



%%---------------------------------------------------------------
\begin{frame}[fragile]
\frametitle{}
El ejemplo anterior es correcto pero no es idiomático.
Para asignar un manejador a un evento, lo
 habitual en JavaScript es
usar funciones anónimas 

  \begin{scriptsize}
  \begin{verbatim}
    $(document).ready(function() {
      $("#boton01").click(function() {
        $("#marco_foto").removeClass("oculto")
      });

      $("#boton02").click(function() {
        $("#marco_foto").addClass("oculto")
      });
    });
  \end{verbatim}
  \end{scriptsize}

\begin{tiny}
\begin{flushright}
\url{http://ortuno.es/hola_jquery02.html}
\end{flushright}
\end{tiny}

\end{frame}




%%---------------------------------------------------------------
\begin{frame}[fragile]
\frametitle{}
Quitar y poner una clase cuando se recibe un evento es muy común, así que
hay una función para ello: \verb|toggleClass()|


  \begin{scriptsize}
  \begin{verbatim}
    $(document).ready(function() {
      $("#boton01").click(function() {
        $("#marco_foto").toggleClass("oculto")
      });
    });
  \end{verbatim}
  \end{scriptsize}

\begin{tiny}
\begin{flushright}
\url{http://ortuno.es/toggle.html}
\end{flushright}
\end{tiny}

\end{frame}




%%---------------------------------------------------------------
\begin{frame}[fragile]
\frametitle{}
Aunque normalmente es preferible quitar y poner clases, también
es posible modificar las propiedades css directamente

  \begin{scriptsize}
  \begin{verbatim}
    $(document).ready(function() {
      $("span").mouseover(function() {
        $(this).css("background-color","grey");
      });

      $("span").mouseleave(function() {
        $(this).css("background-color","white");
      });
    });
  \end{verbatim}
  \end{scriptsize}

\end{frame}



%%---------------------------------------------------------------
\begin{frame}[fragile]
\frametitle{Eventos mouseover, mouseleave}

\begin{itemize}
\item
Cuando el ratón se coloca sobre cierto elemento, este recibe
el evento 
\emph{mouseover}

Para tratarlo, le pasamos a una función con el mismo nombre
(\verb|mouseover()|) el manejador, esto es, la función
que se invocará cuando se reciba el evento

\item
De la misma forma, la retirada del ratón de un elemento es
\emph{mouseleave}
\end{itemize}
Ejemplo:
Destaquemos el párrafo sobre el que se posiciona el ratón

  \begin{scriptsize}
  \begin{verbatim}
    .destacado {
      background-color: LightGrey;
    }
  \end{verbatim}
  \end{scriptsize}

\end{frame}



%%---------------------------------------------------------------
\begin{frame}[fragile]
\frametitle{}


  \begin{scriptsize}
  \begin{verbatim}
    $(document).ready(function() {
      $("p").mouseover(function() {
        $("p").addClass("destacado")
      });

      $("p").mouseleave(function() {
        $("p").removeClass("destacado")
      });
    });
  \end{verbatim}
  \end{scriptsize}


\begin{tiny}
\begin{flushright}
\url{http://ortuno.es/eventos_raton01.html}
\end{flushright}
\end{tiny}

Esto tiene un problema: no estamos destacando el párrafo seleccionado,
sino todos los párrados

\end{frame}


%%---------------------------------------------------------------
\begin{frame}[fragile]
\frametitle{}
Para añadir el manejador al elemento que nos ha devuelvo la consulta
anterior, usamos \verb|this|


  \begin{scriptsize}
  \begin{verbatim}
    $(document).ready(function() {
      $("p").mouseover(function() {
        $(this).addClass("destacado");
      });

      $("p").mouseleave(function() {
        $(this).removeClass("destacado");
      });
    });
  \end{verbatim}
  \end{scriptsize}


\begin{tiny}
\begin{flushright}
\url{http://ortuno.es/eventos_raton02.html}
\end{flushright}
\end{tiny}
\end{frame}





%%---------------------------------------------------------------
\begin{frame}[fragile]
\frametitle{Modificar el texto de un elemento}
Con el método 
\verb|text()| podemos cambiar el texto de un elemento

Como ejemplo, cada vez que se pulse un botón
escribiremos la hora actual dentro de un DIV

La hora estará en formato ISO 8651, para ello
usaremos \emph{moment}, una librería muy popular para procesar fechas


    \begin{itemize}
    \item
Podemos descargarla desde su sitio web
\url{https://momentjs.com}
e incluirla en nuestro directorio

    \item
Podemos descargarla desde un CDN 
y añadirlo en el atributo \verb|src| de un elemento \verb|script|

\url{https://cdnjs.cloudflare.com/ajax/libs/moment.js/2.19.1/moment-with-locales.min.js}

    \end{itemize}

\verb|moment().format()| devuelve la fecha actual en diversos formatos.
Si no se indica ninguno, por omisión devuelve una cadena ISO 8651, 
p.e.

\verb|2017-11-11T22:21:17+01:00|

\end{frame}


%%---------------------------------------------------------------
\begin{frame}[fragile]
\frametitle{}
  \begin{scriptsize}
  \begin{verbatim}
  <button id="boton01">Registrar hora </button>
  <p id="text01">--</p>
  <script>
    $(document).ready(function() {
      $("#boton01").click(function() {
        $("#text01").text(moment().format());
      });
    });
  </script>
  \end{verbatim}
  \end{scriptsize}

\begin{tiny}
\begin{flushright}
\url{http://ortuno.es/text.html}
\end{flushright}
\end{tiny}
\end{frame}

%%---------------------------------------------------------------
\begin{frame}[fragile]
\frametitle{Leer el texto de un elemento}
El método 
\verb|text()| de jQuery devuelve el texto de un elemento

Esto es un patrón común en los método de jQuery:
\begin{itemize}
\item
Invocados sin argumentos, devuelven un valor

\item
Invocados con argumentos, modifican un valor
\end{itemize}

Ejemplo: leer una velocidad en m/s desde un 
elemento 
\verb|<div id="v_in">|

y escribirla expresada en km/h en un 
\verb|<div id="v_out">|

  \begin{scriptsize}
  \begin{verbatim}
      $("#boton02").click(function() {
        let v_text=$("#v_in").text();
        let v=parseInt(v_text,10)
        v_out=v*3.6+"Km/h";
        $("#v_out").text(v_out);
      });
  \end{verbatim}
  \end{scriptsize}

\begin{tiny}
\begin{flushright}
\url{http://ortuno.es/text2.html}
\end{flushright}
\end{tiny}

Este programa es un ejemplo de diseño muy deficiente: lógica
de negocio repartida por los botones

\end{frame}


%%---------------------------------------------------------------
\begin{frame}[fragile]

\frametitle{}
Un diseño mucho más razonable es el que usamos en las
prácticas:
  \begin{scriptsize}
  \begin{verbatim}
      $("#boton02").click(function() {
        let v_text=$("#v_in").text();
        let v_out=calcula_velocidad(v_text, "m/s", "km/h");
        $("#v_out").text(v_out);
      });
  \end{verbatim}
  \end{scriptsize}
\end{frame}


%%---------------------------------------------------------------
\begin{frame}[fragile]
\frametitle{Añadir contenido}
Una vez seleccionado un elemento o elementos
mediante un selector, podemos modificarlo con los siguientes métodos:

    \begin{itemize}
    \item
append()

Inserta contenido al final de la selección, dentro de la selección
    \item
prepend()

Inserta contenido al principio de la selección, dentro de la selección
    \item
after()

Inserta contenido inmediatamente después de la selección, fuera de la selección

    \item
before()

Inserta contenido inmediatamente antes de la selección, fuera de la selección

    \item
replacewith()

Reemplaza la selección
    \end{itemize}

\end{frame}


%%---------------------------------------------------------------
\begin{frame}[fragile]
\frametitle{}
Ejemplo: añadir filas a una tabla
  \begin{scriptsize}
  \begin{verbatim}
<body>
  <button id=boton01>Registrar hora</button>
  <table id=tabla_horas>
    <tr>
      <th>Hora ISO 8651</th>
    </tr>
  </table>

  <script>
    $(document).ready(function() {
      $("#boton01").click(function() {
        let texto;
        texto = '<tr class="hora"><td>' +moment().format() + '</td></tr>';
        $("#tabla_horas").append(texto);
      });
    });
  </script
  \end{verbatim}
  \end{scriptsize}

\begin{tiny}
\begin{flushright}
\url{http://ortuno.es/append.html}
\end{flushright}
\end{tiny}
\end{frame}


%%---------------------------------------------------------------
\begin{frame}[fragile]
\frametitle{Borrar contenido}
El método \verb|remove()| borra los elementos seleccionados

  \begin{scriptsize}
  \begin{verbatim}
    $(document).ready(function() {
      $("#boton01").click(function() {
        var texto
        texto = '<tr class="hora"><td>' + moment().format() + '</td></tr>';
        $("#tabla_horas").append(texto);
      });

      $("#boton02").click(function() {
        $(".hora").remove();
      });
    });
  \end{verbatim}
  \end{scriptsize}
\begin{tiny}
\begin{flushright}
\url{http://ortuno.es/remove.html}
\end{flushright}
\end{tiny}
\end{frame}

%%---------------------------------------------------------------
\begin{frame}[fragile]
\frametitle{tooltip}
Un \emph{tooltip} es un pequeño cuadro emergente con información contextual. Típicamente
describe el elemento sobre el que se posiciona el ratón

Para añadir un 
\emph{tooltip}
a un elemento cualquiera, con Bootstrap y jQuery:

    \begin{itemize}
    \item
Añadimos al elemento la clase
\verb|data-toogle="tooltip"|

    \item
Le añadimos el atributo \verb|title| con el texto del 
\emph{tooltip}

    \item
Posicionamos el 
\emph{tooltip}
con el atributo
\verb|data-placement|, que puede tomar los valores
\verb|top|,
\verb|bottom|,
\verb|left| o
\verb|right| 

    \item
Desde jQuery, activamos los 
\emph{tooltip}
con
\verb|$('[data-toggle="tooltip"]').tooltip();|
    \end{itemize}

\end{frame}

%%----------------------------------------------
\begin{frame}[fragile]

  \begin{scriptsize}
  \begin{verbatim}
<body>
  <div class="container">
    <br>
    <button data-toggle="tooltip" title="Esto es un tooltip" 
     data-placement="bottom" >Al pasar el ratón aparecerá un 
     <span class="italic">tooltip<span></button>
  </div>
  <script>
    $(document).ready(function(){
       $('[data-toggle="tooltip"]').tooltip();
       });
   </script>
</body>
</html>
  \end{verbatim}
  \end{scriptsize}

\begin{tiny}
\begin{flushright}
\url{http://ortuno.es/tooltip.html}
\end{flushright}
\end{tiny}
\end{frame}


%%---------------------------------------------------------------
\begin{frame}[fragile]
\frametitle{Validación de un formulario}

El método
\verb|change()|
recibe una función que será el manejador con el evento que se genera
cuando cambia un input de un formulario 

Una vez seleccionado el input, el método 
\verb|$(this).val()|
contiene su valor

  \begin{scriptsize}
  \begin{verbatim}
    <input type="password" name="contrasenia" id="contrasenia">
    ...
    $("#contrasenia").change(function() {
      if ( $(this).val().length > 5) {
        $("#validacion").text("Contraseña aceptable");
      } else {
        $("#validacion").text("Contraseña muy corta");
      };
    });
  \end{verbatim}
  \end{scriptsize}

\begin{tiny}
\begin{flushright}
\url{http://ortuno.es/validar.html}
\end{flushright}
\end{tiny}
\end{frame}

%%---------------------------------------------------------------
\begin{frame}[fragile]
\frametitle{}
El ejemplo anterior tenía un problema: el evento
\verb|change|
solo se dispara cuando el foco abandona el input
(cuando el usuario ha acabado de editar ese campo y
pasa al siguiente)

Si queremos vincular un manejador con cualquier cambio producido en 
un input, debemos inscribir el manejador a los eventos
\verb|change|,
\verb|keyup|, 
\verb|paste| y 
\verb|mouseup|


    \begin{itemize}
    \item
Para ello empleamos el método \verb|on()|, que recibe como primer agumento
la lista de eventos, y como segundo, el manejador
    \end{itemize}

  \begin{scriptsize}
  \begin{verbatim}
      $("#contrasenia").on('change keyup paste mouseup', function() {
        if ( $(this).val().length > 5) {
          $("#validacion").text("Contraseña aceptable");
        } else {
          $("#validacion").text("Contraseña muy corta");
        };
      });
  \end{verbatim}
  \end{scriptsize}

\begin{tiny}
\begin{flushright}
\url{http://ortuno.es/validar2.html}
\end{flushright}
\end{tiny}

\end{frame}


\end{document}

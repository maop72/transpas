
\documentclass[ucs]{beamer}

\usetheme{GSyC}
%\usebackgroundtemplate{\includegraphics[width=\paperwidth]{gsyc-bg.png}}


\usepackage[spanish]{babel}   
\usepackage[utf8x]{inputenc}
\usepackage{graphicx}
\usepackage{amssymb} % Simbolos matematicos
\usepackage{lmodern,textcomp}  % para usar el carácter € tal cual
%\usepackage{hthtml}
%\usepackage{html}



% Metadatos del PDF, por defecto en blanco, pdftitle no parece funcionar
   \hypersetup{%
     pdftitle={Librería request},
     %pdfsubject={Diseño y Administración de Sistemas y Redes},%
     pdfauthor={GSyC},%
     pdfkeywords={},%
   }
%


% Para colocar un logo en la esquina inferior de todas las transpas
%   \pgfdeclareimage[height=0.5cm]{gsyc-logo}{gsyc}
%   \logo{\pgfuseimage{gsyc-logo}}


% Para colocar antes de cada sección una página de recuerdo de índice
%\AtBeginSection[]{
%  \begin{frame}<beamer>{Contenidos}
%    \tableofcontents[currentframetitle]
%  \end{frame}
%}



\begin{document}

% Entre corchetes como argumento opcional un título o autor abreviado
% para los pies de transpa
\title[request]{Librería request}
%\subtitle{Diseño y Administración de Sistemas y Redes}
\author[GSyC]{Escuela Técnica Superior de Ingeniería de Telecomunicación\\
Universidad Rey Juan Carlos}
\institute{gsyc-profes (arroba) gsyc.urjc.es}
\date[2017]{Noviembre de 2017}


%% TÍTULO
\begin{frame}
  \titlepage
  % Oportunidad para poner otro logo si se usó la opción nologo
  % \includegraphics[width=2cm]{logoesp}  
\end{frame}




%% LICENCIA DE REDISTRIBUCIÓN DE LAS TRANSPAS
%% Nota: la opción b al frame le dice que justifique el texto
%% abajo (por defecto c: centrado)
\begin{frame}[b]
\begin{flushright}
{\tiny
\copyright \insertshortdate~\insertshortauthor \\
  Algunos derechos reservados. \\
  Este trabajo se distribuye bajo la licencia \\
  Creative Commons Attribution Share-Alike 4.0
}
\end{flushright}
\end{frame}



%% ÍNDICE
%\begin{frame}
%  \frametitle{Contenidos}
%  \tableofcontents
%\end{frame}

% $Id$
%

\section{request}

%%---------------------------------------------------------------
\begin{frame}[fragile]
\frametitle{request}
Para hacer peticiones http RESTful en JavaScript / node.js tenemos diversas opciones
\begin{itemize}
\item
http

Librería estándar, de bajo nivel
\item
request

Libería para hacer peticiones http, de más alto nivel que la 
librería http. No es estándar pero es muy popular

Para instalarla:

\verb|npm install request|

\item
Axios

Librería basada en promesas (programación asíncrona). Para el navegador
y para node.js

\item
SuperAgent
Similar a Axios 
\end{itemize}
\end{frame}

%%---------------------------------------------------------------
\begin{frame}[fragile]
\frametitle{}
Mediante consultas basadas en API a diversos servidores podemos construir
sistemas con una gran funcionalidad

\begin{itemize}
\item
Directorio de recursos web accesibles mediante API:

\url{https://www.programmableweb.com}

\item
Como ejemplo, vamos a consultar el tiempo actual y la predicción
que ofrece

\url{http://www.myweather2.com}
    \begin{itemize}
    \item

Necesitamos abrir una cuenta (gratuita), que nos proporcionará
un 
\emph{uac, unique access code}
con el que podremos hacer peticiones http RESTful
    \end{itemize}
\end{itemize}
\end{frame}

%%---------------------------------------------------------------
\begin{frame}[fragile]
\frametitle{Consultas a myweather2}
Al método 
\verb|request.get()| le pasamos
\begin{itemize}
\item
Una cadena con la URI del recurso. Incluye:

    \begin{itemize}
    \item
El uac
    \item
El parámetro 
\emph{output}, que
permite elegir el formato json o el format xml

    \item
El parámetro
\emph{query}, que indica 
latitud y longitud del punto a consultar
    \end{itemize}

\item
La función que manejará la respuesta. Con los parámetros

    \begin{itemize}
    \item
error

Valdrá null si la consulta tuvo éxito

    \item
response

Si hubo respuesta, este objeto contendrá todos los detalles. Muy bajo
nivel
    \item
body

Cuerpo de la respuesta, si la hubo
    \end{itemize}
\end{itemize}
\end{frame}


%%---------------------------------------------------------------
\begin{frame}[fragile]
\frametitle{}
  \begin{scriptsize}
  \begin{verbatim}
'use strict'
let request = require('request');
let uri='http://www.myweather2.com/developer/forecast.ashx?uac=XXXXXX
         &output=json&query=40.28276,-3.81912'

request.get(uri, function(error, response, body){
    console.log("error:");
    console.log(error);
    console.log("body:");
    console.log(body);
});
  \end{verbatim}
  \end{scriptsize}

\begin{tiny}
\begin{flushright}
\url{http://ortuno.es/request.js.txt}
\end{flushright}
\end{tiny}

\end{frame}

%%---------------------------------------------------------------
\begin{frame}[fragile]
\frametitle{}
Respuesta:

  \begin{scriptsize}
  \begin{verbatim}
error:
null
body:
{ "weather": { "curren_weather": [ {"humidity": "50", "pressure": "1020", 
"temp": "11", "temp_unit": "c", "weather_code": "1", "weather_text": 
"Partly cloudy",  "wind": [ {"dir": "W", "speed": "2", "wind_unit": "kph" } 
] } ],  "forecast": [ {"date": "2017-11-18",  "day": [ {"weather_code": "0",
 "weather_text": "Sunny skies",  "wind": [ {"dir": "ENE", "dir_degree": "63", 
"speed": "11", "wind_unit": "kph" } ] } ], "day_max_temp": "19",  "night":
 [ {"weather_code": "0", "weather_text": "Clear skies",  "wind": [ {"dir": 
"NE", "dir_degree": "35", "speed": "11", "wind_unit": "kph" } ] } ], 
"night_min_temp": "11", "temp_unit": "c" }, {"date": "2017-11-19",  "day":
 [ {"weather_code": "0", "weather_text": "Sunny skies",  "wind": [ {"dir": 
"E", "dir_degree": "86", "speed": "11", "wind_unit": "kph" } ] } ], 
"day_max_temp": "16",  "night": [ {"weather_code": "0", "weather_text": 
"Clear skies",  "wind": [ {"dir": "W", "dir_degree": "278", "speed": "4", 
"wind_unit": "kph" } ] } ], "night_min_temp": "9", "temp_unit": "c" } ] }}
  \end{verbatim}
  \end{scriptsize}

\end{frame}


\end{document}


\documentclass[ucs]{beamer}

\usetheme{GSyC}
%\usebackgroundtemplate{\includegraphics[width=\paperwidth]{gsyc-bg.png}}


\usepackage[greek,spanish]{babel}   %greek permite usar euro 
\usepackage[utf8x]{inputenc}
\usepackage{graphicx}
\usepackage{amssymb} % Simbolos matematicos


% Metadatos del PDF, por defecto en blanco, pdftitle no parece funcionar
   \hypersetup{%
     pdftitle={XML},%
     %pdfsubject={Diseño y Administración de Sistemas y Redes},%
     pdfauthor={GSyC},%
     pdfkeywords={},%
   }
%


% Para colocar un logo en la esquina inferior de todas las transpas
%   \pgfdeclareimage[height=0.5cm]{gsyc-logo}{gsyc}
%   \logo{\pgfuseimage{gsyc-logo}}


% Para colocar antes de cada sección una página de recuerdo de índice
%\AtBeginSection[]{
%  \begin{frame}<beamer>{Contenidos}
%    \tableofcontents[currentframetitle]
%  \end{frame}
%}



\begin{document}

% Entre corchetes como argumento opcional un título o autor abreviado
% para los pies de transpa
\title[Antecedentes de REST]{Antecedentes de REST: sockets, RPC, SOAP, WSDL}
%\subtitle{Diseño y Administración de Sistemas y Redes}
\author[GSyC]{Escuela Técnica Superior de Ingeniería de Telecomunicación\\
Universidad Rey Juan Carlos}
\institute{gsyc-profes (arroba) gsyc.urjc.es}
\date[2016]{Marzo de 2016}


%% TÍTULO
\begin{frame}
  \titlepage
  % Oportunidad para poner otro logo si se usó la opción nologo
  % \includegraphics[width=2cm]{logoesp}  
\end{frame}



%% LICENCIA DE REDISTRIBUCIÓN DE LAS TRANSPAS
%% Nota: la opción b al frame le dice que justifique el texto
%% abajo (por defecto c: centrado)
\begin{frame}[b]
\begin{flushright}
{\tiny
\copyright \insertshortdate~\insertshortauthor \\
  Algunos derechos reservados. \\
  Este trabajo se distribuye bajo la licencia \\
  Creative Commons Attribution Share-Alike 4.0\\
}
\end{flushright}  
\end{frame}



%% ÍNDICE
%\begin{frame}
%  \frametitle{Contenidos}
%  \tableofcontents
%\end{frame}



%the practice of system and network administration
%t limoncelli

%principles of network and system administration
%m burgess


\section{Antecedentes de REST}

%%---------------------------------------------------------------
\begin{frame}[fragile]
\frametitle{Antecedentes de REST}
La tecnología más habitual desde la década de 2010 para
intercomunicar aplicaciones en red (no aplicaciones con usuarios),
es REST

En el pasado, se emplearon diversos paradigmas
(que pueden seguir en uso hoy)
\begin{itemize}
\item
Sockets de red (\emph{network sockets})
\item
RPC. \emph{Remote Procedure Call}
\item
SOAP (\emph{Simple Object Access Protocol}) +
 WSDL (\emph{Web Services Description Language})
\end{itemize}

\end{frame}


%%---------------------------------------------------------------
\begin{frame}[fragile]
\frametitle{Network Sockets}
\begin{itemize}
\item
Los más difundidos son los sockets
Berkeley de BSD, en los años 80
\item
Su principal limitación es que son de muy bajo nivel, simplemente
envían y reciben datos entre dos \emph{end points}: dirección IP+Puerto
\item
Se siguen usando hoy cuando es necesario este acceso de bajo a nivel a la red
\item
Para desarrollar aplicaciones suelen resultar poco prácticos, obliga al programador
a \emph{reinventar la rueda} en todo lo relativo a comunicaciones
\end{itemize}
\end{frame}


%%---------------------------------------------------------------
\begin{frame}[fragile]
\frametitle{RPC}
RPC, \emph{Remote Procedure Call}
\begin{itemize}
\item
En los años 90 se popularizan arquitecturas más complejas que los sockets, basadas
en el paradigma RPC

\item
Se populariza el \emph{middleware}

Software de sistemas que proporciona al software de aplicaciones funcionalidad
por encima de la ofrecida por el SO. Facilita a los desarrolladores las tareas
de comunicación y entrada/salida

\begin{itemize}
\item
CORBA: \emph{Common Object Request Broker Architecture}, año 1991
\item
Java RMI, Remote Method Invocation
\item
DCOM (\emph{Distributed Component Object Model}). Protocolo
propietario de Microsoft, competidor de CORBA
\end{itemize}


\end{itemize}
\end{frame}
%%----------------------------------------------
\begin{frame}[fragile]
\frametitle{}
\begin{itemize}

\item
El programador invoca a funciones en máquinas remotas de manera prácticamente igual
a una llamada a una función local. Se supone que el programador puede olvidar
que trabaja en un sistema distribuido

\item
El propósito era ocultar al programador toda la complejidad propia de la red:
(qué recursos hay, dónde están, cómo se accede, qué sucede si fallan)



\item
Todo esto que en principio parece muy conveniente, cae en desuso en la década de 2000
\end{itemize}
\end{frame}


%%---------------------------------------------------------------
\begin{frame}[fragile]
\frametitle{Inconvenientes del modelo RPC}
\begin{itemize}
\item
Las red es compleja. Ocultar esta complejidad inicialmente hace que programar sea sencillo,
pero el software resultante resulta frágil
% mejor que el fallo lo maneje el software de aplicación, cerca
% de la lógica de negocio. No una capa intermedia de software de sistemas
\item
Localizar una función local
es trivial, una función remota, no
\item
Las llamadas locales raramente fallan. Las remotas, a menudo
\item
Las diferencias de latencia son típicamente de varios órdenes de magnitud...
\item
Las tecnologías RPC son muy heterogéneas

\item
Los atributos de las clases normalmente se exportan
directamente, hay un acople muy fuerte entre el lenguaje de programación
y el interfaz que se expone

\end{itemize}
\end{frame}
%%----------------------------------------------
\begin{frame}[fragile]
\frametitle{Inconvenientes del modelo RPC (2)}
\begin{itemize}
\item
Se exportan tipos de datos complejos, propios del modelo de negocio. Proclives a cambiar, generando
incompatibilidad
\item
Los sistemas resultan fuertemente acoplados: en ocasiones es necesario que todo el sistema
emplee el mismo lenguaje de programación, un simple cambio de sistema operativo
puede llegar a ser conflictivo
\item
Mala integración con cortafuegos y otros dispositivos de seguridad en red
\item

Supongamos paradigma cliente-servidor (el más habitual)

\begin{itemize}
\item
El cliente tiene que saber exactamente a qué funciones llamar, con qué parámetros
de entrada y con qué valores de retorno
\item
Típicamente, un pequeño cambio en el cliente o el servidor obliga a cambiar también
el servidor o el cliente
\end{itemize}

\end{itemize}
% webber también es muy crítico con WSDL (pg383) . Lo llama un modelo centrado en métodos
\end{frame}


%%---------------------------------------------------------------
\begin{frame}[fragile]
\frametitle{Web services centradas en métodos}

En los años 2000 aparece la primera generación del paradigma \emph{web service}
con tecnologías como XML-RPC, SOAP y WSDL

Tienen ventajas importantes sobre RPC:
\begin{itemize}
\item
Están basados en HTTP, una de las causas fundamentales del éxito del web,
por ser un estándar sencillo y práctico para el intercambio de documentos
entre máquina y persona

\begin{itemize}
\item
Un \emph{web service} usa el mismo principio y protocolo para la comunicación
máquina-máquina
\end{itemize}

\item
Están basados en XML, un estándar de datos

\end{itemize}

\end{frame}

%%---------------------------------------------------------------
\begin{frame}[fragile]
\frametitle{}
\begin{itemize}
\item
Las ventajas de esta primera generación de web services
consiguen reducir mucho el acoplamiento con el sistema operativo y lenguaje
de programación
\item
La integración con cortafuegos y otros dispositivos similares es muy
sencilla, pues resulta tráfico web a estos efectos
\item
Pero el acoplamiento cliente-servidor sigue siendo fuerte, el cliente tiene que saber
exactamente a qué función llamar, con qué parámetros y qué valores recibir
\end{itemize}

\end{frame}

\section{SOAP}

%%---------------------------------------------------------------
\begin{frame}[fragile]
\frametitle{SOAP}
SOAP (\emph{Simple Object Access Protocol}) es un protocolo que permite la intercomunicación
máquina-maquina,
mediante llamadas a métodos

\begin{itemize}
\item
Desarrollado en 1998 por Dave Winer y otros, con respaldo de Microsoft
\item
SOAP comenzó siendo un protocolo privado de Microsoft, lo que provocó
la aparición de un protocolo muy similar, pero libre, XML-RPC

\item
Normalmente emplea HTTP, aunque también soporta otros protocolos como SMTP y RSS

\end{itemize}
\end{frame}
%%----------------------------------------------
\begin{frame}[fragile]
\frametitle{}
\begin{itemize}
\item
SOAP intercambia documentos XML, tanto para los distintos métodos como para los datos
\item
Los métodos no forman parte de SOAP, sino que los define el programador 
de la aplicación 
\item
Similar a RPC, pero más sencillo al basarse en mensajes http
\item
Muy fácil integración con proxies y cortafuegos
\item

SOAP especifica exactamente cómo codificar las cabeceras HTTP y los documentos XML para
intercambiar toda la información
\end{itemize}
\end{frame}


%%---------------------------------------------------------------
\begin{frame}[fragile]
\frametitle{Mensajes SOAP}
Un mensaje SOAP es un documento XML ordinario que contiente:
\begin{itemize}
\item
Un \emph{sobre} (\emph{envelope}) que identifica el documento XML como mensaje SOAP
\item
Opcionalmente, una cabecera
\item
El cuerpo, con información sobre las llamadas y las respuestas
\item
Opcionalmente, un elemento \emph{fault} con información sobre errores
\end{itemize}

\end{frame}



\section{WSDL}
%%---------------------------------------------------------------
\begin{frame}[fragile]
\frametitle{WSDL}
Web Services Description Language (WSDL) es un lenguaje basado en XML para describir la 
funcionalidad
ofrecida por un servicio web de primera generación, normalmente SOAP

\begin{itemize}
\item
v1.0 año 2000, desarrollada por IBM y Microsoft
\item
v1.1 año 2001, muy similar a 1.1. 
Adoptada por el W3C (\emph{World Wide Web Consortium}).
La más popular
\item
v2.0 año 2007. No muy extendida
\end{itemize}

En su momento (sobre el año 2000) SOAP + WSDL fueron técnicas disruptivas.

Pero resultan protocolos bastante complejos,
es necesario especificar muchos detalles
sobre los métodos a invocar

\end{frame}


%%---------------------------------------------------------------
\begin{frame}[fragile]
\frametitle{Componentes de WSDL}
%https://en.wikipedia.org/wiki/Web_Services_Description_Language
\begin{itemize}

\item
Servicio

Contiene un conjunto de funciones que se exponen a http
\item
Puerto

Dirección donde se ubica el servicio. Típicamente es una cadena con una URL http.
El puerto es siempre el mismo para cualquier llamada
\item
Vínculo (Binding)

Enlace entre la especificación RPC y la especificación SOAP



\end{itemize}
\end{frame}
%%----------------------------------------------
\begin{frame}[fragile]
\frametitle{Componentes de WSDL (2)}
\begin{itemize}
\item
PortType

Define un servicios web, las operaciones que se pueden realizar y los mensajes empleados
para realizar las operaciones
\item
Operation

Equivalente a una función de un lenguaje tradicional
\item
Message

Parámetros de la operación
\item
Types

Descripción de los tipos de los parámetros
\end{itemize}

\end{frame}


%%---------------------------------------------------------------
\begin{frame}[fragile]

  \begin{scriptsize}
  \begin{verbatim}
<?xml version="1.0" encoding="UTF-8"?>
<description xmlns="http://www.w3.org/ns/wsdl"
             xmlns:tns="http://www.tmsws.com/wsdl20sample"
             xmlns:whttp="http://schemas.xmlsoap.org/wsdl/http/"
             xmlns:wsoap="http://schemas.xmlsoap.org/wsdl/soap/"
             targetNamespace="http://www.tmsws.com/wsdl20sample">

<documentation>
    This is a sample WSDL 2.0 document.
</documentation>

<!-- Abstract type -->
   <types>
      <xs:schema xmlns:xs="http://www.w3.org/2001/XMLSchema"
                xmlns="http://www.tmsws.com/wsdl20sample"
                targetNamespace="http://www.example.com/wsdl20sample">

         <xs:element name="request"> ... </xs:element>
         <xs:element name="response"> ... </xs:element>
      </xs:schema>
   </types>
  \end{verbatim}
  \end{scriptsize}

\end{frame}




%%---------------------------------------------------------------
\begin{frame}[fragile]
\frametitle{}

  \begin{scriptsize}
  \begin{verbatim}
<!-- Abstract interfaces -->
   <interface name="Interface1">
      <fault name="Error1" element="tns:response"/>
      <operation name="Get" pattern="http://www.w3.org/ns/wsdl/in-out">
         <input messageLabel="In" element="tns:request"/>
         <output messageLabel="Out" element="tns:response"/>
      </operation>
   </interface>

<!-- Concrete Binding Over HTTP -->
   <binding name="HttpBinding" interface="tns:Interface1"
            type="http://www.w3.org/ns/wsdl/http">
      <operation ref="tns:Get" whttp:method="GET"/>
   </binding>
  \end{verbatim}
  \end{scriptsize}

\end{frame}

%%---------------------------------------------------------------
\begin{frame}[fragile]
\frametitle{}


  \begin{scriptsize}
  \begin{verbatim}
<!-- Concrete Binding with SOAP-->
   <binding name="SoapBinding" interface="tns:Interface1"
            type="http://www.w3.org/ns/wsdl/soap"
            wsoap:protocol="http://www.w3.org/2003/05/soap/bindings/HTTP/"
            wsoap:mepDefault="http://www.w3.org/2003/05/soap/mep/request-response">
      <operation ref="tns:Get" />
   </binding>

<!-- Web Service offering endpoints for both bindings-->
   <service name="Service1" interface="tns:Interface1">
      <endpoint name="HttpEndpoint"
                binding="tns:HttpBinding"
                address="http://www.example.com/rest/"/>
      <endpoint name="SoapEndpoint"
                binding="tns:SoapBinding"
                address="http://www.example.com/soap/"/>
   </service>
</description>
  \end{verbatim}
  \end{scriptsize}


  \begin{scriptsize}
  \begin{flushright}
Fuente: wikipedia
  \end{flushright}
  \end{scriptsize}

\end{frame}


%%---------------------------------------------------------------
\begin{frame}[fragile]
\frametitle{}

\begin{table}[htbp]\center\footnotesize
  \begin{tabular}[h]{|l|l|l|}\hline

 & \bf{Acoplamiento }&  \bf{Acoplamiento}   \\ 
 & \bf{con la plataforma }&  \bf{cliente-servidor}   \\ \hline \hline

Sockets & Muy fuerte & Muy fuerte\\ \hline
RPC & Fuerte (middleware) & Fuerte\\ \hline
Web services 1ª gen. &     &    \\ 
(SOAP, XML-RPC, WSDL) & Débil     & Fuerte    \\ \hline
Web services 2ª gen. &    & \\ 
(REST) & Débil  & Débil\\ \hline


  \end{tabular}
\end{table}



\end{frame}



\end{document}

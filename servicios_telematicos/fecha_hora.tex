
\documentclass[ucs]{beamer}

\usetheme{GSyC}
%\usebackgroundtemplate{\includegraphics[width=\paperwidth]{gsyc-bg.png}}


\usepackage[greek,spanish]{babel}   %greek permite usar euro 
\usepackage[utf8x]{inputenc}
\usepackage{graphicx}
\usepackage{amssymb} % Simbolos matematicos


% Metadatos del PDF, por defecto en blanco, pdftitle no parece funcionar
   \hypersetup{%
     pdftitle={XML},%
     %pdfsubject={Diseño y Administración de Sistemas y Redes},%
     pdfauthor={GSyC},%
     pdfkeywords={},%
   }
%


% Para colocar un logo en la esquina inferior de todas las transpas
%   \pgfdeclareimage[height=0.5cm]{gsyc-logo}{gsyc}
%   \logo{\pgfuseimage{gsyc-logo}}


% Para colocar antes de cada sección una página de recuerdo de índice
%\AtBeginSection[]{
%  \begin{frame}<beamer>{Contenidos}
%    \tableofcontents[currentframetitle]
%  \end{frame}
%}



\begin{document}

% Entre corchetes como argumento opcional un título o autor abreviado
% para los pies de transpa
\title[Fecha y hora en internet ]{Fecha y hora en internet}
%\subtitle{Diseño y Administración de Sistemas y Redes}
\author[GSyC]{Escuela Técnica Superior de Ingeniería de Telecomunicación\\
Universidad Rey Juan Carlos}
\institute{gsyc-profes (arroba) gsyc.urjc.es}
\date[2016]{Abril de 2016}


%% TÍTULO
\begin{frame}
  \titlepage
  % Oportunidad para poner otro logo si se usó la opción nologo
  % \includegraphics[width=2cm]{logoesp}  
\end{frame}



%% LICENCIA DE REDISTRIBUCIÓN DE LAS TRANSPAS
%% Nota: la opción b al frame le dice que justifique el texto
%% abajo (por defecto c: centrado)
\begin{frame}[b]
\begin{flushright}
{\tiny
\copyright \insertshortdate~\insertshortauthor \\
  Algunos derechos reservados. \\
  Este trabajo se distribuye bajo la licencia \\
  Creative Commons Attribution Share-Alike 4.0\\
}
\end{flushright}  
\end{frame}



%% ÍNDICE
%\begin{frame}
%  \frametitle{Contenidos}
%  \tableofcontents
%\end{frame}



%the practice of system and network administration
%t limoncelli

%principles of network and system administration
%m burgess


\section{Fecha y Hora}
%%---------------------------------------------------------------
\begin{frame}[fragile]
\frametitle{datetime}

El lenguaje python tiene un tipo de datos muy conveniente para
el manejo de fechas: datetime

  \begin{footnotesize}
  \begin{verbatim}
#!/usr/bin/python -tt
# -*- coding: utf-8 -*-
import datetime
def main():
    dt=datetime.datetime.now()
    print dt # 2016-03-05 17:39:15.900732
    print dt.year, # 2015
    print dt.month, # 3
    print dt.day, # 5
    print dt.hour, # 17
    print dt.minute, # 39
    print dt.second, # 15
    print dt.microsecond # 900732
if __name__ == "__main__":
    main()
  \end{verbatim}
  \end{footnotesize}
\end{frame}


%%---------------------------------------------------------------
\begin{frame}[fragile]
\frametitle{}

  \begin{footnotesize}
  \begin{verbatim}
#!/usr/bin/python -tt
# -*- coding: utf-8 -*-
import datetime
def main():
    dt1=datetime.datetime(2016,1,1,0,0,0)
    print dt1  # 2016-01-01 00:00:00

    dt2=datetime.datetime.now()
    print dt2  # 2016-03-05 17:51:45.604391

    diferencia= dt2-dt1
    print diferencia  # 64 days, 17:51:45.604391
    print diferencia.total_seconds() # 5593905.60439

if __name__ == "__main__":
    main()
  \end{verbatim}
  \end{footnotesize}

\end{frame}


%%---------------------------------------------------------------
\begin{frame}[fragile]
\frametitle{timestamp}

El tipo de datos datetime es propio del lenguaje python. En otros
lenguajes y en las librerías de Unix y de Windows lo habitual es usar 
\emph{timestamp}, que es el número de segundos transcurridos desde
un instante inicial denominado \emph{epoch}

\begin{itemize}
\item
En Unix el \emph{epoch} es el 1 de enero de 1970 a las 00:00 UTC
\item
En Windows el \emph{epoch} es el 1 de enero de 1601 a las 00:00 UTC
\end{itemize}

El \emph{Unix time} es el \emph{timestamp} basado en el \emph{epoch} Unix


\end{frame}
%%----------------------------------------------
\begin{frame}[fragile]
\frametitle{}

Timestamp es el formato en que se almacenan y procesan las fechas, cuando
se tienen que presentar a las personas se emplea algún formato más legible

\begin{itemize}
\item
El problema es que los formatos humanos para representar fechas son muy
heterogéneos
%#https://en.wikipedia.org/wiki/Date_format_by_country

\begin{itemize}
\item
Dia, mes y año en Europa, parte de Asia, norte de Africa, América Central,
América del Sur
\item
Mes, día y año en Estados Unidos
\item
Años, mes y día en China
\end{itemize}
\item
Una alternativa normalizada es el ISO8601, similar al formato chino

2016-03-05T06:24:57+00:00

\end{itemize}
\end{frame}


%%---------------------------------------------------------------
\begin{frame}[fragile]
\frametitle{}
\begin{itemize}
\item
En python disponemos de la librería \emph{time} para trabajar con 
\emph{timestamps}
\item
Pero \emph{datetime} normalmente resulta más conveniente. Cuando nuestro
programa tenga que aceptar y generar 
\emph{timestamps} es recomendable

\begin{itemize}
\item
En la entrada, convertir el 
\emph{timestamp}
en 
\emph{datetime}
\item
Procesar el 
\emph{datetime}
\item
En la salida, generar o bien un 
\emph{timestamp}
o bien algún formato legible para los humanos, según nuestros requisitos
\end{itemize}

\end{itemize}

\end{frame}



%%---------------------------------------------------------------
\begin{frame}[fragile]
\frametitle{}

  \begin{footnotesize}
  \begin{verbatim}
#!/usr/bin/python -tt
# -*- coding: utf-8 -*-
import time,datetime
def main():
    # Obtenemos la hora actual, como timestamp
    ts=time.time()
    print ts    # 1457197258.88

    # Convertimos el timestamp a datetime naive
    dt=datetime.datetime.utcfromtimestamp(ts)
    print dt  # 2016-03-05 18:00:58.881041

    # Convertimos el datetime naive de nuevo a timestamp
    ts=time.mktime( dt.timetuple())
    print ts # 1457197258.0

if __name__ == "__main__":
    main()
  \end{verbatim}
  \end{footnotesize}

\end{frame}


%%---------------------------------------------------------------
\begin{frame}[fragile]
\frametitle{Zonas horarias}
En cualquier aplicación que funcione en internet es muy importante
tener en cuenta la zona horaria
\begin{itemize}
\item
Los sistemas operativos que usamos actualmente y la mayoría de los lenguajes
de programación actuales son anteriores a internet
\item
Una aplicación tradicional suele
ejecutarse en algún lugar del mundo en concreto,  con lo que no es especialmente
importante especificar la zona horaria. Suele sobreentender \emph{hora local}
\item
En internet no existe la \emph{hora local}. Es imprescindible indicar siempre
la zona horaria
\item
En la actualidad, cualquier representación de una hora que no especifique
claramente su zona horaria, es ambigüa y muy desaconsejable

\end{itemize}
\end{frame}



%%---------------------------------------------------------------
\begin{frame}[fragile]
\frametitle{}
\begin{itemize}

\item
Las horas locales son muy complejas, es necesario tener en cuenta no solo el 
huso horario sino la presencia o ausencia del horario de verano
(adelantar una hora a principios de la primavera y se retrasarla en otoño)

\item
Lo más recomendable es almacenar y procesar todas las fechas en UTC
\emph{Coordinated Universal Time}

\begin{itemize}
\item
Esta hora es casi idéntica a la antigua GMT,
\emph{Greenwich Mean Time}
\end{itemize}

\item
Cualquier programa debería usar siempre horas UTC, y convertirla a hora
local solo inmediatamente antes de presentarla a un humano

\end{itemize}

\end{frame}



%%---------------------------------------------------------------
\begin{frame}[fragile]
\frametitle{}
\begin{itemize}
\item
El tipo de datos 
\emph{timestamp}
no requiere indicar cuál es su zona horaria, se trata de los segundos
transcurridos desde el epoch, y el epoch siempre es UTC

\item
Con el tipo \emph{datetime} sí es necesario indicar zona horaria

\end{itemize}

\end{frame}





%%---------------------------------------------------------------
\begin{frame}[fragile]
\frametitle{pytz}
\begin{itemize}
\item
En python, el formato datetime tradicional no incluye información sobre
la zona horaria
\item
Actualmente se le denomina
\emph{naive datetime} (datetime ingenuo)
\item
En un datetime naive, es necesario sobreentender si se trata de una hora local
o una hora UTC

\begin{itemize}
\item
Obviamente, esto es muy problemático
\end{itemize}

\item
El soporte para zonas horarias en python no es muy bueno. Es recomendable usar
la librería pytz, que no está incluida en la distribución estándar de python,
es necesario instalarla

\verb|pip install pytz|

\end{itemize}
\end{frame}


%%---------------------------------------------------------------
\begin{frame}[fragile]
\frametitle{}
\begin{itemize}
\item
El método strftime del tipo datetime admite una cadena de formato, con
la que podemos indicar que el datetime se muestre indicando zona horaria

  \begin{footnotesize}
  \begin{verbatim}
    fmt = "%Y-%m-%d %H:%M:%S %Z%z"
    print dt.strftime(fmt) # 2016-03-05 19:47:53 UTC+0000
  \end{verbatim}
  \end{footnotesize}


\begin{itemize}
\item
Aunque si el datetime es naive, no mostrará ninguna zona 
\end{itemize}

\item
Un objeto pytz contiene la localización de una zona horaria

  \begin{footnotesize}
  \begin{verbatim}
    madrid=pytz.timezone("Europe/Madrid")
  \end{verbatim}
  \end{footnotesize}


\end{itemize}
\end{frame}


%%---------------------------------------------------------------
\begin{frame}[fragile]
\frametitle{}

  \begin{footnotesize}
  \begin{verbatim}
#!/usr/bin/python -tt
# -*- coding: utf-8 -*-
import datetime,pytz
def main():
    usa_eastern=pytz.timezone("US/Eastern")
    utc=pytz.utc
    fmt = "%Y-%m-%d %H:%M:%S %Z%z"
 
    dt=datetime.datetime(2016,7,4,tzinfo=usa_eastern)
    print dt.strftime(fmt) # 2016-07-04 00:00:00 EST-0500

    dt=datetime.datetime(2016,10,12,12,0, tzinfo=utc)
    print dt.strftime(fmt) # 2016-10-12 12:00:00 UTC+0000
  
if __name__ == "__main__":
    main()
  \end{verbatim}
  \end{footnotesize}

\end{frame}


%%----------------------------------------------
\begin{frame}[fragile]
\frametitle{}
\begin{itemize}


\item
A partir de un datetime naive, podemos añadirle zona horaria con el método 
\emph{localize}
del objeto
\emph{pytz}

  \begin{footnotesize}
  \begin{verbatim}
    dt=madrid.localize(dt)
    print dt.strftime(fmt) # 2016-03-05 20:47:53 CET+0100
  \end{verbatim}
  \end{footnotesize}

\item
Cuando un datetime tiene zona horaria, podemos obtener otro datetime de diferente zona con el método
\emph{astimezone}
del objeto
\emph{pytz}

  \begin{footnotesize}
  \begin{verbatim}
    dt=dt.astimezone(utc)
    print dt.strftime(fmt) # 2016-03-05 19:47:53 UTC+0000
  \end{verbatim}
  \end{footnotesize}

\item
Observa que 
\emph{localize} es un método del 
\emph{timezone} 
que recibe un 
\emph{datetime}, 
mientras que 
\emph{astimezone} es un método del 
\emph{datetime}
que recibe un 
\emph{timezone} 


\end{itemize}

\end{frame}


%%---------------------------------------------------------------
\begin{frame}[fragile]

  \begin{footnotesize}
  \begin{verbatim}
import datetime,pytz
def main():
    madrid=pytz.timezone("Europe/Madrid")
    pekin=pytz.timezone("Asia/Shanghai")
    utc=pytz.utc

    dt=datetime.datetime.now()
    fmt = "%Y-%m-%d %H:%M:%S %Z%z"

    # dt naive, sin información de zona horaria
    print dt.strftime(fmt) # 2016-03-05 20:47:53

    # Añadimos zona horaria
    dt=madrid.localize(dt)
    print dt.strftime(fmt) # 2016-03-05 20:47:53 CET+0100

    # Cambiamos la zona horaria
    dt=dt.astimezone(pekin)
    print dt.strftime(fmt) # 2016-03-06 03:47:53 CST+0800

    dt=dt.astimezone(utc)
    print dt.strftime(fmt) # 2016-03-05 19:47:53 UTC+0000

  \end{verbatim}
  \end{footnotesize}

\end{frame}



%%---------------------------------------------------------------
%\begin{frame}[fragile]
%\frametitle{Incompatibilidad pytz - time.mktime()}
%Un datetime con zona horaria no es un tipo nativo de python, no 
%todas las librerías lo soportan
%\begin{itemize}
%\item
%Supongamos que tenemos un datetime, y queremos pasarlo a timestamp
%con time.mktime()
%
%\begin{itemize}
%\item
%Si le ponemos zona con localize y luego lo convertimos en timestamp, se comporta
%correctamente
%\item
%Pero si le volvemos a cambiar la zona con astimezone() e intentamos convertirlo
%en timestamp, el valor se corrompe
%\end{itemize}
%\end{itemize}
%\end{frame}
%
%
%%%---------------------------------------------------------------
%\begin{frame}[fragile]
%\frametitle{}
%Ejemplo:
%
%  \begin{scriptsize}
%  \begin{verbatim}
%def compara(original,current):
%    if original==current:
%        print "iguales"
%    else:
%         print "distintos"
%    return
%
%def main():
%    ts=int(time.time())
%    original=ts
%    dt=datetime.datetime.fromtimestamp(ts)
%
%    dt=moscu.localize(dt)
%    ts=time.mktime( dt.timetuple())
%    compara(original,ts)  # iguales
%
%    dt=dt.astimezone(utc)
%    ts=time.mktime( dt.timetuple())
%    compara(original,ts)  # distintos
%
%    dt=dt.astimezone(moscu)
%    ts=time.mktime( dt.timetuple())
%    compara(original,ts)  # iguales
%    return
%  \end{verbatim}
%  \end{scriptsize}
%
%\end{frame}



%%---------------------------------------------------------------
%\begin{frame}[fragile]
%\frametitle{}
%Solución
%\begin{itemize}
%\item
%La hora que nuestro programa recibe, convertirla cuanto antes en datetime UTC
%\item
%Guardar y procesar siempre el datetime UTC
%\item
%Convertir el datetime UTC en el último momento al formato/zona requerido
%\begin{itemize}
%\item
%Y  no volver a cambiar este formato/zona
%\end{itemize}
%
%
%\end{itemize}
%\end{frame}
%%----------------------------------------------
%\begin{frame}[fragile]
%\frametitle{}
%Esto es, hacer esto:
%  \begin{footnotesize}
%  \begin{verbatim}
%                                  |--> datetime de otra zona
%cualquier cosa  --> datetime utc -|
%                                  |--> timestamp
%  \end{verbatim}
%  \end{footnotesize}
%
%NUNCA hacer esto:
%
%  \begin{footnotesize}
%  \begin{verbatim}
%cualquier cosa --> datetime zona 1 --> datetime zona 2--> timestamp
%  \end{verbatim}
%  \end{footnotesize}
%
%\end{frame}



%%---------------------------------------------------------------
\begin{frame}[fragile]
\frametitle{Conversión datetime con zona en timestamp}
A partir de un datetime, podemos necesitar convertirlo en un timestamp
\begin{itemize}
\item
Si el datetime tiene zona horaria, ya no es aplicable \emph{time.mktime()}
  \begin{footnotesize}
  \begin{verbatim}
  \end{verbatim}
time.mktime( dt.timetuple())
  \end{footnotesize}

\item
%Debemos usar el método \emph{utctimetimple()} del módulo \emph{calendar}
Debemos usar el método \emph{timegm()} del módulo \emph{calendar}

  \begin{footnotesize}
  \begin{verbatim}
import calendar 

ts=calendar.timegm(dt.utctimetuple())
  \end{verbatim}
  \end{footnotesize}


\end{itemize}

\end{frame}

% receta mágica encontrada en
%http://mostlyhighperformance.blogspot.com.es/2015/03/date-time-and-datetime-conversions.html


%%---------------------------------------------------------------
\begin{frame}[fragile]
\frametitle{Lista completa de zonas horarias}

  \begin{footnotesize}
  \begin{verbatim}
#!/usr/bin/python -tt
import pytz
def main():
    for tz in pytz.all_timezones:
        print tz
if __name__ == "__main__":
    main()
  \end{verbatim}
  \end{footnotesize}

Resultado:
  \begin{tiny}
  \begin{verbatim}
Africa/Abidjan
Africa/Accra
Africa/Addis_Ababa
Africa/Algiers
Africa/Asmara
[...]
US/Mountain
US/Pacific
US/Pacific-New
US/Samoa
UTC
Universal
W-SU
WET
Zulu
  \end{verbatim}
  \end{tiny}

\end{frame}



\end{document}

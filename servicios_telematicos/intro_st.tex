
\documentclass[ucs]{beamer}

\usetheme{GSyC}
%\usebackgroundtemplate{\includegraphics[width=\paperwidth]{gsyc-bg.png}}


\usepackage[greek,spanish]{babel}   %greek permite usar euro 
\usepackage[utf8x]{inputenc}
\usepackage{graphicx}
\usepackage{amssymb} % Simbolos matematicos


% Metadatos del PDF, por defecto en blanco, pdftitle no parece funcionar
   \hypersetup{%
     pdftitle={Introducción a los Servicios Telemáticos},%
     %pdfsubject={Diseño y Administración de Sistemas y Redes},%
     pdfauthor={GSyC},%
     pdfkeywords={},%
   }
%


% Para colocar un logo en la esquina inferior de todas las transpas
%   \pgfdeclareimage[height=0.5cm]{gsyc-logo}{gsyc}
%   \logo{\pgfuseimage{gsyc-logo}}


% Para colocar antes de cada sección una página de recuerdo de índice
%\AtBeginSection[]{
%  \begin{frame}<beamer>{Contenidos}
%    \tableofcontents[currentframetitle]
%  \end{frame}
%}



\begin{document}

% Entre corchetes como argumento opcional un título o autor abreviado
% para los pies de transpa
\title[Introducción a los Servicios Telemáticos]{Introducción a los Servicios Telemáticos }
%\subtitle{Diseño y Administración de Sistemas y Redes}
\author[GSyC]{Escuela Tec. Sup. Ingeniería Telecomunicación}
\institute{gsyc-profes (arroba) gsyc.es}
\date[2016]{Enero de 2016}


%% TÍTULO
\begin{frame}
  \titlepage
  % Oportunidad para poner otro logo si se usó la opción nologo
  % \includegraphics[width=2cm]{logoesp}  
\end{frame}



%% LICENCIA DE REDISTRIBUCIÓN DE LAS TRANSPAS
%% Nota: la opción b al frame le dice que justifique el texto
%% abajo (por defecto c: centrado)
\begin{frame}[b]
\begin{flushright}
{\tiny
\copyright \insertshortdate~\insertshortauthor \\
  Algunos derechos reservados. \\
  Este trabajo se distribuye bajo la licencia \\
  Creative Commons Attribution Share-Alike 4.0\\
}
\end{flushright}  
\end{frame}



%% ÍNDICE
%\begin{frame}
%  \frametitle{Contenidos}
%  \tableofcontents
%\end{frame}



%the practice of system and network administration
%t limoncelli

%principles of network and system administration
%m burgess


%%---------------------------------------------------------------
\begin{frame}[fragile]
\frametitle{Convenciones empleadas}
En español, la conjunción disyuntiva \emph{o} tiene dos significados opuestos
\begin{itemize}
\item
Diferencia, separación, términos contrapuestos

\emph{carne o pescado, blanco o negro}
\item
Equivalencia

%\emph{profesor o docente}
%\emph{alumno o estudiante}, 
\emph{alquiler o arrendamiento},
\emph{arreglar o reparar}
\end{itemize}
Para evitar esta ambigüedad, en esta asignatura usaremos dos siglas muy
comunes en inglés

\begin{itemize}
\item
Para indicar oposición, 
\emph{versus},  
abreviada como
\emph{vs}  


p.e. \emph{comprar vs alquilar}
\item
Para indicar equivalencia, \emph{also known as}, abreviado \emph{aka}

p.e.
el rey 
 \emph{aka}
el monarca

\end{itemize}

\end{frame}



\section{Definiciones}
%%---------------------------------------------------------------
\begin{frame}[fragile]
\frametitle{Definicición de Servicio Telemático}
\begin{itemize}
\item
Según la RAE, servicio es \emph{Organización y personal destinados a cuidar intereses o satisfacer necesidades del público o de alguna entidad oficial o privada. Servicio de correos, de incendios, de reparaciones}

Reemplazando \emph{organización y personal} por \emph{software}, y \emph{entidad} por \emph{sistema informático},
nos puede servir como aproximación

\item
Telemática=TELEcomunicaciones+inforMÁTICA

\item
En esta asignatura el término \emph{servicio telemático} será prácticamente equivalente a 
servicio web (\emph{web service}) o simplemente \emph{servicio}

\end{itemize}

\end{frame}


%%---------------------------------------------------------------
\begin{frame}[fragile]
\frametitle{Definición de web service}
Según el W3C (\emph{ World Wide Web Consortium }), un servicio web es 
\emph{un sistema software diseñado para soportar interacción máquina-máquina sobre una 
red de comunicaciones}

Hay dos categorías principales
\begin{itemize}
\item
Servicios REST (\emph{representational state transfer}) , donde el objetivo principal del servicio
es manipular representaciones de recursos web mediante a un conjunto uniforme de operaciones sin estado
\item
Servicios de cualquier otro tipo, donde el servicio expone un conjunto arbitrario de operaciones

P.e. RPC (\emph{Remote Procedure Call}), XML-RPC, SOAP (\emph{Simplified Object Access Protocol})

\end{itemize}
Por tanto, un servicio web es una clase particular de API (\emph{Application programming interface})

\end{frame}


%%---------------------------------------------------------------
\begin{frame}[fragile]
\frametitle{}
El concepto de \emph{servicio web} está muy ligado con el de \emph{Rich Internet Application, RIA}
\begin{itemize}
\item
Enfoque web tradicional:

Toda la lógica de la aplicación está en el servidor. El web está centrado en el humano
\item
Enfoque RIA

El servidor proporciona datos al cliente, y almacena datos del cliente. Entre servidor y cliente hay API. 
El cliente ofrece la lógica del negocio al humano
\end{itemize}


\end{frame}


%%---------------------------------------------------------------
%\begin{frame}[fragile]
%\frametitle{}
%\begin{itemize}
%\item
%El API de un servicio web es independiente del leguaje de programación y del sistema operativo
%\item
%El servicio web normalmente emplea WSDL (\emph{Web Services Description Language}), una
%definición en lenguaje XML que describe la funcionalidad ofrecida por el servicio
%
%\end{itemize}
%
%\end{frame}



\section{Falacias de la computación distribuida}
%%---------------------------------------------------------------
\begin{frame}[fragile]
\frametitle{Las falacias de la computación distribuida}
Fallacies of distributed computing, Peter Deutsch, 1994

Los desarrolladores de aplicaciones distribuidas con frecuencia parten
de las siguientes premisas, que resultan ser falsas
\begin{enumerate}
\item
La red es fiable
\item
La latencia es nula
\item
El ancho de banda es infinito
\item
La red es segura
\item
La topología no cambia
\item
Hay un administrador único en la red
\item
El coste de las comunicaciones es nulo
\item
La red es homogenea

\end{enumerate}

\end{frame}


%%---------------------------------------------------------------
\begin{frame}[fragile]
\frametitle{La red no es fiable}
\begin{itemize}
\item

No basta con considerar los elementos de red a bajo nivel

Cualquier fallo hardware o software, de cualquier entidad participante en el proceso,
provocará un fallo en la red

\end{itemize}

\end{frame}



%%---------------------------------------------------------------
\begin{frame}[fragile]
\frametitle{La latencia no es nula}

La definición exacta de \emph{latencia} depende del contexto
\begin{itemize}
\item
En un disco duro, es el tiempo que tarda el cabezal en posicionarse en el sector
requerido

Por tanto, es un tiempo muy elevado para el primer bit, nulo para los siguientes
(del mismo sector)
\item
En el nivel de red, p.e. una red de conmutación de paquetes, el tiempo desde que comienza el envío
de un paquete hasta que comienza su recepción. Es frecuente que la latencia del 
primer paquete sea superior a la de paquetes sucesivos
\item
En el nivel físico, p.e. un cable de fibra óptica, es el tiempo que tarda 1 bit
desde que se envía hasta que se recibe
% por tanto no hay que considerar el acceso al medio
%\item
%En simuladores, es el tiempo que transcurre entre la primera entrada 
% En realidad virtual y realidad aumentada, la diferencia de tiempo
% entre un objeto real y su correspondiente virtual

\end{itemize}

% en el último ejemplo (cable), la latencia es muy homogénea,
% en el segundo, menos. p.e. con datagramas no tendría por qué pasar,
% pero pasa
% en el primer ejemplo, hay una diferencia enorme entre primer bit y resto

\end{frame}


%%---------------------------------------------------------------
\begin{frame}[fragile]
\frametitle{}
\begin{itemize}
\item
La latencia suele ser bastante buena en LAN, pero empeora mucho en WAN
\item
Hay mucha diferencia entre accesos locales y accesos LAN
\item
Mejora mucho más despacio que el ancho de banda. Algunos estudios mencionan que desde los años 1990 hasta
los 2000, el ancho de banda se multiplicó por 1400, la latencia por 11
\end{itemize}

\end{frame}


%%---------------------------------------------------------------
\begin{frame}[fragile]
\frametitle{El ancho de banda no es infinito}
\begin{itemize}
\item
Aunque el ancho de banda aumenta continuamente, también las exigencias de las aplicaciones
\item
TCP no es un protocolo especialmente eficiente
\item
En computación móvil el ancho de banda es especialmente costoso
\end{itemize}

\end{frame}


%%---------------------------------------------------------------
\begin{frame}[fragile]
\frametitle{La red no es segura}
\begin{itemize}
\item
El diseño original de TCP/IP no considera participantes maliciosos
\item
Las conexiones globales dejan cualquier sistema muy expuesto
\item
Cualquier sistema podrá sufrir tanto ataque dirigidos intencionadamente contra él
como ataques automatizados indiscriminados, vandalismo, etc
\end{itemize}

\end{frame}


%%---------------------------------------------------------------
\begin{frame}[fragile]
\frametitle{La topología de red cambia continuamente}
\begin{itemize}
\item
Los servidores y su configuración suelen cambiar de vez en cuando
\item
Los clientes cambian continuamente
\end{itemize}

Es importante no depender de \emph{endpoints} fijos (direcciones IP, direcciones de ficheros, etc)
\end{frame}

%%---------------------------------------------------------------
\begin{frame}[fragile]
\frametitle{El administrador de red no es único}
\begin{itemize}
\item
Es normal que en cualquier proyecto haya varias organizaciones implicadas
\item
Las organizaciones cada vez externalizan más los servicios
\item
Los administradores no siempre consideran parte de su trabajo colaborar
con los desarrollos
\item
No es fácil localizar el origen de los problemas
% tendencia a echar balones fuera
\end{itemize}
\end{frame}



%%---------------------------------------------------------------
\begin{frame}[fragile]
\frametitle{Las comunicaciones no son gratuitas}
Las \emph{tarifas planas} son habitual es todos los entornos, pero eso
no implica gratuidad
\begin{itemize}
\item
Es necesario considerar costes de estructura de red, tanto hardware como
software, costes de configuración, de seguridad, etc
\item
El coste del personal implicado suele ser especialmente alto
\end{itemize}

\end{frame}


%%---------------------------------------------------------------
\begin{frame}[fragile]
\frametitle{La red no es homogénea}
\begin{itemize}
\item
Aunque TCP/IP es la norma universal, en los niveles superiores los protocolos son
mucho más variados
\item
En cualquier entorno hay gran cantidad
de hardware, software y sistemas operativos diferentes
\item
Un \emph{host} no solo es un ordenador: puede ser un smartphone, tablet,
impresora, NAS, cámara IP,etc etc
\item
Deberíamos procurar usar siempre protocolos estándar, dentro de lo posible

\end{itemize}

\end{frame}



\section{La abstracción}
%%---------------------------------------------------------------
\begin{frame}[fragile]
\frametitle{La abstracción}
Principio por el que se extrae la información relevante en cierto contexto y se ignora la no relevante. Se obtienen 
las características esenciales y el comportamiento de un objeto
\begin{itemize}
\item
El mismo objeto puede ser abstraido de diferente forma, dependiendo del entorno
\begin{itemize}
\item
Desarrollando aplicaciones, un \emph{ordenador} puede ser: sistema operativo (nombre y versión) + navegador (nombre y versión)
\item
En un servicio técnico, un \emph{ordenador} puede ser: fabricante, placa, video, ram, discos, etc
\end{itemize}

\end{itemize}


\end{frame}


%%---------------------------------------------------------------
\begin{frame}[fragile]
\frametitle{}
Un programa es una descripción abstracta de un procedimiento o fenómeno del mundo real
\begin{itemize}
\item
La abstracción es crucial para comprender cualquier fenómeno
\item
La abstracción es un proceso mental básico, una herramienta muy potente para tratar la complejidad 
\item
La abstracción es clave para diseñar cualquier software, incluyendo por supuesto un servicio web
\end{itemize}

\end{frame}


\end{document}

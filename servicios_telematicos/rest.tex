
\documentclass[ucs]{beamer}

\usetheme{GSyC}
%\usebackgroundtemplate{\includegraphics[width=\paperwidth]{gsyc-bg.png}}


\usepackage[greek,spanish]{babel}   %greek permite usar euro 
\usepackage[utf8x]{inputenc}
\usepackage{graphicx}
\usepackage{amssymb} % Simbolos matematicos


% Metadatos del PDF, por defecto en blanco, pdftitle no parece funcionar
   \hypersetup{%
     pdftitle={XML},%
     %pdfsubject={Diseño y Administración de Sistemas y Redes},%
     pdfauthor={GSyC},%
     pdfkeywords={},%
   }
%


% Para colocar un logo en la esquina inferior de todas las transpas
%   \pgfdeclareimage[height=0.5cm]{gsyc-logo}{gsyc}
%   \logo{\pgfuseimage{gsyc-logo}}


% Para colocar antes de cada sección una página de recuerdo de índice
%\AtBeginSection[]{
%  \begin{frame}<beamer>{Contenidos}
%    \tableofcontents[currentframetitle]
%  \end{frame}
%}



\begin{document}

% Entre corchetes como argumento opcional un título o autor abreviado
% para los pies de transpa
\title[REST]{REST:Representational State Transfer}
%\subtitle{Diseño y Administración de Sistemas y Redes}
\author[GSyC]{Escuela Técnica Superior de Ingeniería de Telecomunicación\\
Universidad Rey Juan Carlos}
\institute{gsyc-profes (arroba) gsyc.urjc.es}
\date[2016]{Abril de 2016}


%% TÍTULO
\begin{frame}
  \titlepage
  % Oportunidad para poner otro logo si se usó la opción nologo
  % \includegraphics[width=2cm]{logoesp}  
\end{frame}



%% LICENCIA DE REDISTRIBUCIÓN DE LAS TRANSPAS
%% Nota: la opción b al frame le dice que justifique el texto
%% abajo (por defecto c: centrado)
\begin{frame}[b]
\begin{flushright}
{\tiny
\copyright \insertshortdate~\insertshortauthor \\
  Algunos derechos reservados. \\
  Este trabajo se distribuye bajo la licencia \\
  Creative Commons Attribution Share-Alike 4.0\\
}
\end{flushright}  
\end{frame}



%% ÍNDICE
%\begin{frame}
%  \frametitle{Contenidos}
%  \tableofcontents
%\end{frame}



%the practice of system and network administration
%t limoncelli

%principles of network and system administration
%m burgess


\section{Introducción a REST}


%%---------------------------------------------------------------
\begin{frame}[fragile]
\frametitle{REST}
REST \emph{Representational State Transfer} 
es la tecnología más empleada actualmente para intercambiar
información \emph{de gestión} entre máquina y máquina

\begin{itemize}
\item
En rigor, REST es una propuesta académica, no una arquitectura concreta.
\item
Define una serie de características que debería cumplir una arquitectura.
\item
Los protocolos que cumplen las restrucciones REST, se les denomina RESTful
\item

Desarrollado por
Roy Fielding a finales de los años 90.
Publica REST en su tesis doctoral, año 2000

\begin{itemize}
\item
Fielding es uno de los autores de HTTP 1.1
\end{itemize}

\item
En la práctica, cuando una aplicación dice ser REST, suele ser ROA
(una arquitectura REST concreta desarrollada L.Richardson y S.Ruby)

\end{itemize}

\end{frame}


%%---------------------------------------------------------------
\begin{frame}[fragile]
\frametitle{Características de una arquitectura REST}
Una arquitectura es REST si cumple estas 6 caracterísitcas
\begin{enumerate}
\item
Cliente-servidor
\item
Sin estado (\emph{stateless})
\item
Admite el uso de caches (\emph{cacheable})
\item
Sistema basado en capas (\emph{layered system})
\item
Generación (opcional) de código bajo demanda
\item
Uso de interfaces uniformes

\begin{itemize}
\item
Identificación de recursos
\item
Manipulación de recursos mendiante representaciones
\item
Mensajes autodescriptivos
\item
Hipermedia como motor del estado de la aplicación (\emph{HATEOAS, 
Hypermedia as the engine of application state})
\end{itemize}

\end{enumerate}

\end{frame}



%%---------------------------------------------------------------
\begin{frame}[fragile]
\frametitle{1. Cliente-servidor}
\begin{itemize}
\item
Clara separación cliente-sevidor en el API, interfaz robusto que permita desarrollo
independiente de cada parte

\begin{itemize}
\item
P.e. El cliente no almacena datos
\end{itemize}

\item
El servidor ignora al usuario: tanto el interfaz de usuario como el estado de usuario
\item
Cliente y servidor se pueden desarrollar por separado
\end{itemize}

\end{frame}




%%---------------------------------------------------------------
\begin{frame}[fragile]
\frametitle{2. Sin estado}

\begin{itemize}
\item
En el servidor no se almacena contexto de las peticiones del cliente
\item
Cada petición del cliente contienen toda la información necesaria para
que el servidor responda
\item
La sesión y el estado del cliente se guardan en el cliente
\item
Todo esto simplifica la lógica del servidor

\begin{itemize}
\item
Garantiza que no llegará
a estados inconsistentes (no hay estado)
\item
Permite replicar los servidores
\item
Facilita el escalado
\end{itemize}
\end{itemize}

\end{frame}



%%---------------------------------------------------------------
\begin{frame}[fragile]
\frametitle{}
Ejemplo de protocolo sin estado:
\begin{itemize}
\item
HTTP
\end{itemize}

Ejemplo de protocolo que sí tiene estado:

\begin{itemize}
\item
FTP

\begin{itemize}
\item
Hay una sesión, con un usuario autenticado, con directorio de trabajo,
con modo de transferencia (texto o binario) preestablecido
\end{itemize}
\end{itemize}

\end{frame}


%%---------------------------------------------------------------
\begin{frame}[fragile]
\frametitle{3. Admite cachés}
\begin{itemize}
\item
Cada respuesta indica si es susceptible de ser almacenada en un cache o no

\begin{itemize}
\item
Si no lo es, los cachés se deshabilitarán,
lo que garantiza que no se usará información
obsoleta o inadecuada
\end{itemize}
\item
Usar caches mejora escalabilidad y el rendimiento
\end{itemize}
\end{frame}



%%---------------------------------------------------------------
\begin{frame}[fragile]
\frametitle{4. Sistema basado en capas}
Entre cliente y servidor puede haber proxys, caches o gateways,
sin que el cliente lo perciba y sin que produzcan resultados incorrectos
\begin{itemize}
\item

Un proxy es un intermediario. Recibe la petición y la vuelve a enviar.

\begin{itemize}
\item
Por motivos de seguridad, para poner un filtro para niños, para conseguir anonimato,
para modificar los documentos sobre la marcha (cambiar formato, idioma..)
para forzar a que una red pase por una máquina y haga cache...
\end{itemize}

\item
Un gateway es como un proxy, pero cambiando de protocolo. P.e. gateway http-ftp
\end{itemize}

\end{frame}


%%---------------------------------------------------------------
\begin{frame}[fragile]
\frametitle{5. Puede generar código bajo demanda}
\begin{itemize}
\item
Opcionalmente, el servidor puede ampliar la funcionalidad del cliente
enviando código, como applets java o JavaScript
\end{itemize}
\end{frame}

%%---------------------------------------------------------------
\begin{frame}[fragile]
\frametitle{6. Interfaz uniforme}
Es la característica principal de cualquier sistema REST. Permite sistemas
débilmente acoplados

\begin{itemize}
\item
6.1 Identificación de recursos
\item
6.2 Manipulación de recursos mendiante representaciones
\item
6.3 Mensajes autodescriptivos
\item
6.4 Hipermedia como motor del estado de la aplicación (\emph{HATEOAS})
\end{itemize}

\end{frame}



%%---------------------------------------------------------------
\begin{frame}[fragile]
\frametitle{6.1 Identificación de recursos}
\begin{itemize}
\item
Los recursos individuales se identifican en las peticiones, típicamente mediante URIs en sistemas web.
\item
El recurso es una cosa y la representación es otra

\begin{itemize}
\item
 Por ejemplo un recurso (un fichero) se puede
servir representado en HTML, en XML o JSON. Pero ninguno de estos formatos tiene por qué ser el
formato en que el recurso es almacenado en la base de datos
\end{itemize}

\end{itemize}

\end{frame}



%%---------------------------------------------------------------
\begin{frame}[fragile]
\frametitle{6.2 Manipulación de recursos mediante representaciones}
En necesario desacoplar el recurso de la representación del recurso.
\begin{itemize}
\item
El recurso tiene una única URI para todos los formatos
\item
En las cabeceras, cliente y servidor negocian qué formatos soportan, prefieren
o solicitan de cada recurso.
\item
No hay una URI diferente para cada formato, cada formato no se entiende como una entidad
separada
\end{itemize}

Recordatorio:


\begin{itemize}
\item
URI: Uniform Resource Identifier
\item
URL: Uniform Resource Locator
\item
URN: Uniform Resource Name
\end{itemize}

Un identificador (URI) puede ser o bien una dirección (URL) o bien
un nombre (URN) o bien ambas cosas

\end{frame}



%%---------------------------------------------------------------
\begin{frame}[fragile]
\frametitle{6.3 Mensajes autodescriptivos}
Cada mensaje contiene toda la información necesaria para ser procesado.
\begin{itemize}
\item
Ejemplo: \emph{Internet Media Type} (antiguamente llamado tipo MIME)

\item
Contraejemplo: Fichero de texto \emph{code page}
% datetime naive
\end{itemize}

\end{frame}


%%---------------------------------------------------------------
\begin{frame}[fragile]
\frametitle{6.4 Hipermedia como motor del estado de la aplicación }
\emph{HATEOAS, Hypermedia as the engine of application state}

\begin{itemize}
\item
Las transiciones entre estados se especifican en un documento HTML, que devuelve
el servidor
\item
El cliente puede descubrir por si mismo los estados a los que puede pasar,
mediante información que le pasa en el servidor, en formato hipermedia
(enlaces html), sin necesidad de instrucciones adicionales \emph{fuera de banda}
\end{itemize}

\end{frame}


\section{ROA: Resource-Oriented Architecture}
%%---------------------------------------------------------------
\begin{frame}[fragile]
\frametitle{ROA: Resource-Oriented Architecture}
L. Richardson y S.Ruby en su libro de 2007 \emph{RESTful Web Services}
proponen una arquitectura concreta,  RESTful, a la que denominan
\emph{ROA: Resource-Oriented Architecture}
%Pautas de Richardson y Ruby para hacer servicios RESTful
% Richardson, pg 79

\begin{itemize}
\item
La mayoría de los servicios RESTful actuales siguen esta arquitectura ROA
\begin{itemize}
\item
Aunque sus usuarios pueden no reconocer este nombre
\end{itemize}
\item
Hay arquitecturas que se autodenominan RESTful, pero que no siguen estas
pautas, resulta discutible que sean verdaderamente RESTful
\end{itemize}

\end{frame}



%%---------------------------------------------------------------
\begin{frame}[fragile]
\frametitle{}
Las pautas de Richardson y Ruby para hacer servicios RESTful son muy similares
a las características definidas por Fielding, pero un poco más concretas
% Richardson, pg 79
\begin{enumerate}
\item
La arquitectura está basada en \emph{recursos} (\emph{resources})
\item
Todo recurso tiene que tener una URI
\item
Las  aplicaciones son direccionables  (\emph{Addressability})
\item
El servidor no tiene estado (es \emph{stateless})
\item
Las aplicaciones usan un interfaz uniforme: GET, PUT, DELETE

\end{enumerate}

\end{frame}


%%---------------------------------------------------------------
\begin{frame}[fragile]
\frametitle{1. La arquitectura está basada en recursos}
Un recurso \emph{resource} 
es cualquier cosa con suficiente entidad como para merecer ser nombrado e indentificado
por sí mismo:

\begin{itemize}
\item
La versión 1.0.3 de cierto software
\item
Un vídeo
\item
Una entrada en un blog
\item
Un mapa de Fuenlabrada
\item
El precio actual del bitcoin en euros en Kraken
\item
...
\end{itemize}

\end{frame}


%%---------------------------------------------------------------
\begin{frame}[fragile]
\frametitle{2. Todo recurso tiene que tener una URI}
Si no tiene URI, no es un recurso

Y la estructura tiene que ser homogénea y predecible
\begin{itemize}
\item
http://www.ejemplo.com/sowtware/releases/1.0.3.tgz
\item
http://www.ejemplo.com/videos/helloworld
\item
http://www.example.com/blog/2016/04/20
\item
http://www.example.com/mapas/ES/MAD/fuenlabrada
\item
http://www.example.com/bitcoin/kraken/BTCEUR
\end{itemize}

\end{frame}



%%---------------------------------------------------------------
\begin{frame}[fragile]
\frametitle{3. Las aplicaciones son direccionables}
\emph{Addressability}

Una aplicación  es direccionable si expone todos sus datos relevantes
como recursos

Ejemplo:
\begin{itemize}
\item
http://www.google.com/search?q=fuenlabrada
\end{itemize}

Esto es muy importante, cuando no existía este concepto, por ejemplo con el protocolo FTP,
había que dar indicaciones como
\emph{abra una sesión FTP con el usuario anonymous, cambie el directorio a pub/files/ y descargue
el fichero file.txt}

\end{frame}


%%---------------------------------------------------------------
\begin{frame}[fragile]
\frametitle{}
Por ser las aplicaciones direccionables, se puede:

\begin{itemize}
\item
Crear un marcador
\item
Crear una URI, ponerla por escrito, en un email, en un QR
\item
Enlazar desde otra página web
\item

Usar una cache

\begin{itemize}
\item
De forma que la segunda vez que se pida el recurso,
no se vuelva a consultar la fuente original sino que se sirve la versión
en cache (tomando las precauciones necesarias para no servir información
anticuada)
\end{itemize}

\end{itemize}

\end{frame}

%%---------------------------------------------------------------
\begin{frame}[fragile]
\frametitle{}

\begin{itemize}
\item
La página que usa Iberia en 2016 para localizar un vuelo, no es RESTful, no es direccionable
  \begin{footnotesize}
  \begin{verbatim}
http://www.iberia.com/es/arrivals-and-departures/
  \end{verbatim}
  \end{footnotesize}
No es posible enviar por correo un enlace a tu vuelo

\item
Para ser REST, debería ser algo así
  \begin{footnotesize}
  \begin{verbatim}
http://www.iberia.com/flights/ib3214/20160329
  \end{verbatim}
  \end{footnotesize}

\end{itemize}

\end{frame}



%%---------------------------------------------------------------
\begin{frame}[fragile]
\frametitle{}
No solo las direcciones web son direccionables, por ejemplo también es direccionable


\begin{itemize}
\item
Un sistema de ficheros 

  \begin{footnotesize}
  \begin{verbatim}
/home/jperez/st/ejemplo.txt
  \end{verbatim}
  \end{footnotesize}

\item
Las celdas de una hoja de cálculo
\end{itemize}

Algo tan sencillo como ponerle una URI a un folleto para que la gente lo
pueda ver en su navegador no sería posible sin el direccionamiento

\end{frame}


%%---------------------------------------------------------------
\begin{frame}[fragile]
\frametitle{}
\begin{itemize}
\item
Si un servicio web es direccionable, los clientes pueden usarlos de formas nuevas,
no previstas en el diseño original
\item
Facilita el ser usado desde distintos dispositivos: aplicaciones de escritorio, clientes
web, clientes nativos para smartphone (iOS, android, BlueBerry...)
\end{itemize}

\end{frame}


%%---------------------------------------------------------------
\begin{frame}[fragile]
\frametitle{4. El servidor no tienen estado}

\begin{itemize}
\item
Cada petición es independiente de las anteriores

\begin{itemize}
\item
Ejemplo: una búsqueda en google devuelve 10 entradas, cada una de ellas contiene toda la información necesaria para hacer la búsqueda

\item
Contraejemplo: Cualquier sistema con \emph{directorio actual}

\end{itemize}
\item
Cada uno de los posibles estados es un recurso, con su propia URI

\item
El estado se tienen que guardar en el cliente, nunca en el servidor
\end{itemize}
\end{frame}


%%---------------------------------------------------------------
\begin{frame}[fragile]
\frametitle{}

El diseño original de HTTP era \emph{stateless}, pero al añadir cookies,
este principio suele romperse

\begin{itemize}
\item
Las cookies que escribe el servidor rompen este principio, no son el estado, son una
clave para identificar un estado guardado en el servidor
\item
Las cookies puden ser RESTful si las genera el cliente y continen toda la información
para hacer una petición (posible, pero no habitual)
\end{itemize}
\end{frame}




%%---------------------------------------------------------------
\begin{frame}[fragile]
\frametitle{5. Interfaz uniforme: GET, PUT, POST, DELETE}
REST dice que el interfaz debe ser uniforme, pero no especifica ninguno en particular

ROA, sí: Las únicas acciones posibles son métodos HTTP:   GET, PUT, POST, DELETE

\begin{itemize}
\item
Atención con el término \emph{método}, es el que emplea HTTP pero no tiene 
nada que ver con los métodos de la programación orientada a objetos
\end{itemize}



\end{frame}
%%----------------------------------------------
\begin{frame}[fragile]
\frametitle{}

\begin{itemize}
\item
GET obtiene un recurso
\item
PUT crea un recurso, creando una nueva URI especificada por el cliente
\item
PUT modifica un recurso, si la URI ya existía
\item
POST añade datos a una URI preexistente

Este añadido puede significar:

\begin{itemize}
\item
Modificar un recurso ya existente
\item
Crear un nuevo recurso, cuya URI crea el servidor, no el cliente. El cliente
solo indica la dirección del recurso \emph{padre}

  (ejemplo típico: añadir entrada a un blog)
\end{itemize}

\item
DELETE borra un recurso
\end{itemize}

\end{frame}



%%---------------------------------------------------------------
\begin{frame}[fragile]
\frametitle{}
En muchas ocasiones, POST se emplea de forma contraria a REST
\begin{itemize}
\item
Es el comportamiento de SOAP/WSDL y XML-RPC, es el POST al estilo RPC

Webber lo llama POST \emph{sobrecargado}


\begin{itemize}
\item
El cliente hace todos los POST a la misma URI
\item
El servidor analiza el contenido y realiza una acción u otra
\end{itemize}
\end{itemize}

\end{frame}


%%---------------------------------------------------------------
\begin{frame}[fragile]
\frametitle{}
Otros métodos de HTTP también pueden resultar útiles, no se
usan mucho pero pueden usarse sin romper REST/ROA
\begin{itemize}
\item
HEAD   

Obtiene los metadatos de un recurso, no los datos
\item
OPTIONS  

Averigua qué métodos soporta un recurso
\end{itemize}

\end{frame}



%%---------------------------------------------------------------
\begin{frame}[fragile]
\frametitle{}

Seguridad e idempotencia
\begin{itemize}
\item
Método seguro (\emph{safety})

No modifica nada en el servidor
\item
Método idempotente

Aplicarlo una vez es equivalente a aplicarlo varias veces

\begin{itemize}
\item
Ejemplo: A=10
\item
Contraejemplo: A=A+1
\end{itemize}

\end{itemize}

\end{frame}



%%---------------------------------------------------------------
\begin{frame}[fragile]
\frametitle{}


Para que una aplicación sea ROA, GET y HEAD tienen que ser seguros 

\begin{itemize}
\item
No pueden provocar ninguna acción excepto tal vez algún efecto lateral
como p.e. actualizar un log
\item
Por ser seguros, también son idempotentes
\end{itemize}

Muchas aplicaciones web no cumplen este requisito

Eso provocó el fracaso de la primer versión, de 2005, del \emph{Web accelerator}
de Google

\end{frame}

%%---------------------------------------------------------------
\begin{frame}[fragile]
\frametitle{}
\begin{itemize}
\item

PUT y DELETE tienen que ser idempotentes
Esto no siempre se cumple, entonces la aplicación ya no es REST, o al menos, no es ROA
\item

A veces los programadores incluso diseñan su sistema de forma que GET provoca
cambios en los recursos. Esto va contra todos los principios no solo de REST,
sino también de  HTTP

\end{itemize}
\end{frame}




\subsection{Ajax}
%%---------------------------------------------------------------
\begin{frame}[fragile]
\frametitle{Ajax}

Es un conjunto de técnicas para enviar información asíncrona entre
servidor web y cliente

\begin{itemize}
\item
Aunque hay antecedentes desde 1996, el Ajax actual lo crea google en 2004, para gmail
y google maps
\item
Evita que el usuario necesite pulsar \emph{enviar}. La aplicación
web recibe información continuamente, resultando una experiencia de usuario similar
a la de una aplicación de escritorio
\item

Originalmente significaba Asynchronous JavaScript and XML, pero cuando
empieza a usarse sin JavaScript y sin XML, el acrónimo AJAX se abandona
para usar la palabra Ajax

\end{itemize}

\end{frame}


%%---------------------------------------------------------------
\begin{frame}[fragile]
\frametitle{Funcionamiento de ajax}
\begin{enumerate}
%   richardson pg 316
\item
El usuario solicita una URL desde su navegador

\item
El servidor web devuelve la página, que contiene un script

\item
El navegador presenta la página y ejecuta el script

\item
El script hace peticiones HTTP asíncronas a una URI del servidor, sin que
el usuario intervenga. El servidor suele ser RESTful

\item
El script interpreta las respuestas HTTP y modifica el HTML, sin que
el usuario intervenga
\end{enumerate}

\end{frame}


%%---------------------------------------------------------------
\begin{frame}[fragile]
\frametitle{¿Es Ajax RESTful?}

\begin{itemize}
\item
Ajax es una arquitectura muy similar a la de una aplicación de escritorio
\item

En una aplicación web ordinaria, el httl solo cambia (la vista solo se refresca) cuando el usuario envía
una actualización (pulsa un botón como aceptar, enviar o similar)

\item

El inconveniente es que toda la aplicación tiene la misma URI

No es direccionable ni stateless

El usuario no puede guardar en un bookmark un estado, ni el botón \emph{back} funciona correctamente

\end{itemize}
\end{frame}
%%----------------------------------------------
\begin{frame}[fragile]
\frametitle{}
\begin{itemize}
\item
El servicio tal vez sí es REST, pero solo el script javascript lo aprovecha

En ocasiones hay \emph{atajos} para que el usuario también lo use, pero no está
diseñado para usarse así

Por ejemplo, buscar todos mis correos con el término \emph{fuenlabrada}
  \begin{footnotesize}
  \begin{verbatim}
https://mail.google.com/mail/?q=fuenlabrada&search=query&view=tl
  \end{verbatim}
  \end{footnotesize}

A esta arquitectura a veces se denomina \emph{REST híbrido}

\end{itemize}

\end{frame}



\section{Requests}
%%---------------------------------------------------------------
\begin{frame}[fragile]
\frametitle{Libería Requests}
En python hay diversas librerías para manejar HTTP

Una de las más convenientes y populares es Requests
\begin{itemize}
\item
  \begin{footnotesize}
  \begin{verbatim}
http://python-requests.org
  \end{verbatim}
  \end{footnotesize}
\item
Desarrollado por Kenneth Reitz con licencia libre Apache2
\item
Extremadamente fácil de usar

\begin{itemize}
\item
El método \verb|get()| hace una petición a la url pasada como parámetro
\item
El atributo \verb|status_code| devuelve el status http
\item
El métido \verb|json()| decodifica la respuesta json y devuelve un valor python
\end{itemize}
\item
Instalación

  \begin{footnotesize}
  \begin{verbatim}
pip install requests
  \end{verbatim}
  \end{footnotesize}
\end{itemize}

\end{frame}


%%---------------------------------------------------------------
\begin{frame}[fragile]
\frametitle{Uso básico de Requests}


  \begin{footnotesize}
  \begin{verbatim}
#!/usr/bin/python -tt
# -*- coding: utf-8 -*-
import requests

# Este servicio de prueba nos devuelve nuestra dirección ip
url="http://httpbin.org/ip"

r=requests.get(url)
if r.status_code==200:
    print r.json() # {u'origin': u'81.37.7.200'}
else:
    print "error "+str(r.status_code)
  \end{verbatim}
  \end{footnotesize}

\end{frame}


%%---------------------------------------------------------------
\begin{frame}[fragile]
\frametitle{Paso de parámetros}
\begin{itemize}
\item
Para hacer peticiones con parámetros como p.e.

  \begin{footnotesize}
  \begin{verbatim}
https://data.btcchina.com/data/ticker?market=all
  \end{verbatim}
  \end{footnotesize}

basta pasar al método get el parámetros \verb|params|, dándole como 
valor un diccionario
\item
Otro parámetro habitual es \verb|timeout|, en segundos

\end{itemize}

\end{frame}


%%---------------------------------------------------------------
\begin{frame}[fragile]
\frametitle{}


  \begin{footnotesize}
  \begin{verbatim}
#!/usr/bin/python -tt
# -*- coding: utf-8 -*-
import requests

url="https://data.btcchina.com/data/ticker"
payload={}
payload["market"]="all"
my_timeout=5

r=requests.get(url,params=payload, timeout=my_timeout)
if r.status_code==200:
    print r.json()
else:
    print "error "+str(r.status_code)
  \end{verbatim}
  \end{footnotesize}


\end{frame}
%%----------------------------------------------
\begin{frame}[fragile]
\frametitle{}

Valor devuelto:

% https://www.btcc.com/apidocs/spot-exchange-market-data-rest-api
  \begin{scriptsize}
  \begin{verbatim}
{u'ticker_ltcbtc': {u'sell': u'0.00780000', u'buy': u'0.00770000',
u'last': u'0.00780000', u'vol': u'16.95500000', u'prev_close':
u'0.0077', u'vwap': u'0.0076', u'high': u'0.00790000', u'low':
u'0.00760000', u'date': 1459699123, u'open': u'0.0077'}, u'ticker_ltccny':
{u'sell': u'21.20', u'buy': u'21.05', u'last': u'21.20', u'vol':
u'5976.87000000', u'prev_close': u'21.16', u'vwap': u'21.07',
u'high': u'21.21', u'low': u'21.00', u'date': 1459699123, u'open':
u'21.16'}, u'ticker_btccny': {u'sell': u'2726.98', u'buy': u'2726.65',
u'last': u'2726.79', u'vol': u'17131.95190000', u'prev_close':
u'2725.22', u'vwap': u'2721', u'high': u'2730.49', u'low': u'2712.49',
u'date': 1459699122, u'open': u'2725.02'}}
  \end{verbatim}
  \end{scriptsize}
\end{frame}



%%---------------------------------------------------------------
\begin{frame}[fragile]
\frametitle{Otras respuestas}
\begin{itemize}
\item
Si la petición devuelve una imagen, un xml o cualquier otro tipo
de recurso, basta leer el atributo \emph{content} de la respuesta

\begin{itemize}
\item
Si el recurso está comprimido, la librería lo descomprimirá automáticamente
\end{itemize}

\item
Las cabeceras están dispoibles en el atributo \emph{headers}

\begin{itemize}
\item
Para saber el tipo de contenido (según la norma MIME):
  \begin{footnotesize}
  \begin{verbatim}
headers["content-type"]
  \end{verbatim}
  \end{footnotesize}

\end{itemize}

\end{itemize}

\end{frame}



%%---------------------------------------------------------------
\begin{frame}[fragile]
\frametitle{}


  \begin{footnotesize}
  \begin{verbatim}
#!/usr/bin/python -tt
# -*- coding: utf-8 -*-
import requests,sys
def main():
    url="http://placehold.it/150/15c072"
    
    r=requests.get(url)
    if r.status_code!=200:
        sys.stderr.write("error"+str(r.status_code))
        raise SystemExit

    print r.headers["content-type"] #   image/png
    fichero=open('mi_imagen','wb')
    fichero.write(r.content)
    fichero.close()
if __name__ == "__main__":
    main()
  \end{verbatim}
  \end{footnotesize}



\end{frame}




\subsection{Flask}
%%---------------------------------------------------------------
\begin{frame}[fragile]
\frametitle{Flask}
Flask es un micro \emph{framework} para hacer aplicaciones web en python
\begin{itemize}
\item
Desarrollado por Armin Ronacher en 2010, ha ido ganando mucha popularidad
\item
Especialmente útil para servicios REST
\item
Alternativa a herramientas más completas y complejas como Django
\item
Ofrece un uso muy sencillo, con la funcionalidad mínima para aceptar peticiones HTTP y ofrecer respuestas,
sin integrar  gestión de formularios, bases de datos, autenticación de usuarios, etc

\begin{itemize}
\item
Aunque todo esto se puede añadir mediante extensiones
\end{itemize}

\item
Instalación de flask:
\verb|pip install flask|

\end{itemize}

\end{frame}

%%---------------------------------------------------------------
\begin{frame}[fragile]
\frametitle{}


  \begin{footnotesize}
  \begin{verbatim}
#!/usr/bin/python -tt
# -*- coding: utf-8 -*-
import flask 
app=flask.Flask(__name__)

@app.route('/')
def index():
    return "hola mundo"

if __name__ == '__main__': 
    app.run(debug=True)
  \end{verbatim}
  \end{footnotesize}

Para depurar el código, flask incorpora un servidor web, que 
por omisión atiende peticiones en
  \begin{footnotesize}
  \begin{verbatim}
http://127.0.0.1:5000
  \end{verbatim}
  \end{footnotesize}

\end{frame}

%%---------------------------------------------------------------
\begin{frame}[fragile]
\frametitle{}

  \begin{footnotesize}
  \begin{verbatim}
#!/usr/bin/python -tt
# -*- coding: utf-8 -*-
from flask import Flask 
app=Flask(__name__)
@app.route('/')
def index():
    return "hola mundo"

@app.route('/user/<name>') 
def user(name):
    return "hola, "+name

if __name__ == '__main__': 
    app.run(debug=True)
  \end{verbatim}
  \end{footnotesize}

\end{frame}



%%---------------------------------------------------------------
\begin{frame}[fragile]
\frametitle{}
El ejemplo anterior lo invocamos con 
  \begin{footnotesize}
  \begin{verbatim}
http://127.0.0.1:5000/user/pedro
  \end{verbatim}
  \end{footnotesize}

Obtendremos en el navegador

  \begin{footnotesize}
  \begin{verbatim}
hola, pedro
  \end{verbatim}
  \end{footnotesize}

Y en la shell donde lanzamos el script python
  \begin{footnotesize}
  \begin{verbatim}
 * Running on http://127.0.0.1:5000/ (Press CTRL+C to quit)
 * Restarting with stat
 * Debugger is active!
 * Debugger pin code: 128-685-262
127.0.0.1 - - [13/Apr/2016 20:20:44] "GET /user/pedro HTTP/1.1" 200 -
  \end{verbatim}
  \end{footnotesize}

\end{frame}

%%---------------------------------------------------------------
\begin{frame}[fragile]
\frametitle{Respuesta json}

  \begin{footnotesize}
  \begin{verbatim}
#!/usr/bin/python -tt
# -*- coding: utf-8 -*-
from flask import Flask 
import json

app=Flask(__name__)
@app.route('/')
def index():
    return "hola mundo"

@app.route('/user/<name>') 
def user(name):
    respuesta=["hola","mundo"]
    return json.dumps(respuesta)

if __name__ == '__main__': 
    app.run(debug=True)
  \end{verbatim}
  \end{footnotesize}
\end{frame}


%%---------------------------------------------------------------
\begin{frame}[fragile]
\frametitle{Query string}
\begin{itemize}
\item
En http, cuando el cliente desea pasar parámetros no jerárquicos,
usa la \emph{query string}

\item
Ejemplo:
  \begin{footnotesize}
  \begin{verbatim}
http://127.0.0.1:5000/user/juanpedro?city=toledo&profile=basic
  \end{verbatim}
  \end{footnotesize}
\item
Flask nos permite acceder a los argumentos de la query string,
a través del diccionario \verb|args| del objeto global
\verb|request| 
\end{itemize}

\end{frame}

%%---------------------------------------------------------------
\begin{frame}[fragile]


  \begin{footnotesize}
  \begin{verbatim}
#!/usr/bin/python -tt
# -*- coding: utf-8 -*-
import flask
app=flask.Flask(__name__)

@app.route('/user/<name>') 
def user(name):
    query_string= flask.request.args
    respuesta="Solicitas "+name
    respuesta=respuesta+" de la ciudad "+query_string["city"]
    respuesta=respuesta+" con perfil "+query_string["profile"]
    return json.dumps(respuesta)

if __name__ == '__main__': 
    app.run(debug=True)
  \end{verbatim}
  \end{footnotesize}

Resultado:

  \begin{footnotesize}
  \begin{verbatim}
Solicitas juanpedro de la ciudad toledo con perfil basic
  \end{verbatim}
  \end{footnotesize}

\end{frame}


%%---------------------------------------------------------------
\begin{frame}[fragile]
\frametitle{}
El servidor flash puede devolver una tupla, cuyo segundo parámetro
es el status http

  \begin{footnotesize}
  \begin{verbatim}
#!/usr/bin/python -tt
# -*- coding: utf-8 -*-
import flask

app=flask.Flask(__name__)

@app.route('/user/<name>') 
def user(name):
    if name=="juan" or name =="maria":
        respuesta="Datos de "+name+": ..."
    else:
        respuesta="Usuario desconocido",404
    return respuesta

if __name__ == '__main__': 
    app.run(debug=True)
  \end{verbatim}
  \end{footnotesize}

\end{frame}


\end{document}




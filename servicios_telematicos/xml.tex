
\documentclass[ucs]{beamer}

\usetheme{GSyC}
%\usebackgroundtemplate{\includegraphics[width=\paperwidth]{gsyc-bg.png}}


\usepackage[greek,spanish]{babel}   %greek permite usar euro 
\usepackage[utf8x]{inputenc}
\usepackage{graphicx}
\usepackage{amssymb} % Simbolos matematicos


% Metadatos del PDF, por defecto en blanco, pdftitle no parece funcionar
   \hypersetup{%
     pdftitle={XML},%
     %pdfsubject={Diseño y Administración de Sistemas y Redes},%
     pdfauthor={GSyC},%
     pdfkeywords={},%
   }
%


% Para colocar un logo en la esquina inferior de todas las transpas
%   \pgfdeclareimage[height=0.5cm]{gsyc-logo}{gsyc}
%   \logo{\pgfuseimage{gsyc-logo}}


% Para colocar antes de cada sección una página de recuerdo de índice
%\AtBeginSection[]{
%  \begin{frame}<beamer>{Contenidos}
%    \tableofcontents[currentframetitle]
%  \end{frame}
%}



\begin{document}

% Entre corchetes como argumento opcional un título o autor abreviado
% para los pies de transpa
\title[XML]{XML }
%\subtitle{Diseño y Administración de Sistemas y Redes}
\author[GSyC]{Escuela Técnica Superior de Ingeniería de Telecomunicación\\
Universidad Rey Juan Carlos}
\institute{gsyc-profes (arroba) gsyc.urjc.es}
\date[2016]{Marzo de 2016}


%% TÍTULO
\begin{frame}
  \titlepage
  % Oportunidad para poner otro logo si se usó la opción nologo
  % \includegraphics[width=2cm]{logoesp}  
\end{frame}



%% LICENCIA DE REDISTRIBUCIÓN DE LAS TRANSPAS
%% Nota: la opción b al frame le dice que justifique el texto
%% abajo (por defecto c: centrado)
\begin{frame}[b]
\begin{flushright}
{\tiny
\copyright \insertshortdate~\insertshortauthor \\
  Algunos derechos reservados. \\
  Este trabajo se distribuye bajo la licencia \\
  Creative Commons Attribution Share-Alike 4.0\\
}
\end{flushright}  
\end{frame}



%% ÍNDICE
%\begin{frame}
%  \frametitle{Contenidos}
%  \tableofcontents
%\end{frame}



%the practice of system and network administration
%t limoncelli

%principles of network and system administration
%m burgess


\section{XML}
%%---------------------------------------------------------------
\begin{frame}[fragile]
\frametitle{Lenguajes de marcado}
Un \emph{lenguaje de marcado} es un sistema que permite incluir metainformación en un documento, esto
es, información sobre la información

\begin{itemize}
\item
La metainformación tiene que
distinguirse sintácticamente del texto.


\item
Es la evolución del \emph{lapiz azul} con el que tradicionalmente se editaban
documentos cuando la tecnología era analógica
\begin{itemize}
\item
Ejemplo de lenguaje de marcado muy elemental: redacto un documento en un procesador de textos, 
lo imprimo y alguien lo revisa, incluyendo anotaciones a mano

Las anotaciones (metainformación) se distingue fácilmente del texto original

Para hacer algo semejante de forma digital, es necesaria una sintaxis que separe el texto
de la metainformación
\end{itemize}


\item
Ejemplos de lenguajes de marcado: troff, LaTeX, JsonML, SGML, XML


\end{itemize}

\end{frame}



%%---------------------------------------------------------------
\begin{frame}[fragile]
\frametitle{}
\begin{itemize}
\item
XML \emph{Extensible Markup Language}
%Resumen de XML: http://www.diveintopython3.net/xml.html

Es una forma de describir datos jerárquicamente.
Estándar para transferir información entre distintos sistemas sin tener
que adaptarlos a cada plataforma concreta, y de forma que sea fácil
de leer por un humano y fácil de procesar por un ordenador

\item
Creado en 1996 por el W3C (\emph{World Wide Web Consortium})

Dos versiones: XML 1.0 y XML 1.1
%1.0 (Fifth Edition) (November 26, 2008; )
% 1.1 (Second Edition) (August 16, 2006; )
\item
Algunos autores lo consideran un lenguaje de marcado, otros, un metalenguaje
de marcado
\item
Proviene de
SGML, \emph{Standard Generalized Markup Language}, norma ISO 8879:1986


SGML es un metalenguaje, un lenguaje para definir lenguajes de marcado,
\end{itemize}
\end{frame}




%%---------------------------------------------------------------
\begin{frame}[fragile]
\frametitle{}
\begin{itemize}
\item
XML tiene una sintaxis similar a la de HTML porque ambos provienen de SGML
\item
XML y HTML no son lenguajes alternativos

\begin{itemize}
\item
XML está diseñado para describir y comunicar datos datos de máquina a máquina
\item
HTML está diseñado para presentar en pantalla datos con formato. De máquina a persona 
\end{itemize}

\end{itemize}

\end{frame}


%%---------------------------------------------------------------
\begin{frame}[fragile]
\frametitle{Estructura de un documento XML}
\begin{itemize}

\item
%<?xml version="1.1"?>
En la primera línea es recomendable incluir un \emph{prólogo} aka \emph{declaración}, indicando
la versión de xml y, opcionalmente, la codificación empleada
  \begin{footnotesize}
  \begin{verbatim}
<?xml version="1.1" encoding="UTF-8"?>
  \end{verbatim}
  \end{footnotesize}

\item
A continuación puede aparecer un
DTD, \emph{Document Type Definition}. Aunque es casi obsoleto

  \begin{footnotesize}
  \begin{verbatim}
CTYPE html PUBLIC "-//W3C//DTD XHTML 1.0 Transitional//EN"
"http://www.w3.org/TR/xhtml1/DTD/xhtml1-transitional.dtd">
  \end{verbatim}
  \end{footnotesize}

\item
A continuación debe aparecer el cuerpo, formado por un
\emph{elemento raiz}. Siembre habrá
uno y solo uno, que podrá tener elementos anidados



\end{itemize}
\end{frame}
%%----------------------------------------------
\begin{frame}[fragile]
\frametitle{}
\begin{itemize}

\item
XML puede usar cualquier caracter unicode, por omisión codificado en UTF-8
% se pueden usar otras codificaciones, pero es obligatorio que cualquier
% herramienta admita al menos utf-8 y utf-16

\item
XML es \emph{case sensitive}

\item
Se pueden poner comentarios en cualquier lugar del documento
  \begin{footnotesize}
  \begin{verbatim}
<!-- Esto es un comentario -->
  \end{verbatim}
  \end{footnotesize}
\end{itemize}

\end{frame}




%%---------------------------------------------------------------
\begin{frame}[fragile]
\frametitle{Elementos XML}
\begin{itemize}
\item
Un documento XML está formado por uno o varios elementos

\item
Cada elemento está delimitado por una etiqueta inicial
y una etiqueta final (start tag, end tag)
\item
Una etiqueta inicial está formada por el signo de menor, el nombre y el signo de mayor
\item
Una etiqueta final está formada por el signo de menor, una barra (slash), el nombre y el signo de mayor
\item
No puede haber espacios ni a la izquierda ni dentro del nombre. Solo se admiten después del nombre
\item
El nombre puede usar cualquier carácter unicode
\item
El nombre puede incluir el guión, la barra baja y números, excepto en la primera posición
\end{itemize}

\end{frame}

%%---------------------------------------------------------------
\begin{frame}[fragile]
\frametitle{}

Ejemplo correcto:
  \begin{footnotesize}
  \begin{verbatim}
<holamundo></holamundo>
  \end{verbatim}
  \end{footnotesize}

Ejemplos incorrectos
  \begin{footnotesize}
  \begin{verbatim}
< holamundo></ holamundo>

<hola mundo></hola mundo>

<1mundo></1mundo>
  \end{verbatim}
  \end{footnotesize}

Ejemplos correctos

  \begin{footnotesize}
  \begin{verbatim}
<holamundo ></holamundo >

<holamundo2></holamundo2>

<holamundo_2></holamundo_2>
  \end{verbatim}
  \end{footnotesize}

\end{frame}



%%---------------------------------------------------------------
\begin{frame}[fragile]
\frametitle{}
Es frecuente que un elemento contenga una lista de elementos, con la misma
etiqueta

  \begin{footnotesize}
  \begin{verbatim}
<grupo>
    <alumno>
        Juan González
    </alumno>
    <alumno>
        María Fernández
    </alumno>
</alumno>
  \end{verbatim}
  \end{footnotesize}

\end{frame}
%%----------------------------------------------
\begin{frame}[fragile]
\frametitle{}

Numerar los elementos normalmente no tendrá sentido. Aunque sintácticamente
puede ser válido, estamos forzando la creación de elementos distintos

  \begin{footnotesize}
  \begin{verbatim}
<grupo>
    <!-- ¡¡MAL EJEMPLO!! -->
    <alumno1>
        Juan González
    </alumno1>
    <alumno2>
        María Fernández
    </alumno2>
</alumno>
  \end{verbatim}
  \end{footnotesize}


\end{frame}

%%---------------------------------------------------------------
\begin{frame}[fragile]
\frametitle{Anidamiento de los elementos}
\begin{itemize}
\item
Los elementos se pueden anidar hasta cualquier nivel.
\item
El elemento raiz de un documento XML es el elemento de mayor nivel
jerárquico. Tiene que haber exactamente uno
\item
El elemento raiz puede tener
uno o más elementos anidados (nunca solapados), que se pueden
anidar hasta cualquier nivel
\item
Los elementos están ordenados, se garantiza que el orden se mantiene
\item
Dentro de los elementos hay \emph{character data}, normalmente
llamado simplemente \emph{text}, texto.
\end{itemize}
\end{frame}


%%---------------------------------------------------------------
\begin{frame}[fragile]
\frametitle{}
  \begin{footnotesize}
  \begin{verbatim}
<holamundo>
    <hola_europa>
        <hola_españa>
            Texto de ejemplo del elemento hola_españa
        </hola_españa>
        <hola_portugal>
        </hola_portugal>
    </hola_europa>
    <hola_asia>
        Texto de ejemplo del elemento hola_asia
    </hola_asia>
</holamundo>
  \end{verbatim}
  \end{footnotesize}

\end{frame}


%%---------------------------------------------------------------
\begin{frame}[fragile]
\frametitle{Etiquetas autocerradas}
Cuando un elemento no tiene texto, hay dos alternativas posibles
\begin{itemize}
\item
Usar una etiqueta de cierre y otra de apertura

  \begin{footnotesize}
  \begin{verbatim}
<holamundo></holamundo>
  \end{verbatim}
  \end{footnotesize}
\item
Usar una etiqueta \emph{auto cerrada}
  \begin{footnotesize}
  \begin{verbatim}
<holamundo/>
  \end{verbatim}
  \end{footnotesize}
Signo de menor, nombre, barra, signo de mayor
\end{itemize}

\end{frame}



%%---------------------------------------------------------------
\begin{frame}[fragile]
\frametitle{Atributos}
Un elemento puede tener \emph{atributos}


  \begin{footnotesize}
  \begin{verbatim}
<log date="2016-02-07 17:01:05+00:00" lang="es">
No hay actividad
</log>
  \end{verbatim}
  \end{footnotesize}

\begin{itemize}
\item
Los atributos
están dentro de la etiqueta inicial,  separados por espacios
\item
Un atributo
es un par formado por un nombre y un valor

\item
El nombre del atributo no se puede repetir dentro del mismo elemento. Sí puede aparece
el mismo nombre de atributo en un elemento distinto
\item
A continuación del nombre va el signo igual
y el valor, entre comillas simples o dobles
\item
El valor es texto
\item
Los atributos no están ordenados, no hay garantía de que se mantenga
el orden
\end{itemize}

\end{frame}

%%---------------------------------------------------------------
\begin{frame}[fragile]
\frametitle{}
Una misma información puede presentarse o bien como texto o bien como
atributo, las siguientes cuatro formas son correctas y equivalentes
\begin{itemize}
\item
  \begin{tiny}
  \begin{verbatim}
<log>
    <date>
        2015-07-10 21:01:05+00:00
    </date>
    <lang>
        es
    </lang>
    <text>
        Pass
    </text>
</log>
  \end{verbatim}
  \end{tiny}
\item
  \begin{tiny}
  \begin{verbatim}
<log date="2015-07-10 21:01:05+00:00" lang="es" text="Pass">
</log>
  \end{verbatim}
  \end{tiny}

\item
  \begin{tiny}
  \begin{verbatim}
<log date="2015-07-10 21:01:05+00:00" lang="es">
    Pass
</log>
  \end{verbatim}
  \end{tiny}


\item
  \begin{tiny}
  \begin{verbatim}
<log date="2015-07-10 21:01:05+00:00" lang="es" text="Pass"/>
  \end{verbatim}
  \end{tiny}
\end{itemize}

\end{frame}


%%---------------------------------------------------------------
\begin{frame}[fragile]
\frametitle{}
Usar un enfoque y otro es decisión del diseñador, no hay reglas fijas
pero sí algunas recomendacions
%fawcett, beginning xml 5ed, pg 37
\begin{itemize}
\item
Es preferible usar atributos para información breve, de una sola pieza
\item
Es preferible usar texto dentre de un nuevo elemento para valores 
de cierta complejidad. Por ejemplo un mensaje de cierta longitud,
una dirección, una descripción, etc
\item
En caso de duda, se recomienda el atributo
\end{itemize}

\end{frame}


%%---------------------------------------------------------------
\begin{frame}[fragile]
\frametitle{}
Cuando una información conste de varias unidades que pueden repetirse,
es obligado usar elementos, porque los atributos no pueden repetirse

\begin{itemize}
\item
Ejemplo correcto:

  \begin{footnotesize}
  \begin{verbatim}
<cliente>
    <telefono>
        91 00 0000
    </telefono>
    <telefono>
        91 11 1111
    </telefono>
</cliente>
  \end{verbatim}
  \end{footnotesize}
\item
¡¡INCORRECTO!!: 

  \begin{footnotesize}
  \begin{verbatim}
<cliente
    telefono="91 00 0000"
    telefono="91 11 1111"
/>
  \end{verbatim}
  \end{footnotesize}
\end{itemize}

\end{frame}


%%---------------------------------------------------------------
\begin{frame}[fragile]
\frametitle{}
Esto sí sería correcto

  \begin{footnotesize}
  \begin{verbatim}
<cliente
    telefono1="91 00 0000"
    telefono2="91 11 1111"
/>
  \end{verbatim}
  \end{footnotesize}
Pero tiene inconvenientes
\begin{itemize}
\item
Es más complicado
\item
No sirve para 3 teléfonos
\end{itemize}

\end{frame}


%%---------------------------------------------------------------
\begin{frame}[fragile]
\frametitle{}
Naturalmente, también es correcto que un nombre de atributo se repita en un elemento distinto

  \begin{footnotesize}
  \begin{verbatim}
<cliente
    telefono="91 00 0000"
/>

<cliente
    telefono="91 11 1111"
/>

<proveedor
    telefono="91 22 2222"
/>

  \end{verbatim}
  \end{footnotesize}

\end{frame}



%%---------------------------------------------------------------
\begin{frame}[fragile]
\frametitle{Gramáticas}
Un documento XML puede incluir una referencia a una gramática.

La gramática indica qué etiquetas, qué atributos y qué texto se permiten
en un documento XML.
\begin{itemize}
\item
Originalmente la gramática se indicaba mediante
DTD, \emph{Document Type Definition}

La referencia al DTD se coloca entre el prólogo y el elemento raiz
\item
En la actualidad es más habitual emplear XSD, \emph{(XML Schema Definition)}

La referencia al XSD se indica como atributo del elemento raíz

\end{itemize}

\end{frame}


%%---------------------------------------------------------------
\begin{frame}[fragile]
\frametitle{Referencia a un DTD}

  \begin{footnotesize}
  \begin{verbatim}
<?xml version="1.0" encoding="utf-8"?>
<!DOCTYPE html PUBLIC "-//W3C//DTD XHTML 1.0 Transitional//EN"
"http://www.w3.org/TR/xhtml1/DTD/xhtml1-transitional.dtd">
<!-- the XHTML document body starts here-->
<html xmlns="http://www.w3.org/1999/xhtml">
 ...
</html>
  \end{verbatim}
  \end{footnotesize}

  \begin{scriptsize}
  \begin{flushright}
Fuente: wikipedia
  \end{flushright}
  \end{scriptsize}

\end{frame}



%%---------------------------------------------------------------
\begin{frame}[fragile]
\frametitle{Ejemplo de gramática XSD}

%https://en.wikipedia.org/wiki/XML_Schema_(W3C)
  \begin{tiny}
  \begin{verbatim}
<?xml version="1.0" encoding="utf-8"?>
<xs:schema elementFormDefault="qualified" xmlns:xs="http://www.w3.org/2001/XMLSchema">
  <xs:element name="Address">
    <xs:complexType>
      <xs:sequence>
        <xs:element name="Recipient" type="xs:string" />
        <xs:element name="House" type="xs:string" />
        <xs:element name="Street" type="xs:string" />
        <xs:element name="Town" type="xs:string" />
        <xs:element name="County" type="xs:string" minOccurs="0" />
        <xs:element name="PostCode" type="xs:string" />
        <xs:element name="Country" minOccurs="0">
          <xs:simpleType>
            <xs:restriction base="xs:string">
              <xs:enumeration value="DE" />
              <xs:enumeration value="ES" />
              <xs:enumeration value="UK" />
              <xs:enumeration value="US" />
            </xs:restriction>
          </xs:simpleType>
        </xs:element>
      </xs:sequence>
    </xs:complexType>
  </xs:element>
</xs:schema>
  \end{verbatim}
  \end{tiny}

  \begin{scriptsize}
  \begin{flushright}
  \end{flushright}
  \end{scriptsize}

\end{frame}


%%---------------------------------------------------------------
\begin{frame}[fragile]
\frametitle{Documento XML referenciando al XSD anterior}

  \begin{footnotesize}
  \begin{verbatim}
<?xml version="1.0" encoding="utf-8"?>
<Address xmlns:xsi="http://www.w3.org/2001/XMLSchema-instance"
         xsi:noNamespaceSchemaLocation="SimpleAddress.xsd">
  <Recipient>Mr. Walter C. Brown</Recipient>
  <House>49</House>
  <Street>Featherstone Street</Street>
  <Town>LONDON</Town>
  <PostCode>EC1Y 8SY</PostCode>
  <Country>UK</Country>
</Address>
  \end{verbatim}
  \end{footnotesize}

  \begin{scriptsize}
  \begin{flushright}
Fuente:wikipedia
  \end{flushright}
  \end{scriptsize}

\end{frame}



\section{ElementTree: procesamiento de XML desde python}
%%---------------------------------------------------------------
\begin{frame}[fragile]
\frametitle{ElementTree: procesamiento de XML desde python}
En python hay varias librerías para procesar xml
\begin{itemize}
\item
Tal vez la más habitual y casi \emph{estándar} es ElementTree
\item
También se puede usar cElementTree, cuya API es idéntica, pero es más
eficiente por estar implementada en C
\item
Otra librería interesante, más avanzada, es lxml. Tiene un API similar a
ElementTree, con más funcionalidad
\end{itemize}

\end{frame}


%%---------------------------------------------------------------
\begin{frame}[fragile]
\frametitle{ejemplo:biblioteca.xml}

  \begin{tiny}
  \begin{verbatim}
<?xml version="1.1"?>
<biblioteca>
    <libro isdn="978-1-118-16213-2"  edicion="5" fecha="july 2012" editorial="Wiley" >
        <autor>
            Joe Fawcett
        </autor>
        <autor>
            Danny Ayers
        </autor>
        <autor>
            Liam R. E. Quin
        </autor>
        <titulo>
            Beginning XML
        </titulo>
    </libro>
    <libro isdn="978-1-484202-03-6"   edicion="1" fecha="march 2015" editorial="Apress" >
        <autor>
            Ben Smith
        </autor>
        <titulo>
            Beginning JSON
        </titulo>
    </libro>
</biblioteca>
  \end{verbatim}
  \end{tiny}
\end{frame}



%%---------------------------------------------------------------
\begin{frame}[fragile]
\frametitle{}

  \begin{footnotesize}
  \begin{verbatim}
#!/usr/bin/python -tt
# -*- coding: utf-8 -*-
import xml.etree.cElementTree as ET
nombre_fichero="biblioteca.xml"

def imprime_elemento(elemento):
    print "etiqueta:",elemento.tag
    print "atributos:",elemento.attrib
    print "texto:", elemento.text
    print "elementos incluidos en ",elemento.tag
    for subelemento in elemento:
        imprime_elemento(subelemento)

def main():
    arbol=ET.ElementTree(file=nombre_fichero)
    root=arbol.getroot()
    imprime_elemento(root)

if __name__ == "__main__":
    main()
  \end{verbatim}
  \end{footnotesize}
\end{frame}


%%---------------------------------------------------------------
\begin{frame}[fragile]
\frametitle{}

  \begin{tiny}
  \begin{verbatim}
etiqueta: biblioteca
atributos: {}
texto: 
    
elementos incluidos en  biblioteca
etiqueta: libro
atributos: {'fecha': 'july 2012', 'edicion': '5', 'editorial': 'Wiley', 'isdn': '978-1-118-16213-2'}
texto: 
        
elementos incluidos en  libro
etiqueta: autor
atributos: {}
texto: 
            Joe Fawcett
        
elementos incluidos en  autor
etiqueta: autor
atributos: {}
texto: 
            Danny Ayers
        
elementos incluidos en  autor
etiqueta: autor
atributos: {}
texto: 
            Liam R. E. Quin
        
elementos incluidos en  autor
etiqueta: titulo
atributos: {}
texto: 
            Beginning XML
        
[...]
  \end{verbatim}
  \end{tiny}

\end{frame}


%%---------------------------------------------------------------
\begin{frame}[fragile]
\frametitle{}
Este programa produce un resultado idéntico

  \begin{footnotesize}
  \begin{verbatim}
#!/usr/bin/python -tt
# -*- coding: utf-8 -*-
import xml.etree.cElementTree as ET

nombre_fichero="biblioteca.xml"
def main():
    root=ET.ElementTree(file=nombre_fichero).getroot()
    for elemento in root.iter():
       print "etiqueta:",elemento.tag
       print "atributos:",elemento.attrib
       print "texto:", elemento.text

if __name__ == "__main__":
    main()
  \end{verbatim}
  \end{footnotesize}
\end{frame}


%%---------------------------------------------------------------
\begin{frame}[fragile]
\frametitle{}
Para leer desde la entrada estándar


  \begin{footnotesize}
  \begin{verbatim}
import sys
    [...]
    root=ET.ElementTree(file=sys.stdin).getroot()
    [...]

  \end{verbatim}
  \end{footnotesize}

\end{frame}


%%---------------------------------------------------------------
\begin{frame}[fragile]
\frametitle{}
A la función iter() se le puede añadir un parámetro como filtro, para que
solo devuelva los elementos con cierta etiqueta

  \begin{footnotesize}
  \begin{verbatim}
#!/usr/bin/python -tt
# -*- coding: utf-8 -*-
import xml.etree.cElementTree as ET
nombre_fichero="biblioteca.xml"

def main():
    root=ET.ElementTree(file=nombre_fichero).getroot()
    for elemento in root.iter("libro"):
       print "etiqueta:",elemento.tag
       print "atributos:",elemento.attrib
       print "texto:", elemento.text

if __name__ == "__main__":
    main()
  \end{verbatim}
  \end{footnotesize}

\end{frame}



%%---------------------------------------------------------------
\begin{frame}[fragile]
\frametitle{Modificación del documento}

  \begin{footnotesize}
  \begin{verbatim}
#!/usr/bin/python -tt
# -*- coding: utf-8 -*-
import xml.etree.cElementTree as ET
nombre_fichero="biblioteca.xml"

def main():
    root=ET.ElementTree(file=nombre_fichero).getroot()
    for elemento in root.iter("titulo"):
       print "etiqueta:",elemento.tag
       print "atributos:",elemento.attrib
       elemento.text=elemento.text.upper()
       print "texto:", elemento.text
    nuevo_arbol=ET.ElementTree(root)
    x=ET.tostring(root, encoding="utf-8", method="xml")
    print x

if __name__ == "__main__":
    main()
  \end{verbatim}
  \end{footnotesize}

\end{frame}

%%---------------------------------------------------------------
\begin{frame}[fragile]
\frametitle{Creación de nuevos documentos}
\begin{itemize}
\item
Para crear un elemento raiz, invocamos el método \verb|Element|, pasando
como argumento su etiqueta
  \begin{footnotesize}
  \begin{verbatim}
    root = ET.Element(u"holamundo")
  \end{verbatim}
  \end{footnotesize}


\item
Para crear un elemento, invocamos el método \verb|SubElement|, pasando como
primer argumento el elemento raiz, y como segundo elemento la etiqueta

  \begin{footnotesize}
  \begin{verbatim}
    elemento=ET.SubElement(root, u"hola_europa")
  \end{verbatim}
  \end{footnotesize}

\end{itemize}
\end{frame}


%%---------------------------------------------------------------
\begin{frame}[fragile]
\frametitle{}

  \begin{footnotesize}
  \begin{verbatim}
#!/usr/bin/python -tt
# -*- coding: utf-8 -*-
import xml.etree.cElementTree as ET
def main():
    root = ET.Element(u"holamundo")
    root.attrib={u"fecha":u"febrero 2016"}
    root.text=u"¡Hola, mundo!"

    elemento=ET.SubElement(root, u"hola_europa")
    atributos={u"fecha":u"marzo 2016"}
    elemento.attrib=atributos
    elemento.text=u"¡Hola, Europa!"

    elemento=ET.SubElement(root, u"hola_asia")
    elemento.text=u"¡Hola, asia!"
    atributos={u"fecha":u"marzo 2016"}
    elemento.attrib=atributos

    print ET.tostring(root, encoding="utf-8",  method="xml")

if __name__ == "__main__":
    main()
  \end{verbatim}
  \end{footnotesize}

\end{frame}


%%---------------------------------------------------------------
\begin{frame}[fragile]
\frametitle{}

  \begin{footnotesize}
  \begin{verbatim}
<holamundo fecha="febrero 2016">
    ¡Hola, mundo!
    <hola_europa fecha="marzo 2016">
        ¡Hola, Europa!
    </hola_europa>
    <hola_asia fecha="marzo 2016">
        ¡Hola, asia!
    </hola_asia>
</holamundo>
  \end{verbatim}
  \end{footnotesize}
\end{frame}

\end{document}





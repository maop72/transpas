
\documentclass[ucs]{beamer}

\usetheme{GSyC}
%\usebackgroundtemplate{\includegraphics[width=\paperwidth]{gsyc-bg.png}}


\usepackage[spanish]{babel}
\usepackage[utf8x]{inputenc}
\usepackage{graphicx}
\usepackage{amssymb} % Simbolos matematicos


% Metadatos del PDF, por defecto en blanco, pdftitle no parece funcionar
   \hypersetup{%
     pdftitle={La shell (II)},%
     %pdfsubject={Diseño y Administración de Sistemas y Redes},%
     pdfauthor={GSyC},%
     pdfkeywords={},%
   }
%


% Para colocar un logo en la esquina inferior de todas las transpas
%   \pgfdeclareimage[height=0.5cm]{gsyc-logo}{gsyc}
%   \logo{\pgfuseimage{gsyc-logo}}


% Par colocar antes de cada sección una página de recuerdo de índice
%\AtBeginSection[]{
%  \begin{frame}<beamer>{Contenidos}
%    \tableofcontents[currentsection]
%  \end{frame}
%}


\definecolor{darkred}{rgb}  {1.0, 0.0, 0.0}
\definecolor{darkgreen}{rgb}{0.0, 0.4, 0.0}
\definecolor{darkblue}{rgb} {0.0, 0.0, 0.8}

% for resalted text
\newcommand{\res}[1]{\textcolor{darkred}{#1}}
% for different text
\newcommand{\dif}{\textsl}
% for reserved words
\newcommand{\rw}[1]{\textrm{\textbf{#1}}}
% for commands
\newcommand{\com}[1]{\textrm{\textbf{#1}}}





\begin{document}

% Entre corchetes como argumento opcional un título o autor abreviado
% para los pies de transpa
\title[La Shell (II)]{La Shell (II)}
%\subtitle{Diseño y Administración de Sistemas y Redes}
\author[GSyC]{Departamento de Sistemas Telemáticos y Computación (GSyC)}
\institute{gsyc-profes (arroba) gsyc.es}
\date[2013]{Octubre de 2013}


%% TÍTULO
\begin{frame}
  \titlepage
  % Oportunidad para poner otro logo si se usó la opción nologo
  % \includegraphics[width=2cm]{logoesp}  
\end{frame}



%% LICENCIA DE REDISTRIBUCIÓN DE LAS TRANSPAS
%% Nota: la opción b al frame le dice que justifique el texto
%% abajo (por defecto c: centrado)
\begin{frame}[b]
\begin{flushright}
{\tiny
\copyright \insertshortdate~\insertshortauthor \\
  Algunos derechos reservados. \\
  Este trabajo se distribuye bajo la licencia \\
  Creative Commons Attribution Share-Alike 3.0\\
}
\end{flushright}  
\end{frame}



%% ÍNDICE
\begin{frame}
  \frametitle{Contenidos}
  \tableofcontents
\end{frame}


%%---------------------------------------------------------------
\section{Usos no estándar de la barra}
%%---------------------------------------------------------------
\begin{frame}[fragile]
\frametitle{Usos no estándar de la barra}
Un principio básico para hacer buenos programas es 

\emph{se laxo con lo que aceptas y estricto con lo que generas}
\begin{itemize}
\item
\verb|/d1//d2///d3/d4|  

En rigor es un nombre incorrecto. Aunque normalmente se admite, porque 
la shell y las librerías lo 
 \emph{limpian} y generan

\verb|/d1/d2/d3/d4|

No hay garantía de que funcione siempre, es mucho mejor evitarlo
\item

\verb|/d1/|

Algunas órdenes y algunos documentos muestran una barra al final de 
un directorio para indicar que se trata de un directorio y no un
fichero ordinario (de la misma manera que puede usarse un color distinto)

Algunas órdenes pueden esperar que un nombre acabado en barra sea un directorio
 
Pero no es un nombre estándar, es preferible evitarlo
% ls -F

\end{itemize}


\end{frame}
%%----------------------------------------------
\begin{frame}[fragile]

\begin{itemize}


\item

Para la orden cp de Mac OS

\verb|cp -r d/ . |

significa

\verb|cp -r d/* .  |

(pero solo para cp -r y solo para Mac OS)
% pj el touch, no
\end{itemize}
\end{frame}

%cp -r d1/ .
%-> copia el contenido del directorio d1 al directorio actual
% SOLO cp, SOLO MAC OS

%cp -r d1 .
%-> copia el directorio d1 al directorio actual


%%---------------------------------------------------------------
\section{Ordenes internas}
%%---------------------------------------------------------------
\begin{frame}[fragile]
  \frametitle{Ordenes internas}
La mayoría de las órdenes son externas



Pero todas las shell
interpretan ciertas órdenes por sí mismas: Las
órdenes internas (\emph{builtin commands}) 
\begin{itemize}
\item
Por
razones de eficiencia: \verb|echo, kill, pwd, test...|

Son internas aunque también tienen versión externa
\item
Necesariamente internas: \verb|cd, export, alias, unset, exit...|

Realizan funciones que tienen que hacerse
forzosamente en el proceso de la shell, harían algo completamente diferente si
se implementan como ejecutables externos 
\end{itemize}

\begin{verbatim}
koji@mazinger:~$ type echo
echo es una función integrada en la shell
\end{verbatim}

\end{frame}

% algunas ordenes están "hashed". La orden hash dice cuáles. significa que
% la primera vez que se encuentran, se guardan en una tabla hash y posteriormente
% no se buscan en ningún otro sitio

%%---------------------------------------------------------------
\begin{frame}[fragile]
\frametitle{alias}
Reemplaza una cadena por otra
\begin{itemize}	
\item \texttt{alias c='clear'}

Expande c, se convierte en \emph{clear}
\item \texttt{alias }

Muestra todos los alias
\item \texttt{unalias c }

Deshace el alias
\end{itemize}
alias suele definirse en \verb|.bashrc|

Hay ataques/bromas basados en alias
\end{frame}


%%---------------------------------------------------


\begin{frame}[fragile]
\frametitle{Funcionamiento de la shell}
\begin{enumerate}
\item 
La shell lee texto de cierto fichero (stdin). Frecuentemente el texto lo
está escribiendo el usuario, así que aporta algunas facilidades (borrar, autocompletar, history)
\item 
Analiza el texto (expande metacaracteres, variables, alias) 
\item 
Busca la primera palabra, para ver si se trata de un ejecutable

\begin{itemize}
\item 
Primero la busca entre las órdenes internas
\item
Si no es interna, busca el ejecutable en 
ciertos directorios (los indicados en el PATH) 
\end{itemize}

\item
Aplica las redirecciones que correspondan
\item
Ejecuta, pasando el resto de palabras como argumento
\item 
Duerme
\begin{itemize}
\item 
A menos que lancemos el ejecutable en \emph{background}\\
\texttt{acroread file.1 \&}
\end{itemize}
\end{enumerate}
\end{frame}


%%---------------------------------------------------------------
\begin{frame}[fragile]
\frametitle{History}
Facilita la entrada de líneas

\begin{itemize}
\item
(cursor arriba y abajo)

Muestra, una a una, las órdenes introducidas
\item
\verb|!<cadena>|

Repite la última orden que empiece por \verb|<cadena>|
\item
\verb|history|

Muestra el historial de órdenes introducidas
\item
\verb|!<n>|

Repite la órden \verb|<n>|



\end{itemize}

\end{frame}



%%---------------------------------------------------------------
\section{Permisos especiales}

\subsection{SUID}
%%---------------------------------------------------------------
\begin{frame}[fragile]
\frametitle{SUID}
Sea un fichero perteneciente a un usuario
  \begin{footnotesize}
  \begin{verbatim}
-rwxr-xr-x 1 koji koji 50 2009-03-24 12:06 holamundo
  \end{verbatim}
  \end{footnotesize}
Si lo ejecuta un usuario distinto
  \begin{footnotesize}
  \begin{verbatim}
invitado@mazinger:~$ ./holamundo
  \end{verbatim}
  \end{footnotesize}
El proceso pertenece al usuario que lo ejecuta, no al dueño del fichero

  \begin{scriptsize}
  \begin{verbatim}
koji@mazinger:~$ ps -ef |grep holamundo
invitado  2307  2260 22 12:16 pts/0    00:00:00 holamundo
koji      2309  2291  0 12:16 pts/1    00:00:00 grep holamundo
  \end{verbatim}
  \end{scriptsize}

Este comportamiento es el normal y es lo deseable habitualmente
\end{frame}


%%---------------------------------------------------------------
\begin{frame}[fragile]
Pero en ocasiones deseamos que el proceso se ejecute con los
permisos del dueño
del ejecutable, no del usuario que lo invoca
\begin{itemize}
\item
Esto se consigue activando el bit SUID  (\emph{set user id})

\verb|chmod u+s fichero|

\verb|chmod u-s fichero|

En un listado detallado aparece una \verb|s| en lugar de la \verb|x| del
dueño (o una \verb|S| si no había \verb|x|)
\item
El bit SUID permite que ciertos usuarios modifiquen un fichero, pero
no de cualquier manera sino a través de cierto ejecutable

  \begin{footnotesize}
  \begin{verbatim}
-rwsr-xr-x 1 root root 29104 2008-12-08 10:14 /usr/bin/passwd
-rw-r--r-- 1 root root  1495 2009-03-23 19:56 /etc/passwd
  \end{verbatim}
  \end{footnotesize}

\end{itemize}
\end{frame}
%%----------------------------------------------
\begin{frame}[fragile]
\begin{itemize}
\item
El bit SUID también puede ser un problema de seguridad

\item
En el caso de los scripts, lo que se ejecuta no es
el fichero con el script, sino el intérprete

Un intérprete con bit SUID es muy peligroso, normalmente
la activación del SUID en un script no tiene efecto

\item
Para buscar ficheros con SUID activo:

\verb|find / -perm +4000|

\item
El bit SGID es análogo, cambia el GID

\verb|chmod g+s fichero|

\end{itemize}

\end{frame}

%%---------------------------------------------------------------
\subsection{Sticky bit}
%%---------------------------------------------------------------
\begin{frame}[fragile]
\frametitle{Sticky bit}
%sticky bit
\begin{itemize}	
\item
En ficheros ya no se usa
\item
En un directorio, hace que sus ficheros solo puedan ser borrados o 
renombrados por el dueño del fichero, del directorio o el \emph{root}
%\item Solo el \emph{root} puede activarlo (aunque parezca lo contrario)
\end{itemize}

Se representa con una \verb|t|, en el listado y en chmod

\begin{footnotesize}
\verb|chmod [+-]t directorio|

\verb|drwxrwxrwt 15 root root 4096 2007-02-21 13:36 /tmp/|
\end{footnotesize}


Si el directorio no tuviera permiso de ejecución, aparecería \verb|T|

\begin{footnotesize}
\verb|drwxrwx-wT|
\end{footnotesize}

\end{frame}



%%---------------------------------------------------------------
\section{Umask}
%%---------------------------------------------------------------

\begin{frame}[fragile]
\frametitle{Umask}
Orden interna que muestra y cambia la variable umask 

(\emph{user file creation mode mask})
\begin{itemize}	
\item \texttt{umask }

Devuelve el valor umask                
\item \texttt{umask nuevo-valor }

Cambia el valor umask   
\end{itemize}




¿Qué permisos tiene por omisión un fichero recién creado?
\begin{itemize}
\item
Ficheros:\verb|      666 and not umask|
\item
Directorios:\verb|    777 and not umask|
\end{itemize}

\end{frame}


%%---------------------------------------------------------------
\begin{frame}[fragile]
Ejemplo. Creación de un fichero

Calculamos el valor de umask negado
\begin{verbatim}
umask       022            000    010    010
not umask   755            111    101    101


\end{verbatim}

Hacemos \emph{and lógico} entre 666 y el valor de umask negado

\begin{verbatim}
            666            110    110    110
                  and
not umask   755            111    101    101
--------------------------------------------
            644            110    100    100 
                           rw-    r--    r--
\end{verbatim}

\end{frame}



%%---------------------------------------------------------------
\section{source}
%%---------------------------------------------------------------


\begin{frame}[fragile]
\frametitle{source}

Ejecuta un fichero en el entorno de la shell actual, que no muere.
Las variables usadas en el
fichero importado serán por tanto variables del proceso actual

El mandato \emph{punto} (.) es equivalente, (aunque puede resultar menos legible)

\begin{itemize}	
\item
\begin{verbatim}
. ~/.bashrc       # Ejecuta el código de .bashrc 
                  # en el entorno actual
 
\end{verbatim}
\item 
\begin{verbatim}
source ~/.bashrc # Forma equivalente 
\end{verbatim}
\end{itemize}
\end{frame}


%%---------------------------------------------------------------
\section{Invocación de la shell}
%%---------------------------------------------------------------

% http://tldp.org/LDP/abs/html/intandnonint.html


%%---------------------------------------------------------------
\begin{frame}[fragile]
\frametitle{Invocación de la shell}
\begin{itemize}
\item
Es frecuente desear que todas nuestras sesiones ejecuten o configuren algo,
sin necesidad de teclearlo a mano cada vez. Para hacer esto necesitamos saber
cómo funciona la \emph{invocación de la shell}
\item
Cada vez que se invoca una shell, esta ejecuta (con source) cierto
fichero 
\item
Típicamente esto se emplea para definir y exportar variables de entorno, modificar
el prompt, declarar alias...
\item
Cada tipo de shell ejecuta un fichero diferente
\begin{itemize}
\item
Una shell puede ser de login o no de login
\item
Una shell puede ser interactiva o no interactiva
\end{itemize}
\end{itemize}

\end{frame}





%%---------------------------------------------------------------
%\begin{frame}[fragile]
%\frametitle{Tipos de shell}
%\begin{enumerate}
%\item
%Encendemos el ordenador. Un proceso cuyo dueño es el root (getty, xdm, 
%gdm, kdm, ...) espera a que un usuario se identifique con su \emph{login}
%y contraseña
%% este proceso no es una shell, la shell no sabe nada de logins,
%% una shell ya se crea con el id del usuario
%\item
%Si la autenticación tiene éxito, se crea un shell con el id de este usuario,
%con \verb|stdin| redirigido desde la consola de este usuario y con
%\verb|stdout| y \verb|stderr| rediridos a la consola de este usuario
%
%Esta  \textcolor{red}{es un shell de login}
%
%\begin{tiny}
%(También se puede forzar a que una shell se consider de login con
%la opción \verb|--login| o pasando \verb|-| como primer caracter
%de su argumento cero)
%\end{tiny}
%
%\item
%
%Posteriormente, a partir de esta shell el usuario generará nuevos procesos,
%incluyendo nuevas shell. Estas,  normalmente,
%\textcolor{red}{no son shell de login}
%
%% en Mac OS sí lo son: no te vuelve a pedir el login porque ya
%% te has autenticado, pero considera que cada ventanita es una 
%% sesión nueva. Mac OS pagó por el derecho a llamarse unix,
%% pero no sigue muchos de los convenios de un unix tradicional,
%% así que considera cada terminal como una sesión nueva
%
%\end{enumerate}

%\end{frame}


% en un entorno gráfico, la shell de login no la veo, está oculta.
% Pero se ha ejecutado en algún momento, porque el bash_profile
% tiene que ejecutarse al comienzo de sesión, y eso solo puede
% hacerlo una shell

% FALSO: en octubre de 2011 pruebo en ubuntu en fluxbox y en gnome,
% y no se ejecuta el bash_profile en shells "normales"
% (con ssh, sí)

% toda shell o bien es de login, o bien alguno de sus antepasados
% lo es.




%%---------------------------------------------------------------
\begin{frame}[fragile]


\begin{itemize}
\item
Una 
\textcolor{red}{shell de login}
es aquella en la que el usuario ha introducido login y contraseña
\item
En general, una 
\textcolor{red}{shell interactiva}
es aquella que tiene 
\verb|stdin| redirigida desde la consola de un usuario,
y \verb|stdout| y \verb|stderr| redirigidos a la consola de un usuario
% An interactive shell is one started without non-option arguments, unless `-s' is specified, without specifiying the `-c' option, and whose input and output are both connected to terminals (as determined by isatty(3)), or one started with the `-i' option. 
% http://www.faqs.org/docs/bashman/bashref_65.html#SEC72
\end{itemize}
\end{frame}

%%---------------------------------------------------------------

%%---------------------------------------------------------------
\begin{frame}[fragile]
\frametitle{ Bash interactivo y de login }



Ejemplos: 
\begin{itemize}
\item
Una sesión en una máquina sin gráficos (p.e. un Unix antiguo, un router...)
\item
Una sesión sin gráficos en una máquina con gráficos, que se inicia pulsando  Ctrl+Alt+F1
\item
Entrar por ssh en una máquina
\end{itemize}


En este caso, la shell
\begin{itemize}
\item
Lee y ejecuta \verb|/etc/profile|
\item
Después, ejecuta el primero que encuentre de
\begin{verbatim}
  ~/.bash_profile 
  ~/.bash_login
  ~/.profile
\end{verbatim}
No se ejecuta \verb|.bashrc|, a menos que \verb|.bash_profile| lo llame.
%%\item \texttt{alias mils='ls -la'}

Al terminar ejecuta
\begin{verbatim}
  ~/.bash_logout
\end{verbatim}

\end{itemize}

\end{frame}


%%---------------------------------------------------------------
\begin{frame}[fragile]
\frametitle{Bash interactivo, no de login }

Ej: Un terminal en Gnome o en Fluxbox 

Se ejecuta
\begin{itemize}
\item
\begin{verbatim}
~/.bashrc
\end{verbatim}
\end{itemize}

No se ejecuta \verb|~/.bash_profile|

\end{frame}



%%---------------------------------------------------------------
\begin{frame}[fragile]
\frametitle{Bash no interactivo, no de login}

Ej: Un script
% pj un script,  aunque parece no funcionar con comilla invertida
\begin{itemize}
\item Se ejecuta el fichero \verb|$BASH_ENV|
\end{itemize}

%\textcolor{red}{Atención},hablamos del inicio del shell, no de unix

\end{frame}

% con umask 077 por omisión cierro todo para el resto

%%---------------------------------------------------------------
\begin{frame}[fragile]
\begin{itemize}
\item
Antes del \verb|.bashrc| de cada usuario, se ejecuta \verb|/etc/bash.bashrc|,
común para todos los usuarios

\item
Cuando se crea un usuario con \verb|adduser|, se copia en
su \emph{home} todos los fichero que haya en \verb|/etc/skel| (aquí
se guardan los ficheros de configuración por omisión para cada usuario)

\item
Hablamos siempre del inicio de la shell.
No debemos confundir todo esto con los niveles de ejecución, 
que se refienen al inicio de la máquina
(directorios \verb|/etc/rc2.d|, \verb|/etc/rcS.d|, etc)

\end{itemize}

\end{frame}



%%---------------------------------------------------------------
\begin{frame}[fragile]
\frametitle{}
\begin{itemize}
\item
Actualmente la diferencia entre shell de login y shell no de login
es algo artificial
\footnote{En un linux con gráficos, una sesión ordinaria no ejecuta
ninguna shell de login, mientras que en MacOS todas las shell 
que ejecuta el usuario
son de login}
\item
Hoy no suele resultar conveniente tener un fichero para
las de login (\verb|~/.bash_profile|) y otro distinto
para las que son no de login (\verb|~/.bashrc|)
\item
Por tanto, lo normal es configurar todo lo necesario en \verb|~/.bashrc|
y tener en \verb|~/.bash_profile| únicamente una llamada a \verb|~/.bashrc|, de la siguiente manera:


  \begin{footnotesize}
  \begin{verbatim}
if [ -f ~/.bashrc ]; then   # si existe .bashrc
    . ~/.bashrc             # ejecuta .bashrc
fi
  \end{verbatim}
  \end{footnotesize}
O lo que es lo lo mismo
  \begin{footnotesize}
  \begin{verbatim}
if test  -f ~/.bashrc ; then   # si existe .bashrc
    source ~/.bashrc             # ejecuta .bashrc
fi
  \end{verbatim}
  \end{footnotesize}

\end{itemize}

\end{frame}



%%---------------------------------------------------------------
\section{Tareas}
%%---------------------------------------------------------------
\begin{frame}[fragile]
\frametitle{Control de tareas (jobs)}


\begin{itemize}
\item
Para lanzar varios procesos que se ejecuten en paralelo lo más
cómodo suele ser abrir varias shells (una nueva terminal o una
nueva pestaña en el terminal o un multiplexor de terminales
como screen)

\item
Pero también es posible desde una única shell manejar varios
procesos simultáneamente: mediante el control de tareas (jobs)
\end{itemize}


\end{frame}
%%----------------------------------------------


%%---------------------------------------------------------------
\begin{frame}[fragile]
\frametitle{}
Un proceso puede ejecutarse en primer o en segundo plano
\begin{itemize}
\item
En primer plano (\emph{foreground}) recibe
órdenes desde el teclado, como Ctrl Z  (detener
temporalmente) o Ctrl C (finalizar)

Cada shell solo puede tener un proceso en primer plano
\item
En segundo plano no tiene vinculada su entrada
estándar desde el teclado, no recibe las señales Ctrl Z o Ctrl D.
Es necesario emplear \emph{kill}

Puede haber varios procesos en segundo plano
\end{itemize}

De la misma manera, un proceso detenido puede estar tanto en primer
como en segundo plano
\end{frame}


%%---------------------------------------------------------------
\begin{frame}[fragile]
\frametitle{}
\begin{itemize}
\item
La orden \emph{jobs} indica, en cada línea, número de tarea, estado y nombre
\end{itemize}

  \begin{footnotesize}
  \begin{verbatim}
koji@mazinger:~$ jobs
[1]   Ejecutando              xcalc &
[2]-  Ejecutando              evince &
[3]+  Detenido                gedit
  \end{verbatim}
  \end{footnotesize}

\begin{itemize}
\item
El signo \verb|+| indica tarea por omisión, aquella que
se sobreentiende si no se indica número de tarea.
Si la tarea por omisión muere, la siguiente será la marcada
con el signo \verb|-|
% cuando muera el último, el penultimo pasa a ser último
\item

\verb|kill %n| envía señal al proceso con el job \verb|n|
(El símbolo de porcentaje indica nº de job, su ausencia indica pid)
\item
Algunos programas multiproceso, complejos, aunque los lancemos desde una shell no son
hijos de esa shell y no figurarán en la lista de tareas. Por ejemplo firefox o nautilus
\end{itemize}
\end{frame}


%%---------------------------------------------------------------
\begin{frame}[fragile]
\frametitle{}
\begin{itemize}
\item
\verb|fg n| 

pone la tarea \verb|n| en ejecución en primer plano
\item
\verb|bg n| 

pone la tarea \verb|n| en ejecución en segundo plano 

El resultado es el mismo que si hubiéramos lanzado la orden con el símbolo \verb|&|
\end{itemize}
Las órdenes 
\verb|bg| y
\verb|fg| 
pueden lanzarse sin indicar \verb|n|, entonces se sobreentiende la tarea
por omisión. 

La orden \verb|kill| necesita que se le indique siempre explícitamente
el número de tarea o el numero de pid

\end{frame}


%%----------------------------------------------
\begin{frame}[fragile]
  \begin{footnotesize}
  \begin{verbatim}
vmstat 1 Lanzo vmstat, indicando que se actualice cada 1 segundo.
Ctrl Z   Detengo el proceso. La shell me indica su número de trabajo. 
fg 1     El trabajo 1 vuelve a primer plano. No puedo usar la shell.
Ctrl Z   Vuelvo a detenerlo.
jobs     Listado de todos los trabajos.
bg 1     El trabajo 1 se ejecuta en segundo plano. Sigue escribiendo
         en stdout, pero puedo usar la shell.
         En este momento no puedo matarlo con ctrl C.
fg       El trabajo pasa a primer plano, puedo matarlo.
  \end{verbatim}
  \end{footnotesize}



\end{frame}


%%---------------------------------------------------------------
\section{Screen}
%%---------------------------------------------------------------

%%---------------------------------------------------------------
\begin{frame}[fragile]
\frametitle{nohup}
\begin{itemize}
\item
Normalmente, cuando un usuario cierra una sesión, todos sus procesos reciben
la señal SIGHUP y mueren
\item
Si tenemos procesos que queremos que se continúen ejecutando aunque el
usuario cierre la sesión, podemos usar \verb|nohup|
\begin{itemize}
\item
\verb|nohup <orden>|

\verb|<orden>| ignorará la señal SIGHUP. Escribirá stdout en ./nohup.out (o en \verb|~/nohup.out|)
\item
Si necesitamos stdin, es necesario redirigirla desde un fichero 
\end{itemize}
\end{itemize}

\end{frame}


%%---------------------------------------------------------------
\begin{frame}[fragile]
\frametitle{Screen}
\begin{itemize}
\item
Screen es una alternativa a \verb|nohup| mucho más potente: Además
de mantener el proceso vivo cuando el usuario se desconecta, posteriormente
se puede seguir usando interactivamente stdin y stdout
\item
Otra ventaja:

\begin{itemize}
\item
Normalmente, si deseo tener n sesiones en una máquina remota, es necesario abrir
n conexiones mediante ssh
\item
Usando screen, puedo abrir una única conexión ssh a una sesión screen, y en ella usar n ventanas
\end{itemize}
\item
Inconventes:
\begin{itemize}
\item
No es POSIX
\item
Es necesario memorizar media docena de atajos de teclado
\end{itemize}
\end{itemize}

\end{frame}

%%---------------------------------------------------------------
\begin{frame}[fragile]
\frametitle{}
Screen maneja \emph{sesiones} 

\begin{itemize}
\item
Una sesión de screen
permite que un usuario se desasocie de ella (dettach). El
usuario puede desconectarse y la sesión permanece (todos los procesos
se siguen ejecutando).
Cuando el usuario vuelva a conectarse (típicamente por ssh)
puede reasociarse (reattach) 
\end{itemize}

En cada sesión puede haber
diferentes \emph{ventanas}
\begin{itemize}
\item
No son ventanas al estilo Windows / X Window ni incluso ncurses
\item
Se parece a tener varias pestañas en un gnome-terminal, o a diferentes sesiones en alt F1, alt F2
\end{itemize}


\end{frame}


%%---------------------------------------------------------------
\begin{frame}[fragile]
\frametitle{}
Uso típico
  \begin{footnotesize}
  \begin{verbatim}
screen       Creamos una sesión de screen y nos asociamos a ella
screen -ls   Vemos listado de sesiones
screen -d    Nos desasociamos de la sesión actual
  \end{verbatim}
  \end{footnotesize}
Desconectamos ssh o cerramos el terminal. Volvemos a conectarnos

  \begin{footnotesize}
  \begin{verbatim}
screen -ls        Vemos listado de sesiones
screen -r nombre  Nos reasociamos a la sesión
  \end{verbatim}
  \end{footnotesize}
\end{frame}


%%---------------------------------------------------------------
\begin{frame}[fragile]
\frametitle{Uso de ventanas}

Ordenes básicas para el uso de ventanas en screen
  \begin{footnotesize}
  \begin{verbatim}
ctrl a "     Ver todas las ventanas de la sesión a las que estoy 
             asociado, con los cursores elijo una.
ctrl a c     Crear una nueva ventana dentro de la sesión actual
             (c minúscula)

  \end{verbatim}
  \end{footnotesize}

Otras órdenes para el uso de ventanas
  \begin{footnotesize}
  \begin{verbatim}
ctrl a A     Cambiar el nombre a la ventana actual
             (A mayúscula)
ctrl a ?     Ayuda
  \end{verbatim}
  \end{footnotesize}

\end{frame}
%%----------------------------------------------
\begin{frame}[fragile]
\frametitle{Observaciones}


Si queremos desconectarnos de la máquina manteniendo la sesión de screen
para usarla en otro momento
\begin{itemize}
\item
Nos desasociamos de la sesión (screen -d)
\item
Cerramos la shell (exit/Ctrl d)
\end{itemize}

Si quieremos cerrar una sesión de screen
\begin{itemize}
\item
Cerramos ordenadamente todas las ventanas (todas las shell, con exit o ctrl d).
Esto cierra de forma definitiva todos los procesos
\end{itemize}



\end{frame}

%%---------------------------------------------------------------
\begin{frame}[fragile]
\frametitle{}
Si vemos una sesión como 

  \begin{footnotesize}
  \begin{verbatim}
koji@mazinger:~$ screen -ls
There is a screen on:
        4680.pts-3.mazinger      (18/01/11 12:54:05)     (Attached)
1 Socket in /var/run/screen/S-koji.
  \end{verbatim}
  \end{footnotesize}
Esto significa que la sesión de screen 4680 ya tiene un terminal asociado, pero no sabemos si es el nuetro o es otro

Podemos saber si  nuestro terminal está en alguna sesión de screen con  \verb|ctrl a "|
\begin{itemize}
\item
Si estamos en screen,  veremos un listado de sus ventanas
\item
Si no, veremos \verb|"|

\end{itemize}

\end{frame}


%%---------------------------------------------------------------
\begin{frame}[fragile]
\frametitle{}
Multi display mode
\begin{itemize}
\item

  \begin{footnotesize}
  \begin{verbatim}
screen -r -x <nombre_sesion>   
  \end{verbatim}

  \end{footnotesize}
Nos asocia a una sesión de screen aunque ya haya otra sesión asociada

(Podremos usar ambos terminales)
%. (Los bucles no se detectan
\end{itemize}

Encontraremos un buen tutorial sobre screen \emph{googleando}:
\begin{itemize}
\item
\emph{Jeff linux screen tutorial}
\end{itemize}

Una vez familiarizados con screen, podemos usar \emph{byobu}, un
recubrimiento de screen con algunas mejoras en el interfaz de usuario

\end{frame}

\end{document}
%%----------------------------------------------
%-------------------------------------END DOCUMENT
%%----------------------------------------------






\begin{frame}[fragile]
  \section{Inicio de sesión}
{\flushleft\fontsize{8pt}{8pt}\selectfont
\verb|Debian GNU/Linux 3.1 blas tty1|\\
\verb| |\\
\verb|blas login: |\texttt{\textbf{cespedes}}\\
\verb|Password:|\\
\verb| |\\
\verb|Last login: Sat Jan 29 17:25:03 2005 on tty4|\\
\verb| |\\
\verb|cespedes@blas:~$ |\texttt{\textbf{id}}\\
\verb|uid=1000(cespedes) gid=1000(cespedes) groups=20(dialout),24(cdrom),|\\
\verb|25(floppy),29(audio),44(video),1000(cespedes)|\\
\verb|cespedes@blas:~$ |\texttt{\textbf{su}}\\
\verb|Password:|\\
\verb|blas:/home/cespedes# |\texttt{\textbf{id}}\\
\verb|uid=0(root) gid=0(root) groups=0(root)|\\
\verb|blas:/home/cespedes# |\texttt{\textbf{exit}}\\
\verb|exit|\\
\verb|cespedes@blas:~$ |\texttt{\textbf{id}}\\
\verb|uid=1000(cespedes) gid=1000(cespedes) groups=20(dialout),24(cdrom),|\\
\verb|25(floppy),29(audio),44(video),1000(cespedes)|\\
}
\end{frame}

\begin{frame}[fragile]
  \section{El intérprete de órdenes}
  El intérprete, o \emph{shell}, es el programa encargado de
  leer las órdenes dadas por un usuario y ejecutarlas una a una

  Hay muchas \emph{shell}s distintas. Hoy en día, la más común es \texttt{bash}.
  \begin{itemize}
    \item Órdenes internas: control de la shell y programación:\\
	    \texttt{cd}, \texttt{echo}, \texttt{exit}, \texttt{if}, \texttt{for}, \texttt{while}, \texttt{break}, \texttt{return}\ldots
    \item Órdenes externas: ejecución de otro programa
    \item Variables: comienzan con \verb+$+.\\
	    Predefinidas: \verb+$HOME+, \verb+$PATH+, \verb+$$+\ldots\\
	    \verb+cespedes@blas:~$ +\texttt{\textbf{echo \$PATH}}\\
	    \verb+/home/cespedes/bin:/usr/local/bin:/usr/bin:+\\
	    \verb+/bin:/usr/bin/X11:/usr/games+
  \end{itemize}
\end{frame}

\begin{frame}[fragile]
  \section{Páginas de manual}
  \texttt{\$ \textbf{man} \emph{[sección] página}} \hspace{\fill} (\texttt{\textbf{man} \emph{sección} \textbf{intro}})
  \begin{itemize}
    \item[]\textbf{Sección 1.} Órdenes de usuario
    \item[]\textbf{Sección 2.} Llamadas al sistema
    \item[]\textbf{Sección 3.} Llamadas a funciones de biblioteca
    \item[]\textbf{Sección 4.} Ficheros especiales
    \item[]\textbf{Sección 5.} Ficheros, formatos y protocolos
    \item[]\textbf{Sección 6.} Juegos
    \item[]\textbf{Sección 7.} Convenciones, miscelánea
    \item[]\textbf{Sección 8.} Órdenes de administrador y privilegiadas
  \end{itemize}
\end{frame}

\begin{frame}[fragile]
  \section{Operaciones con ficheros: consulta}
  \begin{itemize}
    \item \texttt{pwd}$^*$ $\rightarrow$ Indica cuál es el directorio actual
    \item \texttt{cd}$^*$ $\rightarrow$ Cambia el directorio actual
    \item \texttt{ls} $\rightarrow$ Muestra el contenido de un directorio
    \item \texttt{cat} $\rightarrow$ Muestra el contenido de un fichero
    \item \texttt{find} $\rightarrow$ Busca ficheros que cumplan
          determinadas catacterísticas
    \item \texttt{du} $\rightarrow$ Indica el tamaño de un fichero
    \item \texttt{cmp} $\rightarrow$ Compara dos ficheros
    \item \texttt{diff} $\rightarrow$ Muestra diferencias entre dos ficheros
  \end{itemize}
\end{frame}

\begin{frame}[fragile]
  \section{Operaciones con ficheros: creación, borrado}
  \begin{itemize}
    \item \texttt{cp} $\rightarrow$ Copia un fichero en otro destino
    \item \texttt{mv} $\rightarrow$ Renombra un fichero y/o lo mueve
          de un directorio a otro
    \item \texttt{ln} $\rightarrow$ Crea un enlace de un fichero a otro
    \item \texttt{rm} $\rightarrow$ Borra un fichero
    \item \texttt{rmdir} $\rightarrow$ Borra un directorio
    \item \texttt{mkdir} $\rightarrow$ Crea un directorio
    \item \texttt{mknod}$^*$ $\rightarrow$ Crea un fichero ``especial''
    \item \texttt{tar} $\rightarrow$ Manipula conjuntos de ficheros
    \item \texttt{gzip} $\rightarrow$ Comprime un fichero
  \end{itemize}
\end{frame}

\begin{frame}[fragile]
  \section{Operaciones con ficheros: manipulación}
  \begin{itemize}
    \item \texttt{touch} $\rightarrow$ Modifica la fecha de última
          modificación de un fichero
    \item \texttt{chmod} $\rightarrow$ Cambia los permisos de acceso
          a un fichero
    \item \texttt{chown}$^*$ $\rightarrow$ Cambia el dueño de un fichero
    \item \texttt{chgrp}$^*$ $\rightarrow$ Cambia el grupo de un fichero
  \end{itemize}
\end{frame}

\begin{frame}[fragile]
  \section{Manejo de ficheros de texto}
  \begin{itemize}
    \item Editores: \texttt{ed}, \texttt{vi} (\texttt{vim}, \texttt{elvis},
		    \texttt{nvi}), \texttt{emacs} (\texttt{xemacs},
		    \texttt{jove}), \texttt{gedit}, \texttt{kedit},
		    \texttt{nano}, \texttt{jed}, \texttt{wily},
		    \texttt{the}, \texttt{aee}\ldots
    \item Búsqueda de cadenas: \texttt{grep}
{\flushleft\fontsize{6pt}{6pt}\selectfont
\verb|cespedes@blas:~$ |\texttt{\textbf{grep Debian /etc/motd}}\\
\verb|The programs included with the Debian GNU/Linux system are free software;|\\
\verb|Debian GNU/Linux comes with ABSOLUTELY NO WARRANTY, to the extent|\\
}
    \item Sustitución de cadenas: \texttt{sed}
{\flushleft\fontsize{6pt}{6pt}\selectfont
\verb|cespedes@blas:~$ |\texttt{\textbf{grep Debian /etc/motd | sed s/GNU/Ñu/}}\\
\verb|The programs included with the Debian Ñu/Linux system are free software;|\\
\verb|Debian Ñu/Linux comes with ABSOLUTELY NO WARRANTY, to the extent|\\
}
    \item Manipulación general: \texttt{awk}
  \end{itemize}
\end{frame}

\begin{frame}[fragile]
  \section{Usuarios conectados}
{\flushleft\fontsize{8pt}{8pt}\selectfont
\verb|cespedes@blas:~$ |\texttt{\textbf{who}}\\
\verb|rover    pts/0        Feb 16 09:14 (:0.0)|\\
\verb|cespedes pts/1        Feb 16 10:56 (:0.0)|\\
\verb|cespedes@blas:~$ |\texttt{\textbf{who am i}}\\
\verb|cespedes pts/1        Feb 16 10:56 (:0.0)|\\
\verb|cespedes@blas:~$ |\texttt{\textbf{tty}}\\
\verb|/dev/pts/1|\\
\verb|cespedes@blas:~$ |\texttt{\textbf{finger}}\\
\verb|Login     Name               Tty      Idle  Login Time   Office|\\
\verb|rover     Roberto Lumbreras  pts/0    1:59  Feb 16 09:14 (:0.0)|\\
\verb|cespedes  Juan Cespedes      pts/1          Feb 16 10:56 (:0.0)|\\
\verb|cespedes@blas:~$ |\texttt{\textbf{w}}\\
\verb| 10:56:15 up 18:12,  2 users,  load average: 0,00, 0,12, 0,22|\\
\verb|USER     TTY      FROM              LOGIN@   IDLE   JCPU   PCPU WHAT|\\
\verb|rover    pts/0    :0.0             09:14    1:59m  0.02s  0.02s -bash|\\
\verb|cespedes pts/1    :0.0             10:56    0.00s  0.03s  0.01s w|\\
\verb|cespedes@blas:~$|\\}
\end{frame}

\begin{frame}[fragile]
  \section{Procesos}
{\flushleft\fontsize{8pt}{8pt}\selectfont
\verb|cespedes@blas:~$ |\texttt{\textbf{ps}}\\
\verb|  PID TTY          TIME CMD|\\
\verb| 3086 pts/1    00:00:00 bash|\\
\verb| 3087 pts/1    00:00:00 ps|\\
\verb|cespedes@blas:~$ |\texttt{\textbf{ps f}}\\
\verb|  PID TTY      STAT   TIME COMMAND|\\
\verb| 3086 pts/1    Ss     0:00 -bash|\\
\verb| 3088 pts/1    R+     0:00  \_ ps f|\\
\verb+cespedes@blas:~$ +\texttt{\textbf{ps af}}\\
\verb|  PID TTY      STAT   TIME COMMAND|\\
\verb|  623 tty1     Ss+    0:00 /sbin/getty 38400 tty1|\\
\verb|  624 tty2     Ss+    0:00 /sbin/getty 38400 tty2|\\
\verb|  625 tty3     Ss+    0:00 /sbin/getty 38400 tty3|\\
\verb|  626 tty4     Ss+    0:00 /sbin/getty 38400 tty4|\\
\verb|  627 tty5     Ss+    0:00 /sbin/getty 38400 tty5|\\
\verb|  628 tty6     Ss+    0:00 /sbin/getty 38400 tty6|\\
\verb| 2812 pts/0    Ss+    0:00 -bash|\\
\verb| 3086 pts/1    Ss     0:00 -bash|\\
\verb| 3282 pts/1    R+     0:00  \_ ps af|\\}
\end{frame}

\begin{frame}[fragile]
  \section{Procesos (2)}
{\flushleft\fontsize{7pt}{7pt}\selectfont
\verb|cespedes@blas:~$ |\texttt{\textbf{top}}\\
\verb|top - 11:55:44 up 18:52, 31 users,  load average: 0.49, 0.31, 0.25|\\
\verb|Tasks: 177 total,   3 running, 174 sleeping,   0 stopped,   0 zombie|\\
\verb|Cpu(s): 11.8% us,  2.9% sy,  0.0% ni, 85.3% id,  0.0% wa,  0.0% hi,  0.0% si|\\
\verb|Mem:   1035924k total,   687348k used,   348576k free,    89188k buffers|\\
\verb|Swap:  2457904k total,        0k used,  2457904k free,   292060k cached|\\
\verb||\\
\verb|  PID USER      PR  NI  VIRT  RES  SHR S %CPU %MEM    TIME+  COMMAND           |\\
\verb|12118 cespedes  15   0  9500 2964 2640 S  8.7  0.3   0:04.92 mpg123             |\\
\verb|12806 cespedes  16   0  2188 1128  848 R  8.7  0.1   0:00.10 top                |\\
\verb|15888 cespedes  15   0  5824 3184 2212 S  2.9  0.3   0:00.30 xterm              |\\
\verb|    1 root      16   0  1500  516  456 S  0.0  0.0   0:00.44 init               |\\
\verb|    2 root      34  19     0    0    0 S  0.0  0.0   0:00.06 ksoftirqd/0        |\\
\verb|    3 root       5 -10     0    0    0 S  0.0  0.0   0:04.20 events/0           |\\
\verb|    4 root       5 -10     0    0    0 S  0.0  0.0   0:00.00 khelper            |\\
\verb|   16 root       5 -10     0    0    0 S  0.0  0.0   0:00.00 kacpid             |\\
\verb|  107 root       5 -10     0    0    0 S  0.0  0.0   0:00.07 kblockd/0          |\\
\verb|  138 root      20   0     0    0    0 S  0.0  0.0   0:00.00 pdflush            |\\
\verb|  139 root      15   0     0    0    0 S  0.0  0.0   0:00.15 pdflush            |\\
\verb|  141 root      15 -10     0    0    0 S  0.0  0.0   0:00.00 aio/0              |\\
\verb|  140 root      17   0     0    0    0 S  0.0  0.0   0:00.82 kswapd0            |\\
\verb|  728 root      15   0     0    0    0 S  0.0  0.0   0:00.00 kseriod            |\\
}
\end{frame}

\begin{frame}[fragile]
  \section{Procesos (3)}
  \begin{itemize}
    \item Pausa en un proceso: \texttt{sleep}
    \item Terminación y avisos a procesos: \texttt{kill}
  \end{itemize}
  \vspace{.1cm}
{\flushleft\fontsize{8pt}{8pt}\selectfont
\verb|cespedes@blas:~$ |\texttt{\textbf{sleep 3600 \&}}\\
\verb|[1] 17093|\\
\verb|cespedes@blas:~$ |\texttt{\textbf{kill -l}}\\
\verb| 1) SIGHUP       2) SIGINT       3) SIGQUIT      4) SIGILL|\\
\verb| 5) SIGTRAP      6) SIGABRT      7) SIGBUS       8) SIGFPE|\\
\verb| 9) SIGKILL     10) SIGUSR1     11) SIGSEGV     12) SIGUSR2|\\
\verb|13) SIGPIPE     14) SIGALRM     15) SIGTERM     17) SIGCHLD|\\
\verb|18) SIGCONT     19) SIGSTOP     20) SIGTSTP     21) SIGTTIN|\\
\verb|22) SIGTTOU     23) SIGURG      24) SIGXCPU     25) SIGXFSZ|\\
\verb|26) SIGVTALRM   27) SIGPROF     28) SIGWINCH    29) SIGIO|\\
\verb|30) SIGPWR      31) SIGSYS      33) SIGRTMIN    34) SIGRTMIN+1|\\
\verb|cespedes@blas:~$ |\texttt{\textbf{kill -6 17093}}\\
\verb|[1]+ Aborted                  sleep 3600|\\
\verb|cespedes@blas:~$ |\\
}
\end{frame}

\begin{frame}[fragile]
  \section{Herramientas de gestión de discos}
  \begin{itemize}
    \item \texttt{fdisk} $\rightarrow$ Muestra y modifica la tabla
          de particiones de un disco
    \item \texttt{mkfs} $\rightarrow$ Crea un sistema de ficheros
          (\emph{formatea} una partición)
    \item \texttt{fsck} $\rightarrow$ Comprueba la integridad de un
          sistema de ficheros
    \item \texttt{df} $\rightarrow$ Muestra tablas de montaje y uso
          de cada partición
    \item \texttt{mount} $\rightarrow$ Muestra tablas de montaje,
          monta una partición
    \item \texttt{umount} $\rightarrow$ Desmonta una partición
  \end{itemize}
\end{frame}

\begin{frame}[fragile]
  \section{Herramientas de gestión de memoria}
  \begin{itemize}
    \item \texttt{top} $\rightarrow$ Información general acerca de
          los procesos del sistema y de su uso de memoria y de CPU
    \item \texttt{free} $\rightarrow$ Resumen del estado de la memoria
          en el sistema
{\flushleft\fontsize{6pt}{6pt}\selectfont
\verb|cespedes@blas:~$ |\texttt{\textbf{free}}\\
\verb|             total       used       free     shared    buffers     cached|\\
\verb|Mem:       1035924     735480     300444          0      94620     333940|\\
\verb|-/+ buffers/cache:     306920     729004|\\
\verb|Swap:      2457904          0    2457904|\\
}
  \end{itemize}
  Gestión del \emph{swap} (área de intercambio):
  \begin{itemize}
    \item \texttt{mkswap} $\rightarrow$ Crea un área de intercambio
    \item \texttt{swapon} $\rightarrow$ Activa un área de intercambio
    \item \texttt{swapoff} $\rightarrow$ Desactiva un área de intercambio
  \end{itemize}
\end{frame}

\begin{frame}[fragile]
  \section{Gestión de red (TCP/IP)}
  \begin{itemize}
    \item \texttt{ifconfig} $\rightarrow$ Muestra/modifica información
          de las tarjetas de red del sistema
{\flushleft\fontsize{6pt}{6pt}\selectfont
\verb|root@orion:~$ |\texttt{\textbf{ifconfig eth0}}\\
\verb|eth0      Link encap:Ethernet  HWaddr 00:0D:9D:D1:6A:27  |\\
\verb|          inet addr:193.147.71.90  Bcast:193.147.71.255  Mask:255.255.255.128|\\
\verb|          UP BROADCAST RUNNING MULTICAST  MTU:1500  Metric:1|\\
\verb|          RX packets:2219198 errors:0 dropped:0 overruns:0 frame:0|\\
\verb|          TX packets:949209 errors:0 dropped:0 overruns:0 carrier:0|\\
\verb|          collisions:0 txqueuelen:1000 |\\
\verb|          RX bytes:553335922 (527.7 MiB)  TX bytes:357891938 (341.3 MiB)|\\
}
    \item \texttt{route} $\rightarrow$ Muestra/modifica información
          de la tabla de rutas del sistema
{\flushleft\fontsize{6pt}{6pt}\selectfont
\verb|root@orion:~$ |\texttt{\textbf{route}}\\
\verb|Kernel IP routing table|\\
\verb|Destination     Gateway         Genmask         Flags Metric Ref    Use Iface|\\
\verb|10.3.0.2        *               255.255.255.255 UH    0      0        0 tun0|\\
\verb|193.147.71.0    *               255.255.255.128 U     0      0        0 eth0|\\
\verb|192.168.64.0    *               255.255.255.0   U     0      0        0 eth1|\\
\verb|default         193.147.71.1    0.0.0.0         UG    0      0        0 eth0|\\
}
  \end{itemize}
\end{frame}

\begin{frame}[fragile]
  \section{Gestión de red (TCP/IP) (2)}
  \begin{itemize}
    \item \texttt{netstat} $\rightarrow$ Muestra el estado de las conexiones
          TCP/IP de la máquina
{\flushleft\fontsize{6pt}{6pt}\selectfont
\verb|root@orion:~$ |\texttt{\textbf{netstat -nt}}\\
\verb|Active Internet connections (w/o servers)|\\
\verb|Proto Recv-Q Send-Q Local Address           Foreign Address         State      |\\
\verb|tcp        0      0 193.147.71.90:137       193.147.62.27:32887     ESTABLISHED|\\
\verb|tcp        0      0 193.147.71.90:36054     193.147.71.88:80        TIME_WAIT|\\
\verb|tcp6       0     48 ::ffff:193.147.71.90:22 ::ffff:193.147.62:33050 ESTABLISHED|\\
\verb|tcp6       0      0 ::ffff:193.147.71.90:22 ::ffff:193.147.62:32983 ESTABLISHED|\\
}
  \end{itemize}
  Información acerca de protocolos, servicios, puertos, etc. en los ficheros
\texttt{/etc/protocols} y \texttt{/etc/services}
\end{frame}



\end{document}




\end{document}



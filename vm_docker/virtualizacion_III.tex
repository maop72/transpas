
\documentclass[ucs]{beamer}

\usetheme{GSyC}
%\usebackgroundtemplate{\includegraphics[width=\paperwidth]{gsyc-bg.png}}


\usepackage[spanish]{babel}
\usepackage[utf8x]{inputenc}
\usepackage{graphicx}
\usepackage{amssymb} % Simbolos matematicos


% Metadatos del PDF, por defecto en blanco, pdftitle no parece funcionar
   \hypersetup{%
     pdftitle={Virtualización III},%
     %pdfsubject={Diseño y Administración de Sistemas y Redes},%
     pdfauthor={GSyC},%
     pdfkeywords={},%
   }
%


% Para colocar un logo en la esquina inferior de todas las transpas
%   \pgfdeclareimage[height=0.5cm]{gsyc-logo}{gsyc}
%   \logo{\pgfuseimage{gsyc-logo}}


% Para colocar antes de cada sección una página de recuerdo de índice
%\AtBeginSection[]{
%  \begin{frame}<beamer>{Contenidos}
%    \tableofcontents[currentframetitle]
%  \end{frame}
%}


%\newboolean{detallado}
%\setboolean{detallado}{false}
%\newcommand{\opcional}[1]{\ifthenelse{\boolean{detallado}}{#1}{}}


\definecolor{darkred}{rgb}  {1.0, 0.0, 0.0}
\definecolor{darkgreen}{rgb}{0.0, 0.4, 0.0}
\definecolor{darkblue}{rgb} {0.0, 0.0, 0.8}

% for resalted text
\newcommand{\res}[1]{\textcolor{darkred}{#1}}
% for different text
\newcommand{\dif}{\textsl}
% for reserved words
\newcommand{\rw}[1]{\textrm{\textbf{#1}}}
% for commands
\newcommand{\com}[1]{\textrm{\textbf{#1}}}






\begin{document}

% Entre corchetes como argumento opcional un título o autor abreviado
% para los pies de transpa
\title[Virtualización III]{Virtualización III}

%\subtitle{Diseño y Administración de Sistemas y Redes}
\author[GSyC]{Escuela Técnica Superior de Ingeniería de Telecomunicación\\
Universidad Rey Juan Carlos}
\institute{gsyc-profes (arroba) gsyc.urjc.es}
\date[2017]{Noviembre de 2017}



%% TÍTULO
\begin{frame}
  \titlepage
  % Oportunidad para poner otro logo si se usó la opción nologo
  % \includegraphics[width=2cm]{logoesp}  
\end{frame}



%% LICENCIA DE REDISTRIBUCIÓN DE LAS TRANSPAS
%% Nota: la opción b al frame le dice que justifique el texto
%% abajo (por defecto c: centrado)
\begin{frame}[b]
\begin{flushright}
{\tiny
\copyright \insertshortdate~\insertshortauthor \\
  Algunos derechos reservados. \\
  Este trabajo se distribuye bajo la licencia \\
  Creative Commons Attribution Share-Alike 4.0\\
}
\end{flushright}  
\end{frame}



%% ÍNDICE
\begin{frame}
  \frametitle{Contenidos}

%\begin{tiny}
  \tableofcontents
%\end{tiny}

\end{frame}


\section{Docker: prerrequisitos}

%%---------------------------------------------------------------
\begin{frame}[fragile]
\frametitle{Plataformas para ejecuta docker}
Docker tiene arquitectura cliente servidor
\begin{itemize}
\item
El cliente acepta la entrada del usuario, le muestra la salida,
maneja los ficheros con los que preparar las imágenes

\item
El servidor ejecuta el contenedor
\end{itemize}


El servidor solo está disponible para Linux 64 bits

Hay versiones para MacOS y Microsoft Windows, donde el cliente
se ejecuta en nativo contra un servidor dentro de una máquina virtual

\begin{itemize}
\item
Esta virtualización es transparente para el usuario
\end{itemize}


\end{frame}
%%----------------------------------------------
\begin{frame}[fragile]
\frametitle{Docker dentro de una máquina virtual}

Si vamos a ejecutar docker dentro de una máquina virtual,

\begin{itemize}
\item
guest y host deben tener arquitectura 64 bits
\item
Es necesario que el host tenga soporte para Intel VT-x

\begin{itemize}
\item
Los equipos antiguos no lo permiten (VT-x es del año 2006, pero no se generaliza
en los equipos de gama básica/media hasta varios años después)
\item
Muchos equipos actuales tienen  esta opción deshabilitada por omisión en la BIOS/UEFI
\end{itemize}
\end{itemize}
\end{frame}


%%---------------------------------------------------------------
\begin{frame}[fragile]
\frametitle{Algunos conceptos}
Imágenes:
\begin{itemize}
\item
La imagen de un contenedor (o simplemente \emph{imagen})
es un fichero en el sistema de ficheros del host

\item
Un contenedor se ejecuta a partir de una imagen
\end{itemize}

\end{frame}
%%----------------------------------------------
\begin{frame}[fragile]


Manejaremos diversos identificadores, que no debemos confundir


\begin{itemize}
\item
Nombre de la imagen. 
Ejemplos:  

\emph{debian} 

\emph{test/c01} 

\item
Identificador de la imagen. Ejemplo:

cc8393a39248

\item
Nombre de contenedor. Si no lo indicamos explícitamente,
docker usará nombres aleatorios como \emph{focused\_yonath} 
o
\emph{wonderfuld\_goldberg}

\item
Identificador de contenedor

Ejemplo:
18009dd9f349

\item
Nombre de host

Nombre de máquina que se percibirá dentro del contenedor. (variable
de entorno \verb|$HOST|, nombre en el 
\emph{prompt},
fichero
\verb|/etc/hostname|, etc)

Atención: Este \emph{host} de Docker se corresponde con lo que en VirtualBox 
sería el \emph{guest}

\end{itemize}


\end{frame}
%%----------------------------------------------
\begin{frame}[fragile]

\frametitle{Nombres de imagen}


El nombre de la imagen


\begin{itemize}
\item
Un nombre sin prefijo, por ejemplo
\emph{debian} 
indica una imagen oficial
aprobada por docker

\item
Un nombre con prefijo, por ejemplo
\emph{test/c01}
es una imagen no oficial. El prefijo puede ser una etiqueta que hayamos
definido o un nombre de usuario en un registro de imágenes
\end{itemize}


\end{frame}





\section{Instalación de Docker}

%%---------------------------------------------------------------
\begin{frame}[fragile]
\frametitle{Instalación de docker}
\begin{itemize}
\item
Podemos instalar el paquete incluido en ubuntu, aunque la versión instalada
podrá ser bastante antigua (actualmente docker evoluciona rápido)


  \begin{scriptsize}
  \begin{verbatim}
apt-get update; apt-get upgrade -y ; apt-get install docker.io
  \end{verbatim}
  \end{scriptsize}




\item
Mejor usamos el script disponible en
\verb|https://get.docker.com|


  \begin{scriptsize}
  \begin{verbatim}
wget https://get.docker.com -O get-docker.sh  #letra "O" mayúscula
bash get-docker.sh
  \end{verbatim}
  \end{scriptsize}


\end{itemize}

\end{frame}



\section{Ejecución de imágenes}
%%---------------------------------------------------------------
\begin{frame}[fragile]
\frametitle{Lanzar una imagen}

Para ejecutar docker, tenemos dos opciones
\begin{itemize}
\item
Añadir nuestro usuario al grupo docker

  \begin{scriptsize}
  \begin{verbatim}
addgrp docker
adduser $USER docker
# (abrir una nueva sesión)
  \end{verbatim}
  \end{scriptsize}

\item
Ejecutar docker con sudo
\end{itemize}

Para comprobar que la instalación ha sido correcta, lanzamos una
imagen sencilla

  \begin{scriptsize}
  \begin{verbatim}
docker run debian echo "hola,mundo"
  \end{verbatim}
  \end{scriptsize}

Esto busca en el \emph{registry} oficial de docker una imagen
llamada \emph{debian}, ejecuta en ella la orden indicada, muestra
su salida por stdout y concluye


\end{frame}


%%---------------------------------------------------------------
\begin{frame}[fragile]
\frametitle{Otro holamundo}

  \begin{scriptsize}
  \begin{verbatim}
koji@mazinger:~$ docker run hello-world
Unable to find image 'hello-world:latest' locally
latest: Pulling from library/hello-world
5b0f327be733: Pull complete 
Digest: sha256:1f19634d26995c320618d94e6f29c09c6589d5df3c063287a00e6de8458f8242
Status: Downloaded newer image for hello-world:latest

Hello from Docker!
This message shows that your installation appears to be working correctly.

To generate this message, Docker took the following steps:
 1. The Docker client contacted the Docker daemon.
 2. The Docker daemon pulled the "hello-world" image from the Docker Hub.
 3. The Docker daemon created a new container from that image which runs the
    executable that produces the output you are currently reading.
 4. The Docker daemon streamed that output to the Docker client, which sent it
    to your terminal.

To try something more ambitious, you can run an Ubuntu container with:
 $ docker run -it ubuntu bash
  \end{verbatim}
  \end{scriptsize}

\end{frame}


%%---------------------------------------------------------------
\begin{frame}[fragile]
\frametitle{Repositorio de imágenes}
Además de guardarse localmente, las imágenes están disponibles
en los \emph{registry}
\begin{itemize}
\item
%(using docker)
Registro (Registry)

Servicio responsable de almacenar y distribuir imágenes. El registro
por omisión es \verb|https://hub.docker.com|

Aunque hay otros similares, públicos. Y quien lo desee puede establecer su
propio registro

\item
Repositorio (Repository)

Una colección de imágenes relacionadas, normalmente ofrecen diferentes versiones
de la misma aplicación o servicio

\item
Etiqueta (Tag)

Identificador alfanumérico asociado a una única imagen

\end{itemize}

\end{frame}




%%---------------------------------------------------------------
\begin{frame}[fragile]
\frametitle{Docker run}

Esta instrucción lanza un contenedor a partir de una imagen

\verb|docker run <opciones> <imagen>|


\begin{itemize}
\item
La imagen se puede identificar mediante su nombre o mediante su id

\item
Las opciones \verb|-i| y \verb|-t| normalmente se usan juntas, para indicar
que queremos una sesión interactiva en un terminal

\item
\verb|--name <nombre_contenedor>|


\item
\verb|-h <nombre_host>|


\verb|--hostname=<nombre_host>|

%nos permite indicar el nombre de host dentro del contenedor. De lo contrario se
%usará el identificador de contenedor, una secuencia como p.e.  4641824a3ff6

%\footnotesize{Los nombres entre paréntesis triangulares representan un valor indicado por el usuario}
\end{itemize}

Ejemplo

  \begin{scriptsize}
  \begin{verbatim}
docker run -it --name c01 -h c01 test/im01
  \end{verbatim}
  \end{scriptsize}



\end{frame}




%%---------------------------------------------------------------
\begin{frame}[fragile]
\frametitle{Consulta de imágenes y contenedores}
\begin{itemize}
\item
\verb|docker ps|

Muestra los contenedores

\item
\verb|docker images|

Muestra las imágenes

\item
\verb|docker inspect <imagen>|

Muestra un json con descripción detallada del contenedor

\item
\verb|docker diff <imagen>|

Muestra los cambios en el sistema de ficheros del contenedor

\item
\verb|docker logs <imagen>|

Muestra las instrucciones ejecutadas en el contenedor


\end{itemize}

\end{frame}


%%---------------------------------------------------------------
\begin{frame}[fragile]
\frametitle{Exited containers}
Cuando un contenedor finaliza su ejecución, queda en estado 
\emph{exited}, al que informalmente se suele llama
\emph{parado}

\begin{itemize}
\item
\verb|docker ps -a|

Muestra los contenedores, incluyendo los parados

%\item
%\verb|docker start <contenedor>|
%
%Reinicia un contenedor parado

\item
\verb|docker rm <contenedor>|

Borra un contenedor


\item
\verb|docker rmi <imagen>|

Borra una imagen


\end{itemize}

Si el contenedor se lanza con la opción
\verb|-rm|, se borrará automáticamente al concluir

\end{frame}



%%---------------------------------------------------------------
\begin{frame}[fragile]
\frametitle{Borrado de imágenes y contenedores}
\begin{itemize}
\item
Borrar todas las imágenes (que no estén siendo usadas)

\verb|docker rmi $(docker images -a -q)|


\item
Borrar todos los contenedores detenidos

\verb|docker rm $(docker ps -a -f status=exited -q)|

\item
Borrar todos los contenedores creados (y nunca ejecutados)

\verb|docker rm $(docker ps -a -f status=created -q)|

\end{itemize}

\end{frame}





\section{Creación de imágenes}
%%---------------------------------------------------------------
\begin{frame}[fragile]
\frametitle{Creación de imágenes}
%https://docs.docker.com/engine/reference/builder/

La orden  \verb|docker build| nos permite construir imágenes.

Para construir una imagen, normalmente usaremos tres cosas:

\begin{itemize}
\item
Un directorio contexto, que será un directorio vacio en el host, donde iremos añadiendo
los ficheros necesario para construir la imagen

%(Podría ser cualquier directorio incluso no vacío, pero sería poco práctico porque el cliente docker
%tiene que pasar el directorio completo al servidor docker)



\item
Un fichero 
\verb|Dockerfile|
dentro del directorio contexto, con las instrucciones para crear la imagen

%Podría usarse otro nombre en otro directorio, usando la orden  \verb|docker build -f fichero_instruciones|


\item
Un fichero
\verb|entrypoint.sh|, 
que será un script de shell que 

\begin{itemize}
\item
Crearemos en el directorio contexto

\item
Llevaremos a la imagen 

\item
Se ejecutará cada vez que se lance un contenedor con esa imagen
\end{itemize}


\end{itemize}

\end{frame}



%%---------------------------------------------------------------
\begin{frame}[fragile]
\frametitle{}
\begin{itemize}
\item
Si la imagen es muy sencilla, puede que no necesite 
\verb|entrypoint.sh|

Ejemplo:

  \begin{scriptsize}
  \begin{verbatim}
FROM ubuntu:16.04
RUN apt-get update && apt-get upgrade -y
ENTRYPOINT /bin/bash
  \end{verbatim}
  \end{scriptsize}

\item
También es posible crear una imagen sin usar un fichero Dockerfile

Para ello basta con

    \begin{enumerate}
    \item
Entrar en el contenedor

    \item
Configurarlos: instalar paquetes, añadir ficheros, modificar ficheros...

    \item

\verb|docker commit <CONTENEDOR> <IMAGEN>|

\verb|<CONTENEDOR>|: Nombre o id del contenedor que será punto de partida da la imagen

\verb|<IMAGEN>|: Nombre que tendrá la imagen
    \end{enumerate}
\end{itemize}



\end{frame}





%%---------------------------------------------------------------
\begin{frame}[fragile]
\frametitle{Ejemplo: banner}
Vamos a crear una imagen llamada
\emph{test/banner}
basada en la orden
\emph{banner}
que al ejecutarse mostrará los siguiente:

  \begin{tiny}
  \begin{verbatim}
koji@mazinger:~/lagrs/banner$ docker run -h c01 --name c01 test/banner

 #####      #    ######  #    #  #    #  ######  #    #     #    #####    ####
 #    #     #    #       ##   #  #    #  #       ##   #     #    #    #  #    #
 #####      #    #####   # #  #  #    #  #####   # #  #     #    #    #  #    #
 #    #     #    #       #  # #  #    #  #       #  # #     #    #    #  #    #
 #    #     #    #       #   ##   #  #   #       #   ##     #    #    #  #    #
 #####      #    ######  #    #    ##    ######  #    #     #    #####    ####


   ##
  #  #
 #    #
 ######
 #    #
 #    #

          ###      #
  ####   #   #    ##
 #    # # #   #  # #
 #      #  #  #    #
 #      #   # #    #
 #    #  #   #     #
  ####    ###    #####
  \end{verbatim}
  \end{tiny}


\end{frame}




%%---------------------------------------------------------------
\begin{frame}[fragile]
\frametitle{}
Creamos un 
\emph{directorio contexto} 
en el host, y en él escribimos un fichero
\verb|entrypoint.sh|

  \begin{scriptsize}
  \begin{verbatim}
#!/bin/bash
banner bienvenido
banner a
banner $HOSTNAME
  \end{verbatim}
  \end{scriptsize}


Cuando sea posible, es muy conveniente probar este script
antes de construir la imagen, los errores aquí
son uno de los problemas más habituales preparando contenedores


\end{frame}





%%---------------------------------------------------------------
\begin{frame}[fragile]
\frametitle{}
En el 
directorio contexto 
también creamos un directorio
\verb|Dockerfile|

  \begin{scriptsize}
  \begin{verbatim}
FROM ubuntu:16.04
RUN apt-get update && apt-get upgrade -y && apt-get install -y sysvbanner
COPY entrypoint.sh /
ENTRYPOINT ["/entrypoint.sh"]
  \end{verbatim}
  \end{scriptsize}


\begin{itemize}
\item
La instrucción FROM indica la imagen de partida

\item
La instrucción RUN indica las modificaciones a realizar en la imagen

Puede haber más de un RUN, pero eso crea imágenes intermedias, por lo
que lo habitual es encadenar varias órdenes de shell con \verb|&&|

\item
La opción 
\verb|-y|
en

\verb|apt-get upgrade|

\verb|apt-get install| 

es imprescindible (contesta a todas las preguntas con \emph{yes})

\end{itemize}

\end{frame}



%%---------------------------------------------------------------
\begin{frame}[fragile]
\frametitle{}
\begin{itemize}
\item
La instrucción COPY copia un fichero desde el directorio contexto
(que está en el host) hasta el sistema de ficheros del futuro contenedor
que se ejecute a partir de la imagen

\item
La instrucción ENTRYPOINT especifica el fichero que se ejecutará al iniciar
cada contenedor

Es habitual llamarlo 
\verb|entrypoint.sh|
y colocarlo en el directorio raiz del contenedor

\item
En el Dockerfile se pueden crear comentarios con el caracter \verb|#|

\item
El contenido del Dockerfile  es
\emph{case insensitive},
aunque el convenio es usar mayúsculas para las instrucciones

\item

Si no existe un fichero
\verb|Dockerfile|,
docker busca un fichero
\verb|dockerfile| 
% con -f podemos cambiar el nombre del fichero
\end{itemize}


\end{frame}
%%----------------------------------------------
\begin{frame}[fragile]
Una vez preparados los ficheros, construimos la imagen


%\verb|docker build -t <nombre_imagen>  <directorio_contexto>|


\begin{itemize}
\item
Desde el directorio padre del directorio contexto ejecutamos

\verb|docker build -t test/banner directorio_contexto|
\end{itemize}

Recuerda que los nombres de las imágenes que crearemos siempre 
llevarán prefijo (puesto que no son imágenes oficiales)

Almacenamiento de la configuración:

\begin{itemize}
\item
La configuración de docker se guarda en /var/lib/docker

\item
Las imágenes, depende del driver que docker use para el almacenamiento.
Por omisión se usa aufs, que guarda  las imágenes en
\verb|/var/lib/docker/aufs|
\end{itemize}

%https://stackoverflow.com/questions/19234831/where-are-docker-images-stored-on-the-host-machine


\end{frame}



%%---------------------------------------------------------------
\begin{frame}[fragile]
\frametitle{Gestión de datos en docker}

El sistema de ficheros interior al contenedor es volátil

\begin{itemize}
\item
Todo lo escrito
durante la ejecución del contenedor se pierde al borrar el contenedor
\item
Es complicado acceder a esos datos sin usar el mismo contenedor

\end{itemize}
Podríamos guardar datos en una nueva capa creando una nueva imagen,
pero sería poco práctico, no es recomendable

\end{frame}



%%---------------------------------------------------------------
\begin{frame}[fragile]

Un contenedor no debería tener estado. O en su defecto, el mínimo
estado posible

\frametitle{}
Docker ofrece 3 mecanismos para la persistencia de datos
\begin{itemize}
\item
Bind mounts
\item
Volumes
\item
tmpfs
\end{itemize}

Por supuesto, dentro del contenedor se puede usar cualquier
otro protocolo o servicio no específico de Docker:
NFS, sshfs, SMB, rsync, almacenamiento en la nube,
bases de datos relacionales, 
bases de datos no relacionales...
\end{frame}



%%---------------------------------------------------------------
\begin{frame}[fragile]
\frametitle{Bind mounts}
Un \emph{bind mount} es un directorio del host que se comparte con uno (o varios) contenedores

\begin{itemize}
\item
Muy eficientes

\item
Muy prácticos para compartir datos con el host

\item
Dependen del sistema de ficheros del host y de su estructura,
con lo que tienen problemas de portabilidad

\item
Evidentes problemas potenciales de seguridad, al tener el contenedor
acceso directo al sistema de ficheros del host
\end{itemize}

\end{frame}


%%---------------------------------------------------------------
\begin{frame}[fragile]
\frametitle{}


Para hacer un bind mount, basta añadir los siguientes parámetros
a la orden \verb|docker run|


\begin{itemize}
\item
Sintaxis tradicional

\verb|--v <DIR_ORIGEN>:<DIR_DESTINO>|

\item
Sintaxis moderna, disponible a partir de Docker 17.06

\verb|--mount type=bind,source=<DIR_ORIGEN>,target=<DIR_DESTINO>|
\end{itemize}

\begin{itemize}
\item
\verb|DIR_ORIGEN| es el directorio en el host

\item
\verb|DIR_DESTINO| es el directorio en el contenedor

\begin{itemize}
\item
En el montaje de ficheros tradicional en Unix, es necesario que exista
el punto de montaje. Aquí, no
\end{itemize}

\item
Ambos directorios deben estar especificados con path absoluto

\item
No puede haber espacios antes ni después de la coma
\end{itemize}


Ejemplo:


  \begin{scriptsize}
  \begin{verbatim}
docker run -it -h c02  --name c02 --rm \
   -v $HOME:/home/$USER  \
   jperez/imagen01
  \end{verbatim}
  \end{scriptsize}

\end{frame}








%%---------------------------------------------------------------
\begin{frame}[fragile]
\frametitle{Volumen}

Es un disco virtual creado y gestionado por docker.

Se puede almacenar
\begin{itemize}
\item

Como subdirectorio del host
(en linux, por omisión en
\verb|/var/lib/docker/volumes|)

Aunque no se recomienda que el host acceda directamente
al volumen

\item
En host remotos o en la nube, Docker ofrece para ello
diferentes drivers
\end{itemize}


Características:

\begin{itemize}
\item
Son más fáciles de transportar y respaldar que los bind mounts
\item
Tienen mejores prestaciones para ser compartidos entre varios
contenedores
\item
Se pueden cifrar
\end{itemize}

\end{frame}



%%---------------------------------------------------------------
\begin{frame}[fragile]
\frametitle{tmpfs}

Un montaje de tipo \emph{tmpfs} se usa para datos temporales

\begin{itemize}
\item
Es un sistema de ficheros especialmente eficiente porque se almacena en RAM

\item
Si creamos una imagen a partir del contenedor, el contenido de los montajes
tmpfs no se almacena


\end{itemize}

\end{frame}



%%---------------------------------------------------------------
\begin{frame}[fragile]
\frametitle{Uso de sshfs}
Como hemos visto, los bind mounts permiten montar dentro de un
contenedor directorios ubicados en el host docker

\begin{itemize}
\item
Para montar directorios en cualquier otro lugar de internet, podemos usar
por ejemplo sshfs

\item
Para ello es necesario añadir a la orden 
\verb|docker run| los siguientes parámetros

\begin{itemize}
\item
En docker 17.10

\verb|--privileged|

\item
En versiones más modernas de  docker 

\verb|--cap-add SYS_ADMIN --device /dev/fuse|
\end{itemize}

Para averiguar tu versión de docker:
\verb|docker --version|



\begin{itemize}
\item
En un entorno de producción habría que usar estas opciones con precaución,
puesto que incremeta mucho los privilegios del contenedor dentro del host
\end{itemize}



\end{itemize}

\end{frame}




%%---------------------------------------------------------------
\begin{frame}[fragile]
\frametitle{docker hub}
Para subir nuestras imágenes al registro docker hub
\begin{enumerate}
\item
Creamos una cuenta en
\verb|hub.docker.com|

\item
Creamos nuestras imágenes usando como prefijo nuestro login en dockerhub


\verb|docker build -t mi_usuario/mi_imagen|

\item
Abrimos una sesión en docker hub desde la shell con la orden

\verb|docker login|


\item
Subimos la imagen

\verb|docker push mi_usuario/mi_imagen|

\end{enumerate}

\end{frame}

\subsection{Networking}
%%---------------------------------------------------------------
\begin{frame}[fragile]
\frametitle{Configuración de red}
Al instalar docker se crean 3 redes 
\begin{itemize}
\item
bridge

Segmento privado dentro del host,
\verb|172.17.0.0/16|,
 al que se
conectan por omisión todos los contenedores

\item
null

Red nula, aisla los contenedores de la red

\item
host

El contenedor comparte la red con el host, mismos interfaces,
direcciones y puertos
\end{itemize}

  \begin{scriptsize}
  \begin{verbatim}
koji@mazinger:~$ docker network ls
NETWORK ID          NAME                DRIVER              SCOPE
787cf305d42c        bridge              bridge              local
256d470b6133        host                host                local
086e801223bb        none                null                local
  \end{verbatim}
  \end{scriptsize}
\end{frame}




%%---------------------------------------------------------------
\begin{frame}[fragile]
\frametitle{}
\begin{itemize}
\item
Para conectar un contenedor a una red, basta lanzarlo con
\verb|--network=<nombre_red>|

Ejemplo

  \begin{footnotesize}
  \begin{verbatim}
docker run -it -h c03 --name c03 --rm  --network=host test/im03
  \end{verbatim}
  \end{footnotesize}

\item
Para crear una nueva red (un nuevo segmento)

  \begin{footnotesize}
  \begin{verbatim}
docker network create --subnet 192.168.12.1/24  mired
  \end{verbatim}
  \end{footnotesize}
\end{itemize}

\end{frame}



%%---------------------------------------------------------------
\begin{frame}[fragile]
\frametitle{Configuración en español}
Las imágenes base de las distribuciones son esqueletos mínimos,
normalmente tendremos que personalizarlas 


Por ejemplo, para configurar el idioma. En nuestro caso, español.
Instalaremos en la imagen el paquete 
\verb|locales|,
invocaremos a localedef con los parámetros adecuados y definiremos
la variable de entorno LANG

  \begin{scriptsize}
  \begin{verbatim}
FROM ubuntu:16.04
RUN apt-get update && apt-get upgrade -y && \
  apt-get install -y  locales  && \
  localedef -i es_ES -c -f UTF-8  \
  -A /usr/share/locale/locale.alias es_ES.UTF-8
ENV LANG es_ES.UTF-8
COPY entrypoint.sh /
ENTRYPOINT ["/entrypoint.sh"]
  \end{verbatim}
  \end{scriptsize}

\begin{itemize}
\item
La instrucción ENV del Dockerfile define variables de entorno dentro de la imagen
\end{itemize}

\end{frame}

%%---------------------------------------------------------------
\begin{frame}[fragile]
\frametitle{}
Lo habitual es usar una única instrucción RUN en cada Dockerfile,
para evitar las imágenes intermedias.

Pero también podemos usar varias instrucciones, para que resulte
más legible.

Ejemplo:
  \begin{scriptsize}
  \begin{verbatim}
FROM ubuntu:16.04
RUN apt-get update && apt-get upgrade -y 
RUN apt-get install -y  locales  
RUN localedef -i es_ES -c -f UTF-8  \
    -A /usr/share/locale/locale.alias es_ES.UTF-8
ENV LANG es_ES.UTF-8
COPY entrypoint.sh /
ENTRYPOINT ["/entrypoint.sh"]
  \end{verbatim}
  \end{scriptsize}


Observa que 
\begin{itemize}
\item
El slash invertido al final de línea (\verb|\|) une dos líneas físicas
en una misma línea lógica

\item
El doble ampersand (\verb|&&|) separa sentencias dentro de la misma
instrucción RUN

\end{itemize}

\end{frame}


\section{Servidor docker remoto}
%%---------------------------------------------------------------
\begin{frame}[fragile]
\frametitle{Servidor docker remoto}
\begin{itemize}
\item
En la configuración más sencilla, el servidor de docker está en la misma
máquina que el cliente, el ejecutable incluye ambas funciones

\item
Pero también puede ubicarse en una máquina remota. Esto es útil, por ejemplo

\begin{itemize}
\item
Cuando el cliente no es linux 64 bits

\item
Cuando el cliente no tiene privilegios de root en la máquina local 
(nuestro caso en el laboratorio)

\item
Para arquitecturas distribuidas, equilibrio de carga, en la nube, etc

\end{itemize}
\end{itemize}





\end{frame}
%%----------------------------------------------
\begin{frame}[fragile]
Configuración del servidor


(Es tarea del administrador de red, en el laboratorio normalmente no tendrás que preocuparte de esto)
\begin{itemize}
\item
1.1 Instalación de docker-machine
\item
1.2 Configuración del servidor
\item
1.3 Comprobación
\item
1.4 Obtención de variables de entorno
\end{itemize}


Configuración del cliente

\begin{itemize}
\item
2.1 Obtención de credenciales 
\item
2.2 Ubicación de credenciales 
\item
2.3 Asignación de variables de entorno 
\item
2.4 Comprobación final
\end{itemize}


\end{frame}



\subsection{Configuración del servidor}
%%---------------------------------------------------------------
\begin{frame}[fragile]
\frametitle{1.1 Instalación de docker-machine}

El servidor se configura con
\verb|docker-machine|, una herramienta para crear, gestionar y desplegar
servidores

Instalación de docker-machine en el cliente del administrador de red

  \begin{tiny}
  \begin{verbatim}
curl -L \
https://github.com/docker/machine/releases/download/v0.12.2/docker-machine-`uname -s`-`uname -m` \
>/tmp/docker-machine ;
chmod +x /tmp/docker-machine  ;
sudo cp /tmp/docker-machine /usr/local/bin/docker-machine ;
  \end{verbatim}
  \end{tiny}


\end{frame}



%%---------------------------------------------------------------
\begin{frame}[fragile]
\frametitle{1.2 Configuración del servidor}

El administrador de red ejecuta este script en su cliente
  \begin{scriptsize}
  \begin{verbatim}
#!/bin/bash
IP=  xxx.xxx.xxx.xxx 
USUARIO=admin
PRIVATE_KEY=/home/admin/.ssh/id_ed25519
HOST_NAME= xxxx

docker-machine create --driver generic \
    --generic-ssh-user $USUARIO \
    --generic-ip-address=$IP  \
    --generic-ssh-key $PRIVATE_KEY \
      $HOST_NAME
  \end{verbatim}
  \end{scriptsize}

El servidor puede ser una máquina linux 64 bits cualquiera, no necesita ningún software instalado
(ni siquiera docker, docker-machine se encarga de instalarlo)

\end{frame}


%%---------------------------------------------------------------
\begin{frame}[fragile]
\frametitle{}
\begin{itemize}
\item
IP

Dirección de la máquina donde estará el servidor de contenedores

\item
USUARIO

Usuario en el servidor para administrar Docker. Necesita privilegios
para ejecutar sudo,  sin necesidad de teclear ninguna contraseña
\begin{itemize}
\item
Para ello, añadir a la última línea del fichero  \verb|/etc/sudoers| del servidor

\verb|nombre_usuario ALL=NOPASSWD: ALL|



\end{itemize}

\item
PRIVATE\_KEY

Clave privada ssh del usuario administrador en el servidor


\item
HOST\_NAME


Nombre del servidor
\end{itemize}

\end{frame}





%%---------------------------------------------------------------
\begin{frame}[fragile]
\frametitle{1.3 Comprobación}

El administrador comprueba que el servidor se está ejecutando
  \begin{scriptsize}
  \begin{verbatim}
admin@hostadmin:~$ docker-machine ls
NAME   ACTIVE   DRIVER    STATE     URL                        SWARM   DOCKER    ERRORS
xxx    -        generic   Running   tcp://xxx.xxx.xx.xx:2376           v1.12.6   
  \end{verbatim}
  \end{scriptsize}

\end{frame}



%%---------------------------------------------------------------
\begin{frame}[fragile]
\frametitle{1.4 Obtención de Variables de entorno}
El administrador obtiene las variables de entorno con las que los clientes
se conectarán al servidor, así como la ubicación de las credenciales

  \begin{scriptsize}
  \begin{verbatim}
admin@hostadmin:~$ docker-machine env xxxx
export DOCKER_TLS_VERIFY="1"
export DOCKER_HOST="tcp://xxx.xxx.xx.xx:2376"
export DOCKER_CERT_PATH="/home/admin/.docker/machine/machines/xxxx"
export DOCKER_MACHINE_NAME="xxxx"
# Run this command to configure your shell: 
# eval $(docker-machine env xxxx)
  \end{verbatim}
  \end{scriptsize}
\end{frame}


\subsection{Configuración del cliente}
%%---------------------------------------------------------------
\begin{frame}[fragile]
\frametitle{2.1 Obtención de credenciales }


El administrador de red facilita a los usuarios las credenciales
para usar el servidor



\begin{itemize}
\item
Las credenciales son los siguiente ficheros:

  \begin{scriptsize}
  \begin{verbatim}
ca.pem	cert.pem  config.json  id_rsa  id_rsa.pub  
key.pem  server-key.pem  server.pem
  \end{verbatim}
  \end{scriptsize}

\end{itemize}


\end{frame}
%%----------------------------------------------
\begin{frame}[fragile]
\frametitle{2.2 Ubicación de credenciales }



Los usuarios dejan las credenciales en el lugar convenido, por omisión será

  \begin{scriptsize}
  \begin{verbatim}
~/.docker/machine/machines/xxxx
  \end{verbatim}
  \end{scriptsize}

(Donde xxxx es el nombre del servidor)


\end{frame}
%%----------------------------------------------
\begin{frame}[fragile]
\frametitle{2.3 Asignación de variables de entorno }


Los usuarios asignan las variables de entorno que les ha facilitado
el administrador. Un lugar conveniente es el fichero \verb|~/.bashrc|



  \begin{scriptsize}
  \begin{verbatim}
export DOCKER_TLS_VERIFY="1"
export DOCKER_HOST="tcp://xxx.xxx.xx.xx:2376"
export DOCKER_CERT_PATH="$HOME/.docker/machine/machines/xxxx"
export DOCKER_MACHINE_NAME="xxxx"
  \end{verbatim}
  \end{scriptsize}

\end{frame}


%%---------------------------------------------------------------
\begin{frame}[fragile]
\frametitle{2.4 Comprobación final}

Comprobación de las variables de entorno

  \begin{scriptsize}
  \begin{verbatim}
mgarcia@alpha:~$ env |grep DOCKER
DOCKER_HOST=tcp://xxx.xxx.xxx.x:2376
DOCKER_MACHINE_NAME=xxxx 
DOCKER_TLS_VERIFY=1
DOCKER_CERT_PATH=/home/alumnos/mgarcia/.docker/machine/machines/xxxx
  \end{verbatim}
  \end{scriptsize}


Ejecución de un contenedor


  \begin{scriptsize}
  \begin{verbatim}
magarcia@alpha:~$ docker run ubuntu echo holamundo
holamundo
  \end{verbatim}
  \end{scriptsize}


\end{frame}




%%---------------------------------------------------------------
\begin{frame}[fragile]
\frametitle{Servidor de SSH en el contenedor}
Para un contenedor en producción, no es recomendable habilitar 
el demonio de ssh
% https://jpetazzo.github.io/2014/06/23/docker-ssh-considered-evil/
\begin{itemize}
\item
Implica tener un segundo proceso, que no es natural en Docker

\item
No es buena idea dejar contraseñas dentro de un contenedor
¿cómo actualizarlas?

\item
El código dentro del contenedor es responsabilidad del equipo de desarrollo.
Pero el acceso y las políticas, compete a explotación
\end{itemize}


En esta asignara sí lo haremos, porque el objetivo es enseñar
cómo funciona el acceso por ssh, que es lo habitual en máquinas
físicas y máquinas virtuales tradicionales


\end{frame}


%%---------------------------------------------------------------
\begin{frame}[fragile]
\frametitle{}
¿Es necesario acceder por ssh?
\begin{itemize}
\item
Para actualizar el sistema

No. El contenedor entonces tendría estado (las actualizaciones).
Lo recomendable es crear un nuevo contenedor con la actualización. 
\item
Para ver logs

No. El contenedor tendría estado. Lo recomendable es llevar los logs
a un volumen

\item
Para iniciar y detener demonios

No. Se pueden enviar señales


\item
Para editar la configuración

No. Lo recomendable es crear un nuevo contenedor

\item
Para depurar el servicio

No. Se puede abrir una shell desde el servidor de contenedores

\end{itemize}

\end{frame}



%%---------------------------------------------------------------
\begin{frame}[fragile]
\frametitle{}

Si realmente queremos instalar el servidor en un contenedor:

Dockerfile 
  \begin{tiny}
  \begin{verbatim}
FROM ubuntu:16.04
RUN apt-get update && apt-get install -y openssh-server
RUN mkdir /var/run/sshd
# Permitir que los usuarios hagan login mediante el demonio sshd
RUN sed 's@session\s*required\s*pam_loginuid.so@session optional pam_loginuid.so@g' -i /etc/pam.d/sshd

# Con sshd, ENV no funciona. Para fijar una variable de entorno:
# RUN echo "export MI_VARIABLE=mivalor" >> /etc/profile

COPY entrypoint.sh /

EXPOSE 22
ENTRYPOINT ["/entrypoint.sh"]
  \end{verbatim}
  \end{tiny}

entrypoint.sh
  \begin{scriptsize}
  \begin{verbatim}
#!/bin/bash
/usr/sbin/sshd 
/bin/bash
  \end{verbatim}
  \end{scriptsize}


\begin{itemize}
\item
La instrucción EXPOSE indica en qué puerto (TCP) atiende peticiones el contenedor
\item
Realmente esta instrucción no hace nada, es solo un \emph{mensaje} del autor 
del contenedor para quien vaya a usar el contenedor
\end{itemize}

\end{frame}



%%---------------------------------------------------------------
\begin{frame}[fragile]
\frametitle{Sesiones gráficas}
En un contenedor podemos lanzar aplicaciones gráficas
\begin{itemize}
\item
Si el cliente y el servidor de Docker están en la misma máquina y ambas son
Unix, podemos usar \emph{X11 Forwarding}

  \begin{scriptsize}
  \begin{verbatim}
docker run -ti --rm \
       -e DISPLAY=$DISPLAY \
       -v /tmp/.X11-unix:/tmp/.X11-unix \
       mi_imagen
  \end{verbatim}
  \end{scriptsize}
\item
Una solución más general es VNC

Aquí se describe:

\url{http://gsyc.urjc.es/~mortuno/vnc.pdf}


\end{itemize}

\end{frame}




%%---------------------------------------------------------------
\end{document}
